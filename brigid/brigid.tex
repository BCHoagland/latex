\documentclass[twoside,10pt]{article}
\usepackage{url}
\usepackage{structure}
\usepackage{indentfirst}

\usepackage[
	backend=biber,
	autocite=footnote,
	style=authortitle-ibid
]{biblatex}
\addbibresource{brigid.bib}

%--------------------------------------------------------------------------------
% Title
%--------------------------------------------------------------------------------

\begin{document}

\begin{center}
	\vspace*{20mm}
        {\huge \spacedsc{An Education in Reality}}\\
	\vspace{5mm}
	{\Large \spacedsc{The Priesthood and the Sacramentality of Creation}}\\
	\vspace*{\fill}
        {\large \spacedsc{Brigid Hoagland}}\\
	\vspace{2mm}
	{\large \spacedsc{5 May 2021}}\\
	\vspace{2mm}
	{\large \spacedsc{Dr. Bieler}}\\
	\vspace{2mm}
	{\large \spacedsc{Sacramentality of the Fathers}}
\end{center}

\thispagestyle{empty}
\newpage

%--------------------------------------------------------------------------------
% The Actual Document
%--------------------------------------------------------------------------------

%%%%%%%%%%%%%%%%%%%%
% Introduction
%%%%%%%%%%%%%%%%%%%%

\section{Introduction}

The father has the responsibility of introducing his child to the world. The child awakens to the world in the context of an overwhelming positivity. Upon opening its eyes to reality, the child is first met with his mother’s smile, an affirmation of the goodness of his being and the goodness of the world. In these primary moments, the father stands at a distance from the child. While the mother reveals a likeness and union with the child, the father represents that which is other. In the process of development, the child must learn to differentiate himself from his mother in order to explore the world, a process in which the father places an essential role. The father has the responsibility to open the child to the totality of existence, to point him towards the world that is beyond his immediate context. The father places the child in light of the whole. 

In order to understand the role of the priest in the lives of the faithful, it is first necessary to ground the priesthood in a proper understanding of fatherhood. In Christ, human fatherhood is transformed and the priest is offered a share in the fulfillment of fatherhood which takes a virginal form. Through his fatherhood, the priest opens up a reality that is beyond him. His spiritual fatherhood opens the faithful to the eschatological reality. In his being, the priest points the faithful towards the fulfillment of their humanity in communion with God in the eschaton. 

In this paper, I will argue that the priest reveals the truth of the created order through his fatherhood. I will address this by (1) discussing the relationship between the priesthood and fatherhood. Next, I will (2) propose that the priest has a responsibility to educate the faithful to understand the sacramental structure of reality, revealing the meaning and order in creation. Finally, I will (3) explain how the priest reveals the fulfillment of creation in Christ, particularly by discussing the place of the created order in the liturgy. Through this paper, I hope to clarify the significance of the role of the priest in the introduction of the faithful to the reality of Christ and the truth of creation. Furthermore, I would like to suggest that the priestly vocation has a crucial task to heal the break in modernity between the material and spiritual by presenting an understanding of sacramental ontology. The purpose of this paper is to argue that the priesthood educates the faithful in a sacramental worldview, opening them to see the mystery present in created reality that finds its fulfillment in Christ.

%%%%%%%%%%%%%%%%%%%%
% Revelation of Fatherhood: the Sacramentality of the Priesthood
%%%%%%%%%%%%%%%%%%%%

\section{Revelation of Fatherhood: the Sacramentality of the Priesthood}

The priest, by way of Holy Orders, sacramentally represents Christ to the Church. By acting \textit{in persona Christi}, he puts aside his individuality to reveal that he ``stands in place of Another- Christ."\autocite[137]{rat2} The vestments of the priest remind him that he has put on Christ, who transformed his humanity through incorporation into His Body.\footnote{Cf. Romans 13:14; Galatians 3:27.} Joseph Ratzinger writes, ``Vestments are a challenge to the priest to surrender himself to the dynamism of breaking out of the capsule of self and being fashioned anew by Christ and for Christ."\autocite[138]{rat2} In representing Christ to the Church, the priest also reveals Christ as the sacrament of the Father. John 17:3 states, ``And this is eternal life, that they know thee the only true God, and Jesus Christ whom thou hast sent."  Because Christ manifests the Trinity to the world, this manifestation must be reflected in the priesthood. Christ came that man may know the Father. Thus, the priest must reveal the Fatherhood of God by nature of his sacramental office. In his article, ``Priestly Ministry at the Service of Ecclesial Communion," Marc Ouellet writes, ``Through Baptism, one is introduced into the mystery of divine filiation. Through Confirmation, one is inserted into the mission of the Spirit. If this is so, would it not be true that the sacrament of Holy Orders is a particular sign of the divine paternity that is given to the Church in Jesus Christ?"\autocite[683]{ouellet} By representing Christ, the sacrament of the Father, the priest reveals both the mystery of divine filiation and paternity. In revealing the Father, he witnesses to the transcendent origin of love in God. As Granados explains, ``The priest helps us grasp…that it is God who loved us first and, as the original source of communion, allowed us to love in return."\autocite[208]{granados} By virtue of the sacrament of Holy Orders, the priest reveals Christ who points to a relation to the Father. The authority given in the sacrament has its source in the Father and is joined to Christ’s power. The priest gives what he himself cannot give, and thus is the mediator of an order that he has received.\autocite[Cf.][620-1, ``No human being can declare himself to be a priest…It can only be received by a sacrament, which is God’s. A mission can be received only from the Sender, from Christ in his sacrament, by which someone becomes Christ’s voice and hands in the world."]{rat1}  The priest ``actualizes the ministry of Christ" and thus reveals the Father, by whom the Son was sent.\autocite[685]{ouellet} Granados continues, ``The sacrament, by identifying the priest with Christ the head, refers him to Christ’s fatherhood, that is, to his work inasmuch as he gives new life to humanity, and makes visible God the Father’s original love.'' \autocite[208]{granados} The priesthood draws its originating source in the Father’s love who remains faithful to humanity by calling men in his mercy to continue the work and mission of his Son in the life of the Church.\autocite[Cf.][15, ``In the Church and on behalf of the Church, priests are a sacramental representation of Jesus Christ - the head and shepherd - authoritatively proclaiming his word, repeating his acts of forgiveness and his offer of salvation - particularly in baptism, penance and the Eucharist, showing his loving concern to the point of a total gift of self for the flock, which they gather into unity and lead to the Father through Christ and in the Spirit."]{jp2} The priest, as an image of Christ, is a visible manifestation of the Father’s love. 

In addition to revealing the Father, the priest is also granted a participation in a new fatherhood won by Christ. In natural fatherhood, the identity of a man is transformed through the birth of his child. He recognizes that he has been entrusted with the gift of the life of another who will one day surpass him. The child opens the father towards transcendence, to an awareness of the Other who gives existence to man. Furthermore, the child opens him to his own future, to the time when he will be replaced by his children. As Charles Péguy writes, ``He thinks tenderly of the time when he will be no longer and his children will take his place. On earth. Before God."\autocite[16]{peguy} Through his fatherhood, the man is opened to that which is beyond him. Fatherhood is brought to fulfillment in Christ, opening the priest to the ever-new horizons of eternal life. His fatherhood is wrought with an eschatological meaning that gives him the sacramental character of his office.\autocite[209]{granados} Through this new fatherhood, the priest is given the capacity to bear sons into eternity.\autocite[210, ``The capacity of generating in such a way, into eternity, leaves an indelible character on the being of the person who receives it…the priest is identified with Christ precisely as the source of life eternal." Through his office, the priest shares in a power that he has received, the power to lead men to their salvation.]{granados} As Fr. Antonio López writes, ``Priestly fatherhood, then, runs deeper than a spiritual influence on a few people. It is the sacramental collaboration with the Spirit of the Father in the Son’s birth in the believer."\autocite[267]{lopez} As Adam generated men into sin, Christ, the new father, generates men into eternal life. The spiritual fatherhood of the priest not only imprints a character on his soul; it also affects the priest’s corporeality. Spiritual fatherhood is, thus, intimately linked with the body. As Granados writes, ``…it is the body that makes possible the openness of the father’s life toward the child."\autocite[210]{granados} A spiritual father gives to the child what he has lived in his concrete experience on earth. The Kingdom that Jesus brought was revealed in and through his very self. As Ratzinger writes, ``Jesus does not communicate content that is independent of his person."\autocite{rat3} The words of Jesus reveal a deeper meaning in that they bear ``the reality of the Incarnation and theme of the Cross and resurrection. And in this deep way, word and action are combined."\autocite{rat3} He lived and suffered the words He proclaimed. The priest is called to bear the Word in his body. His preaching and ministry must come forth from his own experience of Christ in the world. The Word has to become incarnate in his life for him to communicate the gift he has received.\autocite[Cf.][499, ``A good rule is never to invent anything: we can only communicate something that has come out of our own flesh, that we have learned through hard experience. We must not invent answers that we do not have." Man can only truly share what he has rejoiced or suffered through.]{camisasca} The content of the faith is not merely intellectual knowledge, but ``the fullness of God’s bodily communication in his Son."\autocite[214]{granados} Thus, bodiliness is essentially involved in the priest’s fatherhood.

The spiritual fatherhood of the priest is properly understood in light of eschatology. Granados explains, ``It is in the body that fatherhood is able to speak about the future, about the ultimate goal of human existence. This openness of the body toward establishing a relationship with another, thus opening up a new future, is brought to fulfillment in Christ’s action, because he communicates the eschatological future, the final embrace of God the Father."\autocite[204]{granados} In and through his bodily reality, the priest points the faithful to that which is beyond him, namely, eternal life. While the priest cannot answer all of the problems of his people, he can serve as a reference point that helps people see their lives in light of the eternal. Through his vocation, he reveals the fulfillment and elevation of all things in Christ. The priest stands before the Church as a sign of the fulfillment of man in Christ, witnessing to the origin of man in the Father’s eternal love and pointing man towards his fulfillment in communion with the Father in life eternal. Through his person, the priest opens the faithful to the eternal order of love that informs all of reality and points it towards its fulfillment. 


%%%%%%%%%%%%%%%%%%%%
% An Education in Seeing: The Priesthood and the Sacramentality of Creation
%%%%%%%%%%%%%%%%%%%%

\section{An Education in Seeing: The Priesthood and the Sacramentality of Creation}

In his book, The Portal of the Mystery of Hope, Charles Péguy eloquently writes about the revelatory nature of creation, ``Yes, I am so resplendent in my creation. Upon the face of the mountains and on the face of the plains. In bread and in wine and in the man who tills and in the man who sows and in the harvest of grain and in the harvest of grapes."\autocite[4]{peguy} The world speaks of God. Through a horizontal encounter with God in creation, man is opened to transcendence. Thus, the world has a sacramental structure. The world reveals a symbolical reality that is capable of opening towards an encounter with Christ. Most Christians understand the idea of ‘sacrament’ yet fail to see how it relates to the whole of life. Ratzinger writes, ``To him a sacrament seems to be something strange that he is inclined to relegate to a magical or mythical age of mankind; he cannot quite figure out where it belongs in a rational and technological world."\autocite[169]{rat2} To understand the concept of sacrament, it is first essential to understand the symbolical nature of the cosmos. The natural signs themselves reveal a deeper meaning that opens to the sacred. The world is not brute matter to be manipulated for man’s purpose; it is already oriented and ordered towards Christ. The sacred is discovered within the profane. When the veil of the temple was torn, the boundary between the sacred and profane disappeared. As Ratzinger explains, ``The cult is no longer set apart from ordinary life, but holiness dwells in everyday things."\autocite[400]{rat2} Because the Christ-event reveals the meaning of the whole of reality, the priest has a responsibility to reveal the sacramental character of the world. Consequently, to present the meaning of the whole, the priest must first have a proper understanding of reality.\autocite[Cf.][Ratzinger links the crisis in the priesthood and the lack of vocations to a loss of understanding of reality. In modernity, the sacred is replaced by the functional and the world is seen simply as brute matter. This conception of the world alters the meaning of life. In order to understand the identity and mission of the priest, it is first essential to recover a more adequate understanding of the order and nature of the world so as to be able to recognize the priest’s role within that order.]{rat3} As Christ entered the depths of reality in the Incarnation, the priest too must immerse himself in the world so as to bring it to its fulfillment.\autocite[Cf.][John Chrysostom exemplifies this in his Baptismal Instructions. In order to explain the reality taking place, he draws analogies to real-world examples. In his Ninth Baptismal Instruction, he likens the renewal of nature in Baptism to a gold statue covered in dirt that must be melted down in order to be purified. He says that Christ does the same in baptism when ``He plunges it into the waters as into the smelting furnace and lets the grace of the Spirit fall on it instead of the flames." John Chrysostom uses this image to emphasize that Baptism is not just an exterior cleansing but a total renewal.]{chrysostom}

The priest is tasked with the responsibility of re-establishing the mysterious character of the world. The rationalist modern view eliminates the realm of mystery hidden within the created order. This must be reclaimed if the priest is to open up an encounter with the mystery through the world as Christ did. Faith’s object is reality. Through faith, man is able to see what is. In the visible, man learns to apprehend the invisible, the hidden mystery written within things. Henri de Lubac emphasizes this point:

\begin{quote}
	The Mysterion…designates precisely the mutual relationship of the one with the other (=of the sensible sign to that which it signifies), a relationship which is hidden from the eyes of the profane (wherein lies the mystery), but progressively revealed to the believers who choose to submit themselves deliberately to the school of the Logos. This mystery lies in the aptitude for being revealed, with regard to the thing itself, and in the aptitude for revealing effectively with regard to the sign.\autocite[52]{lubac}
\end{quote}

The natural sign draws man into the mystery because the sign itself participates in a higher reality. Faith forces reason to pay attention to the visible. It helps man to recognize the ``more" that is taking place beyond the signs. Through faith, ``…all these hidden things can be understood and deduced from the things that are seen."\autocite[221]{origen} Thus, faith is both reliant on the created order and essential to understand the sacraments. If man can only see God outside of things, He is not a living God. Because God is present here, man can encounter Him in the world, but in a hidden way. This hiddenness requires faith, a faith that must grow and expand through the guidance of the priest. By helping his people to grow in faith, the priest teaches them how to see God in the world and thus to receive Him there.

The symbolic nature of the cosmos allows for the world to be a revelatory sign of the mystery. Because the mystery is written within nature, natural signs can be used to point to deeper realities. The logic of the sacraments is understood in Christ’s self-communication by means of metaphors and symbols. Because man is a unity of body and soul, he is moved more by truth presented in symbols. By having to discern the symbols, man becomes involved with them and thus incorporated into them. In order to understand the analogies used in Scripture, it is first essential to have knowledge of earthly things.\autocite[Cf.][219-220, In order to understand the parable of the mustard seed, one must have knowledge of the reality of the seed itself. Origen writes that the ``nature of the seed is such that, though it is the least of all seeds…it nevertheless becomes greater than all herbs." The natural understanding of the mustard seed is used to describe the perfection of faith and the growth of the kingdom of heaven. These analogies are possible because of the nature of the mustard seed itself.]{origen} The earthly instructs man to contemplate the heavenly. For example, ``‘Water’ is not just H2O, a chemical compound," it is a ``mystery of refreshment that creates new life in the midst of despair."\autocite[161]{rat2} Water has properties that can refresh, purify, cleanse, and even destroy. Because the cosmos is symbolical, water is more than water. It points to a greater mystery. Ambrose elaborates, ``You have seen water: not all water cures, but the water which has the grace of Christ cures."\autocite[274]{ambrose} In the Baptism of Christ, the Holy Spirit descends upon the water and consecrates it. Thus, water is placed under its fulfillment in Christ. The natural elements of water are elevated and given new meaning in their relationship to Christ. In Baptism, man’s attachment to sin is destroyed and his soul is purified so as to share in the life of Christ. Water now has a role in revealing the mystery of man’s salvation. Through the sacrament, the priest brings out a deeper understanding of the world and reveals the meaning of the natural order and its orientation towards God. The sacred becomes interesting because it takes up and resembles the human. If all of reality is seen as an image of heaven, then everything becomes fascinating. To reach the invisible, man has to pass through the visible. Through the sacraments, man makes his ascent. The symbols given in and through the world point to a future reality that man, through the sacraments, can participate in during the present. 

Modernity is experiencing a crisis in sacramentality. Due to the loss of a true sense of reality, man is alienated from the sacraments now more than ever. Ratzinger explains: 

\begin{quote}
	In a time when we have grown accustomed to seeing in the substance of things nothing but the material for human labor – when, in short, the world is regarded as matter and matter as material – initially there is no room left for that symbolic transparency of reality toward the eternal on which the sacramental principle is based.\autocite[153-4]{rat2}
\end{quote}

The sacraments are grounded on a symbolic understanding of the world. Because modernity views the world as brute matter waiting for human manipulation, it is difficult for the modern mind to comprehend ``how a ‘thing’ can become a ‘sacrament’."\autocite[154]{rat2} Contemporary thinkers, such as Hobbes and Kant, assert that there is a radical dichotomy between the material and the spiritual. Man is told not to accept the world as it reveals itself, but, rather, to determine it through one’s ideas. Modern man thus decides the meaning of the world rather than receiving it. These thinkers reject any real relations between the visible and the invisible, thus abandoning the metaphysical idea that the earthly and heavenly are linked.\autocite[7]{boersma} Since man is seen as an autonomous spirit who constructs himself and the world out of his own choices, it becomes difficult to understand why God would require that man discover Him in and through the material world.\autocite[154, If someone has a functionalist view of the cosmos, rather than a symbolic view, he is unable to see how matter can play a decisive role in man’s encounter with the divine. If matter is merely a brute fact, it cannot be revelatory of God or His actions. Thus, modern man questions if God is really working through the sacraments and fails to comprehend how sacramental actions can be decisive for his life. ``Is it not an unwarranted assumption to imagine that pouring a little water on a person should be existentially decisive for him? Or the imposition of hands by a bishop, which we call Confirmation? Or the anointing with a bit of consecrated oil that is given by the Church to a sick person to accompany him on the last stretch of his journey? And here and there even priests are starting to ask whether the laying on of hands by the bishop, which is called Holy Orders, really can signify an irrevocable commitment of a man’s life down to his final hour and whether in this case the significance of the rite has not been overrated…"]{rat2} In the modern conception, the world is not revelatory of any meaning unless man puts it there himself. But, if man is not autonomous, but corporeal, then it becomes clear that man must transcend the world in and through the world itself. Ratzinger writes, ``…his relationship to God, if it is to be a human relationship to God, must be just as man is: corporeal, fraternal, historical. Or there is no such thing."\autocite[166]{rat2} When man views the world from a functionalist perspective, believing that matter is simply for his enterprises, he is unable to let the world speak to him and thus to discover its symbolic character. In order to understand the sacraments, man must first be taught how to properly view himself and his place in the world. 


%%%%%%%%%%%%%%%%%%%%
% All Things in Christ: Christ as the Fulfillment of the Natural Order
%%%%%%%%%%%%%%%%%%%%

\section{All Things in Christ: Christ as the Fulfillment of the Natural Order}

Through homilies, catechetical instruction, and the witness of his life, the priest is given the opportunity to bridge the gap in modernity between the spiritual and material and thus to posit a sacramental worldview. The unity between the spiritual and material is necessary for God to act in His Church. God put a sign-character into the nature of things. To deny the symbolism in creation is to deny the work that Christ wants to do in it. If man does not allow for the world to speak Christ, then the world cannot be integrated into the mystery that Christ brings. The dichotomy between the spiritual and material puts the union of Christ and creation into jeopardy. It is of crucial importance, then, that the priest shows the faithful the significance of the world and reveals how Christ works within the created order. The priest must point to what is human and then explain how the human moves towards the divine, henceforth revealing that the world is founded on something greater than itself. The sacramental structure of the world reveals both the essence of man and his relation to God. For example, in the sharing of a meal, man realizes that his existence is one of receptivity. He receives the fruits of the earth and the work of his fellow man. Furthermore, man becomes aware of a transcendent quality in the meal. The meal is more than just a means of nourishment; it allows him to experience a sense of communion that lies beyond the mere material aspects of the meal. Ratzinger states, ``And when he experiences the foundation of his existence in a meal, then he knows that things give him more than they themselves have and are. In this way, however, the meal becomes for him a sign of the divine and the eternal that supports him and all things and men and is the real foundation of his existence."\autocite[158]{rat2} The divine element meets him through his humanity, and particularly through his corporeality. In the sacraments, ``God encounters man in a human way."\autocite[158]{rat2} The sacraments are fundamental because they reveal the significance of both the transcendent and the human. God takes up created matter and leads it to its fulfillment. He shows that the mystery hidden within creation itself is intrinsically ordered towards its fulfillment in Christ. In ``Against All Heresies," Irenaeus describes this: 

\begin{quote}
	And as we are His members, we are also nourished by means of the creation (and He Himself grants the creation to us, for He causes His sun to rise, and sends rain when He wills). He has acknowledged the cup (which is a part of creation) as His own blood, from which He bedews our blood; and the bread (also a part of the creation) He has established as His own body, from which He gives increase to our bodies.\autocite[\nopp V. 2. 2]{irenaeus}
\end{quote}

The sacraments express man’s origin and end in God while also holding the significance of the created order and history. In the sacraments, man discovers the sacramental ontology of creation and its ordination to God. 

In the sacraments, Christ recapitulates creation in Himself, revealing that creation finds its meaning and unity in Him. In the Incarnation, Christ receives his body from the Virgin and takes on flesh and blood and human history. Christ takes up the full spectrum and depth of human experience. As Irenaeus points out, ``…For why did He come down into her if He were to take nothing of her?"\autocite[\nopp III. 22. 2]{irenaeus} He recapitulated the entirety of human experience into Himself, illuminating that He is the fulfillment and meaning of all things. God uses matter to communicate His blessing and the Son becomes these very elements.\autocite[\nopp III. 11. 5]{irenaeus} In Christ’s recapitulation of all things in Himself, man discovers the true meaning of creation. If the priest is the image of Christ on earth, then he has something to reveal about the meaning of things. The sacramental character of his office and the witness of his life points to the fulfillment of man in the eschaton. He lives the heavenly reality on earth. The priest, by acting in persona Christi, makes present this act of recapitulation. As Christ received his body from the Virgin, the priest receives the Body, the Church, on behalf of Christ. In this reception, all of man’s human and earthly existence is received and raised towards its fulfillment. In the gesture of offering, all of mankind is incorporated into Christ and offered to the Father. The actions of the priest in the liturgy disclose the meaning and end of creation. The sacraments give man a ``guarantee of a divine answer in which the open question of being human arrives at its goal and comes to its fulfillment."\autocite[168]{rat2} The priest does not accomplish this through his own actions but through the action of Christ. The fulfillment of all things in Christ, accomplished in part through the priest who acts in persona Christi, is seen most clearly in the context of the liturgy. The liturgy is the place where hope is born. In the liturgy, man ``lives in advance the life to come, the only true life, which initiates us into authentic life…Thus it would imprint on the seemingly real life of daily existence the mark of future freedom, break open the walls that confine us, and let the light of heaven shine down on earth."\autocite[14]{rat4} Worship gives man a share in heavenly existence and this informs his life. Man does not create worship himself; he receives the liturgy and thus the divine comes down to man. Man is united to Christ and drawn into his self-offering to the Father. This reveals the meaning of his life, that he is called to be a son in the Son and to share in Trinitarian life and love. This understanding enlightens the whole of man’s life. Ratzinger states, ``…life becomes real life only when it receives its form from looking toward God. Cult exists in order to communicate this vision and to give life in such a way that glory is given to God."\autocite[18]{rat4} Just as the liturgy is a concrete form of hope for the faithful, so too is the priest a sign of hope. The priest lives in the present the human fulfillment of man in Christ. Through his priestly office he teaches the faithful to see all things in light of Christ, to see how He transforms all of creation and unites all things in Himself. Furthermore, the witness of the priest opens man to see the heavenly reality on earth. The priest opens man to see beyond himself and points him to eternal life. Through the priest, the faithful can come to understand how earthly realities are taken up in the life of the Church and raised and elevated to their fulfillment in Christ. Ratzinger writes:

\begin{quote}
	…the sacraments no longer work by foreshadowing and asking; rather, they are effective as a result of what has already happened, and therein is manifest the act of liberation accomplished by Christ. Man no longer has to rely on his own doing and going toward some undisclosed thing that is yet to come; instead, he can entrust himself to the reality that is already waiting for him and approaches him as something that has already happened.\autocite[180]{rat2}
\end{quote}

Through the sacrament of Holy Orders, the priest makes present to his people the reminder of salvation in Christ. He discloses through his vocation the ways in which Christ brings unity and fulfillment to the whole of his life. His affectivity, skills, and personality are gathered up and offered at the service of Christ on behalf of the Church. Through his concrete human experience, the priest is able to enter the lives of his people and open them to the heavenly reality for which they were made. The Word is fleshed out in his own corporeality and thus his whole human experience is informed by Christ and lived in light of eternity. He witnesses through His life the redeeming power of Christ who gathers all of human existence and raises it to new heights. Furthermore, the priest is called to share in the continuance of Christ’s mission on earth by bringing all of creation into the Body of Christ, the Church. Through his teaching and ministry, he is tasked with going out to the ends of the earth to ensure that all of the created order is ensured a place in the Body of Christ, in the One in whom man finds his fulfillment. 

%%%%%%%%%%%%%%%%%%%%
% Conclusion
%%%%%%%%%%%%%%%%%%%%

\section{Conclusion}

In his spiritual fatherhood, the priest is given the responsibility of introducing his people to the symbolic reality of the world. In his distance from his people, he reveals the distance of God who reaches down in love towards His creature for the sake of his salvation. The priest stands in place of another, of Christ, who opens man to see the meaning of the whole of reality. He must open man to see the totality of his existence, to point him towards his end in eternal life, an end that will inform the whole of his earthly existence. The priest, by living the heavenly reality now, places his people in light of the whole meaning of existence. He has a responsibility to teach and to witness to the symbolic nature of the cosmos that finds its fulfillment in Christ. The priest is tasked with revealing the unity between God and creation and ensuring that all men know from whom they came and to whom they are going. By educating the faithful to see the mystery written into reality and the ordination of all things to Christ, the priest seeks to ensure, as Charles Péguy says, that ``the eternal lacks nothing of the temporal…that the spiritual lacks nothing of the carnal…that eternity lacks nothing of time…that the spirit lacks nothing of the flesh. That the soul so to speak lacks nothing of the body. That Jesus lacks nothing of the Church, of his Church. [H]e must go all the way: That God lacks nothing of his creation."\autocite[66]{peguy}


\newpage
\printbibliography

\end{document}
