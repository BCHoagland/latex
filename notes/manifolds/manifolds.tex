\documentclass[twoside,10pt]{report}
\usepackage{/Users/bradenhoagland/latex/styles/toggles}
\toggletrue{sectionbreaks}
%\toggletrue{sectionheaders}
\newcommand{\docTitle}{Manifolds}
\usepackage{/Users/bradenhoagland/latex/styles/common}
\importStyles{modern}{rainbow}{boxy}

%\renewcommand{\theenumi}{\alph{enumi}}

\begin{document}

\title{Braden Hoagland}{Based on \textit{An Introduction to Manifolds} by L. Tu.}

\tableofcontents

%+-------------------+
%| +---------------+ |
%| |    Chapter    | |
%| +---------------+ |
%+-------------------+
% Foundations: Euclidean Space

\chapter{Foundations: Euclidean Space}

%--------------------------------------------------------------------------------
% Reminders
%--------------------------------------------------------------------------------
\section{Reminders}

We say $f$ is \textbf{real analytic} at $p$ if it's equal to its Taylor series at $p$ in some neighborhood of $p$. Note that if $f$ is real analytic, then it's also $C^{\infty}$ (the converse isn't true in general, though).

\begin{prop}[Baby Taylor's Theorem with Remainder]
	Let $U$ be open in $\R^{n}$ and star-convex wrt $p$. If $f$ is $C^{\infty}$ on $U$, then there are $C^{\infty}$ functions $g_1, \dots, g_{n}$ on $U$ such that
	\[
		f(x) = f(p) + \sum_{i}(x^{i}-p^{i}) g_{i}(x)
	\] and $g_{i}(p) = \frac{\p f}{\p x^{i}} (p)$ for all $i$.
\end{prop}
\begin{proof}
	Since $U$ is star-convex wrt $p$, we can draw a straight line from $p$ to any $x \in U$. Intuitively, $f(x)$ should be $f(p)$ plus all the changes in $f$ along this line. We can use the FToC to formalize this:
	\[
		f(x) - f(p) = \int_{0}^{1} \frac{d}{dt} f(p+t(x-p))\;dt.
	\]
	We can use the chain rule to evaluate $\frac{d}{dt} f(p+t(x-p))$, giving
	\[
		f(x) - f(p) = \sum (x^{i} - p^{i}) \int_{0}^{1} \frac{\p f}{\p x^{i}} (p+t(x-p)) \;dt.
	\] 
	Set $g_{i}(x)$ to be its respective integral in the above sum.
\end{proof}

%--------------------------------------------------------------------------------
% Tangent Vectors as Derivations
%--------------------------------------------------------------------------------
\section{Tangent Vectors as Derivations}



\end{document}
