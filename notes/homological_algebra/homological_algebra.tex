\documentclass[twoside,10pt]{report}
\usepackage{/Users/bradenhoagland/latex/styles/toggles}
\toggletrue{sectionbreaks}
%\toggletrue{sectionheaders}
\newcommand{\docTitle}{Homological Algebra}
\usepackage{/Users/bradenhoagland/latex/styles/common}
\importStyles{modern}{rainbow}{boxy}

%\renewcommand{\theenumi}{\alph{enumi}}

\begin{document}
\tableofcontents

%+-------------------+
%| +---------------+ |
%| |    Chapter    | |
%| +---------------+ |
%+-------------------+
% Chain Complexes

\chapter{Chain Complexes}

%--------------------------------------------------------------------------------
% Chain Complexes
%--------------------------------------------------------------------------------
\section{Chain Complexes}

\warn{Want a more intuitive view of left/right exact functors, maybe in terms of lifts/ extensions.}

\begin{defn}[]
	A \textbf{chain complex} $\mathcal{C}$ is a sequence of $R$-morphisms
\[
	\begin{tikzcd}
	\cdots \arrow[r, "d_{i+2}"] & C_{i+1} \arrow[r, "d_{i+1}"]  & C_i \arrow[r, "d_i"]  & C_{i-1} \arrow[r, "d_{i-1}"]  & \cdots
	\end{tikzcd}
\]
such that $d^2=0$ for all $i$. \textbf{Cochain complexes} are the same, except the boundary maps take you up a level instead of down.
\[
	\begin{tikzcd}
		\cdots \rar{d_{i-1}} & C^{i-1}\rar{d_{i}} & C^{i} \rar{d_{i+1}} & C^{i+1}\rar{d_{i+2}} & \cdots
	\end{tikzcd}
\] 
\end{defn}

The map $d_i$ is the \textbf{boundary operator}, as it is a generalization of the geometric concept of a boundary (note $d^2=0$). Thus an element of $\im d$ is a \textbf{boundary}. Since usual geometric cycles have no boundary, we call the elements of $\ker d$ \textbf{cycles}.

\begin{ex}[]
	Chain complexes generalize the concept of boundaries to objects that don't necessarily have clear cyclic geometric properties. Let $\Omega_n(M)$ denote the space of differential $n$-forms on a manifold $M$, then we have a cochain complex
\begin{center}
\begin{tikzcd}
	\Omega_0(M) \arrow[r, "d"] & \Omega_1(M) \arrow[r, "d"] & \Omega_2(M) \arrow[r, "d"] & \cdots
\end{tikzcd}
\end{center}
where $d$ is the exterior derivative. From this we see that the cycles of $\Omega_0(M)$ (the space of differentiable functions on $M$) are the constant functions.
\end{ex}

A \textbf{morphism of complexes/chain morphism} $f:\mathcal{C}\to \mathcal{D}$ is a sequence of morphisms $f_i:C_i\to D_i$ respecting the boundary map, i.e. making the following diagram commute.
\[
	\begin{tikzcd}
		C_i \arrow[d, "f_i"'] \arrow[r, "d_\mathcal{C}"] & C_{i-1} \arrow[d, "f_{i-1}"] \\
		D_i \arrow[r, "d_\mathcal{D}"']                & D_{i-1}
	\end{tikzcd}
\]

%--------------------------------------------------------------------------------
% Chain Homotopies
%--------------------------------------------------------------------------------

\section{Chain Homotopies}

\begin{defn}[]
Given two chain complexes $\mathcal{A},\mathcal{B}$, two chain morphisms $f,g:\mathcal{A}\to \mathcal{B}$ are \textbf{(chain) homotopic}, written $f \simeq g$, if there are morphisms $s_i:A_i\to B_{i-1}$ such that
\[
	d's + sd = f-g.
\]
If $\mathcal{A}, \mathcal{B}$ are cochain complexes instead, then $s_i:A_{i}\to B_{i+1}$.
\end{defn}
\[
\begin{tikzcd}
	A_{i-1} \rar{d} & A_i \rar{d}\dar[shift right, "f_i"']\dar[shift left]{g_i}\arrow[dl,dashed,"s_i"'] & A_{i+1} \arrow[dl,dashed,"s_{i+1}"] \\
	B_{i-1} \rar["d'"'] & B_i \rar["d'"'] & B_{i+1}
\end{tikzcd}
\] 
\warn{Motivation for this?}

\begin{defn}[]
	A chain morphism $f:\mathcal{A}\to \mathcal{B}$ is a \textbf{homotopy equivalence} if there's another chain morphism $g:\mathcal{B}\to \mathcal{A}$ such that $fg \simeq 1_{B}$ and $gf \simeq 1_{A}$.
	\[
	\begin{tikzcd}
		\mathcal{A} \rar[bend left]{f} & \mathcal{B} \lar[bend left]{g}
	\end{tikzcd}
	\] 
\end{defn}

\begin{prop}
	\label{functors-preserve-homotopy}
Additive functors preserve homotopy equivalence.
\end{prop}
\begin{proof}
	Let $f\simeq g$. If $\mathcal{F}$ is additive and covariant, then $d's+sd = f-g \implies \mathcal{F}(d')\mathcal{F}(s)+\mathcal{F}(s)\mathcal{F}(d) = \mathcal{F}f - \mathcal{F}g$. Thus $\mathcal{F}f \simeq \mathcal{F}g$. If $\mathcal{G}$ is additive and contravariant, then $\mathcal{G}(d)\mathcal{G}(s) = \mathcal{G}(s)\mathcal{G}(d') = \mathcal{G}f-\mathcal{G}g$. Since all the arrows are reversed, the LHS is the right form, so $\mathcal{G}f \simeq \mathcal{G}g$.
\end{proof}


%--------------------------------------------------------------------------------
% Homology
%--------------------------------------------------------------------------------
\section{Homology}

\begin{note}[]
	Big idea: given some module, we always have a way of getting a chain complex (take a resolution). Chain complexes by themselves aren't nice, though: they might be unwieldy or unnatural, similar objects might have dissimilar complexes, etc. Passing to homology makes these problems go away, though, giving us access to nice algebraic invariants.
\end{note}

\begin{defn}[]
	The $n$-th \textbf{homology group} $H_{n}(\mathcal{C})$ of a chain complex $\mathcal{C}$ is the kernel of the map going out of $C_{n}$ quotiented by the image of the map coming into $C_{n}$. Cochain complexes have \textbf{cohomology groups} $H^{n}(\mathcal{C})$ instead.
\end{defn}

\begin{prop}
A chain/cochain complex is exact $\iff$ all its homology/cohomology groups are trivial.
\end{prop}

Thus the (co)homology groups of a (co)chain complex measure how much it fails to be exact.

\begin{prop}
	A morphism of complexes $f:\mathcal{A}\to \mathcal{B}$ induces group morphisms between the complexes' homology/cohomology groups given by $[a] \mapsto [f_{n}(a)]$.
\[
\begin{aligned}
	\begin{tikzcd}
		A^{n-1} \rar\dar & A^{n} \rar\dar & A^{n+1} \dar \\
		B^{n-1} \rar & B^{n} \rar & B^{n+1}
	\end{tikzcd}
	\quad \leadsto \quad
	\begin{tikzcd}
		H^{n}(\mathcal{A}) \dar \\
		H^{n}(\mathcal{B})
	\end{tikzcd}
\end{aligned}
\] 
\end{prop}
\begin{proof}
	This follows from the morphism of complexes respecting the boundary map and thus mapping the kernels and images of the first complex to the kernels and images of the second.
\end{proof}

\begin{prop}
	If $f_{*}$ is the induced (co)homology map of $f$, then $(gf)_{*}=g_{*}f_{*}$.
\end{prop}

\begin{defn}[]
$0\to \mathcal{A}\to \mathcal{B}\to \mathcal{C}\to 0$ is a \textbf{short exact sequence of complexes} if each $0\to A^{n}\to B^{n}\to C^{n}\to 0$ is short exact.
\end{defn}

\newpage

\begin{lem}[Snake Lemma]
	If the following diagram has exact rows,
	\[
	\begin{tikzcd}
		& A \rar\dar{\alpha} & B \rar[two heads]\dar{\beta} & C \rar\dar{\gamma} & 0 \\
		0 \rar & A' \rar[tail] & B' \rar & C'
	\end{tikzcd}
	\] 
	then there is an induced exact sequence
	\[
		\ker \alpha \to \ker \beta \to \ker \gamma \to \cok \alpha \to \cok \beta \to \cok \gamma.
	\] 
\end{lem}

\begin{thrm}[Long Exact Sequence in Cohomology]
If $0\to \mathcal{A}\to \mathcal{B}\to \mathcal{C}\to 0$ is a short exact sequence of complexes, then there is a long exact sequence of cohomologies
\begin{align*}
	0 &\to H^{0}(\mathcal{A})\to H^{0}(\mathcal{B})\to H^{0}(\mathcal{C}) \\
	  &\to H^{1}(\mathcal{A})\to H^{1}(\mathcal{B})\to H^{1}(\mathcal{C}) \\
	  &\to H^{2}(\mathcal{A})\to \cdots
\end{align*}
where the morphisms $H^{n}(\mathcal{C})\to H^{n+1}(\mathcal{A})$ are the \textbf{connecting morphisms}.
\end{thrm}
\begin{proof}
	\warn{Intuition?}
	\warn{Use snake lemma (have proof of this in spectral sequences paper).}
\end{proof}

\begin{cor}
If $0\to \mathcal{A}\to \mathcal{B}\to \mathcal{C}\to 0$ is exact and any 2 of the complexes are exact themselves, then so is the third.
\end{cor}
\begin{proof}
	The LES of cohomologies becomes all 0, except for each $H^{n}(\mathcal{X})$, where $\mathcal{X}$ is the third complex. Now $0\to H^{n}(\mathcal{X})\to 0$ exact $\implies H^{n}(\mathcal{X}) \cong 0$, so $\mathcal{X}$ is exact.
\end{proof}

\begin{defn}[]
	A morphism of complexes is a \textbf{quasi-isomorphism} if the (co)homology maps it induces are all iso.
\end{defn}

\begin{lem}
	If $f\simeq g$, then they induce the same (co)homology maps, i.e. $f_{*}=g_{*}$.
\end{lem}
\begin{proof}
	Suppressing subscripts, suppose $f = d's + sd$, then the induced map is
	\[
		[a] \mapsto [f(a)] = [(d's)(a) + (sd)(a)] = [d'(s(a)) + s(0)] = [0].
	\] Then if $f \simeq g$, we have $[f(a)] = [(g+d's+sd)(a)] = [g(a)]$.
\end{proof}

\begin{prop}
	\label{homotopy-quasi}
A homotopy equivalence is a quasi-iso.
\end{prop}
\begin{proof}
	Suppose $f$ and $g$ are inverse chain homotopies, then by the lemma, $f_{*}g_{*} = (fg)_{*} = (1_{B})_{*} = 1_{H(\mathcal{B})}$ and, similarly, $g_{*}f_{*}=1_{H(\mathcal{A})}$. Thus $H^n(\mathcal{A}) \cong H^n(\mathcal{B})$ for all $n$.
\end{proof}


%+-------------------+
%| +---------------+ |
%| |    Chapter    | |
%| +---------------+ |
%+-------------------+
% Derived Functors

\chapter{Derived Functors}

%--------------------------------------------------------------------------------
% Resolutions
%--------------------------------------------------------------------------------
\section{Resolutions}

\begin{defn}[]
Suppose $A$ is an $R$-module, then a \textbf{projective resolution} over $A$ is an exact sequence of projective $R$-modules
\[
\begin{tikzcd}
	\cdots \rar & P_{n}\rar{d_{n}} & P_{n-1}\rar{d_{n-1}} & \cdots \rar{d_1} & P_0 \rar{\varepsilon} & A \rar & 0
\end{tikzcd}
\] 
and a \textbf{injective resolution} over $A$ is an exact sequence of injective $R$-modules
\[
\begin{tikzcd}
	0 \rar & A \rar{\varepsilon} & I_{0} \rar{d_{1}} & \cdots \rar{d_{n-1}} & I_{n-1}\rar{d_{n}} & I_{n} \rar \rar & \cdots.
\end{tikzcd}
\] 
\end{defn}

\begin{thrm}[Existence]
	Every $R$-module has a projective \warn{and injective} resolution.
\end{thrm}
\begin{proof}
	Let $A$ be an $R$-module, and let $P_0$ be free on $A$. Then there's a unique $R$-morphism $\varepsilon:P_0\to A$ extending $1_{A}$. It's clearly epic, so $P_0 \stackrel{\varepsilon}{\to } A\to 0$ is exact. Now let $P_1$ be free on $\ker \varepsilon$, then there's a unique $R$-morphism $d_1:P_1\epi \ker \varepsilon$, so $P_1\stackrel{d_1}{\to } P_0\stackrel{\varepsilon}{\to } A$ is exact. Continue inductively, with $P_{n+1}$ free on $\ker d_{n}$. Since each $P_i$ is free, each is projective.
\end{proof}

\begin{prop}
$R$-morphisms lift to morphisms of projective/injective resolutions that are unique up to chain homotopy.
\[
\begin{tikzcd}
	\cdots \rar & P_1 \rar\dar[dashed]{f_1} & P_0 \rar\dar[dashed]{f_0} & A \rar\dar{f} & 0 \\
	\cdots \rar & P_1' \rar & P_0' \rar & A' \rar & 0
\end{tikzcd}
\]
\[
	\begin{tikzcd}
		0 \rar & A \rar\dar{g} & I_{0} \rar\dar[dashed]{g_0} & I_{1} \rar\dar[dashed]{g_1} & \cdots \\
		0 \rar & A' \rar & I_{0}' \rar & I_1' \rar & \cdots
\end{tikzcd}
\] 
\end{prop}
\begin{proof}
	Inductively use the fact that all the $P_i,P_i'$ are projective and $I_{i}, I_{i}'$ are injective to get each $f_n$ and $g_{n}$. \warn{Chain homotopy bit.}
\end{proof}

\begin{cor}
	\label{resolutions-homotopy-equiv}
Any two projective/injective resolutions of the same module are homotopy equivalent.
\end{cor}
\begin{proof}
	Consider the following lifts to projective resolutions (the case of injective resolutions is similar).
	\[
		\begin{tikzcd}
			\mathcal{P} \rar\dar[dashed]{f_{\bullet}} & A \rar\dar{1_A} & 0 \\
			\mathcal{P}' \rar\dar[dashed]{g_{\bullet}} & A \rar\dar{1_{A}} & 0 \\
			\mathcal{P} \rar & A \rar & 0
		\end{tikzcd}
	\] 
	The identity map is a valid candidate for each $g_n f_n$, so $g_n f_n \simeq 1$. Now flip the diagram upside down and use the same $f_n$ and $g_n$ maps, yielding $f_n g_n \simeq 1$.
\end{proof}

\begin{cor}
	Any two projective/injective resolutions have isomorphic (co)homology groups.
\end{cor}
\begin{proof}
	They're homotopy equivalent, so they're quasi-isomorphic by \Cref{homotopy-quasi}.
\end{proof}

\warn{Maybe start this section more generally, with left and right resolutions...}

%--------------------------------------------------------------------------------
% Derived Functors
%--------------------------------------------------------------------------------
\section{Derived Functors}

\begin{note}[]
Big idea: a left exact covariant functor $\mathcal{F}$ can turn a SES into a short left exact sequence, but there is only one canonical way to extend this to a LES, and that's with right derived functors. Left derived functors do the same for right exact functors. \warn{Why do we want a LES, though?}
\end{note}

Suppose we apply an additive functor $\mathcal{F}$ to some projective/injective resolution of a module $A$. The (co)homology groups of the resulting complex are unique up to isomorphism:
\begin{itemize}
	\item Let $\mathcal{X},\mathcal{Y}$ be projective/injective resolutions of $A$, then they're homotopy equivalent by \Cref{resolutions-homotopy-equiv}.
	\item By \Cref{functors-preserve-homotopy}, additive functors preserve homotopy equivalence, so $\mathcal{F}\mathcal{X}$ and $\mathcal{F}\mathcal{Y}$ are also homotopy equivalent.
	\item By \Cref{homotopy-quasi}, homotopy equivalent complexes have isomorphic (co)homology groups.
\end{itemize}

Thus we can define the (co)homology groups of the sequence gotten by applying an additive functor $\mathcal{F}$ to a projective/injective resolution of $A$ using \textit{any} resolution.

\begin{defn}[]
Let $\mathcal{F}$ be a functor and $A$ an $R$-module, then choose a resolution of $A$ from the following chart based on $\mathcal{F}$.
\begin{center}
	\begin{tabular}{r l}
		Left exact, covariant & \qquad injective \\
		Left exact, contravariant & \qquad projective \\
		Right exact, covariant & \qquad projective \\
		Right exact, contravariant & \qquad injective
	\end{tabular}
\end{center}
Apply $\mathcal{F}$ to the resolution, remove the $\mathcal{F}A$ term from it, then take (co)homologies. If $\mathcal{F}$ is left exact, the cohomologies $R^{i}\mathcal{F}$ are the \textbf{right derived functors} of $\mathcal{F}$. If $\mathcal{F}$ is right exact, the homologies $L_{i}\mathcal{F}$ are the \textbf{left derived functors} of $\mathcal{F}$.
\end{defn}

\warn{I don't think the derived functors depend on $A$ at all, they can just be applied to $A$, etc. to get new objects...}

With left exact functors, we end up with induced sequences of the form
\[
\begin{tikzcd}
	0 \to \mathcal{F}X_0 \to \mathcal{F}X_1 \to \mathcal{F}X_2 \to \cdots,
\end{tikzcd}
\] thus why the derived functors are ``right". Similarly, for right exact functors, we end up with induced sequences of the form
\[
\begin{tikzcd}
	\mathcal{F}X_2 \to \mathcal{F}X_1\to \mathcal{F}X_0\to 0,
\end{tikzcd}
\] thus why the derived functors are ``left". As examples, see the next two propositions.

\warn{Can you get both sets of sequences at once if $\mathcal{F}$ is exact?}

\warn{They're actually functors.... b/c (co)homology is a functor.}

\warn{More detail about \textit{why} we choose inj or proj resolution.}

\begin{prop}
	\label{left-exact-right-0}
	If $\mathcal{F}$ is left exact, then $R^0\mathcal{F}=\mathcal{F}$.
\end{prop}
\begin{proof}
	\textbf{Covariant:} If $0\to A\stackrel{f}{\mono } I_0 \stackrel{g}{\to } I_1$ is exact, then so is $0\to \mathcal{F}A\stackrel{\mathcal{F}f}{\mono } \mathcal{F}I_0\stackrel{\mathcal{F}g}{\to } \mathcal{F}I_1 $. Then $R^{0}\mathcal{F}(A) = \ker (\mathcal{F}g) = \im (\mathcal{F}f) \cong \mathcal{F}(A)$ (by the 1st iso theorem since $\mathcal{F}f$ monic).

	\textbf{Contravariant:} Use a projective resolution instead. The process is the same.
\end{proof}

\begin{prop}
	\label{right-exact-left-0}
	If $\mathcal{F}$ is right exact, then $L_0\mathcal{F} = \mathcal{F}$.
\end{prop}
\begin{proof}
	\textbf{Covariant:} If $P_1\stackrel{f}{\to } P_0\stackrel{g }{\epi } A\to 0$ is exact, then so is $\mathcal{F}P_1\stackrel{\mathcal{F}f}{\to } \mathcal{F}P_0\stackrel{\mathcal{F}g}{\epi } \mathcal{F}A\to 0$. Then $L_0\mathcal{F}(A) = \frac{\mathcal{F}P_0}{\im \mathcal{F}f} = \frac{\mathcal{F}P_0}{\ker \mathcal{F}g} \cong \mathcal{F}A$ (by the 1st iso theorem since $\mathcal{F}g$ epic).

	\textbf{Contravariant:} Use an injective resolution instead. The process is the same.
\end{proof}

\warn{Does ``left derived functor of left exact functor" make sense? Are they all just trivial or something? To prove something like that, would you use a left exact variant of ``exact functors preserves LES's"?}

\begin{thrm}[LES of derived functors]
	\warn{Do this.}
\end{thrm}

If derived functors measure the extent to which a functor fails to be exact, then an exact functor should have trivial derived functors. This turns out to be true.

\begin{prop}
If $\mathcal{F}$ is exact, then $R^{i}\mathcal{F}=L_{i}\mathcal{F}=0$ for all $i>0$.
\end{prop}
\begin{proof}
	\warn{This uses the fact that exact functors preserves LES's... prove this.}
	\textbf{Covariant:} Exact functors preserve exactness, so $0\to A\mono I_0\to I_1\to \cdots$ exact implies $0\to \mathcal{F}A\to \mathcal{F}I_0\to \mathcal{F}I_1\to \cdots$ exact. Chopping off the $\mathcal{F}A$ term and taking cohomologies gives $L_{i}\mathcal{F} = 0$ when $i>0$. Now repeat the argument with a projective resolution for the $R^{i}$.

	\textbf{Contravariant:} Similar argument.
\end{proof}

\begin{prop}
	Fix a functor $\mathcal{F}$. If $A$ is projective/injective (depending on the type of $\mathcal{F})$, then $R^{i}\mathcal{F}(A)$ or $L_{i}\mathcal{F}(A)$ (whichever is correct for $\mathcal{F}$) is trivial when $i>0$.
\end{prop}
\begin{proof}
	We consider the case when $\mathcal{F}$ is left exact and covariant, but the other three cases are similar. Suppose $A$ is injective, then $0\to A\stackrel{\text{id}}{\to }  A\to 0$ is an injective resolution of $A$. This induces the exact sequence $0\to \mathcal{F}A \stackrel{\text{id}}{\to }  \mathcal{F}A \to 0$. Chopping off the first $\mathcal{F}A$ term and taking cohomologies gives $R^{i}\mathcal{F}(A)=0$.
\end{proof}



%--------------------------------------------------------------------------------
% The Ext Functor
%--------------------------------------------------------------------------------
\section{The Ext Functor}

\begin{note}[]
Big idea: the hom functors are left exact, but we can use cohomology to measure how much they fail to be right exact.
\end{note}

\begin{defn}[]
	The \textbf{Ext functors} are the (right) derived functors of the hom functors.
\end{defn}

Given an $R$-module $A$, there are two equivalent ways to construct them:
\begin{enumerate}
	\item \textbf{Using $\hom(-,M)$ (contravariant):} Take a projective resolution
\[
\begin{tikzcd}
	\cdots \rar & P_{n}\rar{d_{n}} & P_{n-1}\rar{d_{n-1}} & \cdots \rar{d_1} & P_0 \rar{\varepsilon} & A \rar & 0
\end{tikzcd}
\] 
and apply $\hom(-,M)$ to it. Removing $\hom(A,M)$ from the sequence gives
\[
	\begin{tikzcd}
		0 \rar & \hom(P_0,M) \rar{d_1^{*}} & \hom(P_1,M)\rar{d_2^{*}} \rar & \cdots.
	\end{tikzcd}
\] 
This is a cochain complex since for any map $f$, applying $d^{*}$ twice gives ${d^{*}}^2(f) = fd^2=0$. Then $\ext_{R}^{n}(A,M)$ is the $n$-th cohomology group of this complex.

	\item \textbf{Using $\hom(M,-)$ (covariant):} Take an injective resolution, apply $\hom(M,-)$, remove $\hom(M,A)$, then take homology. \warn{Is this actually equivalent?}
\end{enumerate}

\begin{prop}
	$\ext_{R}^{0}(A,M)\cong \hom_{R}(A,M)$.
\end{prop}
\begin{proof}
	The hom functors are left exact, so apply \Cref{left-exact-right-0}.
\end{proof}

\warn{Ext(A,B) is contra in A, cov in B.}


\end{document}
