\documentclass[10pt]{report}
\usepackage{/Users/bradenhoagland/latex/math}

\lhead{Braden Hoagland}
\chead{Homological Algebra}
\rhead{}

\begin{document}
%\tableofcontents

%%%%%%%%%%%%%%%%%%%%
% Chain Complexes
%%%%%%%%%%%%%%%%%%%%

\section{Chain Complexes}

\begin{defn}[]
	A \textbf{chain complex} is a sequence of abelian group homomorphisms {\color{red}(change to modules?)}
\begin{center}
	\begin{tikzcd}
	\cdots \arrow[r, "d_{i+2}"] & C_{i+1} \arrow[r, "d_{i+1}"]  & C_i \arrow[r, "d_i"]  & C_{i-1} \arrow[r, "d_{i-1}"]  & \cdots
	\end{tikzcd}
\end{center}
such that $d_i \circ d_{i+1}=0$ for all $i$.
\end{defn}

We can also consider \textbf{cochain complexes}, which are the same except that the maps take you up a level instead of down.
\begin{center}
	\begin{tikzcd}
	\cdots                      & C_{i+1} \arrow[l, "d_{i+1}"'] & C_i \arrow[l, "d_i"'] & C_{i-1} \arrow[l, "d_{i-1}"'] & \cdots \arrow[l, "d_{i-2}"']
	\end{tikzcd}
\end{center}

The map $d_i$ is the \textbf{boundary operator}, as it is a generalization of the geometric concept of a boundary (note $d^2=0$). Thus an element in the image of $d$ is a \textbf{boundary}. Since usual geometric cycles have no boundary, we call the elements of the kernel of $d$ \textbf{cycles}.


\begin{ex}[]
Chain complexes generalize the concept of boundaries to objects that don't necessarily have clear cyclic geometric properties. Let $\mathfrak{X}(\mathbb{R}^3)$ denote the smooth vector fields on $\mathbb{R}^3$, then consider the chain complex
\begin{center}
\begin{tikzcd}
	0 \arrow[r] & \mathbb{R} \arrow[r] & C^\infty(\mathbb{R}^3) \arrow[r, "\text{grad}"] & \mathfrak{X}(\mathbb{R}^3) \arrow[r, "\text{curl}"] & \mathfrak{X}(\mathbb{R}^3) \arrow[r, "\text{div}"] & C^\infty(\mathbb{R}^3) \arrow[r] & 0
\end{tikzcd}
\end{center}
If we consider the $\text{grad}$ map, we see that its ``cycles" are actually just constant functions.
\end{ex}

A map $f:C\to D$ between chain complexes is a sequence of maps \[f_i:C_i\to D_i\] that respect the boundar operators. This means the following diagram commutes.
\begin{center}
	\begin{tikzcd}
		C_i \arrow[d, "f_i"'] \arrow[r, "d_C"] & C_{i-1} \arrow[d, "f_{i-1}"] \\
		D_i \arrow[r, "d_D"']                & D_{i-1}
\end{tikzcd}
\end{center}

\begin{defn}[]
The $n$-th \textbf{homology group} of a chain complex $\mathcal{C}$ is
\[
	H_{n}(\mathcal{C}) = \frac{\ker d_{n}}{\im d_{n+1}} .
\] 
\end{defn}

Similarly, the $n$-th \textbf{cohomology group} is $H^{n}(C) = \ker d_{n+1}/\im d_{n}$. In both cases the quotient represents the $n$-th dimensional holes in the complex, as it is the cycles that do not arise as boundaries of higher dimensional objects.

A chain complex $C$ is exact if and only if $H_{n}(C)$ is trivial for all $n$.

Given a map $f:C \to D$ of chain complexes, the commutativity of the above diagram shows that $f$ sends cycles to cycles and boundaries to boundaries, thus inducing a map $H_{n}(C) \to H_{n}(D)$.

\begin{defn}[]
	A \textbf{quasi-isomorphism} is a map $f:C\to D$ of chain complexes where the induced map $H_{n}(C)\to H_{n}(D)$ is an isomorphism.
\end{defn}


\end{document}
