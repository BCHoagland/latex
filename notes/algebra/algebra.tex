\documentclass[10pt]{report}
\usepackage{/Users/bradenhoagland/latex/math}

\lhead{Braden Hoagland}
\chead{Algebraic Structures II}
\rhead{}

\begin{document}
\tableofcontents

%+-------------------+
%| +---------------+ |
%| |    Chapter    | |
%| +---------------+ |
%+-------------------+
% Field Extensions

\chapter{Field Extensions}

%%%%%%%%%%%%%%%%%%%%
% Fields
%%%%%%%%%%%%%%%%%%%%

\section{Fields}

A \textbf{field} is a tuple $(F,+,\cdot)$ such that $(F,+)$ and $(F^{\times},\cdot)$ are abelian groups and multiplication distributes over addition, where $F^{\times} \doteq F-\left\{ 0 \right\}$.

Equivalently, a field is a commutative ring with unity (i.e. has a multiplicative identity) where every nonzero elt has a multiplicative inverse (i.e. is a unit). Since units can't be zero divisors, fields have no zero divisors.

\[
	\text{Fields } \subset \text{ Euclidean Domains } \subset \text{ PIDs } \subset \text{ UFDs } \subset \text{ Integral Domains}.
\] 

\begin{prop}
	Any nonzero field homomorphism is injective.
\end{prop}
\begin{proof}
	Let $\varphi$ be a field homomorphism with domain $F$. Now $\ker \varphi$ is an ideal of $F$, but the only ideals of a field are 0 and itself. Since $\varphi$ is nonzero, $\ker\varphi=0$, so $\varphi$ is injective.
\end{proof}

\begin{defn}[]
	The \textbf{characteristic} $\text{ch}(F)$ of a field $F$ is the smallest positive integer $p$ such that $p \cdot 1_{F}=0$. If no such $p$ exists, we say $\text{ch}(F)=0$.
\end{defn}

\begin{prop}
	The characteristic of a field is either 0 or prime.
\end{prop}
\begin{proof}
	If $n$ is composite and $n \cdot 1 = 0$, then we can decompose this into its prime factorization and get that its smallest prime factor is the characteristic.
\end{proof}

Fields don't have interesting ideals (it's either 0 or the entire field), so instead we study subfields and field extensions.

\begin{defn}[]
	The \textbf{prime subfield} of a field $F$ is the subfield generated by $1 \in F$.
\end{defn}

\begin{prop}
	The prime subfield of a field $F$ is isomorphic to $\mathbb{Q}$ if $\text{ch}(F)=0$ and isomorphic to $\mathbb{F}_{p}$ if $\text{ch}(F)=p$.
\end{prop}

\begin{defn}[]
	A field $K$ is a \textbf{(field) extension} of $F$ if $F$ is a subfield of $K$. Denote this by $K\nwarrow F$.
\end{defn}

\begin{defn}
	If $K$ is an extension of $F$, then the \textbf{degree} $[K:F]$ of $K$ over $F$ is the dimension of $K$ as an $F$-vector space. An extension is \textbf{finite} if its degree is finite, and its \textbf{infinite} otherwise.
\end{defn}

\begin{ex}[]
	$[\mathbb{C}:\mathbb{R}]=2$ because $\left\{ 1,i \right\}$ is a basis for $\mathbb{C}$ over $\mathbb{R}$.
\end{ex}

{\color{red}Field of fractions (DF sec 7.5). Since $\mathbb{Q}$ is the field of fractions of $\mathbb{Z}$, any field containing $\mathbb{Z}$ must also contain $\mathbb{Q}$.}

%%%%%%%%%%%%%%%%%%%%
% Polynomial Rings over Fields
%%%%%%%%%%%%%%%%%%%%

\section{Polynomial Rings over Fields}

Many field extensions arise from trying to solve polynomial equations, so we gotta review that.

\begin{thrm}[]
	Let $F$ be a field, then $F[x]$ is a Euclidean Domain.
\end{thrm}

This means that any polynomial ring over a field has a division algorithm, i.e. for all $f(x)$ and nonzero $g(x)$, there exist \textit{unique} $q(x), r(x)$ such that
\[
	f(x)=q(x)g(x)+r(x),
\] where $\deg r(x) < \deg g(x)$. Here, we take the degree of the zero polynomial to be 0. It should also be clear that degree is the norm of $F[x]$.

\begin{cor}
	$F[x]$ is also a principal ideal domain (PID) and a unique factorization domain (UFD).
\end{cor}

If $E \nwarrow F$ and $f(x), 0 \neq g(x) \in F[x]$, then the result of the division algorithm in $F[x]$ is the same in $E[x]$ by the uniqueness bit. {\color{red}paragraph at end of sec 9.2.}

Often, even if $R$ is not a field (but \textit{is} a UFD), then we can say something about factorization in $R$ by looking at its field of fractions {\color{red}(the smallest field containing $R$, see sec 7.5, think $\mathbb{Z}$ to $\mathbb{Q}$).}

\begin{lem}[Gauss' Lemma]
	Let $R$ be a UFD with field of fractions $F$. Let $p(x) \in R[x]$ have coefficients with $\gcd$ 1, then $p(x)$ is irreducible in $R[x]$ if and only if it's irreducible in $F[x]$.
\end{lem}

Note that this works for all monic polynomials.

\begin{prop}
	Let $p(x) \in F[x]$, where $F$ is a field. Then $p(x)$ has a root $a \in F$ if and only if $(x-a)$ divides $p(x)$.
\end{prop}
\begin{proof}
	{\color{red}Do this.}
\end{proof}

\begin{cor}
	Any $p(x) \in F[x]$ has at most $\deg p$ roots in $F$ (including with multiplicity).
\end{cor}
\begin{proof}
	Use induction on the proposition above.
\end{proof}

\begin{cor}
	\label{cor:2-3-reducible}
	If $p(x) \in F[x]$ has degree 2 or 3, then it's reducible if and only if it has a root in $F$.
\end{cor}

The above corollary should be relatively obvious, but note that it doesn't hold in 4 dimensions or higher because a reducible polynomial could reduce into two other polynomials that have dimension 2+.

\begin{ex}[]
	We claim that $p(x) = x^3+x+1$ is irreducible in $\mathbb{F}_{2}[x]$. Using Corollary \ref{cor:2-3-reducible}, we check that $p(0)$ and $p(1)$ are nonzero, so $p$ has no roots in $\mathbb{F}_{2}$.
\end{ex}

\begin{prop}
	Let $R$ be a UFD and let $p(x) = \sum_i a_i x^i \in R[x]$. If $c$ and $d$ are relatively prime with $d$ nonzero and $p(c/d) = 0$, then $c \;|\; a_0$ and $d \;|\; a_n$.
\end{prop}

This is very useful in limiting the candidates for the roots of a particular polynomial.

\begin{ex}[]
	We claim that $p(x) = x^3-x-1$ is irreducible in $\mathbb{Z}[x]$. By Gauss' Lemma and Corollary \ref{cor:2-3-reducible}, it suffices to show that $p$ has no rational roots. By the above proposition, the only possibilities of rational roots are $\pm 1$. But $p(1)$ and $p(-1)$ are both nonzero, so $p$ is irreducible.
\end{ex}

\begin{thrm}[Eisenstein's Criterion]
	Let $R$ be a UFD with field of fractions $F$ and let $f(x) = \sum_i a_i x^i \in R[x]$ with $n \geq 1$ (i.e. non-constant) and $a_n \neq 0$. If there is some irreducible $p \in R$ such that
	\begin{enumerate}
		\item $p$ does \textit{not} divide $a_n$,
		\item $p$ divides $a_i$ for all $i < n$, and
		\item $p^2$ does \textit{not} divide $a_0$,
	\end{enumerate}
	then $f(x)$ is irreducible in $F[x]$.
\end{thrm}

This is usually used when $R=\mathbb{Z}$ (so the field of fractions is $\mathbb{Q}$) and $p$ is prime.

\begin{ex}[]
	$x^{12}-10x^{4}+4x-6$ is irreducible in $\mathbb{Q}[x]$ by Eisenstein's criterion for $p=2$.
\end{ex}

\begin{thrm}[]
	The multiplicative group of any finite field is cyclic.
\end{thrm}
\begin{proof}
	Let $F$ be a finite field, then $(F^\times, \cdot)$ is a finite abelian group. By the fundamental theorem of finitely generated abelian groups, there exist positive integers $m_1 \;|\; m_2 \;|\; \cdots \;|\; m_k$ such that
	\[
	F^{\times} \cong \mathbb{Z}_{m_1} \oplus \cdots \oplus \mathbb{Z}_{m_k}.
	\] In particular, every element of $F^{\times}$ has order dividing $m_k$, i.e. $\alpha^{m_k} = 1$ for all $\alpha\in F^{\times}$. Thus every element of $F^{\times}$ is a root of $x^{m_k}-1$. Since this polynomial can have at most $m_k$ roots, $|F^{\times}| \leq m_{k}$; however, if $F^{\times}$ is isomorphic to $\mathbb{Z}_{m_1} \oplus \cdots \oplus \mathbb{Z}_{m_k}$, then $|F^{\times}| = m_1 \cdots m_k$. But this is only true if $k=1$, so $F^{\times}\cong \mathbb{Z}_{m_1}$, so it is cyclic.
\end{proof}

%%%%%%%%%%%%%%%%%%%%
% Constructing Field Extensions with Polynomials
%%%%%%%%%%%%%%%%%%%%

\section{Constructing Field Extensions with Polynomials}

The main idea of all this is to take an irreducible polynomial $p(x)$ over a field $F$, take its {\color{red}(maximal)} ideal $(p(x))$, and use that to create the field $F[x]/(p(x))$. As it turns out, this field will contain a root of $p$, so we can use this technique to construct field extensions that contain the roots of certain polynomials.

\begin{defn}[]
	Suppose $K \nwarrow F$, and let $a_1, \dots, a_n \in K$. Then the extension $F(a_1, \dots, a_n)$ is the smallest subfield of $K$ containing $F$ and all the $a_i$.

	Let $R$ be a subring of $K$, then $R[a_1, \dots, a_n]$ is the smallest subring of $K$ containing $R$ containing $R$ and all the $a_i$.
\end{defn}

If we have a set $A$, we might denote the extension that it generates over $F$ by $F(A)$.

We say $K$ is a \textbf{simple extension} of $F$ if $K=F(\alpha)$ for some $\alpha \in K$.

\begin{defn}[]
	Let $K \nwarrow F$. We say $\alpha \in K$ is \textbf{algebraic} over $F$ if it's the root of \textit{some} polynomial over $F$. Otherwise it's \textbf{transcendental} over $F$.

	$K$ is an \textbf{algebraic extension} of $F$ if every element of $K$ is algebraic over $F$.
\end{defn}

\begin{ex}[]
$\mathbb{C}$ is algebraic over $\mathbb{R}$, but $\mathbb{R}$ is not algebraic over $\mathbb{Q}$.
\end{ex}

\begin{ex}[]
	Every element $\alpha$ of a field $F$ is algebraic over $F$ since $(x-\alpha)$ is a polynomial over $F$.
\end{ex}

Let $K \nwarrow F$ with $\alpha \in K$ algebraic over $F$, and consider the ``evaluation at $\alpha$" map $\phi_\alpha:F[x]\to K$ given by $F \stackrel{\text{id}}{\mapsto } F$, $x \mapsto \alpha$, and $\phi_a$ a ring homomorphism.

\begin{defn}[]
	The \textbf{minimal polynomial} $m_{\alpha,F}(x)$ of $\alpha$ over $F$ is the unique irreducible monic generator of $\ker\phi_a \subset F[x]$, i.e. it generates all the polynomials over $F$ that have $\alpha$ as a root.

	The \textbf{degree} of $\alpha$ over $F$ is the degree of $m_{\alpha,F}(x)$.
\end{defn}

Minimal polynomials are handy because they allow us to construct field extensions that contain one of their roots. If we take $F[x]$ and mod out everything generated by $m_{\alpha}(x)$, then what we get is a field where everything ``related to" $\alpha$ becomes 0. {\color{red}Replace this with actual good intuition. Use the theorem about the form of elements of $F(\alpha_1, \dots)$ to show this.}

\begin{thrm}[]
	If $K \nwarrow F$ and $\alpha \in K$ is algebraic over $F$ with minimal polynomial $m_{\alpha}(x)$, then
	\begin{enumerate}
		\item $F(\alpha) = F[\alpha]$,
		\item $F(\alpha) \cong F[x]/m_{\alpha}(x)$,
		\item $[F(\alpha):F] = \deg m_{\alpha}(x)$, and
		\item $\left\{ 1,\alpha,\dots, \alpha^{n-1} \right\}$ is a basis for $F(\alpha)$ over $F$, where $n = \deg m_{\alpha}(x)$.
	\end{enumerate}
\end{thrm}
\begin{proof}
	{\color{red}Do this.}
\end{proof}

\begin{ex}[]
If $\alpha \in \mathbb{C}$ has minimal polynomial $x^3+x+3$ over $\mathbb{Q}$, then $\mathbb{Q}(\alpha)$ has basis $\left\{ 1,\alpha,\alpha^2 \right\}$ over $\mathbb{Q}$.
\end{ex}

We can use this theorem to construct any field of order $p^n$, where $p$ is a prime. If we take a monic irreducible polynomial $f(x)$ of degree $n$ over the finite field $\mathbb{F}_{p}$, then the extension $\mathbb{F}_{p}[x]/(f(x))$ as a vector space over $\mathbb{F}_{p}$ has degree $n$, so there are $p^n$ elements of the extension.

{\color{red}GO OVER SECTION 13.1 FOR ALL THE PROOFS.}

The roots of an irreducible polynomial $p(x)$ are algebraically indistinguishable in the sense that they generate the same extensions. If $\alpha,\beta$ are roots of $p(x)$ over $F$, then
\[
	F(\alpha) \cong F[x]/(p(x)) \cong F(\beta).
\] 
We can extend this idea slightly by considering field extensions generated by isomorphcially related polynomials. In this case, the field extensions are themselves isomorphic.

\begin{note}[]
	If we have a map $\phi:F\to E$ and I write something like $\phi(f(x))$, this means we're applying $\phi$ to each coefficient of $f(x)$ and returning a new polynomial over $E$.
\end{note}

\begin{thrm}[]
	Suppose $\phi:F\to E$ is a field isomorphism. Let $\alpha$ be the root of minimal polynomial $f(x)$ over $F$, and let $\beta$ be a root of $\phi(f(x))$. Then we can extend $\phi$ to an isomorphism $\hat{\phi} : F(\alpha)\to E(\beta)$ such that $\hat{\phi}(\alpha)=\beta$.
\end{thrm}
\begin{proof}
	{\color{red}Do this.}
\end{proof}

This theorem can be represented with the following diagram.
\begin{center}
\begin{tikzcd}
\hat{\phi}: & F(\alpha) \arrow[r, "\cong"] \arrow[d, no head] & E(\beta) \arrow[d, no head] \\
\phi:       & F \arrow[r, "\cong"]                            & E                          
\end{tikzcd}	
\end{center}

%%%%%%%%%%%%%%%%%%%%
% Algebraic Extensions
%%%%%%%%%%%%%%%%%%%%

\section{Algebraic Extensions}

\begin{defn}[]
	$K \nwarrow F$ is \textbf{finitely generated} if $K = F(\alpha_1, \dots, \alpha_N)$.
\end{defn}

\begin{note}[]
	A field extension might be finitely generated without being a finite extension. Consider $\mathbb{Q}(\pi)$, which is clearly finitely generated. Since $\pi$ is transcendental over $\mathbb{Q}$, $\mathbb{Q}(\pi)$ is an infinite extension over $\mathbb{Q}$.
\end{note}



\end{document}
