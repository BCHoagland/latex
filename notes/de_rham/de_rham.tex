\documentclass[twoside,10pt]{report}
\usepackage{/Users/bradenhoagland/latex/toggles}
\toggletrue{sectionbreaks}
%\toggletrue{sectionheaders}
\newcommand{\docTitle}{De Rham Theory}
\usepackage{/Users/bradenhoagland/latex/math2}

%\renewcommand{\theenumi}{\alph{enumi}}

\begin{document}
%\tableofcontents

%%%%%%%%%%%%%%%%%%%%
% The De Rham Complex on R^n
%%%%%%%%%%%%%%%%%%%%

\section{The De Rham Complex on \texorpdfstring{$\mathbb{R}^n$}{Rn}}

Denote the space of $k$-forms on an $n$-dimensional manifold $N$ by $\Omega^{k}(N)$, then the $C^{\infty}$ differential forms on $N$ form the vector space
\[
	\Omega^{*}(N) \doteq \bigoplus_{k=0}^{n} \Omega^{k}(N).
\] The exterior derivative is defined as usual: if $f$ is a smooth function, then $df \doteq \sum \p_{i}{f} \;dx_i$, and if $\omega = \sum f_{I}dx_{I}$ is a differential form, then $d\omega \doteq \sum df_{I} \wedge dx_{I}$.

\begin{defn}[]
$\left( \Omega^{*}(N), d \right)$ is the \textbf{de Rham complex} on $N$, which we represent by the cochain complex
\[
\begin{tikzcd}
	0\rar &\Omega^{0}(N) \rar{d}&\Omega^{1}(N)\rar{d}&\cdots\rar{d}&\Omega^{n}(N)\rar&0.
\end{tikzcd}
\] The $k$-th \textbf{de Rham cohomology} of $N$ is then the vector space
\[
	H^{k}(N) \doteq \frac{\ker d \isct\Omega^{k}(N)}{\im d \isct \Omega^{k}(N)} .
\] 
\end{defn}
Since our complex is finite, the $0$-th and $n$-th cohomologies will always be a bit simpler:
\begin{align*}
	H^{0}(N) &= \ker d \isct \Omega^{0}(N),\\
	H^{n}(N) &= \frac{\Omega^{n}(N)}{\im d \isct \Omega^{n}(N)} .
\end{align*}
Any differential form in the kernel of $d$ is \textbf{closed}, and any in the image of $d$ is \textbf{exact}. Note that since $d^{2}=0$, an exact form must also be closed.

%%%%%%%%%%%%%%%%%%%%
% Functoriality of de Rham Cohomology
%%%%%%%%%%%%%%%%%%%%

\section{Functoriality of de Rham Cohomology}

Suppose we have a smooth map of manifolds $f:M\to N$, then this induces a pullback
\begin{align*}
	f^{*}:\Omega^{*}(N)&\to \Omega^{*}(M)\\
	g&\mapsto g\circ f,
\end{align*}
which is easily seen from the following diagram.
\[
\begin{tikzcd}
	M\rar{f}\arrow[dr,"g\circ f"']&N\dar{g}\\
				     &\mathbb{R}
\end{tikzcd}
\]
Given smooth maps between manifolds $A,B,C$, we can show that the pullbacks satisfy a reversed composition law: $g^{*}\circ f^{*}=(f\circ g)^{*}$. {\color{red}It's straightforward} to do this calculation, but the following picture makes it clear.
\[
\begin{tikzcd}
	A\rar{f}&B\rar{g}&C\\
	\Omega^{*}(A)&\Omega^{*}(B)\arrow[l,"f^{*}"']&\Omega^{*}(C)\arrow[l,"g^{*}"']
\end{tikzcd}
\] All this shows that $\Omega^{*}$ is a contravariant functor from the category of smooth manifolds to the category of commutative differential graded algebras. The commutativity bit refers to the identity
\[
	\tau \wedge \omega = (-1)^{\deg \tau \deg \omega} \; \omega \wedge \tau.
\] 

We can check that $f^{*}$ commutes with the exterior derivative: $f^{*}(d_{N}\omega) = d_{M}(f^{*}\omega)$ for any differential form $\omega$ on $N$. {\color{red}(Do this)} This means $f^{*}$ is a chain map $\Omega^{*}(N) \to \Omega^{*}(M)$, so it induces homomorphisms $H^{k}(N)\to H^{k}(M)$ for all $k$.
\[
\begin{tikzcd}
	0\rar&\Omega^{0}(N)\rar{d_{N}}\dar{f^{*}}&\cdots\rar{d_{N}}&\Omega^{k}(N)\rar{d_{N}}\dar{f^{*}}&\cdots\\
	0\rar&\Omega^{0}(M)\rar{d_{M}}&\cdots\rar{d_{M}}&\Omega^{k}(M)\rar{d_{M}}&\cdots
\end{tikzcd}
\] 

Then since taking the induced homological structure is functorial {\color{red}(check)}, this means that $H^{*}$ is also a contravariant functor {\color{red}(be specific about the category it's going to)}.


\end{document}
