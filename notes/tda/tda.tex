\documentclass[twoside,10pt]{report}
\usepackage{/Users/bradenhoagland/latex/styles/toggles}
\toggletrue{sectionbreaks}
%\toggletrue{sectionheaders}
\newcommand{\docTitle}{Topological Data Analysis}
\usepackage{/Users/bradenhoagland/latex/styles/common}
\importStyles{modern}{rainbow}{boxy}

%\renewcommand{\theenumi}{\alph{enumi}}

\begin{document}
\tableofcontents

%+-------------------+
%| +---------------+ |
%| |    Chapter    | |
%| +---------------+ |
%+-------------------+
% Simplicial Homology

\chapter{Simplicial Homology}

%--------------------------------------------------------------------------------
% Graphs
%--------------------------------------------------------------------------------
\section{Graphs}

\begin{defn}[]
	A \textbf{(simple undirected) graph} $G$ is a set of vertices $V$ and undirected edges $E$, where $E$ has no self-loops or duplicate edges.
\end{defn}

\begin{defn}[]
	A \textbf{path} between vertices $x$ and $y$ is a sequences of vertices
	\[
	x=u_0, \quad u_1, \quad \dots, \quad u_m=y
\] such that $[u_i,u_{i+1}]$ is an edge for all $i$.
\end{defn}

\begin{defn}[]
A graph is \textbf{connected} if there is a path between every pair of vertices.
\end{defn}

A \textbf{separation} of $G$ is two nonempty subsets of $G$ with no edges going between them. We can then equivalently define a graph to be connected if it has no separation.

\begin{prop}
Let $x \sim_{p} y$ if there is a path from $x$ to $y$. Then $\sim_{p}$ is an equivalence relation.
\end{prop}

We call the equivalence classes of $\sim_{p}$ \textbf{connected components}. Since equivalence relations naturally form partitions, the connected components of a graph union to the entire graph.

\begin{ex}[]
	Let $V$ be a vector space with subspace $N$, then $x \sim y \iff x - y \in N$ is an equivalence relation (since $N$ has 0 and is closed under addition and additive inverse). The quotient $V/N$ can then be defined as the equivalence classes of $\sim$, which is also a vector space with the operations $\alpha[x] = [\alpha x]$ and $[x]+[y] = [x+y]$.
\end{ex}

%--------------------------------------------------------------------------------
% Simplicial Complexes
%--------------------------------------------------------------------------------
\section{Simplicial Complexes}

\begin{defn}[]
	An \textbf{abstract simplicial complex} is a collection $K$ of nonempty subsets of a finite set $V$ such that
	\begin{enumerate}
		\item If $\left\{ v \right\} \in K$, then $v \in V$;
		\item If $\sigma \in K$ and $\tau \subset \sigma$ is nonempty, then $\tau \in K$.
	\end{enumerate}
\end{defn}

In this setting, each singleton $\left\{ v \right\}$ represents a vertex, and $V$ represents all the vertices in the complex. Then for a simplex $\sigma$, any subset $\tau \subset \sigma$ is one of its \textbf{faces}. Note that a face need not be 1 dimension lower (e.g. a vertex is a face of a tetrahedron).

We say that the \textbf{dimension} of a complex $\sigma \in K$ is $\dim \sigma = |\sigma|-1$. The dimension of the whole simplex $K$ is then $\max_{\sigma\in K}\left\{ \dim \sigma \right\}$. We need the $-1$ in the definition to make edges 1-dimensional, triangles 2-dimensional, etc.

%--------------------------------------------------------------------------------
% Simplicial Homology
%--------------------------------------------------------------------------------
\section{Simplicial Homology}

\begin{defn}[]
	Suppose $X$ is a simplicial complex, then let $C_{n}(X)$ be the vector space over $\mathbb{Z}_{2}$ with basis the $n$-simplices in $X$. Elements of $C_{n}(X)$ are called \textbf{$n$-chains}.
\end{defn}

\begin{itemize}
	\item $C_{0}$: vertices
	\item $C_{1}$: edges
	\item $C_{2}$: triangles
\end{itemize}

\begin{note}[]
	The choice to do this over $\mathbb{Z}_2$ was pretty arbitrary. It makes lots of things nicer from a computation viewpoint, and it simplifies a few things later on since $-1=1$, but we can also get more information by taking homology with coefficients in more complicated rings. In general, $C_{n}$ will be some module. Then if they're over $\mathbb{Z}$, they're abelian groups.
\end{note}

\begin{defn}[]
The \textbf{boundary map} $\p_n$ is the linear map
\begin{align*}
	C_n(X) &\to C_{n-1}(X) \\
	[v_0,\dots,v_n] &\mapsto \sum_{i} \; [v_0, \dots, \hat{v}_{i}, \dots, v_{n}],
\end{align*}
where $\hat{v_{i}}$ indicates that $v_{i}$ has been removed from the simplex.
\end{defn}

\begin{prop}
$\p^2 = 0$.
\end{prop}
\begin{proof}
	Applying the defintion of $\p$ gives
	\begin{align*}
		\p^2([v_0, \dots, v_{n}]) &= \sum_{i} \p([v_0, \dots, \hat{v}_{i}, \dots, v_{n}]) \\
					  &= \sum_{j<i} [v_0, \dots, \hat{v}_{j}, \dots, \hat{v}_{i}, \dots, v_{n}] + \sum_{i < j} [v_0, \dots, \hat{v}_{i}, \dots, \hat{v}_{j}, \dots, v_{n}].
					  \intertext{Now we can swap the roles of $i$ and $j$ in the second sum to get a sum identical to the first. This gives}
					  &= 2 \sum_{j<i} [v_0, \dots, \hat{v}_{j}, \dots, \hat{v}_{i}, \dots, v_{n}] \\
					  &= 0
	\end{align*}
	since we're working over $\mathbb{Z}_2$.
\end{proof}

This result shows that
\[
\begin{tikzcd}
	\cdots \rar & C_2(X) \rar{\p_2} & C_1(X) \rar{\p_1} & C_0(X) \rar{\p_0} & 0
\end{tikzcd}
\] is a chain complex. Thus we call $Z_k(X) \doteq \ker \p_k$ the space of \textbf{$k$-cycles} and $B_{k}(X) \doteq \im \p_{k+1}$ the space of \textbf{$k$-boundaries}.

\begin{defn}[]
	The \textbf{$k$-th homology} of $X$ is $H_{k}(X) \doteq Z_k(X) / B_k(X)$, and its dimension $\beta_{k}$ is the \textbf{$k$-th Betti number}.
\end{defn}

\begin{prop}
	$\beta_0$ is the number of connected components of $X$. \warn{Infinite case?}
\end{prop}
\begin{proof}
	Suppose $X$ has connected components $X_1, \dots, X_{\ell}$. Then since the homology functor commutes with direct sums,
	\[
		H_{0}(X) = H_{0}\left( \bigoplus_{i=1}^{\ell}X_{i} \right) = \bigoplus_{i=1}^{\ell}H_{0}(X_{i}).
	\] 
	{\color{blue}Show that $\beta_0=1$ when $X$ is connected.}
	Then since $\beta_0$ of a connected complex is 1,
	\[
		\beta_0 = \dim \left( \bigoplus_{i=1}^{\ell}H_{0}(X_i) \right) = \sum_{i=1}^{\ell} \dim H_0(X_{i}) = \sum_{i=1}^{\ell}1 = \ell.
	\] 
\end{proof}


%--------------------------------------------------------------------------------
% Functoriality of Simplicial Homology
%--------------------------------------------------------------------------------
\section{Functoriality of Simplicial Homology}

\warn{I think I can build in the chain map-ness of all the functors better.}

Suppose $K$ and $L$ are simplicial complexes, then we can turn morphisms between them into morphisms between homologies. This boils down to defining two covariant functors, then composing them.

\begin{defn}[]
Let $K,L$ be simplicial complexes. We say $f:K\to L$ is a \textbf{simplicial map} if
\begin{enumerate}
	\item $f:V(K)\to V(L)$;
	\item it's determined by where it sends vertices: $f([x_0, \dots, x_p]) = [f(x_0), \dots, f(x_{p})]$;
	\item $f(\sigma) \in L$ when $\sigma \in K$.
\end{enumerate}
\end{defn}

\begin{thrm}[]
	Fix $p \in \mathbb{N}_{0}$, then taking $p$-th chain groups is a functor. It sends simplicial maps $f:K\to L$ to the morphism $f_{\#}:C_{p}(K)\to C_{p}(L)$ uniquely determined by
	\[
		\sigma \mapsto 
		\begin{cases}
			f(\sigma) & \text{ if } |f(\sigma)|=|\sigma|, \\
			0 & \text{ else}.
		\end{cases}
\]
\end{thrm}
\begin{proof}
	\warn{Do this.}
\end{proof}

\begin{prop}
	$f_{\#}$ respects $\p$, i.e. the following diagram commutes.
	\[
	\begin{tikzcd}
		C_{p}(K) \rar{\p}\dar{f_{\#}} & C_{p-1}(K) \dar{f_{\#}} \\
		C_{p}(L) \rar{\p} & C_{p-1}(L)
	\end{tikzcd}
	\] 
\end{prop}
\begin{proof}
	\warn{Do this.}
\end{proof}

\begin{thrm}[]
	Taking $p$-th homology is a functor. It sends $f_{\#}:C_{p}(K)\to C_{p}(L)$ to the morphism $f_{*}:H_{p}(K) \to H_{p}(L)$ uniquely determined by
	\[
		[x] \mapsto [f_{\#}(x)].
	\] 
\end{thrm}

\warn{Want a similar commutative diagram prop.}

%+-------------------+
%| +---------------+ |
%| |    Chapter    | |
%| +---------------+ |
%+-------------------+
% Persistent Homology

\chapter{Persistent Homology}

%--------------------------------------------------------------------------------
% Persistence Modules
%--------------------------------------------------------------------------------
\section{Persistence Modules}

The following definition of a persistence module is restricted, but it's all we'll need for our purposes (in general, you can define it to be a functor from a specific poset to the category of $R$-modules). \warn{You should read more about this.}

\begin{defn}[]
A \textbf{persistence module} $V$ is a collection of vector spaces $\left\{ V_{r} \right\}_{r \in \mathbb{R}}$ over $\mathbb{Z}_2$ with functorial maps $v_{i}^{j}:V_{i}\to V_{j}$ whenever $i \leq j$. \warn{Covariant functor from $\mathbb{R}$ to Vector spaces over $\mathbb{Z}_2$...}
\end{defn}
The functoriality condition means
\begin{enumerate}
	\item $v_{i}^{i} = \id$;
	\item $v_{i}^{k} = v_{j}^{k} v_{i}^{j}$ whenver $i \leq j \leq k$.
\end{enumerate}
Suppose we're indexing our vectors spaces via $\mathbb{Z}$ instead of $\mathbb{R}$, then the situation looks as follows.
\[
\begin{tikzcd}
	\cdots \rar & V_{i-1} \rar{v_{i-1}^{i}} \arrow[rr, bend right, "v_{i-1}^{i+1}"'] & V_{i} \rar{v_{i}^{i+1}} & V_{i+1} \rar & \cdots
\end{tikzcd}
\] 

%--------------------------------------------------------------------------------
% Interval Decomposition
%--------------------------------------------------------------------------------
\section{Interval Decomposition}




%--------------------------------------------------------------------------------
% Idk where to put this stuff
%--------------------------------------------------------------------------------
\section{Idk where to put this stuff}

Given a function $f:G\to \mathbb{R}$, we can think of $f(x)$ as the time at which $x$ appears.

\begin{defn}[]
	$F:G\to \mathbb{R}$ is \textbf{monotonic} if $f(v) \leq f(e)$ whenever $e$ is an edge containing vertex $v$. \warn{gen to complexes...}
\end{defn}

Thus for monotonic functions, no edge will appear until both its vertices have also appeared.

\warn{0 dim Persistent Homology stuff...}

Note that every birth-death pair is an element of
\[
	\overline{\mathbb{R}}_{<}^2 \doteq \left\{ (x,y) \;|\; x \in \mathbb{R}, \; y \in \mathbb{R} \uni \left\{ \infty \right\} \right\}.
\] 
\warn{Figure.}

\begin{defn}[]
A \textbf{partial mapping} between {\color{red}multisets} $P,Q \subset \overline{\mathbb{R}}_{<}^2$ is a bijection $\eta: P_0 \to Q_0$, where $P_0 \subset P$ and $Q_0 \subset Q$. We denote it
\[
\eta: P\bij Q.
\] 
\end{defn}

We define the cost of a partial matching $\eta:P\bij Q$ \warn{finish...}





\end{document}
