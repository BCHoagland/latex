\documentclass[twoside,10pt]{report}
\usepackage{toggles}
%\toggletrue{sectionbreaks}
\newcommand{\docTitle}{Categorical Topology}
\usepackage{common}
\importStyles{formal}{rainbow}{lined}

%\renewcommand{\theenumi}{\alph{enumi}}

\begin{document}
\toc
%\header{text}{text}
%\footer

%+-------------------+
%| +---------------+ |
%| |    Chapter    | |
%| +---------------+ |
%+-------------------+
% Foundations

\chapter{Foundations}

%--------------------------------------------------------------------------------
% Topology
%--------------------------------------------------------------------------------
\section{Topology}

\begin{defn}[]
Given a set $X$, a \textbf{topology} $\mathcal{T}$ on $X$ is a set of subsets of $X$ satisfying:
\begin{enumerate}
	\item $\varnothing, X \in \mathcal{T}$;
	\item it's closed under arbitrary unions; and
	\item it's closed under finite intersections.
\end{enumerate}
The pair $(X, \mathcal{T})$ is then a \textbf{topological space}.
\end{defn}

\noindent If $\mathcal{T} \subseteq \mathcal{T}'$, then we say $\mathcal{T}$ is \textbf{coarser} than $\mathcal{T}'$, or $\mathcal{T}'$ is \textbf{finer} than $\mathcal{T}$.

\begin{defn}[]
A set $\mathcal{B}$ of subsets of $X$ is a \textbf{basis} for $\mathcal{T}$ on $X$ if
\begin{enumerate}
	\item $\mathcal{B}$ covers $X$; and
	\item if $x \in A, B \in \mathcal{B}$, then there's at least one $C \in \mathcal{B}$ such that $x \in C \subseteq A \isct B$.
\end{enumerate}
\end{defn}

\noindent The topology generated by a basis $\mathcal{B}$ is defined to be the coarsest topology containing $\mathcal{B}$, which ends up being the set of all unions of elements of $\mathcal{B}$.

%--------------------------------------------------------------------------------
% Category Theory
%--------------------------------------------------------------------------------
\section{Category Theory}

\begin{defn}[]
A \textbf{category} $\cat{C}$ consists of a class of \textbf{objects} and, for all objects $X, Y \in \cat{C}$, a set of \textbf{morphisms} $\cat{C}(X,Y)$. Morphisms have an associative composition rule
\[
f(gh) = (fg)h,
\] 
and each object has an identity morphism: for all $f: X\to Y$,
\[
f = f\;\id_{X} = \id_{Y}\;f.
\] 
\end{defn}

\noindent Identity morphisms are unique: Suppose $\id_{X}$ and $\id_{X}'$ are both identity morphisms $X \to X$, then $\id_{X} = \id_{X}\id_{X}' = \id_{X}'\id_{X} = \id_{X}'$.

Take any category and reverse the morphisms to get an \textbf{opposite/dual category} $\cat{C}^{op}$. More formally, $\cat{C}^{op}$ has the same objects as $\cat{C}$, but $\cat{C}^{op}(X,Y) = \cat{C}(Y,X)$.

\begin{defn}[]
Let $f: X\to Y$.
\begin{itemize}
	\item $f$ is \textbf{left invertible} if there's a $g: Y \to X$ such that $gf = \id_{X}$.
	\item $f$ is \textbf{right invertible} if there's an $h: Y \to X$ such that $fh = \id_{Y}$.
	\item If both $g$ and $h$ exist, then $g=h$ is the \textbf{inverse} of $f$, and $f$ is an \textbf{isomorphism} between $X$ and $Y$.
\end{itemize}
\end{defn}

\noindent If both left and right inverses exist, they're the same: $h = \id_{X} h = gfh = g \id_{Y} = g$. Inverses are unique: if $g$ and $g'$ are both inverses of $f$, then $g = \id_{X}g = (g'f)g = g'(fg) = g' \id_{X} = g'$.

Isos are reflexive, symmetric, and associative, so they're an equivalence relation. The equivalence classes induced by an iso in a category are \textbf{isomorphism classes} of that category.

%-------------------
% Functors
%-------------------
\subsection{Functors}

\warn{Functors}

\begin{prop}
Functors map isos to isos.
\end{prop}

\noindent Consequentially, functors encode invariants of iso classes, since if $FX \ncong FY$, then $X \ncong Y$.

\begin{ex}[]
Algebraic topology considers functors $\cat{Top}\to \cat{C}$, where $\cat{C}$ is some algebraic category. For instance, homology is a functor $H : \cat{Top} \to R\cat{Mod}$, so if $HX \ncong HY$, then $X \ncong Y$.
\end{ex}

%-------------------
% Hom Functors
%-------------------
\subsection{Hom Functors}

\begin{defn}[]
Let $f: X \to Y$. The \textbf{pushforward} $f_{*}: \cat{C}(\bullet,X) \to \cat{C}(\bullet,Y)$is given by postcomposition with $f$.
\[\begin{tikzcd}
	X && Y \\
	\\
	\bullet
	\arrow["g", from=3-1, to=1-1]
	\arrow["f", from=1-1, to=1-3]
	\arrow["{f_*(g) = fg}"', dashed, from=3-1, to=1-3]
\end{tikzcd}\]
The \textbf{pullback} $f^{*}: \cat{C}(Y,\bullet) \to \cat{C}(X,\bullet)$ is given by precomposition with $f$.
\[\begin{tikzcd}
	X && Y \\
	\\
	&& \bullet
	\arrow["f", from=1-1, to=1-3]
	\arrow["g", from=1-3, to=3-3]
	\arrow["{f^*(g) = gf}"', dashed, from=1-1, to=3-3]
\end{tikzcd}\]
\end{defn}

As a first step toward the Yoneda Lemma, the following attempts to formalize the idea that understanding an object is equivalent to understanding all maps into or out of it.
\begin{prop}
TFAE:
\begin{enumerate}
	\item $f: X \to Y$ is an iso.
	\item $f_{*}: \cat{C}(\bullet,X) \to \cat{C}(\bullet,Y)$ is an iso of sets for all $\bullet$.
	\item $f^{*}: \cat{C}(Y,\bullet) \to \cat{C}(X,\bullet)$ is an iso of sets for all $\bullet$.
\end{enumerate}
\end{prop}

\warn{Hom functors}

%-------------------
% Natural Transformations
%-------------------
\subsection{Natural Transformations}

\warn{natural transformations}

%--------------------------------------------------------------------------------
% The Yoneda Lemma
%--------------------------------------------------------------------------------
\section{The Yoneda Lemma}

\warn{Motivation}

\begin{thrm}[Yoneda Lemma]
	Fix a category $\cat{C}$.

	\textbf{Contravariant:} For all $X \in \cat{C}$ and functors $F: C^{op}\to \cat{Set}$,
	\[
	\text{Nat}( \cat{C}(-,X), F) \cong FX.
	\] 

	\textbf{Covariant:} For all $X \in \cat{C}$ and functors $F: C\to \cat{Set}$,
	\[
	\text{Nat}( \cat{C}(X,-), F) \cong FX.
	\] 
	Note that these are isomorphisms of sets (bijections).
\end{thrm}
\begin{proof}
	This is a proof of the contravariant version, as the covariant proof is analogous. Since $F$ is a contravariant functor, the following diagram commutes for any $f: \bullet \to X$. Note that $\eta f$ is completely determined by where $\eta$ sends $\id_{X}$, which I've denoted $\phi$; since $f$ is arbitrary, this means the choice of $\phi$ completely determines $\eta$ as a whole. There's a clear bijection between choices of $\phi$ and elements of $FX$.
	\[\begin{tikzcd}
	{\cat{C}(\bullet,X)} &&&& {\cat{C}(X,X)} \\
	& {\text{id}_X \; f = f} && {\text{id}_X} \\
	\\
	& {\eta f = (Ff)\phi} && {\phi := \eta \; \text{id}_X} \\
	F\bullet &&&& FX
	\arrow["{\eta_\bullet}", from=1-1, to=5-1]
	\arrow["{\eta_X}", from=1-5, to=5-5]
	\arrow["{f^*}"{description}, from=1-5, to=1-1]
	\arrow["Ff"{description}, from=5-5, to=5-1]
	\arrow[maps to, from=2-4, to=2-2]
	\arrow[maps to, from=2-4, to=4-4]
	\arrow[maps to, from=4-4, to=4-2]
	\arrow[maps to, from=2-2, to=4-2]
	\end{tikzcd}\]
\end{proof}

\begin{cor}
	Let $F$ be the compatible hom functor to get the following.

\noindent Contravariant:
\[
\text{Nat}\left( \cat{C}(-,X), \cat{C}(-,Y) \right) \cong \cat{C}(X,Y).
\] 
Covariant:
\[
\text{Nat}\left( \cat{C}(X,-), \cat{C}(Y,-) \right) \cong \cat{C}(Y,X).
\] 
\end{cor}

%--------------------------------------------------------------------------------
% Set Theory
%--------------------------------------------------------------------------------
\section{Set Theory}


%+-------------------+
%| +---------------+ |
%| |    Chapter    | |
%| +---------------+ |
%+-------------------+
% Constructing Spaces

\chapter{Constructing Spaces}

We can completely determine the topology on a space by determining the continuous maps into/out of it. We'll use this throughout this section to define topologies on new spaces. Since the uniqueness of the universal properties is shown in the below proposition, all that's left to show is existence.

\begin{lem}
Fix a space $A$. Then $(A, \mathcal{T}_1) \cong (A, \mathcal{T}_2)$ via $\id_{A} \iff \mathcal{T}_1 = \mathcal{T}_2$.
\end{lem}
\begin{proof}
	Since isos are necessarily continuous, $\id_{A}^{-1}(U_2) = U_2 \in \mathcal{T}_{1}$ for all $U_2 \in \mathcal{T}_{2}$. Thus $\mathcal{T}_{2} \subseteq \mathcal{T}_{1}$. Similarly $\mathcal{T}_{1} \subseteq \mathcal{T}_{2}$. The backward direction is clear.
\end{proof}

\begin{prop}
For any space $A$, determining $\cat{Top}(\bullet,A)$ and/or $\cat{Top}(A,\bullet)$ uniquely determines the topology on $A$.
\end{prop}
\begin{proof}
	Let $A_1 := (A,\mathcal{T}_{1})$ and $A_2:=(A,\mathcal{T}_{2})$, and suppose $\cat{Top}(A_1,\bullet) = \cat{Top}(A_2,\bullet)$ for all $\bullet$. Then $\id_{A} \in \cat{Top}(A_1,A_1) = \cat{Top}(A_2,A_1)$ and $\id_{A}\in \cat{Top}(A_2,A_2) = \cat{Top}(A_1,A_2)$. Then $A_1 \cong A_2$ via the identity, so by the previous lemma, $\mathcal{T}_1 = \mathcal{T}_{2}$. It's similar when we've determined $\cat{Top}(\bullet,A_{i})$ instead.
\end{proof}

%--------------------------------------------------------------------------------
% The Subspace Topology
%--------------------------------------------------------------------------------
\section{The Subspace Topology}

\begin{prop}[UP of the subspace topology]
	Let $i:A \inj B$ be an injective map. The subspace topology (induced by $i$) on $A$ is the unique topology on $A$ such that for all maps $f$ into $A$, the composition $if$ is continuous $\iff$ $f$ is continuous.
\[\begin{tikzcd}
	\bullet \\
	\\
	A && B
	\arrow["i", hook, from=3-1, to=3-3]
	\arrow["{ f}"', dashed, from=1-1, to=3-1]
	\arrow["if"{description}, dashed, from=1-1, to=3-3]
\end{tikzcd}\]
\end{prop}
This topology has the form
\[
\left\{ i^{-1}(U) \;|\; U \text{ open in } B  \right\} = \left\{ U \isct iA \;|\; U \text{ open in } B \right\}.
\]
It's the coarsest topology on $A$ for which $i: A \inj B$ is continuous. When $i$ is the natural inclusion, this all coincides with the usual definition of the subspace topology.

\warn{discrete topology makes i continuous (and is trivially the most fine top doing this), but we can find a coarser one}

It's fine to do this even though $A$ might not be a literal subset of $B$. If $i: A \inj B$ is an injective map, then $A$ is isomorphic as a set to its image $iA \subseteq B$. Using the UP, the space $A$ with the subspace topology induced by $i$ is homeomorphic to the space $iA \subseteq B$ with the subspace topology induced by the natural inclusion; so we can think of $A$ as being embedded in $B$.

\begin{defn}[]
Suppose $f: A \inj B$ is a continuous injection (so $A$ and $B$ are already equipped with topologies). It's an \textbf{embedding} when the topology on $A$ is the same as the subspace topology induced by $f$.
\end{defn}

%--------------------------------------------------------------------------------
% The Quotient Topology
%--------------------------------------------------------------------------------
\section{The Quotient Topology}

Given a set $S$ and surjection $\pi:X \surj S$, there's a bijection of sets
\begin{align*}
	S &\stackrel{\cong}{\longrightarrow} X/\sim\\
	s &\mapsto \pi^{-1}s
\end{align*}
If $X$ is a topological space, we can make $S$ inherit a topology from it via the surjection. If $S$ had the indiscrete topology (only $\varnothing,S$ are open), then $\pi$ is trivially continuous. But we can find a finer topology for $S$, namely
\[
U \text{ open in } S \iff \pi^{-1}U \text{ open in } X.
\] This is the finest topology for which $\pi$ is continuous.

\warn{Can think of the quotient topology as being defined on either $S$ or $X/\sim$ since they're iso as sets.}

\warn{Have to be careful, since the union of topologies isn't necessarily a top. If we had taken an intersection instead, that's fine since that'll always be a topology itself.}

\begin{prop}[UP of the quotient topology]
	Let $\pi: X \surj S$ be a surjective map. The quotient topology (induced by $\pi$) on $S$ is the unique topology on $S$ such that for all maps $f$ out of $S$, the composition $f\pi$ is continuous $\iff f$ is continuous.
	\[\begin{tikzcd}
	X \\
	\\
	S && \bullet
	\arrow["f\pi", dashed, from=1-1, to=3-3]
	\arrow["\pi"', two heads, from=1-1, to=3-1]
	\arrow["f"', dashed, from=3-1, to=3-3]
\end{tikzcd}\]
\end{prop}

One way to interpret this UP is: the continuous maps $S \to \bullet$ are continuous maps $X \to \bullet$ that are constant on the fibers of $\pi:X\surj S$.

\warn{Define quotient map}

\begin{ex}[]
Consider the map
\begin{align*}
	\pi: [0,1] &\to S^{1}\\
	t &\mapsto \left( \cos(2\pi t), \sin(2\pi t) \right).
\end{align*}
$\pi$ is a quotient map, so for any space $\bullet$, the continuous maps $S^{1}\to \bullet$ are the same as continuous maps $[0,1]\to \bullet$ that factor through $\pi$. In simpler terms, the continuous maps $S^{1}\to \bullet$ are the same as the loops in $\bullet$.
\end{ex}

\begin{ex}[]
The \textbf{projective space} $\R\mathbb{P}^{n}$ is the quotient of $\R^{n+1}-\left\{ 0 \right\}$ by $x \sim \lambda x$ for $\lambda \in \R$. In words, it's the set of lines through the origin in $\R^{n+1}$, and we give it a topology via the quotient topology.
\end{ex}

\begin{ex}[]
We can identify sides of a square to get common spaces with topologies defined by the quotient topology: the torus, the M\"obius band, the Klein bottle, and the projective plane $\R\mathbb{P}^{2}$.
\end{ex}

\end{document}
