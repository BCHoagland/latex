\documentclass[twoside,10pt]{article}
\usepackage{/Users/bradenhoagland/latex/styles/toggles}
\toggletrue{sectionbreaks}
%\toggletrue{sectionheaders}
\newcommand{\docTitle}{Cohomology}
\usepackage{/Users/bradenhoagland/latex/styles/common}
\importStyles{modern}{rainbow}{boxy}

%\renewcommand{\theenumi}{\alph{enumi}}

\begin{document}
%\tableofcontents

%--------------------------------------------------------------------------------
% (Co)homology with coefficients
%--------------------------------------------------------------------------------
\section{(Co)homology with coefficients}

Let $\mathcal{C}$ be a chain complex of free $\mathbb{Z}$-modules (free abelian groups)
\[
\cdots \to C_{n+1} \stackrel{\p}{\to }  C_{n}\stackrel{\p}{\to} C_{n-1}\to  \cdots,
\] then we can apply any functor $\mathcal{F}:\cat{Ab} \to \cat{Ab}$ (perhaps contravariant) to get another complex $\mathcal{F}\mathcal{C}$. In particular, we can use the following two functors, where $G$ is some abelian group.
\begin{itemize}
	\item $- \otimes_{} G$ (covariant) maps $C \mapsto C \otimes_{}G$ and $\phi \mapsto \phi \otimes_{} \id_{}$; since $G$ is an abelian group, we're implicitly using $\otimes_{\mathbb{Z}}$;
	\item $\hom(-,G)$ (contravariant) maps $C \mapsto \hom(C,G)$ and $\phi \mapsto \phi^{*}$ (precomposition with $\phi$).
\end{itemize}

\begin{defn}[]
	For a chain complex $\mathcal{C}$, its \textbf{homology with $G$ coefficients} is
	\[
		H_{*}(\mathcal{C}; G) := H_{*}(\mathcal{C} \otimes_{}G).
	\] Its \textbf{cohomology with $G$ coefficients} is
	\[
		H^{*}(\mathcal{C}; G) := H_{*}(\hom(\mathcal{C},G)).
	\] 
\end{defn}

Note that $C \otimes_{\mathbb{Z}}\mathbb{Z} \cong C$ for any abelian group $C$, so $H_{*}(\mathcal{C};\mathbb{Z}) \cong H_{*}(\mathcal{C})$. Also, when dealing with $H^{*}$, we throw ``co-" on the front of all the vocab words, e.g. ``cocyle" instead of ``cycle".

\warn{go over why using TP and hom instead of hom and hom...}

%--------------------------------------------------------------------------------
% Ext and Tor
%--------------------------------------------------------------------------------
\section{Ext and Tor}

Derived functors measure the extent to which a functor fails to preserve exactness. Ext and Tor are two examples of derived functors, which we will use in a bit to formulate the Universal Coefficient Theorem.

\begin{defn}[]
        A covariant functor $\mathcal{F}$ is one of the below if it preserves exactness in the manner depicted.
\begin{equation*}
        \begin{aligned}[c]
                \text{\textbf{exact}} \\
                \text{\textbf{left exact}} \\
                \text{\textbf{right exact}}
        \end{aligned}
        \qquad
        \begin{aligned}[c]
                A\to B\to C \quad&\leadsto\quad 0\to \mathcal{F}A\to \mathcal{F}B\to \mathcal{F}C\to 0\\
                0\to A\to B\to C \quad&\leadsto\quad 0\to \mathcal{F}A\to \mathcal{F}B\to \mathcal{F}C\\
                A\to B\to C\to 0 \quad&\leadsto\quad \qquad \mathcal{F}A\to \mathcal{F}B\to \mathcal{F}C\to 0
        \end{aligned}
\end{equation*}
The following apply to a contravariant functor $\mathcal{G}$ instead.
\begin{equation*}
        \begin{aligned}[c]
                \text{\textbf{exact}} \\
                \text{\textbf{left exact}} \\
                \text{\textbf{right exact}}
        \end{aligned}
        \qquad
        \begin{aligned}[c]
                A\to B\to C \quad&\leadsto\quad 0\to \mathcal{G}C\to \mathcal{G}B\to \mathcal{G}A\to 0\\
                A\to B\to C\to 0 \quad&\leadsto\quad 0\to \mathcal{G}C\to \mathcal{G}B\to \mathcal{G}A\\
                0\to A\to B\to C \quad&\leadsto\quad \qquad \mathcal{G}C\to \mathcal{G}B\to \mathcal{G}A\to 0
        \end{aligned}
\end{equation*}
\end{defn}

\warn{In \cat{Ab}, the above definitions are equivalent to those given by including 0's on the left and right of each LHS, but the forms above are a bit easier to work with since we won't always have things with 0's bookending them.}

\begin{defn}[]
A \textbf{free} resolution of an abelian group $A$ is an exact sequence of abelian groups
\[
\cdots \to F_2\to F_1\to F_0\to A \to 0,
\] where each $F_{i}$ is free.
\end{defn}

\begin{note}[]
We're really only concerned with the derived functors Ext and Tor, which are both formulated in terms of projective resolutions. But that's okay, since a free module is projective. Thus we only need to concern ourselves with free resolutions.
\end{note}

Supose we have a right exact covariant functor $\mathcal{F}$ and a free resolution
\[
\cdots \to F_2\to F_1\to F_0\epi A \to 0,
\] 
then applying $\mathcal{F}$ gives
\[
	\cdots \to \mathcal{F} F_2 \to {\color{blue}\mathcal{F}F_1 \to \mathcal{F}F_0 \epi \mathcal{F}A \to 0.}
\] 
Since $\mathcal{F}$ is right exact, the blue subsequence above is still exact. Removing $\mathcal{F}A$, we get a new sequence
\[
\cdots \to \mathcal{F}F_2 \to \mathcal{F}F_1 \to \mathcal{F}F_0 \to 0,
\] 
Taking homology gives us the \textbf{derived functors} of $\mathcal{F}$. A similar story holds when $\mathcal{F}$ is a contravariant left exact functor instead. \warn{check that still functor, i.e. a morphism $X\to Y$ induces morphism $L_{i}X\to L_{i}Y$.}

\begin{thrm}
	\label{diff-resolutions-same-derived}
	Different free resolutions yield \warn{isomorphic} derived functors.
\end{thrm}
\begin{proof}
	\warn{Do this.}
\end{proof}

\begin{note}
	A nice thing about working with abelian groups is that you can find short free resolutions, which makes calculating derived functors much easier by \Cref{diff-resolutions-same-derived}.
\end{note}

\begin{prop}
	Every abelian group $A$ has a free resolution
	\[
		0 \to \ker \varepsilon \inj \ang{A} \stackrel{\varepsilon}{\epi} A \to 0.
	\] 
\end{prop}
\begin{proof}
	First note that all objects in the sequence are free abelian since the kernel of a free abelian group is itself free abelian. Construct $\varepsilon$ by extenting $\id_{A}$. Exactness is clear.
\end{proof}

\begin{note}[]
To be clear, $\Ang{A}$ is not necessarily the same thing as $A$, since $A$ might have extra relations. None of these relations are in $\Ang{A}$. Thus $\ker \varepsilon$ is generated by the relations of $A$.
\end{note}

\begin{cor}
\warn{calc derived functors for abelian group.}
\end{cor}

\warn{how to turn $0\to A\to B\to C\to 0$ SES into LES using derived functors?}

With all this in place, we can finally define Ext and Tor as the derived functors of particular functors.

\begin{defn}[]
	\textbf{Ext} is the derived functors of $\hom(-,G)$, and \textbf{Tor} is the derived functors of $-\otimes G$.
\end{defn}

Note that both of these use projetive resolutions, as $\hom(-,G)$ is contravariant and left exact and $-\otimes_{}G$ is covariant and right exact. \warn{Go over earlier in more detail why contra/left and cov/right work with free resolutions.}


%--------------------------------------------------------------------------------
% The Universal Coefficient Theorem
%--------------------------------------------------------------------------------
\section{The Universal Coefficient Theorem}

Homology with coefficients is useful for simplifying certain calculations, but as it turns out, it encodes the exact same information that the usual homology with $\mathbb{Z}$ coefficients does. The idea is that although $H_{n}(\mathcal{C} \otimes_{}G) \not\cong H_{n}(\mathcal{C}) \otimes_{}G$ and $H^{*}(\hom(\mathcal{C},G)) \not\cong \hom(H_{n}\mathcal{C},G)$ in general, we can use these as approximations and introduce some correction terms. These corrections are Ext and Tor.

\begin{thrm}[The Universal Coefficient Theorem]
Let $\mathcal{C}$ be a chain complex of free abelian groups, and let $G$ be any abelian group. Then there are short exact sequences
\[
\begin{tikzcd}
	0 \rar & H_{n}\mathcal{C} \otimes_{}G \rar & H_{n}(\mathcal{C};G) \rar & \tor(H_{n-1}\mathcal{C},G) \rar & 0, \\
	0 & \lar \hom(H_{n}\mathcal{C},G) & \lar H^{n}(\mathcal{C};G) & \lar \ext(H_{n-1}\mathcal{C},G) & \lar 0
\end{tikzcd}
\] that are natural and split (although the splitting isn't natural). In other words,
\begin{align*}
	H_{n}(\mathcal{C};G) &\cong (H_{n}\mathcal{C} \otimes_{}G) \oplus \tor(H_{n-1}\mathcal{C},G),\\
	H^{n}(\mathcal{C};G) &\cong \hom(H_{n}\mathcal{C},G) \oplus \ext(H_{n-1}\mathcal{C},G).
\end{align*}
\end{thrm}

hello there. changes.

\end{document}
