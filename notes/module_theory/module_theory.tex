\documentclass[twoside,10pt]{report}
\usepackage{/Users/bradenhoagland/latex/styles/toggles}
\toggletrue{sectionbreaks}
%\toggletrue{sectionheaders}
\newcommand{\docTitle}{Module Theory}
\usepackage{/Users/bradenhoagland/latex/styles/common}
\importStyles{modern}{rainbow}{boxy}

%\renewcommand{\theenumi}{\alph{enumi}}

\usepackage{multicol}
\setlength{\columnseprule}{0.5pt}
\setlength{\columnsep}{1cm}

\begin{document}

\textbf{Notes}
\begin{enumerate}
	\item Done
	\item Done
	\item Done
	\item {\color{red}Nope}
	\item {\color{orange}Partially}
	\item {\color{orange}Partially}
	\item {\color{orange}Partially}
	\item {\color{red}Nope}
	\item {\color{red}Nope}
	\item {\color{red}Nope}
\end{enumerate}

\vspace{5mm}
\textbf{Exercises}
\begin{enumerate}
        \item Done
        \item {\color{yellow}Almost}
        \item {\color{yellow}Almost}
        \item {\color{red}Nope}
	\item {\color{red}Nope}
	\item {\color{red}Nope}
	\item {\color{red}Nope}
        \item {\color{red}Nope}
        \item {\color{red}Nope}
        \item {\color{red}Nope}
\end{enumerate}

\tableofcontents

%+-------------------+
%| +---------------+ |
%| |    Chapter    | |
%| +---------------+ |
%+-------------------+
% Modules

\chapter{Modules}

%--------------------------------------------------------------------------------
% Modules and Algebras
%--------------------------------------------------------------------------------
\section{Modules and Algebras}

Modules are a generalization of vector spaces, replacing the field of scalars with a unital ring of scalars.

\begin{defn}[]
	Let $R$ be a unital ring. A \textbf{(left) $R$-module} is an additive abelian group $M$ with a left action $R \times M \to  M$ satisfying
\begin{enumerate}
	\item $\lambda(x+y) = \lambda x + \lambda y$;
	\item $(\lambda + \mu)x = \lambda x + \mu x$;
	\item $\lambda(\mu x) = (\lambda \mu)x$; and
	\item $1_{R}x = x$.
\end{enumerate}
\textbf{Right $R$-modules} are defined similarly.
\end{defn}

Note that if $R$ is commutative, then any left $R$-module can be given the structure of a right $R$-module by defining $x \lambda \doteq \lambda x$ (commutativity of $R$ is needed to prove 3). Similarly, any right $R$-module is also a left $R$-modules. Thus when $R$ is commutative, left and right $R$-modules are really the same thing, so we don't need to specify.

If $F$ is a field, then an $F$-module is the same thing as an $F$-vector space.

\begin{prop}
Basic properties of modules:
\begin{enumerate}
	\item $\lambda 0_{M} = 0_{M}$;
	\item $0_{R}x = 0_{M}$;
	\item $\lambda(-x) = -(\lambda x) = (-\lambda) x$.
\end{enumerate}
If $R$ is a division ring, then we also have
\begin{enumerate}
	\item[4.] $\lambda x = 0_{M} \implies \lambda=0_{R}$ or $x = 0_{M}$.
\end{enumerate}
\end{prop}

The definition of modules gives us addition and scalar multiplication, but we still don't have a way of multiplying module elements together. Providing this is exactly the role of an algebra, which adds a ring structure to a module. \warn{It seems like there isn't much of a difference between a ring and an algebra, so you should ask someone about this...}

\begin{defn}[]
Let $R$ be a commutative unital ring. An $R$-\textbf{algebra} is an $R$-module $M$ along with a ``multiplication" map
\begin{align*}
	M \times M &\to M \\
	(x,y) &\mapsto xy.
\end{align*}
This map distributes over addition and satisfies
\[
	\lambda(xy) = (\lambda x)y = x (\lambda y).
\] 
\end{defn}
\warn{Why is $R$ commutative?}
We can form more specific types of algebras by putting restrictions on the multiplication map. \textbf{Associative and commutative algebras} have associative and commutative multiplication maps, respectively. A \textbf{unital} algebra has a multiplicative identity. A \textbf{division algebra} is a unital associative algebra in which every nonzero element has a multiplicative inverse.

%--------------------------------------------------------------------------------
% Submodules
%--------------------------------------------------------------------------------
\section{Submodules}

A module is just an abelian group with a left action, so we can define a submodule to be just a subgroup that respects this action.

\begin{defn}[]
A \textbf{submodule} of an $R$-module $M$ is a subgroup of $M$ that is closed under the left action of $R$ on $M$.
\end{defn}

A module $N$ is a submodule of $M$ if and only if $N$ is closed under subtraction and scalar multiplication (the subtraction emcompasses both addition and additive inverses). From this we infer the following simple characterization of a submodule.
\begin{prop}
$N$ is an submodule of $M$ if and only if
\[
\lambda x + \mu y \in N
\] for all $x,y \in N$ and $\lambda,\mu \in R$.
\end{prop}

Thus given any set $S \subseteq M$, we can form a submodule of $M$ by adding in all linear combinations of the elements of $S$ (remember that linear combinations are by definition finite sums, so the induction works). This could be a good enough definition of $\ang{S}$, but we have to make sure that we aren't adding in any unnecessary terms. The following definition ensures this is the case, the next proposition shows that the definition makes sense, and the following theorem shows that our definition is equivalent to the linear combination approach.

\begin{defn}[]
Given a set $S \subseteq M$, let $\ang{S}$ denote the intersection of all submodules of $M$ containing $S$.
\end{defn}

\begin{prop}
If $\left\{ M_{\alpha} \right\}_{\alpha}$ is a family of submodules of $M$, then $\bigcap_{\alpha}M_{\alpha}$ is also a submodule of $M$.
\end{prop}

\begin{thrm}[]
	\label{thrm:LC}
	Let $S \subseteq M$, and let $LC(S)$ denote the set of all linear combinations of $S$. Then
	\[
	\ang{S}=
	\begin{cases}
		\left\{ 0 \right\} & \text{ if } S = \varnothing,\\
		LC(S) & \text{ otherwise}.
	\end{cases}
	\] 
\end{thrm}
\begin{proof}
	The case $S = \varnothing$ is clear since all subgroups must contain 0, so assume $S$ is nonempty. It's clear that $LC(S)$ is a submodule of $M$. Since $S \subseteq LC(S)$, this means $LC(S)$ is a submodule of $M$ containing $S$, i.e. $\ang{S} \subseteq LC(S)$. But every linear combination of $S$ must be in any submodule containing $S$, so $LC(S) \subseteq \ang{S}$ too. Thus $\ang{S}= LC(S)$.
\end{proof}

If \{$M_{\alpha}\}_{\alpha}$ is a family of submodules of $M$, then $\bigcup_{\alpha}M_{\alpha}$ won't be a submodule in general (unlike $\bigcap_{\alpha}M_\alpha$), but it can certainly generate one. $\ang{\cup_{\alpha}M_{\alpha}}$ can be interpreted as the smallest submodule of $M$ containing each of the $M_{\alpha}$, and we can construct it by once again filling in all the missing linear combinations.

\begin{prop}
	Let $\mathcal{A}$ be some index set, and let $\mathbb{P}^{\star}(\mathcal{A})$ denote the set of all nonempty finite subsets of $\mathcal{A}$. Then $\Ang{\;\bigcup_{\alpha}M_{\alpha}\;}$ is all finite sums of the form
	\[
	\sum_{\beta \in \mathcal{B}} m_{\beta},
	\] where $\mathcal{B} \in \mathbb{P}^{\star}(\mathcal{A})$ and $m_{\beta} \in M_{\beta}$.
\end{prop}
\begin{proof}
	All linear combinations of the elements of $\bigcup_{\alpha}M_{\alpha}$ is this form, and $LC = \Ang{\;\bigcup_{\alpha}M_{\alpha}\;}$ by Theorem \ref{thrm:LC} since $\bigcup_{\alpha}M_{\alpha}$ is nonempty (it must contain 0).
\end{proof}
This motivates the notation
\[
\sum_{\alpha}M_{\alpha} \doteq \ang{\; \bigcup_{\alpha}M_{\alpha}\;}
\] 
and the terminology ``sum of the family $\left\{ M_{\alpha} \right\}_{\alpha}$."

\begin{thrm}[Modular Law]
Let $M$ be an $R$-module, and let $A,B,C$ be submodules of $M$ with $C \subseteq A$. Then
\[
	A \uni (B+C) = (A\uni B) + C.
\] 
\end{thrm}
\warn{I have no idea why the book introduced this now.}

%--------------------------------------------------------------------------------
% Morphisms
%--------------------------------------------------------------------------------
\section{Morphisms}

As usual, an $R$-morphism respects the structure of $R$-modules.
\begin{defn}[]
An $R$-\textbf{morphism} is a map $f:M\to N$ between $R$-modules satisfying
\begin{enumerate}
	\item $f(x+y) = f(x) + f(y)$;
	\item $f(\lambda x) = \lambda f(x)$.
\end{enumerate}
\end{defn}

Note that if $R$ is a field, then an $R$-morphism is just a linear map. Also note that if $f:M\to N$ is an $R$-morphism, then $\ker f$ is a submodule of $M$ and $\im f$ is a submodule of $N$.

\begin{prop}
Basic properties an $R$-morphism $f:M\to N$.
\begin{enumerate}
	\item $f(0_{M})=0_{N}$.
	\item $f(-x)=-f(x)$.
\end{enumerate}
\end{prop}
Because we like to be fancy, we'll use categorical language to describe specific types of $R$-morphisms:
\begin{align*}
	R\text{-monomorphism}&: \quad\quad M \mono N,\\
	R\text{-epimorphism}&: \quad\quad M \epi N.
\end{align*}
It's straightforward to show that the inverse of a bijective $R$-morphism is also an $R$-morphism, i.e. an $R$-isomorphism is just a bijective $R$-morphism. The usual properties of composed morphisms of course hold too:
\begin{itemize}
	\item The composition of morphisms/monos/epis is a morphism/mono/epi.
	\item If $g \circ f$ is mono, then so is $f$.
	\item If $g \circ f$ is epi, then so if $g$.
\end{itemize}
As you might expect, a map between modules induces maps between their submodules.

\begin{prop}
	Suppose we have an $R$-morphism $f:M\to N$. Then for any submodule $X$ of $M$, the image $f(X)$ is a submodule of $N$. Additionally, for any submodule $Y$ of $N$, the preimage $f^{-1}(Y)$ is a submodule of $ M$.
\end{prop}
These maps induce maps between the entire submodule lattices $L(M)$ and $L(N)$:
\begin{equation*}
	\begin{aligned}[c]
		\begin{tikzcd}
			L(M) \rar[bend left]{f^{\to }} & L(N) \lar[bend left]{f^{\leftarrow}}
		\end{tikzcd}
	\end{aligned}
	\qquad\qquad
	\begin{aligned}[c]
		f^{\to }: X &\mapsto f(X) \\
	f^{\leftarrow}: Y &\mapsto f^{-1}(Y)
	\end{aligned}
\end{equation*}
Note that $f^{\to }$ and $f^{\leftarrow}$ are inclusion-preserving. We can also show how they interact with each other.
\begin{prop}
	Let $f$ be an $R$-morphism $M\to N$. If $A \in L(M)$ and $B \in L(N)$, then
	\begin{enumerate}
		\item $f^{\to }\left( A \isct f^{\ot}(B) \right) = f^{\to }(A) \isct B$;
		\item $f^{\ot}\left( B + f^{\to }(A) \right) = f^{\ot}(B) + A$.
	\end{enumerate}
\end{prop}
\warn{Prove this.}
\begin{cor}
	If $A \in L(M)$ and $B \in L(N)$, then
	\begin{enumerate}
		\item $f^{\to }(f^{\ot}(B)) = B \isct \im f$;
		\item $f^{\ot}(f^{\to }(A)) = A + \ker f$.
	\end{enumerate}
\end{cor}


\warn{Is there a way to generalize this to something other than modules? If we have a morphism $f:X\to Y$, will $f(x)$ and $f^{-1}(y)$ have that proerty if $x$ and $y$ have the property, respectively?}

\warn{Is the defn of $R$-morphism really just saying that it preserves module-ness by respecting linear combs?}

{\color{blue}
	$f$ inj: There is a map $g:B\to A$ such that $g \circ f = 1_{A}$.

	$f$ surj: There is a map $g:B\to A$ such that $f \circ g = 1_{B}$.
}

%-------------------
% Lifts and Extensions of $R$-Morphisms
%-------------------
\subsection{Lifts and Extensions of \texorpdfstring{$R$}{R}-Morphisms}

It's common to want to extend or lift an $R$-morphism. The following propositions give criteria for when this is possible.

\begin{prop}
	\label{prop:unique-lift}
Suppose $A, B, C$ are nonempty.
\[
\begin{tikzcd}
	& B \dar{f} \\
	C \arrow[ur, dashed, "\exists! \;h"] \rar{g} & A
\end{tikzcd}
\] 
Suppose $f$ is monic. Then there is a unique $R$-morphism $h$ lifting $g$ if and only if $\im g \subseteq \im f$. In this case, $h$ is epic if and only if $\im g = \im f$.
\end{prop}
\begin{proof}
	The forward direction of the first statement is clear. To go backwards, note that any $c$, there is a $b$ such that $g(c) = f(b)$ since $\im g \subseteq \im f$. Define $h$ by $c \mapsto b$, then $f(h(c)) = f(b)=g(c)$, so $h$ lifts $g$. This map is well-defined and unique since $f$ is monic. To show it's an $R$-morphism, use the morphism properties of $f$ and $g$ to show $f(h(\lambda c)) = f(\lambda h(c))$ and $f(h(c_1+c_2)) = f(h(c_1)+h(c_2))$, then use the fact that $f$ is monic.

	If $h$ is epic, it's straightforward to show that $\im f \subseteq \im g$, which proves their equality. Conversely, fix $b$ and suppose $\im f = \im g$. Then $f(b) = g(c) = f(h(c))$ for some $c$, which implies $b = h(c)$ since $f$ is monic.
\end{proof}

\begin{lem}
Suppose $f$ and $g$ are $R$-morphisms. If $\ker f \subseteq \ker g$, then
\[
	f(x) = f(y) \implies g(x) = g(y).
\] 
\end{lem}
\begin{proof}
	If $f(x) = f(y)$, then $f(x-y)=0$, so $x-y \in \ker f \subseteq \ker g$. Thus $g(x-y) = 0$, so $g(x)=g(y)$.
\end{proof}

\begin{prop}
	\label{prop:unique-ext}
	Suppose $A, B, C$ are nonempty.
	\[
	\begin{tikzcd}
		& B \dar[dashed]{\exists! \; h} \\
		A \arrow[ur, "f"] \rar{g} & C
	\end{tikzcd}
	\] 
	Suppose $f$ is epic. Then there is a unique $R$-morphism $h$ extending $g$ if and only if $\ker f \subseteq \ker g$. In this case, $h$ is monic if and only if $\ker f=\ker g$.
\end{prop}
\begin{proof}
	The forward direction of the first statement is clear. To go backwards, since $f$ is epic, any $b$ can be written $b=f(a)$ for some $a$. Then define $h:b \mapsto g(a)$. This clearly lifts $g$, and it is well-defined and unique by the preceding lemma. Showing it's an $R$-morphism is a standard check by writing $b=f(a)$ and using the morphism properties of $f$ and $g$.

	If $h$ is monic, then for $a \in \ker g$, we have $h(f(a)) = g(a)=0$. But since $f$ is monic, this implies $f(a)=0$, so $a \in \ker f$. Thus $\ker g \subseteq \ker f$, and we already know the opposite inclusion. Conversely, using the $b=f(a)$ fact, $h(b_1)=b(b_2)\implies g(a_1)=g(a_2)$, so $a_1-a_2 \in \ker g = \ker  f$, so $b_1=f(a_1) = f(a_2)=b_2$.
\end{proof}


%-------------------
% Consequences of Exactness
%-------------------
\subsection{Consequences of Exactness}

We'll start out by noting some obvious characterizations of morphisms in terms of exact sequences. Quick reminder if if a sequence is exact, the composition of any two subsequent morphisms is 0.
\begin{prop}
	Monos, epis, and isos in terms of exact sequences:
\begin{enumerate}
	\item $f$ is monic $\iff 0\to M \stackrel{f}{\to } N$ is exact.
	\item $f$ is epic $\iff M \stackrel{f}{\to } N \to 0$ is exact.
	\item $f$ is iso $\iff 0 \to M \stackrel{f}{\to } N \to 0$ is exact.
\end{enumerate}
\end{prop}

Now to prove that a bunch of diagrams commute if some exactness condition holds. I don't include any diagram chases, but luckily only the Four Lemma needs one. Also, everything below is implicitly assumed to be using $R$-modules and $R$-morphisms.
\begin{prop}
The diagram commutes if the row is exact and $\theta g = 0$.
\[
\begin{tikzcd}
	& & A \arrow[dl, dashed, "\exists! \;h"'] \dar{g} \\
	0 \rar & X \rar{f} & Y \rar{\theta} & Z
\end{tikzcd}
\] 
\end{prop}
\begin{proof}
	$f$ must be monic and $\im g \subseteq \im f$, so a unique $h$ exists by Proposition \ref{prop:unique-lift}.
\end{proof}

Note that if $X = \ker \theta$ and $f$ is an inclusion map, then the row will always be exact.

\begin{prop}
The diagram commutes if the row is exact and $g\theta=0$.
\[
\begin{tikzcd}
	& A \\
	X \rar{\theta} & Y \uar{g}\rar{f} & Z \rar\arrow[ul, dashed, "\exists!\; h"'] & 0
\end{tikzcd}
\] 
\end{prop}
\begin{proof}
	$f$ must be epic and $\ker f \subseteq \ker g$, so a unique $h$ exists by Proposition \ref{prop:unique-ext}.
\end{proof}

Note that if $Z = X / \im \theta$ and the $f$ is a projection map, then the row will always be exact.

\begin{thrm}[Four Lemma]
Suppose the following diagram commutes and has exact rows.
\[
\begin{tikzcd}
	A \rar\dar{\alpha} & B \rar\dar{\beta} & C\rar\dar{\gamma} & D\dar{\delta} \\
	A'\rar & B'\rar & C'\rar & D'
\end{tikzcd}
\] Then the following hold:
\begin{enumerate}
	\item If $\alpha,\gamma$ are epic and $\delta$ is monic, then $\beta$ is epic.
	\item If $\alpha$ is epic and $\beta,\gamma$ are monic, then $\gamma$ is monic.
\end{enumerate}
\end{thrm}

\begin{thrm}[Five Lemma]
Suppose the following diagram commutes and has exact rows.
\[
\begin{tikzcd}
	A\rar\dar{\alpha_1} & B\rar\dar{\alpha_2} & C\rar\dar{\alpha_3} & D\rar\dar{\alpha_4} & E\dar{\alpha_5} \\
	A'\rar & B'\rar & C'\rar & D'\rar & E'
\end{tikzcd}
\] 
If $\alpha_1,\alpha_2,\alpha_4,\alpha_5$ are iso, then so is $\alpha_3$.
\end{thrm}
\begin{proof}
	Apply the Four Lemma to the first three squares to show that $\alpha_3$ is monic, and to the last three squares to show that $\alpha_3$ is epic. Since it's an $R$-morphism, this is enough to show it's iso.
\end{proof}

\begin{cor}[Short Five Lemma]
Suppose the following diagram commutes and has exact rows.
\[
\begin{tikzcd}
	0\rar & A\rar\dar{\alpha} & B\rar\dar{\beta} & C\rar\dar{\gamma} & 0 \\
        0\rar & A'\rar & B'\rar & C'\rar & 0
\end{tikzcd}
\] 
If $\alpha,\gamma$ are iso, then so is $\beta$.
\end{cor}
\warn{Check D\&F about the case "any two are iso".}

This last corollary is just a special case of the Five Lemma when our two rows are short exact sequences. \warn{If the D\&F case applies here, that would be very useful for determining when a map of SESs is iso.}

\warn{Are all epis split in the category of modules?}

%+-------------------+
%| +---------------+ |
%| |    Chapter    | |
%| +---------------+ |
%+-------------------+
% Constructing Modules

\chapter{Constructing Modules}

%--------------------------------------------------------------------------------
% Quotient Modules
%--------------------------------------------------------------------------------
\section{Quotient Modules}

\warn{Hello there.}

%--------------------------------------------------------------------------------
% Products and Coproducts
%--------------------------------------------------------------------------------
\section{Products and Coproducts}

We can make a \textbf{direct product} of $R$-modules $\prod_{a}M_{\alpha}$ into an $R$-module itself by defining
\begin{align*}
	(x_{\alpha})_{\alpha} + (y_{\alpha})_{\alpha} &\doteq (x_{\alpha} + y_{\alpha})_{\alpha}, \\
	\lambda (x_{\alpha})_{\alpha} &\doteq (\lambda x_{\alpha})_{\alpha}.
\end{align*}
If we add the restriction that only a finite number of the coordinates can be nonzero, then we get the \textbf{(external) direct sum} $\bigoplus_{\alpha} M_{\alpha}$. In this context, $\pi_{\alpha}$ denotes the canonical projection onto the $\alpha$-th coordinate, and $i_{\alpha}$ denotes the $\alpha$-th canonical injection
\[
	x \mapsto (\dots, 0, x, 0, \dots),
\] where the single nonzero coordinate is the $\alpha$-th coordinate.

Instead of worrying about individual elements, we can use the universal properties of the product and coproduct to characterize direct products and sums.

\begin{note}[]
I still use the notation $\pi_{\alpha}$ and $i_{\alpha}$ in the general categorical setting, but unless I'm specifically using them with a direct product or direct sum, they're just ordinary morphisms instead of special projections or injections.
\end{note}


\begin{defn}[]
Fix a category $\cat{C}$ and objects $\left\{ M_{\alpha} \right\}_{\alpha}$. A \textbf{product} of $\left\{ M_{\alpha} \right\}_{\alpha}$ is an object $P$ with morphisms $\pi_{\alpha}:P\to M_{\alpha}$ such that for any other object $N$ and morphisms $f_{\alpha}:N\to M_{\alpha}$, there is a unique morphism $f:N\to P$ lifting each $f_{\alpha}$.
\[
\begin{tikzcd}
	& P \dar{\pi_{\alpha}} \\
	N \arrow[ur,dashed,"f"]\rar["f_{\alpha}"'] & M_{\alpha}
\end{tikzcd}
\] 
Dually, a \textbf{coproduct} of $\left\{ M_{\alpha} \right\}_{\alpha}$ is an object $C$ with morphisms $i_{\alpha}:M_{\alpha}\to C$ such that for any other object object $N$ and morphisms $f_{\alpha}:M_{\alpha}\to N$, there is a unique morphism $f:C\to N$ extending each $f_{\alpha}$.
\[
\begin{tikzcd}
	& C \arrow[dl,dashed,"f"'] \\
	N & M_{\alpha} \lar{f_{\alpha}}\uar["i_{\alpha}"']
\end{tikzcd}
\] 
\end{defn}

\begin{prop}
	If $(P, \left\{ \pi_{\alpha} \right\})$ is a product, then each $\pi_{\alpha}$ is epic. If $(C, \left\{ i_{\alpha} \right\})$ is a coproduct, then each $i_{\alpha}$ is monic.
\end{prop}
\begin{proof}
	Fix $\alpha$, let $N=M_{\alpha}$, and let $f_{\alpha}$ be the identity. Then there are unique $f_{P}, f_{C}$ such that $\pi_{\alpha}f_{P}=1$ and $f_{C}i_{\alpha}=1$, i.e. $\pi_{\alpha}$ is epic and $i_{\alpha}$ is monic.
\end{proof}

\begin{thrm}[Uniqueness]
	If $(P, \left\{ \pi_{\alpha} \right\})$ is a product, then $(Q, \left\{ \phi_{\alpha} \right\})$ is too $\iff$ there is a unique isomorphism $P \cong Q$ such that the first diagram commutes for all $\alpha$. Dually, if $(C, \left\{ i_{\alpha} \right\})$ is a coproduct, then $(D, \left\{ j_{\alpha} \right\})$ is too $\iff$ there is a unique isomorphism $C \cong D$ such that the second diagram commutes for all $\alpha$.
	\[
	\begin{tikzcd}
		P \dar["\pi_{\alpha}"'] & Q \lar["\sim"']\arrow[dl,"\phi_{\alpha}"] & & C \rar{\sim} & D \\
		M_{\alpha} & & & M_{\alpha} \uar{i_{\alpha}}\arrow[ur, "j_{\alpha}"']
	\end{tikzcd}
	\] 
\end{thrm}
\begin{proof}
	We need only prove the case for products, since the coproduct case is dual. The forward direction is a straightforward consequence of setting $N=P$ and $f_{\alpha}=\pi_{\alpha}$, then $N=Q$ and $f_{\alpha}=\phi_{\alpha}$, and analyzing the unique maps that the universal property gives us in these cases. For the backward direction, the unique lift is given by $h^{-1} f$, where $h$ is the isomorphism $P \cong Q$ and $f$ is the unique lift gotten from $P$ being a product.
\end{proof}

\begin{thrm}[Existence]
	$\left(\prod_{\alpha \in \mathcal{A}} M_{\alpha}, \left\{ \pi_{\alpha} \right\}\right)$ is a product of $\left\{ M_{\alpha} \right\}$.
\end{thrm}
\begin{proof}
	Given $N$ and morphisms $f_{\alpha}:N\to M_{\alpha}$, we define $f$ in the obvious way by
	\begin{align*}
		%f: N &\to \prod_{\alpha}M_{\alpha} \\
		x &\mapsto (f_{\alpha}(x))_{\alpha}.
	\end{align*}
	It's an $R$-morphism, it satisfies the universal property, and it clearly must be unique.
\end{proof}

\begin{note}[]
	Thus up to (unique) isomorphism, every family of $R$-modules has a unique product and coproduct. We can then call the direct product (direct sum) \textit{the} product (coproduct).
\end{note}


A consequence of the uniqueness of the product and coproduct is that both $\prod$ and $\bigoplus$ are commutative and associative (no matter what order we do things in, we end up with a product/coproduct, which must be isomorphic to the product/coproduct we got with the original ordering).

\warn{Do proof of associativity for practice.}

\warn{Finish this section.}

%+-------------------+
%| +---------------+ |
%| |    Chapter    | |
%| +---------------+ |
%+-------------------+
% Special Modules

\chapter{Special Modules}

%--------------------------------------------------------------------------------
% Chain Conditions and Towers
%--------------------------------------------------------------------------------
\section{Chain Conditions and Towers}

Any modules can be broken down into some ascending or descending sequences of submodules. If we restrict our attention to only modules with finite such sequences, then we characterize them further.

\begin{defn}[]
An $R$-module $M$ is \textbf{Noetherian} if for all ascending submodule chains
\[
M_0 \subseteq M_1 \subseteq M_2 \subseteq \cdots,
\] there is some $n \in \mathbb{N}$ such that $M_{n+k} = M_{n}$ for all $k \in \mathbb{N}$, i.e. the chain stabilizes at $n$. We say that $M$ is \textbf{Artinian} if for all descending chains
\[
M_0 \supseteq M_1 \supseteq M_2 \supseteq \cdots,
\] there is again some $n$ at which the chain stabilizes. We call these two qualities \textbf{chain conditions}.
\end{defn}

We can also define similar concepts for unordered sets of submodules.
\begin{defn}[]
	An $R$-module $M$ has the \textbf{maximal (minimal) condition} if every nonempty collection of submodules of $M$ has some maximal (minimal) submodule w.r.t. set inclusion.
\end{defn}

Note that we're using maximal/minimal, \textit{not} maximum/minimum. This is important.

\begin{thrm}[]
	TFAE:
	\begin{enumerate}
		\item $M$ is Noetherian.
		\item $M$ satisfies the maximal condition.
		\item Every submodule of $M$ is finitely generated.
	\end{enumerate}
\end{thrm}

\begin{thrm}[]
TFAE:
\begin{enumerate}
	\item $M$ is Artinian.
	\item $M$ satisfies the minimal condition.
\end{enumerate}
\end{thrm}
\warn{Is there any similar thing about being finitely generated, or is that just a property of Noetherian modules?}

A nice property of chain conditions is that they are passed onto submodules and quotient modules. The converse also holds.
\begin{prop}
If $M$ has some chain condition, then each of its submodules and quotient modules has it too. Conversely, if every submodule $N$ of $M$ and every quotient module $M/N$ has the same chain condition, then so does $M$.
\end{prop}

%-------------------
% Simple Modules
%-------------------
\subsection{Simple Modules}

A very extreme case of the above conditions is when a module's only proper submodule is the trivial submodule. These modules are called \textbf{simple}. As you might expect (since $R$-morphisms induce maps between submodules), modules going to or coming from a simple module are pretty restricted.

\begin{prop}
If $f:M\to N$ is a nonzero $R$-morphism, then:
\begin{enumerate}
	\item If $M$ is simple, then $f$ is monic.
	\item If $N$ is simple, then $f$ is epic.
\end{enumerate}
\end{prop}
\begin{proof}
	Since $f \neq 0$, this follows from $\ker f$ and $\im f$ being submodules of $M$.
\end{proof}

\begin{cor}[Schur]
	If $M$ is simple, then $\text{End}_{R}(M)$ is a division ring.
\end{cor}
\begin{proof}
	Every nonzero endomorphism is necessarily iso. Since the natural multiplication on $\text{End}_{R}(M)$ is composition, this means every nonzero element has a multiplicative inverse.
\end{proof}

%-------------------
% Submodule Towers
%-------------------
\subsection{Submodule Towers}

\warn{Stuff here.}












Extra nice modules will be both Noetherian and Artinian, and its these modules that have a special ``height" characterization based on their submodule towers.


%--------------------------------------------------------------------------------
% Free Modules
%--------------------------------------------------------------------------------
\section{Free Modules}

Given a nonempty set $S$ and a unital ring $R$, we can fill in all the missing linear combinations of $S$ to get a module $\ang{S}$. This module is ``free" of any unnecessary relations between its elements: it contains every possible linear combination of terms, with nothing simplified via some other relation.

\begin{defn}[]
A \textbf{free $R$-module} on a set $S$ is an $R$-module $F$ with a map $i:S\to F$ such that for all $R$-modules $M$, every map $f:S\to M$ extends uniquely through $i$ to a morphism $F\to M$.
\[
\begin{tikzcd}
	F \arrow[dr, dashed, "\exists! \; h"] \\
	S \uar{i}\rar["f"'] & M
\end{tikzcd}
\] 
We denote this by $(F, i)$ and say that $F$ is free on $S$.
\end{defn}

\begin{prop}
	If $(F,i)$ is a free module, then $f$ is injective and $\ang{\im i}= F$.
\end{prop}

\begin{thrm}[Uniqueness]
	\label{thrm:free-unique}
	Suppose $(F,i)$ is free on $S$. Then so is $(G,j) \iff$ there is a unique isomorphism $F \cong G$ making the following diagram commute.
	\[
	\begin{tikzcd}
		F \rar[dashed]{\exists!\;\sim} & G \\
		S \uar{i}\arrow[ur,"j"']
	\end{tikzcd}
	\] 
\end{thrm}
\begin{proof}
	To go forwards, plug $F$ into $G$'s universal property, then plug $G$ into $F$'s. The resulting two unique morphisms are isomorphisms that make the diagram commute. To go backwards, lift $i$'s unique extesion by the unique isomorphism $G\to F$ (the inverse of the one in the diagram) to get $j$'s unique extension.
\end{proof}

\begin{thrm}[Existence]
	For every nonempty set $S$ and unital ring $R$, there is a free $R$-module on $S$.
\end{thrm}
\begin{proof}
	Let $F = \bigoplus_{s \in S}Rs$ denote the set of all formal linear combinations of $S$, which has elements of the form $\sum_s r_s s$, where only finitely many of the $r_s$ are nonzero. There's a natural inclusion $i: S \inj F$. Given $M$ and $g$, define $h$ on $i(S)$ by $h(s) = g(s)$, then extend by linearity to all of $F$. It's necessarily a unique $R$-morphism that satisfies the universal property.
\end{proof}

\begin{note}[]
	Thus up to (unique) isomorphism, every nonempty set $S$ has a unique free $R$-module. We can then call $\bigoplus_{s \in S}Rs$ \textit{the} free $R$-module on $S$.
\end{note}

Note that the map $s \mapsto \mathbf{e}_{s}$ shows $\bigoplus_{s}Rs \cong \bigoplus_{s}R$, so we can also describe the free $R$-module on $S$ as a direct sum of copies of $R$, indexed by $S$.

So that we don't have to deal with the map $i$ when describing free modules, we say that a module $M$ is free if there is some free module $(F,i)$ and an isomorphism $M \cong F$. Then the ``inclusion" map for $M$ is $i$ extended by the isomorphism.

%-------------------
% Bases
%-------------------
\subsection{Bases}

\begin{defn}[]
A \textbf{basis} of an $R$-module $M$ is a linearly independent subset of $M$ that generates $M$.
\end{defn}

\begin{thrm}[]
A nonempty subset $S \subseteq M$ is a basis of $M \iff$ each element of $M$ can be uniquely expressed as a linear combination of elements of $S$.
\end{thrm}

\begin{prop}
	If $(F,i)$ is a free module, then $\im i$ is a basis of $F$.
\end{prop}
\begin{proof}
	Suppose $(F,i)$ is free over some nonempty $S$, then we know $F \cong \bigoplus_{s}Rs$, and it's clear that $S$ is a basis of $\bigoplus_{s}Rs$. We can then translate this basis for $\bigoplus_{s}Rs$ into a basis for $F$ since the isomorphism necessarily commutes with both modules' inclusion maps by Theorem \ref{thrm:free-unique}.
\end{proof}

\begin{thrm}[]
A module is free $\iff$ it has a basis.
\end{thrm}
\begin{proof}
	If $F$ is free, then isomorphic to $\bigoplus_{s}Rs$, so its basis is the basis $S$ of $\bigoplus_{s}Rs$ mapped through the isomorphism. Conversely, if $S$ is a basis of $F$, then there is a natural inclusion $i:S \inj F$. Fix another module $M$ and a map $f:S\to M$, then the only way to get an $R$-morphism $h:F\to M$ is to define $h(s) \doteq g(s)$ and then extend by linearity, which is unique. Thus $F$ is free.
\end{proof}

\warn{Fill in other notes here.}

\begin{thrm}[]
The direct sum of free modules is free. Precisely, if $F_{\alpha}$ is free on $S_{\alpha}$ for all $\alpha$, then $\bigoplus_\alpha F_{\alpha}$ is free on $\bigsqcup_{\alpha}S_{\alpha}$.
\end{thrm}
\begin{proof}
	This is essentially just using the universal properties of free modules and coproducts. Fix $\alpha$, and let $M$ and $g:\bigsqcup_\alpha S_{\alpha}\to M$ be arbitrary.
	\[
		\begin{tikzcd}
F_\alpha \arrow[r, "j_\alpha", hook]                                                                  & \bigoplus_\alpha F_\alpha \arrow[rdd, "\exists!\;\phi", dashed, bend left]                                                        &   & F_\alpha \arrow[rrdd, "\exists!\;f", dashed] &  &   \\
S_\alpha \arrow[rrd, "f_\alpha"', bend right] \arrow[u, "i_\alpha", hook] \arrow[r, hook, shift left] & \bigsqcup_\alpha S_\alpha \arrow[u, "i", hook] \arrow[rd, "g" description] \arrow[l, "\pi_\alpha", two heads, shift left] &   &                                              &  &   \\
                                                                                                      &                                                                                                                                   & M &                                              &  & M
\end{tikzcd}
	\] The diagram's got a lot going on, but it's straightforward. All four inclusions and the one projection are the natural ones. $f$ comes from the universal property of free modules, so $fi_{\alpha}=f_{\alpha}$. $\phi$ comes from the universal property of the coproduct, so $\phi j_{\alpha}=f$. The map $g$ induces $f_{\alpha}$ by $f_{\alpha}\pi_{\alpha}=g|_{S_{\alpha}}$.

	To show that $\bigoplus_{\alpha}F_{\alpha}$ is free, we have to show that $\phi$ extends $g$ through $i$. But for any $s \in S_{\alpha}$,
	\[
		(\phi i)(s) = (\phi j_{\alpha}i_{\alpha} \pi_{\alpha})(s) = (f i_{\alpha}\pi_{\alpha})(s) = (f_{\alpha}\pi_{\alpha})(s) = g(s),
	\] so $\phi i=g$. Thus $\bigoplus_{\alpha}F_{\alpha}$ is free on $\bigsqcup_\alpha S_{\alpha}$.
\end{proof}

\warn{Finish this section}

%--------------------------------------------------------------------------------
% Hom Sets and Split Sequences
%--------------------------------------------------------------------------------
\section{Hom Sets and Split Sequences}

Given $R$-modules $M,N$, the set $\hom(M,N)$ is an abelian group under function addition, but the left action $(\lambda, f) \mapsto \lambda f$ doesn't necessarily make $\hom(M,N)$ into an $R$-module ($\lambda f$ might not be a morphism). This is only true if $R$ is commutative.



\end{document}
