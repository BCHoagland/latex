\documentclass[twoside,10pt]{article}
\usepackage{toggles}
\toggletrue{sectionbreaks}
\newcommand{\docTitle}{Measure Theory}
\usepackage{common}
\importStyles{formal}{rainbow}{lined}

%\renewcommand{\theenumi}{\alph{enumi}}

\begin{document}
\tableofcontents

\warn{Go over extended reals}

%--------------------------------------------------------------------------------
% Basic Definitions
%--------------------------------------------------------------------------------
\section{Basic Definitions}

Intuitively, $\sigma$-algebras contain the subsets of a space which we care about measuring.

\begin{defn}[]
Let $P(X)$ denote the power set of $X$. Then $\mathcal{A} \in P(X)$ is a $\sigma$\textbf{-algebra} on $X$ if
\begin{enumerate}
	\item $\varnothing, X \in \mathcal{A}$;
	\item if $A \in \mathcal{A}$, then $A^{c} \in \mathcal{A}$;
	\item if $\left\{ A_{i} \right\} \subset \mathcal{A}$ is countable then $\bigcup_{i}A_{i} \in \mathcal{A}$.
\end{enumerate}
Each $A \in \mathcal{A}$ is a \textbf{measurable set}.
\end{defn}

\begin{prop}
If $\left\{ \mathcal{A}_i \right\}$ is an arbitrary collection of $\sigma$-algebras on $X$, then so is $\bigcap_{i}\mathcal{A}_{i}$.
\end{prop}

This lets us define a $\sigma$-algebra generated by a set of subsets.

\begin{defn}[]
Let $\mathcal{M} \subset P(X)$ be a family of subsets of $X$. Then $\sigma(\mathcal{M})$ is the $\sigma$-algebra \textbf{generated by} $\mathcal{M}$, defined by the intersection of all $\sigma$-algebras on $X$ containing each element of $\mathcal{M}$.
\end{defn}

\begin{ex}[]
Suppose $(X,\mathcal{T}) \in \cat{Top}$, then $B(X) := \sigma(\mathcal{T})$ is the \textbf{Borel $\sigma$-algebra}.
\end{ex}

Measures are maps that measure an element of a $\sigma$-algebra (a measurable subset of a space). We want such maps to have intuitive properties of volume. The main one is that we can calculate the volume of something by breaking it up into (perhaps countably infinite) subvolumes and measuring those volumes instead.

\begin{defn}[]
$(X,\mathcal{A})$ is a \textbf{measurable space}. A map $\mu: \mathcal{A} \to [0,\infty]$ is a \textbf{measure} if
\begin{enumerate}
	\item $\mu(\varnothing)=0$;
	\item $\sum_{i=0}^{\infty} \mu(A_{i}) = \mu\left( \uni_{i=1}^{\infty}A_{i} \right)$ for any countable collection of disjoint $A_{i}$ ($\mu$ is ``$\sigma$-additive").
\end{enumerate}
$(X,\mathcal{A},\mu)$ is a \textbf{measure space}.
\end{defn}

\begin{ex}[]
The \textbf{counting measure} is given by
\[
\mu(A) :=
\begin{cases}
	\text{number of elements in $A$} & \text{if } A \text{ is finite,} \\
	\infty & \text{else}.
\end{cases}
\] 
\end{ex}

\newpage
\begin{ex}[]
The \textbf{Dirac measure} for $p \in X$ is
\[
\delta_{p}(A) :=
\begin{cases}
	1 & p \in A, \\
	0 & \text{else}.
\end{cases}
\] 
\end{ex}

We also want to define the ``normal" measure on $\R^{n}$. It should definitely have the following two properties:
\begin{enumerate}
	\item $\mu([0,1]^{n}) = 1$ (the unit hypercube has measure 1);
	\item $\mu(x+A) = \mu(A)$ for all $x \in \R^{n}$ (measure is translation-invariant).
\end{enumerate}
Note by extension, this first property becomes $\mu([a,b]^{n}) = (b-a)^{n}$ for all $a < b$. As it turns out, it's impossible to construct such a measure on all of $P(\R^{n})$. Instead, we'll have to construct it on a particular $\sigma$-algebra inside of $P( \R^{n})$. We'll call this the \textbf{Lebesgue measure}.

\begin{prop}
\warn{Prove that can't construct the Lebesgue measure on all of $P(\R^{n})$}
\end{prop}

%--------------------------------------------------------------------------------
% The Lebesgue Integral
%--------------------------------------------------------------------------------
\section{The Lebesgue Integral}

\begin{defn}[]
For measurable spaces $(\Omega_i,\mathcal{A}_{i})$, a map $f:\Omega_1\to \Omega_2$ is \textbf{measurable} if
\[
f^{-1}(A_2) \in \mathcal{A}_{1} \text{ for all } A_2 \in \mathcal{A}_{2}.
\]
\end{defn}
\warn{Describe intuition here w/ integral figure}

\begin{ex}[]
The indicator function $\mathbb{I}_{A}$ is measurable if $A$ is a measurable set.
\end{ex}

\begin{prop}
Given measurable maps
\[\begin{tikzcd}
	{\Omega_1} & {\Omega_2} & {\Omega_3}
	\arrow["f", from=1-1, to=1-2]
	\arrow["g", from=1-2, to=1-3]
\end{tikzcd}\]
the composition $g \circ f$ is also measurable.
\end{prop}

\begin{ex}[]
Given measurable $f,g: \Omega \to \R$, the maps $f \pm g, f\cdot g$, and $|f|$ are also measurable.
\end{ex}

To define the Lebesgue integral, we'll start by defining it for simple functions (step functions with finite range), then extend it. We'll use it to integrate measurable functions $f:\Omega \to \R$ (where $\R$ is implicitly equipped with the Borel $\sigma$-algebra).

Take a simple function $h \in \mathbb{S}$ mapping $\Omega \to \R$ by
\[
h = \sum_{i=1}^{N} c_{i}\mathbb{I}_{A_{i}}
\] for disjoint measurable $A_{i}$. Note that $h$ is measurable since $x \mapsto c_{i}$ and $\mathbb{I}_{A_{i}}$ are measurable for all $i$. We can define the integral of $h$ to be
\[
I(h) := \sum_{i=1}^{N} c_{i} \; \mu(A_{i}),
\] which agrees with the intuition that the integral captures the area under the graph of a function (although this is now clearly generalizable to arbitrary dimensions, as long as $\Omega$ has a measure $\mu$).

There's a problem here: suppose $h = 2 \mathbb{I}_{A_{1}} - 3 \mathbb{I}_{A_2}$ and $\mu(A_1) = \mu(A_2) = \infty$, then the integral evaluates to $I(h) = 2\cdot\infty - 3\cdot\infty$, which is undefined behavior so far. We have two main options to deal with this:
\begin{enumerate}
	\item Only define the integral for simple maps with $|c_{i}| < \infty$ and $\mu(A_{i}) < \infty$ for all $i$.
	\item Enforce $c_{i} \geq 0$ for all $i$ (i.e. $h \geq 0$), then we can keep using $\infty$ without this problem.
\end{enumerate}
We'll do the second option.

\warn{The representation of a simple function doesn't matter.}

\begin{defn}[]
Denote the space of non-negative simple maps $\Omega \to \R$ by $\mathbb{S}^{+}$. For $h \in \mathbb{S}^{+}$, choose a representation
\[
h = \sum_{i=1}^{N} c_{i} \mathbb{I}_{A_{i}}
\] with $c_{i} \geq 0$ for all $i$. Then the \textbf{Lebesgue integral} of $h$ wrt $\mu$ is
\[
	\int_{\Omega} f\;d\mu := I(h) = \sum_{i=1}^{N} c_{i}\; \mu(A_{i}).
\] 
\end{defn}

\begin{defn}[]
Let $f: \Omega \to [0,\infty]$ be measurable. Then the Lebesgue integral of $f$ is
\[
\int_{\Omega} f\;d\mu = I(f) := \sup \left\{ I(h) \;|\; h \in \mathbb{S}^{+}, h\leq f \right\}.
\] The map $f$ is \textbf{$\mu$-integrable} if $\int_{\Omega} f\;d\mu < \infty$.
\end{defn}

\warn{Intuition here w/ sketch}

A nice property of the Lesbesgue integral is that sets of measure 0 don't change its value.

\begin{prop}
For $B \subset \Omega$ with $\mu(B) = 0$,
\[
\int_{\Omega} f \;d\mu = \int_{\Omega-B} f \;d\mu
\] for all measurable $f$.
\end{prop}
\begin{proof}
	Let $h \in \mathbb{S}^{+}$, then $h = \sum_{i=1}^{N} c_{i} \mathbb{I}_{A_{i}}$ for disjoint $A_{i}$. We can split this up as \[
	h = \sum_{i=1}^{N} c_{i}\mathbb{I}_{(A_{i} \isct B^{c})} + \sum_{i=1}^{N} c_{i} \mathbb{I}_{(A_{i} \isct B)}.
	\] Since $0 \leq \mu(A_{i}\isct B) \leq \mu(B) = 0$ for all $i$, the integral of $h$ is
	\begin{align*}
		I(h) &= \sum_{i=1}^{N} c_{i} \mu(A_{i}\isct B^{c}) + \sum_{i=1}^{N} c_{i} \mu(A_{i} \isct B) \\
		     &= \sum_{i=1}^{N} c_{i} \mu(A_{i}\isct B^{c}) + \sum_{i=1}^{N} 0 \\
		     &= I(h|_{\Omega-B}).
	\end{align*}
\end{proof}

This result is nice because it lets us care about properties that hold $\mu$-a.e. instead of everywhere. In general, a property holds $\mu$-a.e. if the set of points for which it doesn't hold has measure 0.

\begin{prop}
Properties of the Lebesgue integral for general measurable maps:
\begin{enumerate}
	\item Linearity: $I(\alpha f + \beta g) = \alpha I(f) + \beta I(g)$.
	\item Monotonicity: if $f \leq g \; \mu$-a.e., then $I(f) \leq I(g)$.
	\item If $f=g\;\mu$-a.e., then $I(f) = I(g)$.
	\item $f=0 \;\mu$-a.e. $\iff I(f) = 0$.
\end{enumerate}
\end{prop}

\begin{thrm}[Monotone Convergence Theorem]
For measurable $\left\{ f_{n} \right\}, f$ such that
\begin{enumerate}
	\item $f_{n} \leq f_{n+1}$ for all $n$;
	\item $\lim_{n \to \infty} f_{n} = f \; \mu$-a.e.;
\end{enumerate}
then
\[
\lim_{n \to \infty} \int_{\Omega} f_{n}\;d\mu = \int_{\Omega} f\;d\mu.
\] 
\end{thrm}
\begin{proof}
	Since $f_{n} \leq f_{n+1} \;\mu$-a.e. for all $n$, and since $f_{n} \to f\;\mu$-a.e., we get that $f_{n} \leq f \;\mu$-a.e. Then by monotonicity,
	\[
	\int_{\Omega} f_{n}\;d\mu \leq \int_{\Omega} f\;d\mu \text{ for all } n \implies \lim_{n \to \infty} \int_{\Omega} f_{n}\;d\mu \leq \int_{\Omega} f\;d\mu.
	\] Proving the opposite inequality is a bit more involved. \warn{Write this part down.}
\end{proof}

\end{document}
