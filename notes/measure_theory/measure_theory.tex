\documentclass[twoside,10pt]{article}
\usepackage{toggles}
\toggletrue{sectionbreaks}
\newcommand{\docTitle}{Measure Theory}
\usepackage{common}
\importStyles{formal}{rainbow}{lined}

%\renewcommand{\theenumi}{\alph{enumi}}

\begin{document}
\tableofcontents

\warn{Go over extended reals}

%--------------------------------------------------------------------------------
% Basic Definitions
%--------------------------------------------------------------------------------
\section{Basic Definitions}

Intuitively, $\sigma$-algebras contain the subsets of a space which we care about measuring.

\begin{defn}[]
Let $P(X)$ denote the power set of $X$. Then $\mathcal{A} \in P(X)$ is a $\sigma$\textbf{-algebra} on $X$ if
\begin{enumerate}
	\item $\varnothing, X \in \mathcal{A}$;
	\item if $A \in \mathcal{A}$, then $A^{c} \in \mathcal{A}$;
	\item if $\left\{ A_{i} \right\} \subset \mathcal{A}$ is countable then $\bigcup_{i}A_{i} \in \mathcal{A}$.
\end{enumerate}
Each $A \in \mathcal{A}$ is a \textbf{measurable set}.
\end{defn}

\begin{prop}
If $\left\{ \mathcal{A}_i \right\}$ is an arbitrary collection of $\sigma$-algebras on $X$, then so is $\bigcap_{i}\mathcal{A}_{i}$.
\end{prop}

This lets us define a $\sigma$-algebra generated by a set of subsets.

\begin{defn}[]
Let $\mathcal{M} \subset P(X)$ be a family of subsets of $X$. Then $\sigma(\mathcal{M})$ is the $\sigma$-algebra \textbf{generated by} $\mathcal{M}$, defined by the intersection of all $\sigma$-algebras on $X$ containing each element of $\mathcal{M}$.
\end{defn}

\begin{ex}[]
Suppose $(X,\mathcal{T}) \in \cat{Top}$, then $B(X) := \sigma(\mathcal{T})$ is the \textbf{Borel $\sigma$-algebra}.
\end{ex}

Measures are maps that measure an element of a $\sigma$-algebra (a measurable subset of a space). We want such maps to have intuitive properties of volume. The main one is that we can calculate the volume of something by breaking it up into (perhaps countably infinite) subvolumes and measuring those volumes instead.

\begin{defn}[]
$(X,\mathcal{A})$ is a \textbf{measurable space}. A map $\mu: \mathcal{A} \to [0,\infty]$ is a \textbf{measure} if
\begin{enumerate}
	\item $\mu(\varnothing)=0$;
	\item $\sum_{i=0}^{\infty} \mu(A_{i}) = \mu\left( \uni_{i=1}^{\infty}A_{i} \right)$ for any countable collection of disjoint $A_{i}$ ($\mu$ is ``$\sigma$-additive").
\end{enumerate}
$(X,\mathcal{A},\mu)$ is a \textbf{measure space}.
\end{defn}

\begin{ex}[]
The \textbf{counting measure} is given by
\[
\mu(A) :=
\begin{cases}
	\text{number of elements in $A$} & \text{if } A \text{ is finite,} \\
	\infty & \text{else}.
\end{cases}
\] 
\end{ex}

\newpage
\begin{ex}[]
The \textbf{Dirac measure} for $p \in X$ is
\[
\delta_{p}(A) :=
\begin{cases}
	1 & p \in A, \\
	0 & \text{else}.
\end{cases}
\] 
\end{ex}

We also want to define the ``normal" measure on $\R^{n}$. It should definitely have the following two properties:
\begin{enumerate}
	\item $\mu([0,1]^{n}) = 1$ (the unit hypercube has measure 1);
	\item $\mu(x+A) = \mu(A)$ for all $x \in \R^{n}$ (measure is translation-invariant).
\end{enumerate}
Note by extension, this first property becomes $\mu([a,b]^{n}) = (b-a)^{n}$ for all $a < b$. As it turns out, it's impossible to construct such a measure on all of $P(\R^{n})$. Instead, we'll have to construct it on a particular $\sigma$-algebra inside of $P( \R^{n})$. We'll call this the \textbf{Lebesgue measure}.

\begin{prop}
\warn{Prove that can't construct the Lebesgue measure on all of $P(\R^{n})$}
\end{prop}

%--------------------------------------------------------------------------------
% The Lebesgue Integral
%--------------------------------------------------------------------------------
\section{The Lebesgue Integral}

\begin{defn}[]
For measurable spaces $(\Omega_i,\mathcal{A}_{i})$, a map $f:\Omega_1\to \Omega_2$ is \textbf{measurable} if
\[
f^{-1}(A_2) \in \mathcal{A}_{1} \text{ for all } A_2 \in \mathcal{A}_{2}.
\]
\end{defn}
\warn{Describe intuition here w/ integral figure}

\begin{ex}[]
The indicator function $\mathbb{I}_{A}$ is measurable if $A$ is a measurable set.
\end{ex}

\begin{prop}
Given measurable maps
\[\begin{tikzcd}
	{\Omega_1} & {\Omega_2} & {\Omega_3}
	\arrow["f", from=1-1, to=1-2]
	\arrow["g", from=1-2, to=1-3]
\end{tikzcd}\]
the composition $g \circ f$ is also measurable.
\end{prop}

To define the Lebesgue integral, we'll start by defining it for simple functions (step functions), then extend it. We'll use it to integrate measurable functions $f:\Omega \to \R$ (where $\R$ is implicitly equipped with the Borel $\sigma$-algebra).

\end{document}
