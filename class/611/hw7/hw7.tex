\documentclass[twoside,10pt]{article}
\usepackage{/Users/bradenhoagland/latex/styles/toggles}
%\toggletrue{sectionbreaks}
%\toggletrue{sectionheaders}
\newcommand{\docTitle}{Math 611 - HW 7}
\usepackage{/Users/bradenhoagland/latex/styles/common}
\importStyles{modern}{rainbow}{boxy}

%\renewcommand{\theenumi}{\alph{enumi}}

\begin{document}
%\tableofcontents

% ------------------------------
% 1.2: 2
% ------------------------------
\begin{exer}[1.2: 2]
	Show that the $\Delta$-complex obtained from $\Delta^{3}$ by ${\color{blue}[v_0,v_1]\sim [v_1,v_3]}$ and ${\color{red}[v_0,v_2]\sim [v_2,v_3]}$ deformation retracts onto a Klein bottle. Also, find other pairs of identifications of edges that product $\Delta$-complexes retracting onto a torus, a 2-sphere, and $\R P^2$.
\end{exer}

\begin{center}
	\textit{In this problem, any arrows represent orientation, and colors represent identificationss.}
\end{center}

Suppose our $\Delta^{3}$ is oriented as below. The first identification from the problem statement is shown in blue, and the second identification is in red.

\vspace{2in}

We can deform $\Delta^{3}$ into a solid hemisphere whose boundary looks like a parallelogram, as pictured below. From this, it's obvious that we can deformation retract $\Delta^{3}$ onto just that parallelogram.

\vspace{2in}

As it turns out, that parallelogram is in fact the Klein bottle. We can see this by cutting it down the middle, doing some rearranging, and then gluing it back together again.

\newpage

For the other 3 spaces, the process isn't very different. All that matters is how we identify the red edges and the blue edges. If we do so in a certain way, we don't need any cutting or pasting to realize what the final shape is. For $T^2$, we identify ${\color{blue}[v_3,v_1] \sim [v_2,v_0]}$ and ${\color{red}[v_0,v_1] \sim [v_2,v_3]}$.

\vspace{2in}

For $S^2$, we identify ${\color{blue}[v_0,v_1] \sim [v_3,v_1]}$ and ${\color{red}[v_2,v_0] \sim [v_2,v_3]}$.

\vspace{2in}

For $\R P^2$, we identify ${\color{blue}[v_0,v_2]\sim[v_3,v_1]}$ and ${\color{red}[v_0,v_1]\sim[v_3,v_2]}$.

\vspace{2in}

\newpage

% ------------------------------
% 2.1: 4
% ------------------------------
\begin{exer}[2.1: 4]
What's the simplicial homology of $\Delta^{2}$ with all its vertices identified to a point?
\end{exer}

This space (call it $X$) can be drawn as below, where the three vertices are colored to indicate identification to a single point $v$.

\vspace{2in}

The nontrivial spaces of $n$-chains are then
\begin{align*}
	C_2(X) &= \ang{\sigma} \cong \mathbb{Z}, \\
	C_1(X) &= \ang{a,b,c} \cong \mathbb{Z}^{3}, \\
	C_0(X) &= \ang{v}.
\end{align*}
Now we calculate the kernels of the relevant $\p_{n}{} $.
\begin{align*}
	\p_{2}{\sigma} = a+b-c \neq 0 \quad &\implies \quad \ker \p_{2}{} =0. \\
	\p_{1}{a} =\p_{1}{b} =\p_{1}{c} = 0 \quad & \implies \quad \ker \p_{1}{} =C_1(X) \cong \mathbb{Z}^{3}.
\end{align*}
The relevant images are
\begin{align*}
	\im \p_{2}{} &= \ang{\p_{2}{\sigma} } = \ang{a+b-c} \cong \mathbb{Z},\\
	\im \p_{1}{} &= \ang{\p_{1}{a} ,\p_{1}{b} ,\p_{1}{c} } = 0.
\end{align*}
Then since our simplicial chain complex is
\[
0 \to C_2 \stackrel{\p}{\to } C_1 \stackrel{\p}{\to } C_0 \to 0,
\] the homology groups are
\begin{align*}
	H_2(X) &= \ker \p_2 = 0,\\
	H_1(X) &= \frac{\ker \p_1}{\im \p_2} \cong \frac{\mathbb{Z}^{3}}{\mathbb{Z}} \cong \mathbb{Z}^{2},\\
	H_0(X) &= \frac{C_0(X)}{\im \p_1} \cong \frac{\mathbb{Z}}{0} \cong \mathbb{Z}.
\end{align*}
All other homology groups are trivial.

\newpage

% ------------------------------
% 2.1: 5
% ------------------------------
\begin{exer}[2.1: 5]
Compute the simplicial homology of the Klein bottle using the given $\Delta$-complex structure.
\end{exer}

The $\Delta$-complex structure on the Klein bottle $K$ given by the book is pictured below.

\vspace{2in}

Its chain groups are
\begin{align*}
	C_2(K) &= \ang{U,L} \cong \mathbb{Z}^{3},\\
	C_1(K) &= \ang{a,b,c} \cong \mathbb{Z}^{3},\\
	C_0(K) &= \ang{v} \cong \mathbb{Z}.
\end{align*}
Now we calculate the kernels of the relevant $\p_{n}{} $.
\begin{align*}
        \p_{2}{U} = a+b-c \neq 0, \quad \p_{2}{L} =a-b+c\neq 0 \quad &\implies \quad \ker \p_{2}{} = 0.\\
	\p_{1}{a} =\p_{1}{b} =\p_{1}{c} = 0 \quad & \implies \quad \ker \p_{1}{} =C_1(X) = \ang{a,b,c}.
\end{align*}
The relevant images are
\begin{align*}
        \im \p_{2}{} &= \ang{\p_{2}{U},\p_{2}{L}} = \ang{a+b-c,a-b+c},\\
        \im \p_{1}{} &= \ang{\p_{1}{a} ,\p_{1}{b} ,\p_{1}{c} } = 0.
\end{align*}
Then since our simplicial chain complex is
\[
0 \to C_2 \stackrel{\p}{\to } C_1 \stackrel{\p}{\to } C_0 \to 0,
\] the homology groups are
\begin{align*}
	H_2(K) &= \ker \p_2 = 0,\\
	H_1(K) &= \frac{\ker \p_1}{\im \p_2} \cong \frac{\ang{a,b,c}}{\ang{a+b-c,a-b+c}} \cong \ang{a,b \;|\; 2a} \cong \mathbb{Z}_2 \oplus \mathbb{Z},\\
	H_0(K) &= \frac{C_0(K)}{\im \p_1} \cong \frac{\mathbb{Z}}{0} \cong \mathbb{Z}.
\end{align*}
All other homology groups are trivial.

\newpage

% ------------------------------
% 2.1: 7
% ------------------------------
\begin{exer}[2.1: 7]
Identify pairs of faces of $\Delta^{3}$ to get a $\Delta$-complex structure on $S^{3}$.
\end{exer}

Suppose we orient $\Delta^{3}$ as below, then identify
\[
	[v_0, v_2, v_3] \sim [v_1, v_2, v_3] \quad\text{ and }\quad [v_0, v_1, v_2] \sim [v_0, v_1, v_3].
\]

\vspace{2in}

We can perform these identifications one at a time. The first one gives us a solid cone, which we can deform into $D^{3}$. I colored one of the edges to make it easier to track how things are moving around.

\vspace{2in}

Now we perform the second identification. This identifies each point on $\p D^{3} = S^{2}$ with its image under a reflection over the equator (note that this fixes the equator, as required since $v_0$ and $v_1$ are fixed under this identification).

\newpage

But this is the same as $S^{3}$. After performing the second identification, we have taken $\p D^{3}$ and identified it to a circle, which we can then contract to a point since our space is solid. But this is $S^{3}$, as it is $D^{3}$ with $\p D^{3}$ identified to a point.

\newpage

% ------------------------------
% 2.1: 8
% ------------------------------
\begin{exer}[2.1: 8]
Homology of the lens space made from tetrahedra.
\end{exer}

\textbf{The chain groups:} We first make some notes about how the identifications affect the chain groups. None of the open 3-simplices $\tau_i$ get identified, so there are still $n$ of them. Thus $C_3(X) = \ang{\tau_i, \dots, \tau_n} \cong \mathbb{Z}^{n}$.

\vspace{1.5in}

Each tetrahedron has 2 unique faces, a vertical face $V_i$ and a horizontal face $H_i$ (it loses the other two to its neighbors through the identifications). Thus $C_2(X) = \ang{V_1,\dots,V_n,H_1,\dots,H_n}\cong \mathbb{Z}^{2n}$.

\vspace{1.5in}

Each tetrahedron has one unique edge $c_i$. There's one horizontal edge $b$ going around the whole space, and there's one vertical edge $a$ in the center of the shape. Thus $C_1(X) = \ang{a,b,c_1,\dots,c_n}\cong \mathbb{Z}^{n+2}$.

\vspace{1.5in}

The two endpoints of $a$ are identified to a single vertex $v$, and all vertices on the outside of the shape are identified to a single vertex $w$. Thus $C_0(X) = \ang{v,w}\cong \mathbb{Z}^{2}$.

\newpage

\textbf{Images:} We can calculate the images of each of these generating simplices:
\begin{align*}
	\p_3 \tau_i &= V_i - V_{i+1} - H_i + H_{i+1},\\
	\p_2 V_i &= a + c_{i+1}-c_i ,\\
	\p_2 H_i &= b+c_{i+1}-c_i,\\
	\p_1 a = \p_1 b &= 0, \\
	\p_1 c_i &= v-w.
\end{align*}
Thus the images of the various boundary maps are
\begin{align*}
	\im \p_3 &= \ang{V_i - V_{i+1} - H_i + H_{i+1}},\\
	\im \p_2 &= \ang{a + c_{i+1}-c_i, \quad b+c_{i+1}-c_i},\\
	\im \p_1 &= \ang{v-w}.
\end{align*}

\textbf{Kernels:} The kernels of the boundary maps then follw:
\begin{align*}
	\ker \p_3 &= \ang{\tau_1 + \dots + \tau_n} \cong \mathbb{Z}^{3},\\
	\ker \p_2 &= \ang{V_i - H_i - V_j + H_j \;|\; i \neq j} = \ang{V_i - V_{i+1} - H_i + H_{i+1}}=\im \p_3,\\
	\ker \p_1 &= \ang{a,b,c_i-c_j \;|\; i \neq j} = \ang{a,b,c_i -c_{i+1}}.
\end{align*}

\textbf{Homology:} Calculating the homology groups is then straightforward.
\begin{align*}
	H_3(X) &= \ker \p_3 \cong \mathbb{Z}^{3},\\
	H_2(X) &= \frac{\ker \p_2}{\im \p_3} \cong 0,\\
	H_1(X) &= \frac{\ker \p_1}{\im \p_2} = \frac{\ang{a,b,c_i -c_{i+1}}}{\ang{a + c_{i+1}-c_i, \quad b+c_{i+1}-c_i}} \cong \ang{c_i - c_{i+1}}\cong \mathbb{Z}_{n},\\
	H_0(X) &= \frac{C_0(X)}{\im \p_1} = \frac{\ang{v,w}}{\ang{v-w}} \cong \ang{v}\cong \mathbb{Z}.
\end{align*}
Note that $H_1(X) \cong \mathbb{Z}_{n}$ since $a=b=c_i-c_{i+1} \; \forall i$, and thus $n(c_i-c_{i+1}) = \sum_{i=1}^{n}(c_i-c_{i+1}) = 0$.

\end{document}
