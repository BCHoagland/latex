\documentclass[twoside,10pt]{article}
\usepackage{/Users/bradenhoagland/latex/styles/toggles}
%\toggletrue{sectionbreaks}
%\toggletrue{sectionheaders}
\newcommand{\docTitle}{HW 1}
\usepackage{/Users/bradenhoagland/latex/styles/common}
\importStyles{modern}{rainbow}{boxy}

\renewcommand{\theenumi}{\alph{enumi}}

\begin{document}
%\tableofcontents

% ------------------------------
% 0: 4
% ------------------------------
\begin{exer}[0: 4]
Show that if $X$ deformation retracts to $A$ in the weak sense, then the inclusion $i: A \inj X$ is a homotopy equivalence.
\end{exer}

Suppose $X$ deformation retracts (weakly) onto $A$ via $F$, then this determines a map $r:X\to A$ by $x \mapsto F_1(x)$ (as an element of $A$). Then $F$ is the homotopy showing $i \circ r \simeq \text{id}_{X}$. We can also consider the map $r \circ i : A\to A$, which is homotopic to $\text{id}_{A}$ via $F|_{A}$. Thus $i$ is a homotopy equivalence.

\newpage

% ------------------------------
% 0: 5
% ------------------------------
\begin{exer}[0: 5]
Show that if a space $X$ deformation retracts to a point $x \in X$, then for each neighborhood $U$ of $x$ in $X$, there is a neighborhood $V \subset U$ of $x$ such that the inclusion map $V \inj U$ is nullhomotopic.
\end{exer}

Suppose $X$ deformation retracts onto $x$ via $F$, and fix a neighborhood $U$ of $x$. Now consider the open set $F^{-1}(U)$ in $X\times I$. We'd like to apply the tube lemma to find a nice neighborhood $V$ of $x$ now, but we have to check its two conditions:
\begin{enumerate}
	\item $F^{-1}(U)$ contains $\left\{ x \right\}\times I$ since $F_t(x) =x$ for all $t$ and since $t$ ranges over all of $I$;
	\item $I$ is compact.
\end{enumerate}
Thus applying the tube lemma, we get a neighborhood $V$ of $x$ such that $V \times I \subset F^{-1}(U)$. Since $V \times I$ lies entirely in $F^{-1}(U)$, the map $F\circ i$ (where $i$ is the inclusion $V\inj U$) always maps into $U$. Thus $F\circ i:V \times I \to U$ is a well-defined homotopy showing $i \simeq x$.

\newpage

% ------------------------------
% 0: 6
% ------------------------------
\begin{exer}[0: 6]
\begin{enumerate}
	\item Let $X$ be the subspace of $\mathbb{R}^{2}$ consisting of the horizontal segment $[0,1]\times \left\{ 0 \right\}$ together with all the vertical segments $\left\{ r \right\}\times [0,1-r]$ for $r$ rational in $[0,1]$. Show that $X$ deformation retracts to any point in the segment $[0,1] \times \left\{ 0 \right\}$, but not to any other point.

	\item Let $Y$ be the subspace of $\mathbb{R}^2$ that is the union of an infinite number of copies of $X$ arranged as in the figure below. Show that $Y$ is contractible but does not deformation retract onto any point.

	\item Let $Z$ be the zigzag subspace of $Y$ homeomorphic to $\mathbb{R}$. Show that there is a deformation retraction in the weak sense of $Y$ onto $Z$, but no true deformation retraction.
\end{enumerate}
\end{exer}

\begin{enumerate}
	\item Given a point $(x,0)$, we can form an explicit deformation retraction
		\[
			F( (q,s), t) =
			\begin{cases}
				(q, s(1-2t)) & 0 \leq t \leq 1/2, \\
				( (2t-1)x + (2-2t)q, 0) & 1/2 \leq t \leq 1.
			\end{cases}
		\] 

		Now we want to show that we cannot deformation retract $X$ onto any point on one of the verticle prongs. To do so, we'll use the previous exercise (0: 5). Let $x = (q,s)$, where $q \in \mathbb{Q} \isct [0,1]$ and $s \in [0,1-q]$, and suppose $X$ deformation retracts to $x$. Let $U$ be the open ball of radius $s$ centered at $x$, so that it does not intersect the bottom line of $X$. By the previous exercise, there is some $V \subseteq U$ such that $V \inj U$ is nullhomotopic, say via $F$. In particular, for all $v \in V$, the map $F(v)$ is a path from $v$ to some constant point.

		Since $\mathbb{Q}$ is dense in $\mathbb{R}$, there is some $r \in \mathbb{Q}$ such that the prong $\left\{ r \right\}\times [0,1-r]$ intersects $V$. Then since $V \subseteq U$ doesn't intersect the bottom line of $X$, the open sets $\left\{ x < r \right\}\isct V$ and $\left\{ x > r \right\} \isct V$ disconnect $V$. This contradicts the result of the previous paragraph (where each $v\in V$ could follow a path to $x$ while staying entirely in $U$), so $X$ cannot deformation retract onto $x$.

	\item To show that $Y$ is contractible, we assume the result from part (c) that $Y$ deformation retracts in the weak sense to its zigzag subspace $Z$. Then by Exercise (0: 4), $Y \simeq Z$. This gives the sequence $Y \simeq Z \cong \mathbb{R}$. Since $\mathbb{R}$ is contractible, so is $Y$.

		Now we show that $Y$ does not deformation retract onto any point. Suppose $y$ is on any of the prongs of any of the combs making up $Y$. Then by the same argument as in part (a), $Y$ cannot deformation retract onto $y$. If $y$ is instead on the bottom line of any of the combs, we have a similar situation. Any neighborhood of $y$ intersects the prongs of another comb, so by a disconnection argument similar to that in (a), $Y$ cannot deformation retract onto $y$.

	\item Fix a point $y \in Y$, then consider the following path $F_{y}$ (with everything done at the same constant speed):
		\begin{itemize}
			\item if $y$ is on a prong, move toward the base of the current comb;
			\item if $y$ is in $Z$, move to the right along $Z$.
		\end{itemize}
		The map $F$ given by having every $y \in Y$ follow its path $F_{y}$ is then the weak deformation retraction between $Y$ and $Z$.

		The fact that $Y$ doesn't have any true deformation retraction onto $Z$ is essentially the same argument as in parts (a) and (b). If we replace $x$ in the argument in part (a) with some path connected space $Z$, we get that any neighborhood $U$ of a point in $Z$ must be path connected, which, as discussed in part (b), is not true.
\end{enumerate}

\newpage

% ------------------------------
% 0: 9
% ------------------------------
\begin{exer}[0: 9]
Show that a retract of a contractible space is contractible.
\end{exer}

Suppose $X$ is contractible to a point $x$ via $F$, and suppose $r:X\to A$ is a retraction, i.e. $r \circ i = \id_{A}$. Then consider the map
\begin{align*}
	G: A \times I &\to A \\
	(a,t) &\mapsto r(F_{t}(i(a))).
\end{align*}
At $t=0$, we get $a \mapsto r(i(a)) = a$, so $G_0=\id_{A}$. And at $t=1$, we get $a \mapsto r(x)$. As the composition of continuous maps, $G$ is also continuous. This shows that $A$ is contractible to a single point, namely $r(x)$.

\newpage

% ------------------------------
% 0: 13
% ------------------------------
\begin{exer}[0: 13]
	Show that any two deformation retractions $r_{t}^{0}$ and $r_{t}^{1}$ of $X$ onto a subspace $A$ can be joined by a continuous family of deformation retractions $r_{t}^{s}$.
\end{exer}

All the indices were messing with me, so I let $F \doteq r^{0}$ and $G \doteq r^{1}$. Now define a homotopy $H:X\times I\times I \to X$ by
\[
	H(\;\cdot\;,s,t) \doteq F_{\varphi_2(s,t)} \circ G_{(\varphi_1(s,t))}
\] 
where $\varphi:I\times I\to I\times I$ is given by
\[
	\varphi(s,t) =
	\begin{cases}
		(2st, t) & 0\leq s \leq 1/2, \\
		(t,2(1-s)t) & 1/2 \leq s \leq 1.
	\end{cases}
\] 
$H$ is continuous since it's the composition of continuous functions. It transitions from $F$ to $G$ as $s$ goes from 0 to 1:
\begin{itemize}
	\item $H(\;\cdot\;,0,t) = F_{t}\circ G_0 = F_{t}\circ \id_{X}=F_{t}$;
	\item $H(\;\cdot\;,1,t) = F_{0}\circ G_{t} = \id_{X}\circ \;G_{t}=G_{t}$.
\end{itemize}
Each $H_{s}$ is itself a deformation retraction:
\begin{itemize}
	\item When $t=0$, $H_{s}$ is the identity on $X$ since $H(\;\cdot\;,s,0) = F_{0}\circ G_{0}= \id_{X}\circ \id_{X}=\id_{X}$;
	\item Let $t=1$, then \[
			H(\;\cdot\;,s,1) =
			\begin{cases}
				F_{1} \circ G_{2s} \\
				F_{2(1-s)}\circ G_{1}.
			\end{cases}
		\] In both cases, $H_{s}$ maps $X$ into $A$ since $F_{1}, G_{1}$ map $X$ into $A$ and all $F_{t}$ fix $A$.
	\item Each $H_{s}$ fixes $A$ since every $F_{t}$ and $G_{t}$ fixes $A$.
\end{itemize}

\newpage

% ------------------------------
% 1.1: 6
% ------------------------------
\begin{exer}[1.1: 6]
	We can regard $\pi_1(X,x_0)$ as the basepoint-preserving homotopy classes of maps $(S^{1},s_0)\to (X,x_0)$. Let $[S^{1},X]$ be the set of homotopy classes of maps $S^{1}\to X$, with no conditions on basepoints. Thus there is a natural map $\phi:\pi_1(X,x_0)\to [S^{1},X]$ obtained by ignoring basepoints. Show that $\phi$ is onto if $X$ is path-connected, and that $\phi([f]) = \phi([g]) \iff [f]$ and $[g]$ are conjugate in $\pi_1(X,x_0)$. Hence $\phi$ induces a one-to-one correspondence between $[S^{1},X]$ and the set of conjugacy classes in $\pi_1(X)$, when $X$ is path-connected.
\end{exer}

\begin{enumerate}
	\item First we show that $\phi$ is onto if $X$ is path-connected. Fix $[\tilde{g}] \in [S^{1},X]$. Considering $S^{1}$ as $[0,1] / 0 \sim 1$, there is clearly some $[g] \in \pi_1(X,x_1)$ for some $x_1$ such that $\phi([g]) = [\tilde{g}]$. Since $X$ is path-connected, there is an isomorphism $\pi_1(X,x_0)\to \pi(X,x_1)$ given by the change of basepoint map. Thus there is some $[f] \in \pi_1(X,x_0)$ such that
		\[
			[f] \mapsto [\overline{h}fh] = [g],
		\] where $h$ is a path from $x_0$ to $x_1$. Now on $S^{1}$, we can form a homotopy from $\overline{h}f h$ to $f$ by
		\begin{align*}
			F(x,t) = (\overline{h}fh)\left(\frac{1}{3} tx\right)
		\end{align*}
		(note that this does not preserve basepoints). Thus in $[S^{1},X]$, we have $[g]=[\overline{h}fh] = [fh\overline{h}] = [f]$. This implies $\phi([f]) = \phi([g]) = [\tilde{g}]$, so $\phi$ is onto.

	\item Now we show that $\phi([f])=\phi([g]) \iff [f]$ and $[g]$ are conjugate in $\pi_1(X,x_0)$.

			\textbf{Backward:} Suppose $[h]^{-1}[f][h] = [\overline{h}fh]=[g]$ for soe $[h] \in \pi_1(X,x_0)$. In part (a) we showed that $[\overline{h}fh]=[f]$ in $[S^1,X]$, so $\phi([f])=\phi([g])$.

			\textbf{Forward:} Suppose $\phi([f])=\phi([g])$ for $[f],[g] \in \pi_1(X,x_0)$, then $f \simeq g$ by some homotopy $F$ that doesn't necessarily preserve the basepoint. Then by Lemma 1.19 in Hatcher, the following diagram commutes,
			\[
			\begin{tikzcd}
				\pi_1(S^{1},s_0) \rar{f_{*}} \arrow[rd,"g_{*}"'] & \pi_1(X,x_0) \dar{\beta_{h}} \\
										 & \pi_1(X,x_0))
			\end{tikzcd}
		\] where $\beta_{h}$ is the change of basepoint isomorphism induced by the path $h \doteq F(s_0)$ (note that since $f$ and $g$ are both loops at $x_0$, so is $h$). The commutativity of the diagram then implies $[g] = [\overline{h} f h] = [h]^{-1}[f][h]$, as desired (note that the second equality is valid since $h$ is a loop at $x_0$, and so $[h] \in \pi_1(X,x_0)$).
\end{enumerate}

\newpage

% ------------------------------
% 1.1: 7
% ------------------------------
\begin{exer}[1.1: 7]
Define $f:S^{1}\times I \to S^{1}\times I$ by
\[
	f(\theta,s) = (\theta+2\pi s, s),
\] so $f$ restricts to the identity on the two boundary circles of $S^{1}\times I$. Show that $f$ is homotopic to the identity by a homotopy $f_{t}$ that is stationary on one of the boundary circles, but not by any homotopy $f_{t}$ that is stationary on both boundary circles. [What does $f$ do to the path $s\mapsto (\theta_0,s)$ for fixed $\theta_0 \in S^{1}$?]
\end{exer}

\textbf{Homotopy that fixes one boundary circle:} Consider the map
\begin{align*}
	F: S^{1} \times I \times I &\to S \times I \\
	(\theta,s,t) &\mapsto (\theta + 2\pi s t,s).
\end{align*}
Since $F_0: (\theta,s) \mapsto (\theta,s)$ is the identity and $F_1: (\theta,s) \mapsto (\theta+2\pi s,s)$ is $f$, this is a homotopy between the identity and $f$. Furthermore, $F_t: (\theta,0) \mapsto (\theta,0)$ for all $t$, so this homotopy fixes the boundary circle $S^{1}\times \left\{ 0 \right\}$. It doesn't fix the other boundary circle $S^{1}\times \left\{ 1 \right\}$, though, since $F_{t}:(\theta,1) \mapsto (\theta+2\pi t,1)$.

\textbf{No homotopy exists that fixes both boundary circles:} Suppose we have a homotopy $G$ between $f$ and the identity that fixes both boundary circles. Now fix $\theta_0$ and consider the path
\[
	\varphi(s) \doteq (\theta_0,s).
\] 
Since $G$ fixes the boundary circles, $G\varphi$ is a path homotopy showing $\varphi \simeq f \varphi$. Visually, at the very least, this is clearly impossible: $\varphi$ is a straight line along the length of the cylinder, while $f \varphi$ is a helix, so they can't possibly be homotopic rel their endpoints.

To formalize this, we can post-compose $G\varphi$ with the natural projection $p:S^{1}\times I \epi S^{1}$ to get a new homotopy $pG\varphi$ showing $p\varphi \simeq p f \varphi$. Since the endpoints of $\varphi$ and $f\phi$ all have the same $\theta_0$ coordinate, and since $G\varphi$ fixes the boundary circles throughout the homotopy, $pG\varphi$ is a homotopy of loops with the same basepoint.

Now $p\varphi$ is a constant loop and $pf \varphi$ is a full loop around the circle, so we have that any loop going around the circle once is nullhomotopic. But a loop going around the circle $n$ times can be decomposed into $n$ single loops around the circle. And since loop homotopies respect path multiplication, this implies that all $n$-loops around the circle are nullhomotopic, i.e. $\pi_1(S^{1}) \cong 1$. But is this a contradiction, since we know $\pi_1(S^{1}) \cong \mathbb{Z}$. Thus our original homotopy $G$ could not have existed in the first place.

\newpage

% ------------------------------
% 1.1: 16 (a,b,c,f)
% ------------------------------
\begin{exer}[1.1: 16 (a,b,c,f)]
	Show that there are no retractions in the following cases:
	\begin{enumerate}
		\item $\mathbb{R}^{3}$ onto any subspace $A \cong S^{1}$.
		\item $S^{1}\times D^{2}$ onto its boundary torus $S^{1}\times S^{1}$.
		\item $S^{1}\times D^{2}$ onto the circle in the figure.
		\item[f.] The M\"obius band onto its boundary circle.
	\end{enumerate}
\end{exer}

These problems use the proposition that given a retraction
\[
\begin{tikzcd}
	A \rar[hook,shift right,"i"'] & X \lar[two heads, shift right, "r"'],
\end{tikzcd}
\] $r_{*}$ is surjective and $i_{*}$ is injective.

\begin{enumerate}
	\item An existence of a retraction $\mathbb{R}^{3}\to A$ means there is an injection $i_{*}:\pi_1(A) \to \pi(\mathbb{R}^{3})$. But since $A \cong S^{1}$ and $\mathbb{R}^{n}$ is simply connected, this is an injection from $\pi_1(A) \cong \pi_1(S^{1}) \cong \mathbb{Z}$ into $\pi_1(\mathbb{R}^{3}) \cong 1$, which is clearly impossible.

	\item The argument here is the same. The existence of a retraction means we have an injection from $\pi_1(S^1 \times S^{1}) \cong \mathbb{Z}^{2}$ into $\pi_1(S^{1}\times D^2) \cong \mathbb{Z}$. But this is impossible: suppose the injective homomorphism is $\phi$, then the first isomorphism theorem gives $\mathbb{Z}^{2} \cong \phi(\mathbb{Z}^2) \leq \mathbb{Z}$, but all subgroups of $\mathbb{Z}$ are cyclic and $\mathbb{Z}^2$ is not cyclic.

	\item Once again $i_{*}$ is an injection. But any loop in the circle from the figure becomes nullhomotopic when treated as an element of $S^{1}\times D^2$ due to this torus being filled in, so $i_{*}$ is the zero map, which contradicts its injectivity.
	
	\item[f.] Since the M\"obius band $M$ deformation retracts to its center circle, $\pi_1(M) \cong \mathbb{Z}$. Its boundary circle $A$ is a circle, so $\pi_1(A) \cong \mathbb{Z}$ as well. Now a loop that goes around $A$ once ends up going around $M$ twice, so $i_{*}$ can be described explicitly by
		\begin{align*}
			i_{*}: \mathbb{Z} &\to \mathbb{Z} \\
			1 &\mapsto 2.
		\end{align*}
		This map is simply multiplication by 2. But since $r\circ i = \id_{A}$, we know $r_{*}\circ i_{*} = \id_{\mathbb{Z}}$. Thus we get the following commutative diagram.
		\[
		\begin{tikzcd}
			\mathbb{Z} \rar{\times 2} \arrow[rr,bend right, "\id"'] & \mathbb{Z} \rar{r_{*}} & \mathbb{Z}
		\end{tikzcd}
		\] 
		But this is impossible: the only map undoing multiplication by 2 is division by 2, but this is not a homomorphism $\mathbb{Z}\to \mathbb{Z}$.
\end{enumerate}

\newpage

% ------------------------------
% 1.1: 20
% ------------------------------
\begin{exer}[1.1: 20]
	If $F:X\to X$ is a homotopy such that $F_0=F_1=\id_{X}$, then for any $x_0$, the loop $F(x_0)$ represents an element of the center of $\pi_1(X,x_0).$
\end{exer}

By lemma 1.19, we get the following commutative diagram,
\[
\begin{tikzcd}
	\pi_1(X,x_0) \rar{\id_{}} \arrow[dr, "\id_{}"'] & \pi_1(X,x_0) \dar{\beta_{F(x_0)}} \\
	& \pi_1(X,x_0)
\end{tikzcd}
\] 
where $\beta_{F(x_0)}$ is the change of basepoint isomorphism $[\alpha] \mapsto [F(x_0) \cdot \alpha \cdot \overline{F(x_0)}]$.
This says that $\beta_{F(x_0)}$ is just the identity, i.e.
\[
[F(x_0) \cdot \alpha \cdot \overline{F(x_0)}] = [\alpha]
\] for all $[\alpha]$. Then since $F(x_0)$ is itself a loop at $x_0$, we can decompose the change of basepoint:
\begin{align*}
	[F(x_0) \cdot \alpha \cdot \overline{F(x_0)}] &= [\alpha] \\
	[F(x_0)] [\alpha] [F(x_0)]^{-1} &= [\alpha] \\
	[F(x_0)] [\alpha] &= [\alpha] [F(x_0)].
\end{align*}
Since this statement holds for all $[\alpha]$, the homotopy class $[F(x_0)]$ is in the center of $\pi_1(X,x_0)$.

\newpage

\end{document}
