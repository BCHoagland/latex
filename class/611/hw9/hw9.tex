\documentclass[twoside,10pt]{article}
\usepackage{/Users/bradenhoagland/latex/styles/toggles}
%\toggletrue{sectionbreaks}
%\toggletrue{sectionheaders}
\newcommand{\docTitle}{Math 611 - HW 9}
\usepackage{/Users/bradenhoagland/latex/styles/common}
\importStyles{modern}{rainbow}{boxy}

\renewcommand{\theenumi}{\alph{enumi}}

\begin{document}
%\tableofcontents

% ------------------------------
% 2.1: 20
% ------------------------------
\begin{exer}[2.1: 20]
	Show that $\tilde{H}_{n}(X) \approx \tilde{H}_{n+1}(SX)$ for all $n$, where $SX$ is the suspension of $X$. More generally, thinking of $SX$ as the union of two cones $CX$ with their bases identified, compute the reduced homology groups of the union of any finite number of cones $CX$ with their bases identified.
\end{exer}

\textbf{First part:} Suppose $A \subset B$ is contractible, then the long exact sequence of the pair $(B,A)$ in reduced homology gives the following exact sequence for all $n$.
\[
	0 = \tilde{H}_{n}(A) \to \tilde{H}_{n}(B) \to H_{n}(B,A) \to \tilde{H}_{n-1}(A) = 0
\] Thus $H_{n}(B,A) \cong \tilde{H}_{n}(B)$ when $A$ is contractible. We can apply this to the problem at hand; We know $SX$ is the union of two cones $CX$ and $C'X$. Then since $CX \subset SX$ is clearly contractible,
\[
	\tilde{H}_{n+1}(SX) \cong H_{n+1}(SX,CX) \cong H_{n+1}(C'X,X),
\]
where the second isomorphism follows from Corollary 2.24 in the text with $A=CX$ and $B=C'X$ (a consequence of excision). The long exact sequence of the pair $(C'X,X)$ in reduced homology then gives the following exact sequence.
\[
	0=\tilde{H}_{n+1}(C'X) \to H_{n+1}(C'X,X) \to \tilde{H}_{n}(X) \to \tilde{H}_{n}(C'X) = 0
\] Thus $H_{n+1}(C'X,X) \cong \tilde{H}_{n}(X)$. Composing all our isomorphisms gives
\[
	\tilde{H}_{n+1}(SX) \cong \tilde{H}_{n}(X),
\] as desired.

\textbf{Second part:} Let $S^{k}X$ denote $k$ cones $CX$ with bases all identified together. Since $S^{k}X$ is the union of $S^{k-1}X$ and $CX$ and since $CX$ is contractible, we have
\[
	\tilde{H}_{n+1}(S^{k}X) \cong H_{n+1}(S^{k}X,CX) \cong H_{n+1}(S^{k-1},X),
\] where the second isomorphism once again comes from Corollary 2.24, with $A=CX$ and $B=S^{k-1}X$. We claim that $(S^{k-1}X,X)$ is a good pair: $X$ is clearly closed in $S^{k-1}$, and if we remove all the points where the individual cones are identified to a point, then this open set deformation retracts onto $X$. Since it's a good pair, we have $H_{n+1}(S^{k-1}X,X) \cong \tilde{H}_{n+1}(S^{k-1}X/X)$.

But $S^{i}X/X \cong \bigvee_{j=1}^{i} SX$, as the following illustration illustrates in the case $i=2$.

\newpage

Then by Corollary 2.25 and the first part of the question, we have
\[
	\tilde{H}_{n+1}(S^{k-1}X/X) \cong \tilde{H}_{n+1}\left(\bigvee_{i=1}^{k-1} SX \right) \cong \bigoplus_{i=1}^{k-1} \tilde{H}_{n+1}(SX) \cong \bigoplus_{i=1}^{k-1} \tilde{H}_{n}(X).
\] 
Composing all our isomorphisms together, we get
\[
	\tilde{H}_{n+1}(S^{k}X) \cong \bigoplus_{i=1}^{k-1} \tilde{H}_{n}(X).
\] 

\newpage

% ------------------------------
% 2.1: 22
% ------------------------------
\begin{exer}[2.1: 22]
	Prove by induction on dimension the following facts about the homology of a finite-dimensional CW complex $X$, using the observation that $X^{n}/X^{n-1}$ is a wedge sum of $n$-spheres:
	\begin{enumerate}
		\item If $X$ has dimension $n$ then $H_{i}(X)=0$ for $i>n$ and $H_{n}(X)$ is free.
		\item $H_{n}(X)$ is free with basis in bijective correspondence with the $n$-cells if there are no cells of dimension $n-1$ or $n+1$.
		\item If $X$ has $k$ $n$-cells, then $H_{n}(X)$ is generated by at most $k$ elements.
	\end{enumerate}
\end{exer}

All three parts of the question will use the following result: suppose $\alpha$ indexes the $n$-cells of a CW complex $X$, then $X^{n}/X^{n-1} \cong \bigvee_{\alpha}S^{n}$. We claim that $(X^{n},X^{n-1})$ is a good pair: $X^{n}$ is a union of $D^{n}$ such that each $\text{int} D^{n}$ is disjoint, and $X^{n-1}$ is the border of all the $D^{n}$. If we remove a point from each $D^{n}$, then we get an open set that clearly deformation retracts onto its boundary, i.e. onto $X^{n-1}$. Thus $X^{n},X^{n-1})$ is a good pair for all $n$. Then by Proposition 2.22 and Corollary 2.25 in the text,
\[
	H_{i}(X^{n}, X^{n-1}) \cong \tilde{H}_{i}(X^{n}/X^{n-1}) \cong \tilde{H}_{i}\left(\bigvee_{\alpha} S^{n}\right) \cong \bigoplus_{\alpha}\tilde{H}_{i}(S^{n})
\] for all $i$. We know the reduced homology groups of the $n$-sphere, so this chain of isomorphisms gives us
\[
	H_{i}(X^{n},X^{n-1}) \cong 
	\begin{cases}
		\bigoplus_{\alpha}\mathbb{Z} & \text{if } i=n,\\
		0 & \text{else},
	\end{cases}
	\tag{$\star$}
\] 
where $\alpha$ indexes the $n$-cells. Additionally,
\[
	H_{i}(X) \cong H_{i}(X^{n}) \qquad \text{when } n>i. \tag{$\star\star$}
\] To see this, consider the long exact sequence of the pair $(X^{n+1},X^{n})$, which gives the following exact sequence.
\[
	H_{i+1}(X^{n+1},X^{n}) \to H_{i}(X^{n}) \to H_{i}X^{n+1} \to H_{i}(X^{n+1},X^{n})
\] When $n > i$, the first and last elements in the sequence are both $0$ by $(\star)$, so $H_{i}(X^{n}) \cong H_{i}(X^{n+1})$. Since $X$ is finite dimensional, we can induct on $n$ to get $H_{i}(X^{n}) \cong H_{i}(X)$.

\begin{enumerate}
	\item Suppose $n=0$, then $X=X^{0}$ is just a set of isolated points (0-cells). Since we can decompose the homology of a space into the direct sum of the homology of its path components, and since we know the homology of a point, this means
		\[
			H_{i}(X) \cong 
			\begin{cases}
				\bigoplus_{\alpha}\mathbb{Z} & \text{if } i=0,\\
				0 & \text{else},
			\end{cases}
		\] where $\alpha$ indexes the points (0-cells). This shows $H_{i}(X) = 0$ when $i>0$. Since the direct sum of free modules is free, this also means $H_{0}(X)$ is free. Now we can induct on $n$: suppose this holds for $n-1$, then we want to show it is true for $n$.

		From the long exact sequence of the pair $(X^{n},X^{n-1})$, the following is exact for all $i$.
		\[
			H_{i}(X^{n-1}) \to H_{i}(X^{n}) \to H_{i}(X^{n},X^{n-1})
		\] By $(\star)$ and our inductive hypothesis, this becomes the following when $i > n$.
		\[
			0 \to H_{i}(X^{n}) \to 0
		\] This implies $H_{i}(X^{n}) = 0$ when $i>n$. When $i=n$, this becomes the following instead.
		\[
			0 \to H_{n}(X^{n}) \mono \bigoplus_{\alpha}\mathbb{Z}
		\] Since the second map is injective by exactness, $H_{n}(X^{n})$ is isomorphic to a subgroup of a free group. But the subgroup of a free group is also free, so we can pass the basis of this subgroup to a basis of $H_{n}(X^{n})$ via the isomorphism. Thus $H_{n}(X^{n})$ is free.

	\item We'll need two base cases for this induction. Suppose $n=0$, then we know $H_{0}(X) \cong \bigoplus_{\beta} \mathbb{Z}$, where $\beta$ indexes the path components of $X$. Since $X$ has no 1-cells by assumption, each 0-cell must be isolated, i.e. there is a single 0-cell in each path component. Thus the basis of $H_0(X) \cong \bigoplus_{\gamma}\mathbb{Z}$ is in bijective correspondence with the 0-cells.

		Now suppose $n=1$, then $X$ has no 0-cells by assumption. But a CW complex without any 0-cells cannot have any other cells: any 1-cell must have 0-cells at its boundary, so there cannot be any 1-cells; any 2-cell must have 1- and 0-cells at its boundary, so there cannot be any 2-cells. Since $X$ is finite dimensional, we can repeat this argument up through the dimension of $X$ to show that it is empty. The claim is then trivially true.

		Now we induct on $n$. Suppose $X$ has no $n-1$ or $n+1$ cells, then the long exact sequence of the pair $(X^{n},X^{n-1})$ gives the following exact sequence.
		\[
			H_{n}(X^{n-1}) \to H_{n}(X^{n}) \to H_{n}(X^{n},X^{n-1}) \to H_{n-1}(X^{n-1})
		\] By our assumption that there are no $n-1$ or $n+1$ cells, $X^{n-1}=X^{n-2}$ and $X^{n+1}=X^{n}$, so this becomes the following.
		\[
                        H_{n}(X^{n-1}) \to H_{n}(X^{n+1}) \to H_{n}(X^{n},X^{n-1}) \to H_{n-1}(X^{n-2})
		\] Then by $(\star)$, $(\star\star)$, and (a), this reduces to the following, where $\alpha$ indexes the $n$-cells.
		\[
			0 \to H_{n}(X) \mono \bigoplus_{\alpha}\mathbb{Z} \to 0
		\] 
		Thus $H_{n}(X) \cong \bigoplus_{\alpha}\mathbb{Z}$, so it is free with basis is in bijective correspondence with the $n$-cells.

	\item The long exact sequence of the pair $(X^{n},X^{n-1})$ gives the following exact sequence.
		\[
			H_{n}(X^{n-1}) \to H_{n}(X^{n}) \to H_{n}(X^{n},X^{n-1})
		\] By part (a), the first term above is 0, and by $(\star)$ and our assumption that there are $k$ $n$-cells, the last term is $\mathbb{Z}^{k}$. Thus the exact sequence becomes the following.
		\[
			0 \to H_{n}(X^{n}) \mono \mathbb{Z}^{k}
		\] 
		By exactness, the last map is injective. Since $\mathbb{Z}^{k}$ has exactly $k$ generators and $H_{n}(X^{n})$ is identified with a subgroup of $\mathbb{Z}^{k}$, this means $H_{n}(X^{n})$ has at most $k$ generators.

		The long exact sequence of the pair $(X^{n+1},X^{n})$ gives the following exact sequence.
		\[
			H_{n}(X^{n}) \epi H_{n}(X^{n+1}) \to H_{n}(X^{n+1},X^{n}) = 0.
		\] Exactness makes the first map surjective. By $(\star\star)$, $H_{n}(X^{n+1})\cong H_{n}(X)$, so this becomes the following.
		\[
			H_{n}(X^{n}) \epi H_{n}(X) \to 0
		\] Since we just argued that $H_{n}(X^{n})$ has at most $k$ generators, this means $H_{n}(X)$ must also have at most $k$ generators.
\end{enumerate}

\newpage

% ------------------------------
% 2.1: 27
% ------------------------------
\begin{exer}[2.1: 27]
	Let $f:(X,A)\to (Y,B)$ be a map such that both $f:X\to Y$ and the restriction $f:A\to B$ are homotopy equivalences.
	\begin{enumerate}
		\item Show that $f_{*}:H_{n}(X,A)\to H_{n}(Y,B)$ is an isomorphism for all $n$.
		\item For the case of the inclusion $f:(D^{n},S^{n-1})\inj (D^{n},D^{n}-\left\{ 0 \right\})$, show that $f$ is not a homotopy equivalence of pairs -- there is no $g:(D^{n},D^{n}-\left\{ 0 \right\})\to (D^{n},S^{n-1})$ such taht $fg$ and $gf$ are homotopic to the identity through maps of pairs. [Observe that a homotopy equivalence of pairs $(X,A)\to (Y,B)$ is also a homotopy equivalence for the pairs obtained by replacing $A$ and $B$ by their closures.]
	\end{enumerate}
\end{exer}

\begin{enumerate}
	\item Consider the following diagram, where $f_{\#} : \mathcal{C}(X,A) \to \mathcal{C}(Y,B)$ is an abuse of notation: it's actually induced by the map $f_{\#}$ that sends $\mathcal{C}(X)\to \mathcal{C}(Y)$ and $\mathcal{C}(A) \to \mathcal{C}(B)$.
		\[
		\begin{tikzcd}
			0 \rar & \mathcal{C}(A) \rar{i_{\#}}\dar{f_{\#}} & \mathcal{C}(X) \rar{\pi_{\#}}\dar{f_{\#}} & \mathcal{C}(X,A) \rar\dar{f_{\#}} & 0 \\
			0 \rar & \mathcal{C}(B) \rar{i_{\#}} & \mathcal{C}(Y) \rar{\pi_{\#}} & \mathcal{C}(Y,B) \rar & 0
		\end{tikzcd}
		\] 
		Since $A \subset X$ and $B \subset Y$, each $i_{\#}$ is induced the natural inclusion. Since $\mathcal{C}(X,A)$ and $\mathcal{C}(Y,B)$ are quotients, each $\pi_{\#}$ is induced from the canonical projection.

		It's straightforward to check that this diagram commutes. For any $n$ and any $\sigma \in C_{n}(X)$, we have $(f_{\#}\pi_{\#})(\sigma) = [f\sigma] = (\pi_{\#}f_{\#})(\sigma)$. For any $n$ and any $\sigma \in C_{n}(A)$, we have $(f_{\#}i_{\#})(\sigma) = f\sigma = (i_{\#}f_{\#})(\sigma)$. Thus the diagram commutes.

		Then by naturality, we have the following commutative diagram for all $n$, where both rows are exact and the middle $f_{*}$ is once again a similar abuse of notation.
		\[
		\begin{tikzcd}
			H_{n}(A) \rar\dar{f_{*}} & H_{n}(X) \rar\dar{f_{*}} & H_{n}(X,A) \rar\dar{f_{*}} & H_{n-1}(A) \rar\dar{f_{*}} & H_{n-1}(X) \dar{f_{*}} \\
			H_{n}(B) \rar & H_{n}(Y) \rar & H_{n}(Y,B) \rar & H_{n-1}(B) \rar & H_{n-1}(Y)
		\end{tikzcd}
		\] 
		We'd like to apply the five lemma here, but to do this, we'll need to show that $f_{*}$ is an isomorphism $H_{n}(X)\to H_{n}(Y)$ and $H_{n}(A) \to H_{n}(B)$ for all $n$. Since $f$ is a homotopy equivalence $X \simeq Y$, we know there is a map $g:Y\to X$ such that $fg \simeq \id_{Y}$ and $gf \simeq \id_{X}$. Applying the homology functor for any $n$ then gives the following diagram.
	\begin{equation*}
        \begin{aligned}[c]
		\begin{tikzcd}
			X \rar[bend left]{f} & Y \lar[bend left]{g}
		\end{tikzcd}
        \end{aligned}
        \qquad\leadsto\qquad
        \begin{aligned}[c]
                \begin{tikzcd}
			H_{n}(X) \rar[bend left]{f_{*}} & H_{n}(Y) \lar[bend left]{g_{*}}
                \end{tikzcd}
        \end{aligned}
        \end{equation*}
	By functoriality and the fact that homotopic maps have the same induced map on homology, $g_{*}f_{*}=(gf)_{*} = \id_{*}=\id_{}$. Similarly, $f_{*}g_{*}=\id_{}$ as well. Thus $f_{*}$ is an isomorphism $H_{n}(X) \cong H_{n}(Y)$. Since $f$ restricts to a homotopy equivalence $A \simeq B$ as well, we can repeat this argument to show $H_{n}(A) \simeq H_{n}(B)$.

		Thus we can apply the five lemma, which gives $H_{n}(X,A) \cong H_{n}(Y,B)$ for all $ n$.

	\item First we'll show that any homotopy of pairs $(X,A)\to (Y,B)$ is also a homotopy of pairs $(X,\overline{A})\to (Y,\overline{B})$. If we have a homotopy $H:X\times I \to Y$ such that $H_t:A\to B$ for all $t$, then all we need for this is for $H_t$ to map $\overline{A}$ into $\overline{B}$. But since each $H_{t}$ is necessarily continuous,
		\[
			H_{t}(\overline{A}) \subset \overline{H_{t}(A)} \subset \overline{B}
		\] for all $t$. Thus we have a homotopy of maps $(X,\overline{A})\to (Y,\overline{B})$. Now we can show that the $f$ in the problem statement isn't a homotopy equivalence of pairs.

		If we have a homotopy equivalence of pairs $(X,\overline{A})\to (Y,\overline{B})$, we can clearly restict the homotopy equivalence $X\simeq Y$ to get a homotopy equivalence $\overline{A}\simeq \overline{B}$. In the problem, we have $X=Y=D^{n}$, $\overline{A}=S^{n-1}$, and $\overline{B}=\overline{D^{n}-\{0\}}=D^{n}$, so by part (a), we must have
		\[
			H_{n}(D^{n}, S^{n-1}) \cong H_{n}(D^{n},D^{n}) \cong 0
		\] for all $n$. By the long exact sequence of the pair $(D^{n},S^{n-1})$ in reduced homology, we have the following exact sequence.
		\[
			\tilde{H}_{n}(D^{n}) \to H_{n}(D^{n},S^{n-1}) \to \tilde{H}_{n-1}(S^{n-1}) \to \tilde{H}_{n-1}(D^{n})
		\] Since $D^{n}$ is contractible, it has trivial reduced homology in all dimensions. We also know $\tilde{H}_{n-1}(S^{n-1}) \cong n-1$, so this exact sequence is actually the following.
		\[
		0 \to H_{n}(D^{n},S^{n-1}) \to \mathbb{Z} \to 0
	\] This implies $H_{n}(D^{n},S^{n-1}) \cong \mathbb{Z}$, but $\mathbb{Z} \not\cong 0 \cong H_{n}(D^{n},D^{n})$, so we've arrived at a contradiction. Thus $f$ cannot be a homotopy equivalence of pairs.

\end{enumerate}

\end{document}
