\documentclass[twoside,10pt]{article}
\usepackage{/Users/bradenhoagland/latex/styles/toggles}
%\toggletrue{sectionbreaks}
%\toggletrue{sectionheaders}
\newcommand{\docTitle}{HW 1}
\usepackage{/Users/bradenhoagland/latex/styles/common}
\importStyles{modern}{rainbow}{boxy}

%\renewcommand{\theenumi}{\alph{enumi}}

\begin{document}
%\tableofcontents

% ------------------------------
% 1
% ------------------------------
\begin{exer}[]
	Prove a graph is connected according to Definition 1 (every pair has a path) if and only if it is connected according to Definition 2 (there is no separation).
\end{exer}

\textbf{1 implies 2:} Suppose every pair of points in $G$ has a path between them. Now suppose $U,W$ separate $G$. Then for any $u \in U$, $w \in W$, there is a path connecting them. But this path must cross from $U$ to $W$ at some point, so there is an edge starting in $U$ and ending in $V$. But this contradicts the definition of a separation, so no separation of $G$ exists.

\textbf{2 implies 1:} Suppose there is no separation of $G$, and fix $x,y \in G$. Now suppose there is no path from $x$ to $y$, then $x$ and $y$ are in different (necessarily nonempty) connected components $X$ and $Y$, respectively. Suppose $Z$ is the union of all other connected components, then $(Z \uni X)$ and $Y$ separate $G$. This contradicts our original assumption, so there must be a path from $x$ to $y$.

\newpage

% ------------------------------
% 2
% ------------------------------
\begin{exer}[]
Let $V$ be a finite dimensional vector space with subspace. Let $N$ be a subspace with basis $\mathcal{A} = \{a_1, \dots, a_n\}$, and let $\mathcal{B} = \{[b_1], \dots, [b_m]\}$ be a basis for $V/N$. Prove $\{a_1, \dots, a_n, b_1, \dots, b_m\}$ is a basis for $V$.
\end{exer}

We must show that this basis spans $V$ and is linearly independent.

\textbf{Spans:} Fix $v \in V$. Since $[v] \in V/N$ and $V/N$ has basis $\mathcal{B}$, we know
\[
	[v] = \sum_{j=1}^{m} \lambda_{j}[b_{j}] = \left[ \sum_{j=1}^{m}\lambda_{j}b_{j} \right]
\] for some collection of scalars $\{\lambda_{j}\}$. In particular, this means $v - \sum_{j}\lambda_{j}b_{j} \in N$. Then since $N$ has basis $\mathcal{A}$, this means
\[
v - \sum_{j=1}^{m} \lambda_{j}b_{j} = \sum_{i=1}^{n} \mu_{i}a_{i}
\] for some collection of scalars $\left\{ \mu_{i} \right\}$. Then $v = \sum_{i}\mu_{i}a_{i} + \sum_{j}\lambda_{j}b_{j}$, so the proposed basis spans $V$.

\textbf{Linearly Independent:} Suppose $\sum_{i}\mu_{i}a_{i} + \sum_{j}\lambda_{j}b_{j} = 0$, then we want to show that each $\mu_{i}$ and $\lambda_{j}$ is 0. To start, note that since $\mathcal{B}$ is a basis (and is thus linearly independent),
\[
	\sum_{j}\lambda_{j}[b_{j}] = \left[ \sum_{j}\lambda_{j}b_{j} \right] = [0] = N \quad\implies \quad \lambda_i = 0 \text{ for all } i.
\] In particular, this means that if $\sum_{j}\lambda_{j}b_{j} \in N$, then each $\lambda_j$ is 0. But by our original assumption, $\sum_{j}\lambda_{j}b_{j} = - \sum_{i}\mu_{i}a_{i} \in N$, so $\lambda_{j}=0 $ for all $j$. This leaves us with $\sum_{i}\mu_{i}a_{i}=0$. Then since $\mathcal{A}$ is a basis, this implies that each $\mu_i=0$ as well. Thus the proposed basis is also linearly independent.

\newpage

\end{document}
