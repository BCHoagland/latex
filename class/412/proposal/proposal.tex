\documentclass[10pt]{article}
\newcommand{\docTitle}{Math 412 - Project Proposal}

%\renewcommand{\theenumi}{\alph{enumi}}

\begin{document}
%\tableofcontents
\pagestyle{empty}

\begin{center}
	\textbf{Math 412: Project Proposal} 
\end{center}

\vspace{5mm}

\noindent\textbf{Group members:} Braden Hoagland, Jerry Liu, Mike Montelli

\vspace{5mm}

We will analyze the latent space of styleGAN using persistent homology and attempt to add in topological constraints to the training of such architecture that result in more realistic generated data. We plan on using the CelebA-HQ dataset of celebrity portraits, but could also use less complex datasets like MNIST or CIFAR-10 when conducting smaller experiments throughout the process.

To study the latent space of our GAN, we plan on calculating the persistent homology of the learned representations of the training data. This should give us a notion of how well the learned representations are clustered and whether or not we should expect decent generalization. We will then incorporate homological constraints into the training loss function, ensuring that the latent representations are regularized in some sense that has not yet been decided on. This process will hopefully lead to a more stable and realistic generator.

\end{document}
