\documentclass[twoside,10pt]{article}
\usepackage{/Users/bradenhoagland/latex/styles/toggles}
%\toggletrue{sectionbreaks}
%\toggletrue{sectionheaders}
\newcommand{\docTitle}{Math 412 - HW 6}
\usepackage{/Users/bradenhoagland/latex/styles/common}
\importStyles{modern}{rainbow}{boxy}

%\renewcommand{\theenumi}{\alph{enumi}}

\begin{document}
%\tableofcontents

\begin{exer}[Lesson 10, 5 points]
	Let $\emptyset = K^0 \subset K^1 \subset \dots \subset K^m = K$ be a simplex-wise filtration of $K$ and $\sigma_i$ a negative $(p+1)$-simplex. Prove $\beta_p(K^i) = \beta_p(K^{i-1}) - 1$.
\end{exer}

Since
\begin{align*}
	\beta_p(K^{i}) = \dim H_{p}(K^{i}) &= \dim Z_{p}(K^{i}) - \dim B_{p}(K^{i}),\\
	\beta_p(K^{i-1}) = \dim H_{p}(K^{i-1}) &= \dim Z_{p}(K^{i-1}) - \dim B_{p}(K^{i-1}),
\end{align*}
we can show the desired relationship by calculating the dimensions of $Z_{p}(K^{i}), Z_{p}(K^{i-1}), B_{p}(K^{i}),$ and $B_{p}(K^{i-1})$.

\textbf{Cycles:} Since $\sigma_i \in C_{p+1}$, we know $C_{p}(K^{i-1})=C_{p}(K^{i})$. This implies $Z_{p}(K^{i}) = Z_{p}(K^{i-1})$, so their dimensions are the same.

\textbf{Boundaries:} Since $\sigma_i$ is negative, there is no cycle in $Z_{p+1}(K^{i})$ containing it. This implies that $\p \sigma_i$ cannot be in $B_{p}(K^{i-1})$: if it were, there would be some $(p+1)$-chain $c \in C_{p+1}(K^{i-1})$ such that $\p c = \p \sigma_i$; then by linearity, $\p(c + \sigma_i)=0$, so $c+\sigma_i$ would be a cycle in $K^{i+1}$ containing $\sigma_i$, contradicting $\sigma_i$ being negative.

Now $\p \sigma_i$ is clearly in $B_{p}(K^{i})$, so we know that $B_{p}(K^{i})$ must have a higher dimension than $B_{p}(K^{i-1})$. We must show that the dimension only increases by 1.

Suppose $\left\{ \p \tau_j \right\}_{j}$ is a basis for $B_{p}(K^{i})$. Since we add 1 simplex every timestep, at \textit{most} 1 of these basis elements can be missing from $B_{p}(K^{i-1})$. But we just argued that at \textit{least} one of them must be missing (otherwise $\p \sigma_i$ would be an element of $B_{p}(K^{i-1})$). Thus only 1 of them is added at time $i$, i.e. $\dim B_{p}(K^{i}) = \dim B_{p}(K^{i-1})+1$.

\textbf{Conclusion:} Putting this all together, we get
\begin{align*}
	\beta_{p}(K^{i}) &= \dim Z_{p}(K^{i}) - \dim B_{p}(K^{i}) \\
			 &= \dim Z_{p}(K^{i-1}) - \dim B_{p}(K^{i}) - 1 \\
			 &= \beta_{p}(K^{i-1}) - 1.
\end{align*}


\newpage

\begin{exer}[Lesson 10, 5 points]
	Let $\emptyset = K^0 \subset K^1 \subset \dots \subset K^m = K$ be a simplex-wise filtration of $K$ and $\sigma_i$ a positive $p$-simplex. In class, we proved there exists $c_i \in Z_p(K^i)$ such that $\sigma_i \in c_i$, $c_i \not\in B_p(K^i)$, and $c_i$ does not contain any other positive simplices besides $\sigma_i$. Prove $c_i$ is the unique cycle satisfying these properties. 
\end{exer}

Suppose $\sigma_i$ is a positive $p$-simplex, and suppose there exist $c,c' \in Z_{p}(K^{i})$ both containing $\sigma_i$ and not containing any other positive simplices. Consider $c+c'$ and note that it does not contain $\sigma_i$. By linearity,
\[
	\p(c+c') = \p c + \p c' = 0,
\] so it is a cycle. Take the simplex in $c+c'$ added at the latest time, then $c+c'$ is the cycle showing that this simplex is positive. But this is a contradiction, as $\sigma_i$ was the \textit{only} positive simplex in both $c$ and $c'$. Thus by contradiction, $c$ is unique.

\end{document}
