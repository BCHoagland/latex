\documentclass[10pt]{amsart}
\usepackage{latexsym}
\usepackage{amscd,amsthm,amssymb,amsfonts,amsmath}
\usepackage{xcolor}
\usepackage[matrix,tips,graph,curve]{xy}

\newcommand{\mnote}[1]{${}^*$\marginpar{\footnotesize ${}^*$#1}}
\linespread{1.065}
\setlength{\parskip}{0.5em}

\makeatletter

\setlength\@tempdima  {5.5in}
\addtolength\@tempdima {-\textwidth}
\addtolength\hoffset{-0.5\@tempdima}
\setlength{\textwidth}{5.5in}
\setlength{\textheight}{8.75in}
\addtolength\voffset{-0.625in}

\newtheorem{manualtheoreminner}{Exercise}
\newenvironment{exercise}[1]{%
        \vspace{10mm}
        \renewcommand\themanualtheoreminner{#1}%
  \manualtheoreminner
}\hrulefill{\endmanualtheoreminner}

\newcommand{\p}{\partial}

\begin{document}
\title{Math 531 Homework 7}
\author{Braden Hoagland}
\date{\today}
\maketitle

\begin{exercise}{7}
	Consider a compact set $B \subset \mathbb{R}^n$ and let $f:B\to\mathbb{R}^m$ be continuous and one-to-one. Then prove that $f^{-1}:f(B)\to B$ is continuous. Show be example that this may fail if $B$ is connected but not compact.
\end{exercise}

Let $U \subset B$ be open in $B$, then we must show that $(f^{-1})^{-1}(U) = f(U)$ is open. Since $U$ is open in $B$, $B-U$ is closed in $ B$. A closed subset of a compact set is itself compact, so $B-U$ is compact. Now since $f$ is continuous, $f(B-U)$ is also compact and, subsequently, closed. Then its complement $f(B) - f(B-U) = f(U)$ is open. Thus $f^{-1}$ is continuous.

As a counterexample when $B$ is connected but not compact, take $f:[0,2\pi) \to \mathbb{R}$ defined by $f(t) = (\sin x, \cos x)$. This maps the non-open interval $[0,2\pi)$ to the entire unit circle, which is compact since it is closed and bounded in $\mathbb{R}^2$; however, since the unit circle is compact, any continuous map with the unit circle as its domain would have a compact image. Since $[0,2\pi)$ is not closed in $\mathbb{R}$, it is not compact, so the inverse map of $f$ cannot be continuous.

%============================
\begin{exercise}{13}
	Let $f$ be a bounded continuous function $f:\mathbb{R}^n\to\mathbb{R}$. Prove that $f(U)$ is open for all open sets $U \subset \mathbb{R}^n$ if and only if for all nonempty open sets $V \subset \mathbb{R}^n$,
	\[
		\inf_{x\in V}f(x) < f(y) < \sup_{x\in V}f(x)
	\] for all $y \in V$.
\end{exercise}

\textbf{Forward:} Assume that for all open $U\subset \mathbb{R}^n$, the set $f(U)$ is open in $\mathbb{R}$. Let $V \subset \mathbb{R}^n$ be nonempty and open in $V$, then by assumption, $f(V)$ is open in $\mathbb{R}$. Let $y \in V$, then becuase $f(V)$ is open, there is $\varepsilon>0$ such that $(f(y)-\varepsilon, f(y)+\varepsilon)\subset f(V)$. Thus there is $y_1\in V$ such that $f(y_1)\in(f(y)-\varepsilon, f(y))$, so \[\inf_{x\in V}f(x) \leq f(y_1)<f(y).\] Similarly, there is a point $y_2\in V$ such that \[f(y) < f(y_2) \leq \sup_{x\in V}f(x).\] Chaining these inequalities together gives the desired result.

\textbf{Backward:} Let $V$ be an open set in $\mathbb{R}^n$, then by assumption $\inf_{x\in V}f(x) < f(y) < \sup_{x\in V}f(x)$ for arbitrary $y \in V$. To avoid the case where either the infimum or supremum is unbounded, we can extend our assumption to say that there exist $y_1$, $y_2 \in V$ such that \[\inf_x f(x) \leq f(y_1) < f(y) < f(y_2) \leq \sup_x f(x).\] There are two cases we must consider: when $V$ is connected and when $V$ is disconnected.

When $V$ is connected, $f(V)$ is also connected since $f$ is continuous. Then the open ball $D(y, \min\{d(y,y_1), d(y,y_2)\})$ clearly lies in $f(V)$.

When $V$ is disconnected, it must be made up of connected open components. Each of these components falls into the previous category, so $f(V)$ is the union of open sets and is thus itself open.

%============================
\begin{exercise}{14}
	\begin{enumerate}
		\item Find a function $f:\mathbb{R}^2\to\mathbb{R}$ such that
			\[
				\lim_{x \to 0} \lim_{y \to 0} f(x,y) \text{ and } \lim_{y \to 0} \lim_{x \to 0} f(x,y)
			\] exist but are not equal.
		\item Find a function $f:\mathbb{R}^2\to\mathbb{R}$ such that the two limits in \textbf{(a)} exist and are equal but $f$ is not continuous.
		\item Find a function $f:\mathbb{R}^2\to\mathbb{R}$ that is continuous on every line through the origin but is not continuous.
	\end{enumerate}
\end{exercise}

\begin{enumerate}
	\item Let $f(x,y) = x^y$, then \[
			\lim_{x \to 0} \lim_{y \to 0} f(x,y) = \lim_{x \to 0} 1 = 1,
	\] but
	\[
		\lim_{y \to 0} \lim_{x \to 0} f(x,y) = \lim_{y \to 0} 0 = 0,
	\] so we have found a satisfactory function.

\item Let $f(x,y) = xy/(x^2+y^2)$, with $f(0,0)=0$. The two limits in question are
	\[
		\lim_{x \to 0} \lim_{y \to 0} f(x,y) = \lim_{x \to 0} 0 = 0
	\] and
	\[
		\lim_{y \to 0} \lim_{x \to 0} f(x,y) = \lim_{y \to 0} 0 = 0
	\] so the limits exist and are equal. We claim, however, that $f$ is not continuous. Let $z \neq 0$, then $f(z,z) = 1/2$. Then the limit of $f(z,z)$ as $z$ approaches $0$ along the line $y=x$ is $1/2$, not $0$. Thus $f$ is not continuous.

\item A line through the origin can be either vertical (the y-axis) or of the form $y=mx$. In the former case, restrict $f$ to the y-axis gives
	\[
		f(0,y) = \frac{0}{y^2} = 0,
	\] which is constant everywhere and, subsequently, continuous. In the latter case, restricting $f$ to a line $y=mx$ gives
	\[
		f(x,mx) = \frac{mx^2}{x^2 + m^2x^2} = \frac{m}{m^2 + 1},
	\] which is also constant and, subsequently, continuous. Thus for any line through the origin, $f$ is continuous. We have already shown, though, that $f$ is not continuous.
\end{enumerate}

%============================
\begin{exercise}{23}
	Let $X$ be a compact metric space and $f:X\to X$ an isometry; that is, $d(f(x),f(y)) = d(x,y)$ for all $x,y \in X$. Show that $f$ is a bijection.
\end{exercise}

First we show that $f$ is injective. Let $x,y \in X$ such that $x\neq y$, then $d(x,y)>0$. By assumption, $d(f(x),f(y)) = d(x,y) > 0$, so $f(x) \neq f(y)$.

Now suppose that $f$ is not surjective, i.e. $X - f(X)$ is nonempty. Let $x_0 \in X - f(X)$. Since $X$ is compact, $f(X)$ is closed and $X-f(X)$ is open. Thus there exists $\varepsilon>0$ such that $D(x_0,\varepsilon) \in X-f(X)$. This implies that any element of $f(X)$ is at least $\varepsilon$ away from $x_0$.

Now let $x_1 = f(x_0)$, $x_2 = f(x_1)$, and continue inductively to construct a sequence $\left\{ x_n \right\}_{n=0}^\infty \subset X$. For $x_k$ in this sequence and $l > 0$, the distance between points $x_{k}$ and $x_{k+l}$ is
\begin{align*}
	d(x_{k},x_{k+l}) &= d(f(x_{k-1}), f(x_{k+l-1})) \\
		   &= d(x_{k-1}, x_{k+l-1}) \\
		   & \quad\quad\vdots \\
		   &= d(x_0,x_l).
\end{align*}
Since $x_l = f(x_{l-1}) \in f(X)$, it is at least $\varepsilon$ away from $x_0$. Thus $d(x_k, x_{k+l}) \geq \varepsilon$ for any $k$ with $l > 0$. Since every point in our sequence $\left\{ x_n \right\}$ is at least $\varepsilon$ apart, we cannot find any convergent subsequence. This is a contradiction, as $X$ being compact implies that $X$ is sequentially compact. Thus our original assumption must have been false, so $X-f(X)$ is empty. Equivalently, $f(X)$ is surjective.

%============================
\begin{exercise}{24}
	Let $f:A\subset M\to N$.
	\begin{enumerate}
		\item Prove that $f$ is uniformly continuous on $A$ if and only if for every pair of sequences $x_k,y_k$ of $A$ such that $d(x_k,y_k)\to 0$, we have $\rho(f(x_k),f(y_k)) \to 0$.
		\item Let $f$ be uniformly continuous, and let $x_k$ be a Cauchy sequence of $A$. Show that $f(x_k)$ is a Cauchy sequence.
		\item Let $f$ be uniformly continuous and $N$ be complete. Show that $f$ has a unique extension to a continuous function on $\overline{A}$.
	\end{enumerate}
\end{exercise}

\begin{enumerate}
	\item \textbf{Forward:} Assume $f$ is uniformly continuous on $A$. Let $\left\{ x_k \right\}, \left\{ y_k \right\}$ be sequences such that $d(x_k,y_k) \to 0$. Fix $\varepsilon>0$, then there is a $\delta>0$ such that $\rho(f(x_k),f(y_k)) < \varepsilon$ when $d(x_k,y_k) < \delta$. Since $d(x_k,y_k) \to 0$, there exists $N$ such that $d(x_k,y_k) < \delta$ when $k > N$. Thus for $k > N$, $\rho(f(x_k),f(y_k)) < \varepsilon$, so $\rho(f(x_k),f(y_k)) \to 0$.

		\textbf{Backward:} We will prove this by contrapositive. Assume $f$ is not uniformly continuous, then there exists $\varepsilon>0$ such that for all $\delta>0$, there are $x_k$ and $y_k$ such that $d(x_k,y_k)<\delta$ but $\rho(f(x_k),f(y_k)) \geq \varepsilon$. In particular such $x_k$ and $y_k$ exist for $\delta=1/n$ for all $n \in \mathbb{N}$. We can take these $x_k$ and $y_k$ to form a sequence that, by construction, satisfies $d(x_k, y_k) \to 0$. However, also by construction, $\rho(f(x_k),f(y_k))$ does not converge to $0$. This shows the contrapositive, so the original statement must also be true.

	\item Since $f$ is uniformly continuous, for all $\varepsilon>0$ there exists $\delta>0$ such that for all $x,y$ satisfying $d(x,y) < \delta$, we have $\rho(f(x),f(y)) < \varepsilon$. Since $\left\{ x_k \right\}$ is a Cauchy sequence, there exists $N$ such that $d(x_m,x_n)<\delta$ when $m,n > N$. Putting these together, when $k,l>N$, we have $\rho(f(x_n),(x_m)) < \varepsilon$. Thus $\left\{ f(x_k) \right\}$ is a Cauchy sequence.
	
	\item Let $a \in \overline{A}$, then $a_n \to a$ for some sequence $\left\{ a_n \right\}\subset A$. Since this sequence converges, it is Cauchy. Then by part $(2)$, $\{f(x_k)\}$ is also Cauchy. Since $N$ is complete, this implies that $\left\{ f(x_k) \right\}$ converges to some point which we define to be $f(a)$.

		We claim that this extension is continuous. Let $a \in \overline{A}$. If $a \in A$, then we know by assumption that $f$ is continuous already, so consider the case when $a \not\in A$. In this case, we have by definition $\lim_{n \to \infty} f(a_n) = f(\lim_{n \to \infty} a_n) = f(a)$, so $f$ is continuous on $\overline{A}$.

		We now show that this extension of $f$ is unique. Let $\left\{ a_n \right\}$ and $\left\{ b_n \right\}$ both converge to $a \in \overline{A}$. Then it is clear that $d(a_n,b_n) \to 0$. By part $(1)$, this implies $\rho(f(a_n),f(b_n)) \to 0$ as well, so $\left\{ f(a_n) \right\}$ and $\left\{ f(b_n) \right\}$ both converge to the same element of $\overline{A}$, namely $f(a)$. This shows that $f(a)$ is independent of the convergent sequence chosen, so our extension of $f$ is unique.
\end{enumerate}

%============================
\begin{exercise}{25}
	Let $f:(0,1)\to\mathbb{R}$ be differentiable and let $f'(x)$ be bounded. Show that $\lim_{x \to 0^{+}} f(x)$ and $\lim_{x \to 1^{-}} f(x)$ exist. Do this both directly and by applying exercise 24c. Give a counterexample if $f'(x)$ is not bounded.
\end{exercise}

\textbf{Direct proof:} We start by showing the existence of $\lim_{x \to 0^+} f(x)$. Since $f'$ is bounded, $|f'(x)| \leq M$ for all $x \in (0,1)$. Let $x \in (0, 1/n)$, then by the mean value theorem, for some $c \in (x,1/n)$ we have
\begin{align*}
	|f(x) - f(1/n)| &= |f'(c) (x-1/n)| \\
			&\leq M |x-1/n| \\
			&< M/n.
\end{align*}
Since $1/m \in (0, 1/n)$ when $m>n$, we apply this inequality to show \[|f(1/m) - f(1/n)| < M/n.\] For any $\varepsilon>0$, when $m>n>M/\varepsilon$, we have $|f(1/m) - f(1/n)|<\varepsilon$. Thus $\left\{ f(1/n) \right\}_{n=1}^{\infty}$ is a Cauchy sequence. Since $\mathbb{R}$ is complete, any Cauchy sequence converges, so $f(1/n) \to L$ for some $L \in \mathbb{R}$ as $n \to \infty$. Since $1/n$ converges to $0$ from the right as $n \to \infty$, this is equivalent to saying that $f(x) \to L$ as $x \to 0^+$. Thus $\lim_{x \to 0^+} f(x)$ exists.

Using a similar argument and the sequence $1-(1/n)$ instead of $1/n$, we can show that $\lim_{x \to 1^-} f(x)$ exists.

\textbf{Using (24c):} Since $f$ is differentiable on $(0,1)$ and $f'$ is bounded, say $|f'| \leq M$, by the Mean Value Theorem we have $|f(y)-f(x)| \leq M |y-x|$ for all $x,y \in (0,1)$. Fix $\varepsilon>0$, then set $\delta=\varepsilon/M$. If $|x-y| < \delta$, then $|f(y)-f(x)| \leq M |y-x| < \varepsilon$. Since $\delta$ was independent of $x$ and $y$, this shows that $f$ is uniformly continuous on $(0,1)$.

Since $f$ is uniformly continuous and $\mathbb{R}$ is complete, we can apply $(24c)$ to show that $f$ has a unique continuous extension to $\overline{(a,b)} =[a,b]$. Since a function that is continuous on a set is continuous at each of the points in that set, we know that $\lim_{x \to 0^+} f(x)$ and $\lim_{x \to 1^-} f(x)$ exist.

\textbf{Counterexample:} Now we show a counterexample when $f'$ is not bounded. Let $f:(0,1) \to \mathbb{R}$ be defined by $f(x) = 1/x$, which has unbounded derivative $f'(x) = -1/x^{2}$. Assume $\lim_{x \to 0^+} f(x) = L$, then for any sequence $\left\{ x_n \right\}$ satisfying $x_n \to 0$ and $x_n \neq 0$, the sequence $\left\{f(x_n)\right\} = \left\{ 1/x_n \right\}$ converges to $L$. If we let $x_n = 1/n$, then we have $f(x_n) = n$, so the sequence $\left\{ f(x_n) \right\}$ is unbounded and thus cannot converge. Then by contradiction, no such $L$ exists.

%============================
\begin{exercise}{26}
	Let $f:(a,b]\to\mathbb{R}$ be continuous such that $f'(x)$ exists on $(a,b)$ and the limit $\lim_{x \to a^{+}} f'(x)$ exists. Prove that $f$ is uniformly continuous.
\end{exercise}

Since $\lim_{x \to a^{+}} f'(x) = L$ for some $L$, we have that for $\varepsilon = 1$, there exists some $\delta>0$ such that $|f'(x)-L| < \varepsilon$ when $x \in (a,\delta)$. This allows us to bound $f'$ over this interval:
\[
	|f'(x)|-|L| \leq |f'(x)-L| < 1
\] so
\[
	|f'(x)| < |L| + 1
\] when $x \in (a,\delta)$. Then since $f'$ exists at $\delta$, we can bound $f'$ over the interval $(a,\delta]$ by
\[
	|f'(x)| < \mathcal{L} \doteq \max \left\{ |L|+1, |f'(\delta) \right\}.
\] 
Then by the mean value theorem, for all $x,y \in (a,\delta]$, we have
\[
	|f(y)-f(x)| \leq \mathcal{L} |y-x|.
\] Fix $\varepsilon' > 0$, then set $\delta' = \varepsilon' / \mathcal{L}$. Then for $|x-y| < \delta'$, we have $|f(y)-f(x)| \leq \mathcal{L} |y-x| < \varepsilon'$. Thus $f$ is uniformly continuous on $(a,\delta]$.

The function $f$ is also uniformly continuous on $[\delta,b]$, since this is a compact interval and $f$ is continuous to begin with. Then since $f$ is uniformly continuous over both segments of $(a,b]$, it is uniformly continuous over the whole interval, which we now prove.

Fix $\varepsilon>0$, then we can find $\delta_l$ such that for $x,y \in (a,\delta]$, $|x-y| < \delta_l$ implies $|f(x)-f(y)| < \varepsilon/2$. Let $\delta_r$ be the corresponding value for the interval $[\delta,b]$ instead. Now set $\delta= \min\left\{ \delta_l, \delta_r \right\}$, and let the pair $x,y$ be in the entire interval $(a,b]$. If both points are in $(a,\delta]$ or if both are in $[\delta,b]$, then clearly $|f(x)-f(y)| < \varepsilon/2 < \varepsilon$.

If one point is in $(a,\delta]$ and the other is in $[\delta,b]$, then assume without loss of generality that $x$ is in the former and $y$ is in the latter. In this case $\delta$ lies between $x$ and $y$, so $|x-\delta| \leq |x-y| < \delta$ and $|y-\delta| \leq |y-x| < \delta$. Then we have
\[
	|f(x) - f(y)| \leq |f(x) - f(\delta)| + |f(\delta) - f(y)| < \frac{\varepsilon}{2} + \frac{\varepsilon}{2} = \varepsilon.
\] 
Thus $f$ is uniformly continuous over $(a,b]$.

%============================
\begin{exercise}{29}
	Let $f:\mathbb{R}\to\mathbb{R}$ satisfy $|f(x)-f(y)|\leq |x-y|^2$. Prove that $f$ is a constant.
\end{exercise}

We will show that the derivative of this function is 0. Using the given bound and the fact that $x \mapsto |x|$ is a continuous map, we have
\begin{align*}
	|f'(x)| &= \left| \lim_{h \to 0} \frac{f(x+h)-f(x)}{h} \right| \\
		&= \lim_{h \to 0} \frac{|f(x+h)-f(x)|}{|h|} \\
		&\leq \lim_{h \to 0} \frac{|h|^2}{|h|} \\
		&= \lim_{h \to 0} |h| \\
		&= 0.
\end{align*}
This implies $f'(x) = 0$, which is only true if $f$ is a constant function.

%============================
\begin{exercise}{34}
	Assuming that the temperature on the surface of the earth is a continuous function, prove that on any great circle of the earth there are two antipodal points with the same temperature.
\end{exercise}

Let $C$ denote any great circle of the earth. If $x \in C$, then denote its antipodal point by $x'$. Finally denote the temperature of the earth by the continuous function $T: C \to \mathbb{R}$, and define the continuous function $f(x) = T(x) - T(x')$.

Let $x \in C$ be arbitrary, then we have two cases: $f(x) = 0$ or $f(x) \neq 0$. If the former case holds, the result is trivial, so assume $f(x) \neq 0$. Then we have $f(x') = T(x') - T(x) = -f(x)$, so $f(x)$ and $f(x')$ have opposite signs.

Since $C$ is connected and $f$ is continuous, we then can apply the intermediate value theorem to show that there is some point $z \in C$ such that $f(z) = T(z) - T(z') = 0$, which proves the result.

%============================
\begin{exercise}{38}
	A real-valued function defined on $(a,b)$ is called \textbf{convex} when the following inequality holds for $x,y$ in $(a,b)$ and $t$ in $[0,1]$:
	\[
		f(tx + (1-t)y) \leq tf(x) + (1-t)f(y).
	\] If $f$ has a continuous second derivative and $f''>0$, show that $f$ is convex.
\end{exercise}

Let $z = tf(x) + (1-t)f(y)$, and without loss of generality, assume $y \geq x$. Then we wish to show $tf(x) + (1-t)f(y) \geq f(z)$. We will do so by showing that $tf(x) + (1-t)f(y) - f(z)$ is non-negative. We have
\begin{align*}
	tf(x) + (1-t)f(y) - f(z) &= tf(x) + (1-t)f(y) -tf(z) + (1-t)f(z) \\
				 &= t[f(x)-f(z)] + (1-t)[f(y)-f(z)].
				 \intertext{Let $\alpha\in(x,z)$, $\beta\in(z,y)$, then by the mean value theorem this becomes}
				 &= t[f'(\alpha)(x-z)] + (1-t)[f'(\beta)(y-z)].
				 \intertext{Finally, expand $z$ to get}
				 &= t(1-t)f'(\alpha)(x-y) + t(1-t)f'(\beta)(y-x) \\
				 &= t(1-t)(y-x) [f'(\beta)-f'(\alpha)].
\end{align*}
Since $t\in[0,1]$, we know $t(1-t)\geq 0$. By assumption, $y-x\geq 0$ as well. Since $f''(x) > 0$ for all $x \in (a,b)$, this means $f'$ is always increasing. Then since $\beta>\alpha$, $f'(\beta)-f'(\alpha)$ must also be non-negative. Thus we have
\[
	tf(x) + (1-t)f(y) - f(z) = t(1-t)(y-x) [f'(\beta)-f'(\alpha)] \geq 0,
\] so
\[
	tf(x) + (1-t)f(y) \geq f(z),
\] as desired.


\end{document}
