\documentclass[10pt]{amsart}
\usepackage{latexsym} 
\usepackage{amscd,amsthm,amssymb,amsfonts,amsmath}
\usepackage{xcolor}
%\usepackage{epsfig}
%\usepackage{graphicx}
%\usepackage[dvips]{graphicx}

\usepackage[matrix,tips,graph,curve]{xy}

\newcommand{\mnote}[1]{${}^*$\marginpar{\footnotesize ${}^*$#1}}
\linespread{1.065}
% \setlength{\parskip}{1em}

\makeatletter

\setlength\@tempdima  {5.5in}
\addtolength\@tempdima {-\textwidth}
\addtolength\hoffset{-0.5\@tempdima}
\setlength{\textwidth}{5.5in}
\setlength{\textheight}{8.75in}
\addtolength\voffset{-0.625in}

\makeatother

\makeatletter 
\@addtoreset{equation}{section}
\makeatother


\renewcommand{\theequation}{\thesection.\arabic{equation}}

\theoremstyle{plain}
\newtheorem{theorem}[equation]{Theorem}
\newtheorem{corollary}[equation]{Corollary}
\newtheorem{lemma}[equation]{Lemma}
\newtheorem{proposition}[equation]{Proposition}
\newtheorem{conjecture}[equation]{Conjecture}
\newtheorem{fact}[equation]{Fact}
\newtheorem{facts}[equation]{Facts}
\newtheorem*{theoremA}{Theorem A}
\newtheorem*{theoremB}{Theorem B}
\newtheorem*{theoremC}{Theorem C}
\newtheorem*{theoremD}{Theorem D}
\newtheorem*{theoremE}{Theorem E}
\newtheorem*{theoremF}{Theorem F}
\newtheorem*{theoremG}{Theorem G}
\newtheorem*{theoremH}{Theorem H}

\newtheorem{manualtheoreminner}{Exercise}
\newenvironment{exercise}[1]{%
  \renewcommand\themanualtheoreminner{#1}%
  \manualtheoreminner
}{\endmanualtheoreminner}

\theoremstyle{definition}
\newtheorem{definition}[equation]{Definition}
\newtheorem{definitions}[equation]{Definitions}
%\theoremstyle{remark}

\newtheorem{remark}[equation]{Remark}
\newtheorem{remarks}[equation]{Remarks}
\newtheorem{example}[equation]{Example}
\newtheorem{examples}[equation]{Examples}
\newtheorem{notation}[equation]{Notation}
\newtheorem{question}[equation]{Question}
\newtheorem{assumption}[equation]{Assumption}
\newtheorem*{claim}{Claim}
\newtheorem{answer}[equation]{Answer}

%% my definitions %%%

\newcommand{\End}{\mathrm{End}}
\newcommand{\tr}{\mathrm{tr}}
%\newcommand{\ind}{\mathrm{ind}}

\renewcommand{\index}{\mathrm{index \,}}
\newcommand{\Hom}{\mathrm{Hom}}
\newcommand{\Aut}{\mathrm{Aut}}
\newcommand{\Trace}{\mathrm{Trace}\,}
\newcommand{\Res}{\mathrm{Res}\,}
\newcommand{\rank}{\mathrm{rank}}
%\renewcommand{\dim}{\mathrm{dim}}

\renewcommand{\deg}{\mathrm{deg}}
\newcommand{\spin}{\rm Spin}
\newcommand{\Spin}{\rm Spin}
\newcommand{\erfc}{\rm erfc\,}
\newcommand{\sgn}{\rm sgn\,}
\newcommand{\Spec}{\rm Spec\,}
\newcommand{\diag}{\rm diag\,}
\newcommand{\Fix}{\mathrm{Fix}}
\newcommand{\Ker}{\mathrm{Ker \,}}
\newcommand{\Coker}{\mathrm{Coker \,}}
\newcommand{\Sym}{\mathrm{Sym \,}}
\newcommand{\Hess}{\mathrm{Hess \,}}
\newcommand{\grad}{\mathrm{grad \,}}
\newcommand{\Center}{\mathrm{Center}}
\newcommand{\Lie}{\mathrm{Lie}}


\newcommand{\ch}{\rm ch} % Chern Character

\newcommand{\rk}{\rm rk} 
%\renewcommand{\c}{\rm c}  % Chern class

\newcommand{\sign}{\rm sign}
\renewcommand\dim{{\rm dim\,}}
\renewcommand\det{{\rm det\,}}
\newcommand{\dimKrull}{{\rm Krulldim\,}}
\newcommand\Rep{\mathrm{Rep}}
\newcommand\Hilb{\mathrm{Hilb}}
\newcommand\vol{\mathrm{vol}}

\newcommand\QED{\hfill $\Box$} %{\bf QED}} 

\newcommand\Pf{\nonintend{\em Proof. }}


\newcommand\reals{{\mathbb R}} 
\newcommand\complexes{{\mathbb C}}
\renewcommand\i{\sqrt{-1}}
\renewcommand\Re{\mathrm Re}
\renewcommand\Im{\mathrm Im}
\newcommand\integers{{\mathbb Z}}
\newcommand\quaternions{{\mathbb H}}


\newcommand\iso{{\cong}} 
\newcommand\tensor{{\otimes}}
\newcommand\Tensor{{\bigotimes}} 
\newcommand\union{\bigcup} 
\newcommand\onehalf{\frac{1}{2}}
%\newcommand\Sym[1]{{Sym^{#1}(\complexes^2)}}

\newcommand\lie[1]{{\mathfrak #1}} 
\newcommand\smooth{\mathcal{C}^{\infty}}
\newcommand\trivial{{\mathbb I}}
\newcommand\widebar{\overline}

%%%%%Delimiters%%%%

\newcommand{\<}{\langle}
\renewcommand{\>}{\rangle}

%\renewcommand{\(}{\left(}
%\renewcommand{\)}{\right)}


%%%% Different kind of derivatives %%%%%

\newcommand{\delbar}{\bar{\partial}}
\newcommand{\pdu}{\frac{\partial}{\partial u}}
%\newcommand{\pd}[1][2]{\frac{\partial #1}{\partial #2}}

%%%%% Arrows %%%%%
%\renewcommand{\ra}{\rightarrow}                   % right arrow
%\newcommand{\lra}{\longrightarrow}              % long right arrow
%\renewcommand{\la}{\leftarrow}                    % left arrow
%\newcommand{\lla}{\longleftarrow}               % long left arrow
%\newcommand{\ua}{\uparrow}                     % long up arrow
%\newcommand{\na}{\nearrow}                      %  NE arrow
%\newcommand{\llra}[1]{\stackrel{#1}{\lra}}      % labeled long right arrow
%\newcommand{\llla}[1]{\stackrel{#1}{\lla}}      % labeled long left arrow
%\newcommand{\lua}[1]{\stackrel{#1}{\ua}}      % labeled  up arrow
%\newcommand{\lna}[1]{\stackrel{#1}{\na}}      % labeled long NE arrow

\newcommand{\into}{\hookrightarrow}
\newcommand{\tto}{\longmapsto}
\def\llra{\longleftrightarrow}

\def\d/{/\mspace{-6.0mu}/}
\newcommand{\git}[3]{#1\d/_{\mspace{-4.0mu}#2}#3}
\newcommand\zetahilb{\zeta_{{\text{Hilb}}}}
\def\Fy{\sF \mspace{-3.0mu} \cdot \mspace{-3.0mu} y}
\def\tv{\tilde{v}}
\def\tw{\tilde{w}}
\def\wt{\widetilde}
\def\wtilde{\widetilde}
\def\what{\widehat}

%%%%%%%%%%%%%%%%%%% Mark's definitions %%%%%%%%%%%%%%%%%%%%

\newcommand{\frakg}{\mbox{\frakturfont g}}
\newcommand{\frakk}{\mbox{\frakturfont k}}
\newcommand{\frakp}{\mbox{\frakturfont p}}
\newcommand{\q}{\mbox{\frakturfont q}}
\newcommand{\frakn}{\mbox{\frakturfont n}}
\newcommand{\frakv}{\mbox{\frakturfont v}}
\newcommand{\fraku}{\mbox{\frakturfont u}}
\newcommand{\frakh}{\mbox{\frakturfont h}}
\newcommand{\frakm}{\mbox{\frakturfont m}}
\newcommand{\frakt}{\mbox{\frakturfont t}}
\newcommand{\G}{\Gamma}
\newcommand{\g}{\gamma}
\newcommand{\fraka}{\mbox{\frakturfont a}}
\newcommand{\db}{\bar{\partial}}
\newcommand{\dbs}{\bar{\partial}^*}
\newcommand{\p}{\partial}

%%%%%%%%%%%%% new definitions for the positive mass paper %%%%%%%%%

\newcommand{\sperp}{{\scriptscriptstyle \perp}}

%%%%%%%%%%%%%%%%%%%%%%%

%%%%%%%%%%%%%%%%%%%%%%%%%%%%%%%%%%%%%%%%%%%%%



%
\begin{document}
%

\title{Math 531 Homework 2}
\author{Braden Hoagland}


\date{\today}

\maketitle

%\setcounter{secnumdepth}{1} 


\begin{exercise}{i}
	Prove that every real number has an additive inverse and every nonzero real number has a multiplicative inverse.
\end{exercise}

\begin{lemma}\label{bruh}
	Let $\left\{ x_n \right\}$ be a non-increasing function bounded below, then we can find $\left\{ s_n \right\}$ non-decreasing and bounded above such that $ \left\{ x_n \right\}-\left\{ s_n \right\}\sim i(0)$.
\end{lemma}
\begin{proof}

Let $m_0$ be the largest integer such that $m_0 \leq x_i$ for all $i$. Inductively define $m_i$ to be the largest element of $\left\{ 0,1,\dots,9 \right\}$ such that $\sum_{k=0}^{i} \frac{m_k}{10^k} \leq x_i$ for all $i$. Now let the sequence $\left\{ s_n \right\}$ be defined by $s_n = \sum_{k=0}^{n} \frac{m_k}{10^k}$. We are essentially constructing a non-decreasing sequence that ``converges" to the same point as $x_n$ (we have not yet constructed the real numbers, so any notion of convergence is just intended to informally explain the general method of this proof).

We claim that $\left\{ x_n \right\} - \left\{ s_n \right\} \sim i(0)$. First we must show that if $L$ is an upper bound of $i(0)$, then $L$ is also an upper bound of $\left\{ x_n \right\} - \left\{ s_n \right\}$ (which from now on we abbreviate with $\left\{ z_n \right\}$). Since $s_j \leq x_j$ by construction for all $j$, it follows that $s_j - x_j \leq 0$ for all $j$ as well. Since $L$ is at least 0, it is clear that $z_n \leq L$.

Now we must show that if $L$ is an upper bound of $\left\{ z_n \right\}$, then it is also an upper bound of $i(0)$, which we will prove by contradiction. If $L$ is \textit{not} an upper bound of $i(0)$, then there is some $j$ such that $L + \frac{1}{10^j} < 0$. We also know that $z_j \leq L$ (by definition of an upper bound). Putting these two statements together yields
\[
z_j + \frac{1}{10^j} \leq L + \frac{1}{10^j} < 0
\]
However, by construction we know $z_j + \frac{1}{10^j} $ is an upper bound for 0. This is a contradiction, so it must be the case that $L$ is actually an upper bound of $i(0)$.
\end{proof}

\textbf{Additive Inverse}
Let $\left\{ x_n \right\}$ be a non-decreasing sequence bounded above, then the negation of this sequence, $\left\{ -x_n \right\}$, is non-increasing and bounded below by $-L$. Thus by Lemma \ref{bruh}, we can construct a non-decreasing sequence $\left\{ z_n \right\}$ such that $\left\{ -x_n \right\} - \left\{ z_n \right\} \sim i(0) \implies \left\{ x_n \right\} + \left\{ z_n \right\} \sim i(0)$. This means that the sequence $\left\{ z_n \right\}$ is the additive inverse of $\left\{ x_n \right\}$.

\textbf{Multiplicative Inverse}
For any nonzero $\left\{ x_n \right\}$, we need to find $\left\{ x_n \right\}^{-1}$ such that $\left\{ x_n \right\} \cdot \left\{ x_n \right\}^{-1} \sim i(1)$. This is straightforward if we use additive inverses, which we just proved exist in our construction.

The sequence $\left\{ 1/x_n  \right\}$ clearly multiplies with $\left\{ x_n \right\}$ to yield $i(1)$, but we cannot use it since it is a monotonically decreasing function. However, Lemma \ref{bruh} tells us that since $\left\{ 1/x_n \right\}$ is bounded below by $\min\left\{ 0, 1/L \right\}$, we can find a sequence  $\left\{ b_n \right\}$ such that $\left\{ 1/x_n \right\}-\left\{ b_n \right\}\sim i(0) \implies \left\{ 1 \right\}-\left\{ x_n b_n \right\} \sim i(0) \implies \left\{ x_n b_n \right\} \sim i(1)$. This means that $\left\{ b_n \right\}$ is the mlutiplicative inverse of $\left\{ x_n \right\}$.

\begin{exercise}{ii}
	The naive definition of multiplication does not always take monotone increasing sequences to monotone increasing sequences. Fix the definition so that it does.
\end{exercise}
The problem in the naive definition of multiplication arises when a negative sequence ``flips" the product sequence, making it non-increasing instead of non-decreasing. We can use the existence of additive inverses to fix this. Suppose the product $\left\{ x_n y_n \right\}$ produces a non-increasing sequence, then we can negate one of the sequences to create a non-decreasing sequence $-\left\{ x_n y_n \right\}$. Since we have proven that additive inverses exist, then there must be some sequence $- -\left\{ x_n y_n \right\} = \left\{ x_n y_n \right\}$ that is non-decreasing. In cases where the naive definition of $\left\{ x_n y_n \right\}$ would result in a non-increasing sequences, we can replace the product with this new sequence instead and stay within the set of non-decreasing sequences.


%%%%%%%%%%%%%%%%%%%%%%%%%%%%%%%%%%%%%%%%
\begin{exercise}{1.3.3}
	If $P \subset Q \subset \mathbb{R}$, $P \neq \emptyset$, and  $P$ and $Q$ are bounded aboved, show $\sup P \leq \sup Q$.
\end{exercise}
Suppose $\sup Q < \sup P$, then $\sup Q$ is not an upper bound of $P$. This is because by definition, no upper bound of $P$ can be strictly less than the supremum of $P$.
However, $P \subset Q$ implies that if $p \in P$, then $p \in Q$ as well. Then by the definition of the supremum, $p \leq \sup Q$. This is a contradiction, so our original assumption that $\sup Q < \sup P$ must be incorrect. By the law of trichotomy for the real numbers, it must then be the case that $\sup P \leq \sup Q$.

%%%%%%%%%%%%%%%%%%%%%%%%%%%%%%%%%%%%%%%%
\begin{exercise}{1.3.5}
	Let $S \subset [0,1]$ consist of all infinite decimal expansions $x=0.a_1 a_2 a_3 \cdots$ where all but finitely many digits are 5 or 6. Find $\sup S$.
\end{exercise}

The supremum of this set is 1. Fist we can show that 1 is an upper bound of the set, then we can show that any number less than 1 cannot be an upper bound.

To show that 1 is an upper bound, note that 1 can be written as
\[
\sum_{i=0}^{\infty} \frac{n_i}{10^i}
\]
where $n_0 = 1$ and all other $n_i = 0$ (all $n_i \in \left\{ 0, 1, \dots, 9 \right\}$). From this it is clear that if $x \sum_{i=0}^{\infty} \frac{m_i}{10^i} < 1$, then $m_0$ must be 0. Since all elements of the set $S$ satisfy this, 1 is an upper bound for the set.

Now we must show that any $x < 1$ cannot be an upper bound of $S$. Take an arbitary $x < 1$ and write it in the decimal expansion notation
\[
x = \sum_{i=0}^{\infty} \frac{m_i}{10^i}
\] 
Find the first $i$ such that $m_i \neq 9$. Set this to 9 to get a new number $x'$. Now set all $m_j = 6$ for $j>i$. Then $x' \in S$ and $x' > x$. The only number for which this process does not work is $0.999\dots$, since there are no digits not equal to 9. In this case, we can note that given a point $x$ such that $0.999\dots \leq x \leq 1$, it must be the case that $|1 - x| \leq \frac{1}{10^n} $ for any $n$. By the Archimedean property, we can always find $n$ such that $|1-x| > \frac{1}{10^n} $ when $x$ is fixed and $1-x$ is nonzero. Thus the only solution is $x=1$, so this in fact does not invalidate our earlier process.

This shows that no $x < 1$ can be an upper bound of $S$. Since $x < 1$ implies that $x$ is not an upper bound and since 1 is an upper bound itself, it must be the least upper bound.

%%%%%%%%%%%%%%%%%%%%%%%%%%%%%%%%%%%%%%%%
\begin{exercise}{1.4.1}
	Let $x_n$ satisfy $|x_n - x_{n+1}| < 1/3^n$. Show that $x_n$ converges.
\end{exercise}
We need to show that for some $\varepsilon$, there exists $N$ such that $|x_n - x_m| < \varepsilon$ for all $n,m > N$. We can use the algebraic identity $1+r+\cdots+r^n = \frac{1-r^{n+1}}{1-r} $ to form the following inequality
\begin{align*}
	r^n + r^{n+1} + \cdots + r^{n+k-1} &= \frac{1-r^{n+1}}{1-r} - \frac{1-r^n}{1-r} \\
					   &= \frac{r^n - r^{n+k}}{1-r} \\
					   &< \frac{r^n}{1-r} 
\end{align*}
where the final inequality holds if $0 < r<1$.

Now we can bound the difference between two points in the sequence separated by $k$ points.
\begin{align*}
	|x_n - x_{n+k}| &\leq |x_n - x_{n+1}| + |x_{n+1} - x_{n+2}| + \cdots + |x_{n+k-1} - x_{n+k}| \\
			&< \frac{1}{3^n} + \frac{1}{3^{n+1}} + \cdots + \frac{1}{3^{n+k-1}} \\
			\intertext{Using our derived inequality with $r=1/3$, we can bound this with}
			&< \frac{(1/3)^n}{2/3} \\
			&= \frac{1}{2 \cdot 3^{n-1}}
\end{align*}
We claim that the sequence defined by $y_n = \frac{1}{2} \cdot \frac{1}{3^{n-1}} $ converges to 0. If this is true, then by definition, we can find an $N$ such that $|x_n - x_{n+k}| < \frac{1}{2} \cdot \frac{1}{3^{n-1}} = | \frac{1}{2} \cdot \frac{1}{3^{n-1}} | < \varepsilon$ for any $\varepsilon>0$. Since this is not dependent on the choice of $k$, this holds for any arbitrary choice of $k$ and thus the sequence $x_n$ is a Cauchy sequence and, subsequently, converges.

To show that $y_n$ converges to 0, we can derive the inequality $y_n < \frac{1}{n} $ and use the fact that $\frac{1}{n} \to 0$ in $\mathbb{R}$. The inequality clearly holds when $ n=1$ since $\frac{1}{2} < 1$. Assuming the inequality holds for some given $n$, we can then show that it holds for  $n+1$.
\begin{align*}
\frac{1}{2}  \cdot \frac{1}{3^{n}} < \frac{1}{3^n} = \frac{1}{3 \cdot 3^{n-1}} < \frac{1}{3n} = \frac{1}{n+n+n} < \frac{1}{n+1}  
\end{align*}
Thus by induction, $y_n < \frac{1}{n}$. Since $\frac{1}{n} \to 0$ as $n\to\infty$ and $y_n > 0$, we know  $y_n\to 0$ by the squeeze theorem. By the argument above, it immediately follows that $x_n$ is a Cauchy sequence and converges to some limit.

%%%%%%%%%%%%%%%%%%%%%%%%%%%%%%%%%%%%%%%%
\begin{exercise}{1.4.2}
	Show that the sequence $x_n = e^{\sin(5n)}$ has a convergent subsequence.
\end{exercise}
The $\sin$ function is defined on the codomain $[-1, 1]$, so it is bounded by definition. Since $e^x$ is monotonically increasing in $x$ ($x \leq y \implies e^x \leq e^y$), it must be true that $\frac{1}{e} \leq e^x \leq e$ when $x \in [-1, 1]$. This means that the entire sequence $x_n = e^{sin(5n)}$ is bounded. Then by the Bolzano-Weierstrass property, this sequence is guaranteed to have a convergent subsequence.

%%%%%%%%%%%%%%%%%%%%%%%%%%%%%%%%%%%%%%%%
\begin{exercise}{1.4.3}
	Find a bounded sequence with three subsequences converging to three different numbers.
\end{exercise}
Let the sequence $\left\{ x_n \right\}$ be defined by the pattern $\left\{ 1, 2, 3, 1, 2, 3, \dots  \right\}$. More formally, this sequence can be defined by
\[
	x_n = \begin{cases}
		1 & x = 3n + 1 \text{ for some } n \in \mathbb{N} \\
		2 & x = 3n + 2 \text{ for some } n \in \mathbb{N} \\
		3 & x = 3n \text{ for some } n \in \mathbb{N}
	\end{cases}
\]
This sequence is clearly bounded above by 3 and bounded below by 1. We can define a subsequence of this sequence with the map $\phi_p(n) = 3n + p$, where $n \in \mathbb{N}, p \in \mathbb{N}_0$.

Now take the three subsequences $\left\{ x_{\phi_0(n)} \right\}_{n=1}^\infty = \left\{ 1, 1, \dots \right\}$, $\left\{ x_{\phi_1(n)} \right\}_{n=1}^\infty = \left\{ 2, 2, \dots \right\}$, and $\left\{ x_{\phi_2(n)} \right\}_{n=1}^\infty = \left\{ 3, 3, \dots \right\}$. These clearly converge to 1, 2, and 3, respectively.

%%%%%%%%%%%%%%%%%%%%%%%%%%%%%%%%%%%%%%%%
\begin{exercise}{1.4.4}
	Let $x_n$ be a Cauchy sequence. Suppose that for every $\varepsilon>0$, there is some $n>1/\varepsilon$ such that $|x_n|<\varepsilon$. Prove that $x_n\to 0$.
\end{exercise}
Let $\varepsilon>0$, then by assumption there exists some $N \in \mathbb{N}$ such that $|x_j - x_{j+k}| < \frac{\varepsilon}{2} $ for all $j > N, k > 0$. There are two possible relationships between $\varepsilon$ and $N$. The first is $\frac{1}{\varepsilon} \geq N$, and the second is $\frac{1}{\varepsilon} < N$. In the first case, set an additional variable $m$ equal to $1$. In the second case, find $m$ such that $\frac{2m}{\varepsilon} > N$. The existence of such an $m$ is guaranteed by the definition of an Archimedean ordered field ($\mathbb{R}$ is such a field since it is complete).

Using this value $\frac{2m}{\varepsilon} > N$, we can guarantee two things:
\begin{enumerate}
	\item By assumption, there exists $n > \frac{2m}{\varepsilon}$ such that $|x_n| < \frac{\varepsilon}{2m} $.
	\item Since this $n$ satisfies $n > N$, it is true that $|x_n - x_{n+k}| < \frac{\varepsilon}{2}$ for any natural number $k > 0$.
\end{enumerate}
With these two statements, we can bound the distance of $x_{n+k}$ from $0$ for any $k$.
\begin{align*}
|x_{n+k}| &= |x_k + x_{n+k} - x_k| \\
	  &\leq |x_n| + |x_{n+k} - x_k| \\
	  &< \frac{\varepsilon}{2m} + \frac{\varepsilon}{2} \\
	  &\leq \frac{\varepsilon}{2} + \frac{\varepsilon}{2} \\
	  &= \varepsilon
\end{align*}
Since the choice of $k$ was arbitrary, this holds for any point in the sequence after $x_n$. Thus given $\varepsilon > 0$, we can find an $n$ such that for all $\tilde{n}>n$, $|x_{\tilde{n}}-0| = |x_{\tilde{n}}| < \varepsilon$. By the definition of convergence, $x_n$ then converges to 0.

\end{document}
