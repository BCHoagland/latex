\documentclass[10pt]{amsart}
\usepackage{latexsym} 
\usepackage{amscd,amsthm,amssymb,amsfonts,amsmath}
\usepackage{xcolor}
%\usepackage{epsfig}
%\usepackage{graphicx}
%\usepackage[dvips]{graphicx}

\usepackage[matrix,tips,graph,curve]{xy}

\newcommand{\mnote}[1]{${}^*$\marginpar{\footnotesize ${}^*$#1}}
\linespread{1.065}
% \setlength{\parskip}{1em}

\makeatletter

\setlength\@tempdima  {5.5in}
\addtolength\@tempdima {-\textwidth}
\addtolength\hoffset{-0.5\@tempdima}
\setlength{\textwidth}{5.5in}
\setlength{\textheight}{8.75in}
\addtolength\voffset{-0.625in}

\makeatother

\makeatletter 
\@addtoreset{equation}{section}
\makeatother


\renewcommand{\theequation}{\thesection.\arabic{equation}}

\theoremstyle{plain}
\newtheorem{theorem}[equation]{Theorem}
\newtheorem{corollary}[equation]{Corollary}
\newtheorem{lemma}[equation]{Lemma}
\newtheorem{proposition}[equation]{Proposition}
\newtheorem{conjecture}[equation]{Conjecture}
\newtheorem{fact}[equation]{Fact}
\newtheorem{facts}[equation]{Facts}
\newtheorem*{theoremA}{Theorem A}
\newtheorem*{theoremB}{Theorem B}
\newtheorem*{theoremC}{Theorem C}
\newtheorem*{theoremD}{Theorem D}
\newtheorem*{theoremE}{Theorem E}
\newtheorem*{theoremF}{Theorem F}
\newtheorem*{theoremG}{Theorem G}
\newtheorem*{theoremH}{Theorem H}

\newtheorem{manualtheoreminner}{Exercise}
\newenvironment{exercise}[1]{%
  \renewcommand\themanualtheoreminner{#1}%
  \manualtheoreminner
}{\endmanualtheoreminner}

\theoremstyle{definition}
\newtheorem{definition}[equation]{Definition}
\newtheorem{definitions}[equation]{Definitions}
%\theoremstyle{remark}

\newtheorem{remark}[equation]{Remark}
\newtheorem{remarks}[equation]{Remarks}
\newtheorem{example}[equation]{Example}
\newtheorem{examples}[equation]{Examples}
\newtheorem{notation}[equation]{Notation}
\newtheorem{question}[equation]{Question}
\newtheorem{assumption}[equation]{Assumption}
\newtheorem*{claim}{Claim}
\newtheorem{answer}[equation]{Answer}

%% my definitions %%%

\newcommand{\End}{\mathrm{End}}
\newcommand{\tr}{\mathrm{tr}}
%\newcommand{\ind}{\mathrm{ind}}

\renewcommand{\index}{\mathrm{index \,}}
\newcommand{\Hom}{\mathrm{Hom}}
\newcommand{\Aut}{\mathrm{Aut}}
\newcommand{\Trace}{\mathrm{Trace}\,}
\newcommand{\Res}{\mathrm{Res}\,}
\newcommand{\rank}{\mathrm{rank}}
%\renewcommand{\dim}{\mathrm{dim}}

\renewcommand{\deg}{\mathrm{deg}}
\newcommand{\spin}{\rm Spin}
\newcommand{\Spin}{\rm Spin}
\newcommand{\erfc}{\rm erfc\,}
\newcommand{\sgn}{\rm sgn\,}
\newcommand{\Spec}{\rm Spec\,}
\newcommand{\diag}{\rm diag\,}
\newcommand{\Fix}{\mathrm{Fix}}
\newcommand{\Ker}{\mathrm{Ker \,}}
\newcommand{\Coker}{\mathrm{Coker \,}}
\newcommand{\Sym}{\mathrm{Sym \,}}
\newcommand{\Hess}{\mathrm{Hess \,}}
\newcommand{\grad}{\mathrm{grad \,}}
\newcommand{\Center}{\mathrm{Center}}
\newcommand{\Lie}{\mathrm{Lie}}


\newcommand{\ch}{\rm ch} % Chern Character

\newcommand{\rk}{\rm rk} 
%\renewcommand{\c}{\rm c}  % Chern class

\newcommand{\sign}{\rm sign}
\renewcommand\dim{{\rm dim\,}}
\renewcommand\det{{\rm det\,}}
\newcommand{\dimKrull}{{\rm Krulldim\,}}
\newcommand\Rep{\mathrm{Rep}}
\newcommand\Hilb{\mathrm{Hilb}}
\newcommand\vol{\mathrm{vol}}

\newcommand\QED{\hfill $\Box$} %{\bf QED}} 

\newcommand\Pf{\nonintend{\em Proof. }}


\newcommand\reals{{\mathbb R}} 
\newcommand\complexes{{\mathbb C}}
\renewcommand\i{\sqrt{-1}}
\renewcommand\Re{\mathrm Re}
\renewcommand\Im{\mathrm Im}
\newcommand\integers{{\mathbb Z}}
\newcommand\quaternions{{\mathbb H}}


\newcommand\iso{{\cong}} 
\newcommand\tensor{{\otimes}}
\newcommand\Tensor{{\bigotimes}} 
\newcommand\union{\bigcup} 
\newcommand\onehalf{\frac{1}{2}}
%\newcommand\Sym[1]{{Sym^{#1}(\complexes^2)}}

\newcommand\lie[1]{{\mathfrak #1}} 
\newcommand\smooth{\mathcal{C}^{\infty}}
\newcommand\trivial{{\mathbb I}}
\newcommand\widebar{\overline}

%%%%%Delimiters%%%%

\newcommand{\<}{\langle}
\renewcommand{\>}{\rangle}

%\renewcommand{\(}{\left(}
%\renewcommand{\)}{\right)}


%%%% Different kind of derivatives %%%%%

\newcommand{\delbar}{\bar{\partial}}
\newcommand{\pdu}{\frac{\partial}{\partial u}}
%\newcommand{\pd}[1][2]{\frac{\partial #1}{\partial #2}}

%%%%% Arrows %%%%%
%\renewcommand{\ra}{\rightarrow}                   % right arrow
%\newcommand{\lra}{\longrightarrow}              % long right arrow
%\renewcommand{\la}{\leftarrow}                    % left arrow
%\newcommand{\lla}{\longleftarrow}               % long left arrow
%\newcommand{\ua}{\uparrow}                     % long up arrow
%\newcommand{\na}{\nearrow}                      %  NE arrow
%\newcommand{\llra}[1]{\stackrel{#1}{\lra}}      % labeled long right arrow
%\newcommand{\llla}[1]{\stackrel{#1}{\lla}}      % labeled long left arrow
%\newcommand{\lua}[1]{\stackrel{#1}{\ua}}      % labeled  up arrow
%\newcommand{\lna}[1]{\stackrel{#1}{\na}}      % labeled long NE arrow

\newcommand{\into}{\hookrightarrow}
\newcommand{\tto}{\longmapsto}
\def\llra{\longleftrightarrow}

\def\d/{/\mspace{-6.0mu}/}
\newcommand{\git}[3]{#1\d/_{\mspace{-4.0mu}#2}#3}
\newcommand\zetahilb{\zeta_{{\text{Hilb}}}}
\def\Fy{\sF \mspace{-3.0mu} \cdot \mspace{-3.0mu} y}
\def\tv{\tilde{v}}
\def\tw{\tilde{w}}
\def\wt{\widetilde}
\def\wtilde{\widetilde}
\def\what{\widehat}

%%%%%%%%%%%%%%%%%%% Mark's definitions %%%%%%%%%%%%%%%%%%%%

\newcommand{\frakg}{\mbox{\frakturfont g}}
\newcommand{\frakk}{\mbox{\frakturfont k}}
\newcommand{\frakp}{\mbox{\frakturfont p}}
\newcommand{\q}{\mbox{\frakturfont q}}
\newcommand{\frakn}{\mbox{\frakturfont n}}
\newcommand{\frakv}{\mbox{\frakturfont v}}
\newcommand{\fraku}{\mbox{\frakturfont u}}
\newcommand{\frakh}{\mbox{\frakturfont h}}
\newcommand{\frakm}{\mbox{\frakturfont m}}
\newcommand{\frakt}{\mbox{\frakturfont t}}
\newcommand{\G}{\Gamma}
\newcommand{\g}{\gamma}
\newcommand{\fraka}{\mbox{\frakturfont a}}
\newcommand{\db}{\bar{\partial}}
\newcommand{\dbs}{\bar{\partial}^*}
\newcommand{\p}{\partial}

%%%%%%%%%%%%% new definitions for the positive mass paper %%%%%%%%%

\newcommand{\sperp}{{\scriptscriptstyle \perp}}

%%%%%%%%%%%%%%%%%%%%%%%

%%%%%%%%%%%%%%%%%%%%%%%%%%%%%%%%%%%%%%%%%%%%%



%
\begin{document}
%

\title{Math 531 Homework 3}
\author{Braden Hoagland}


\date{\today}

\maketitle

%\setcounter{secnumdepth}{1} 


\begin{exercise}{1.7}
	For nonempty sets $A,B \subset \mathbb{R}$, let $a+B=\left\{ x+y \;|\; x\in A, y\in B \right\}$. Show that $\sup(A+B) = \sup(A) + \sup(B)$.
\end{exercise}
Let $L_A \doteq \sup A$ and $L_B \doteq \sup B$. For any $a \in A$, we have $a+b \leq L_A + L_B$, which implies that $L_A + L_B$ is an upper bound of $A+B$. By definition, $\sup(A+B) \leq L_A + L_B$.

Now fix $\varepsilon > 0$, then $L_A$ is a least upper bound of $A$, we can find $a \in A$ such that $a > L_A - \varepsilon/2$. Similarly, we can find $b \in B$ such that $b>L_B - \varepsilon/2$. Then $a + b > L_A + L_B - \varepsilon$.

Suppose $a'+b' < L_A + L_B$ is an upper bound of $A+B$, then $a+b \leq a'+b' < L_A + L_B - \varepsilon'$ for all $a\in A, b\in B$ and for some $\varepsilon' > 0$. But we just showed that for any $\varepsilon' > 0$, we can always find $a$ and $b$ such that $a+b > L_A + L_B - \varepsilon'$. Thus by contradiction, nothing strictly less than $L_A + L_B$ can be an upper bound of $A+B$.

This allows us to conclude that $L_A + L_B$ is the least upper bound of $A+B$, also written $\sup(A+B) = \sup A + \sup B$.


\begin{exercise}{1.8}
	For nonempty sets $A,B \subset \mathbb{R}$, determine which of the following statements are true. Prove the true statements and give a counterexample for those that are false:
	\begin{enumerate}
		\item $\sup(A \cap B) \leq \inf\left\{ \sup(A), \sup(B) \right\}$
		\item $\sup(A \cap B) = \inf\left\{ \sup(A), \sup(B) \right\}$
		\item $\sup(A \cup B) \geq \sup\{\sup(A), \sup(B)\}$
		\item $\sup(A \cup B) = \sup\{\sup(A), \sup(B)\}$
	\end{enumerate}
\end{exercise}
\begin{enumerate}
	\item
		False. Let $A$ and $B$ be disjoint sets, then $A \cap B = \varnothing$ and $\sup(A \cap B) = \infty$. The infimum of a set is defined to be either finite or $-\infty$, so the inequality fails.
	\item
		False, by the same counterexample as above.
	\item
		True, which we show by evaluating two possible cases.
		
		\textbf{Case 1:} \textit{$A$ and $B$ are both bounded above.} Without loss of generality, let $\sup A \leq \sup B < \infty$ then $\sup\left\{ \sup A, \sup B \right\}=\sup B$. Let $c \in A \cup B$. If $c \in A$, then $c \leq \sup A \leq \sup B$, and if $c \in B$, then $c \leq \sup B$. Thus $\sup B$ is clearly an upper bound of any element of $A \cup B$. We must now show that it is the least upper bound.

		For any $L < \sup B$, we can find $b \in B$ such that $b > L$ (by the definition of the supremum). Since $b \in B \implies b \in A \cup B$, $L$ cannot be an upper bound of $A \cup B$. Thus $\sup(A \cup B) = \sup\left\{ \sup A, \sup B \right\} = \sup B$.

		\textbf{Case 2:} \textit{Either $A$ or $B$ is unbounded above.} Then $A \cup B$ is unbounded above, which implies that $\sup(A \cup B) = \infty$. Since $\sup\left\{ \sup A, \sup B \right\}$ is clearly also $\infty$, we have $\sup(A \cup B) = \sup \left\{ \sup A, \sup B \right\}$.

		We have proven a stronger condition that what was required.
	\item
		True, by the same proof as above.
\end{enumerate}

\begin{exercise}{1.10}
	Verify that the bounded metric is indeed a metric.
\end{exercise}
We must verify the four properties of a metric.
\begin{enumerate}
	\item \textbf{Non-negativity:} Here we can use the fact that $d(x,y) \geq 0 \implies 1+d(x,y) > 0$. This allows us to form the fraction $d(x,y) / (1 + d(x,y))$ with no possibility of division by zero, and it also allows us to multiply both sides of an inequality by $1 + d(x,y)$ without reversing the inequality.
	\[
		\rho(x,y) \geq 0 \iff \frac{d(x,y)}{1 + d(x,y)} \geq 0 \iff d(x,y) \geq 0
	\]

	\item \textbf{Zero distance implies equal points:} We can use the same property as before.
	\[
		\rho(x,y) = 0 \iff \frac{d(x,y)}{1 + d(x,y)} = 0 \iff d(x,y) = 0 \iff x=y
	\]

	\item \textbf{Symmetry:}
	\[
		\rho(x,y) = \frac{d(x,y)}{1+d(x,y)} = \frac{d(y,x)}{1+d(y,x)} = \rho(y,x)
	\] 

	\item \textbf{Triangle Inequality:}
	\begin{align*}
		\rho(x,y) &= \frac{d(x,y)}{1+d(x,y)} \\
			  &= 1 - \frac{1}{1+d(x,y)} \\
			  \intertext{Using the triangle inequality for the metric $d$, $d(x,y) \leq d(x,z) + d(z,y)$, we can turn this into the inequality} \\
			  &\leq 1 - \frac{1}{1 + d(x,z) + d(z,y)} \\
			  &= \frac{d(x,z)}{1+d(x,z)+d(z,y)} + \frac{d(z,y)}{1+d(x,z)+d(z,y)} \\
			  &\leq \frac{d(x,z)}{1+d(x,z)} + \frac{d(z,y)}{1+d(z,y)} \\
			  &= \rho(x,z) + \rho(z,y)
	\end{align*}
\end{enumerate}
Since all four properties are satisfied, $\rho$ is a metric.

\begin{exercise}{1.12}
	In an inner product space show that
	\begin{enumerate}
		\item $2\Vert{x}\Vert^2+2\Vert{y}\Vert^2=\Vert{x+y}\Vert^2+\Vert{x-y}\Vert^2$ (parallelogram law)
		\item $\Vert{x+y}\Vert \Vert{x-y}\Vert\leq\Vert{x}\Vert^2+\Vert{y}\Vert^2$
		\item $4\left\langle x,y \right\rangle=\Vert{x+y}\Vert^2-\Vert{x-y}\Vert^2$ (polarization identity)
	\end{enumerate}
	Interpret these results geometrically in terms of the parallelogram formed by $x$ and $y$.
\end{exercise}
	These identities will rely on the same expansions of $\Vert{x+y}\Vert^2$ and $\Vert{x-y}\Vert^2$, namely
	\begin{align*}
		\Vert{x+y}\Vert^2 = \left\langle x+y,x+y \right\rangle &= \Vert{x}\Vert^2 + 2\left\langle x,y \right\rangle+\Vert{y}\Vert^2 \\
		\Vert{x-y}\Vert^2 = \left\langle x-y,x-y \right\rangle &= \Vert{x}\Vert^2 -2\left\langle x,y \right\rangle + \Vert{y}\Vert^2
	\end{align*}
\begin{enumerate}
	\item We can add these two identities to get the desired result
	\[
	\Vert{x+y}\Vert^2 + \Vert{x-y}\Vert^2 = 2\Vert{x}\Vert^2 + 2\Vert{y}\Vert^2
	\] 
	Geometrically, this can be interpreted as the sum of the squared edge lengths of the parallelogram (formed by $x$ and $y$) being equal to the sum of the squared lengths of the diagonals.

	\item 
	We can show this by expanding the norms of $x+y$ and $x-y$ and then simplifying.
	\begin{align*}
		\Vert{x+y}\Vert\Vert{x-y}\Vert &= \left[ \left( \Vert{x}\Vert^2 + 2\left\langle x,y \right\rangle + \Vert{y}\Vert^2 \right)\left( \Vert{x}\Vert^2 -2\left\langle x,y \right\rangle+\Vert{y}\Vert^2 \right) \right]^{1/2} \\
					       &= \left[ \Vert{x}\Vert^4 +2\Vert{x}\Vert^2\Vert{y}\Vert^2 - 4\left\langle x,y \right\rangle^2 + \Vert{y}\Vert^4 \right]^{1/2} \\
					       &\leq \left[ \Vert{x}\Vert^4 +2\Vert{x}\Vert^2\Vert{y}\Vert^2 + \Vert{y}\Vert^4 \right]^{1/2} \\
					       &= \Vert{x}\Vert^2 + \Vert{y}\Vert^2
	\end{align*}
	Geometrically, the product of the lenghts of the diagonals of the parallelogram formed by $x$ and $y$ is equal to the sum of the squared lengths of $x$ and $y$.

	\item 
	Subtracting the second of our two original expansions gives the desired result
	\[
	\Vert{x+y}\Vert^2 - \Vert{x-y}\Vert^2 = 4\left\langle x,y \right\rangle
	\] 
	Geometrically, the difference in the lengths of the diagonals of the parallelogram formed by $x$ and $y$ is equal to 4 times the inner product of $x$ and $y$.
\end{enumerate}

\begin{exercise}{1.15}
	Let $\left\{ x_n \right\}$ be a sequence in $\mathbb{R}$ such taht $d(x_n, x_{n+1}) \leq d(x_{n-1}, x_n)/2$. Show that $\left\{ x_n \right\}$ is a Cauchy sequence.
\end{exercise}
We can show that $\left\{ x_n \right\}$ is a Cauchy sequence by first unraveling the recursion and then finding a useful bound on the distance between a point in the sequence and any future point in the sequence. We start by proving the lemma
\begin{lemma}
	If $d(x_1,y_2) = \ell/2$, then $d(x_n, x_{n+1}) \leq \frac{\ell}{2^{n} } $.
\end{lemma}
\begin{proof}
	We can prove this inductively. The base case is trivial since $\ell/2 \leq \ell/2$. Assuming the hypothesis holds for $d(x_n, x_{n+1})$, we can show it also holds for $d(x_{n+1}, x_{n+2})$.
	\[
		d(x_{n+1}, x_{n+2}) \leq \frac{d(x_{n}, x_{n+1})}{2} \leq \frac{\ell / 2^{n}}{2} = \frac{\ell}{2^{n+1}}.
	\] 
Thus the inequality holds for all $n \geq 1$.
\end{proof}

With this inequality in place, we can unravel the recursion of the distance between a point $x_n$ and $x_{n+k}$.
\begin{align*}
	d(x_{n}, x_{n+k}) &\leq d(x_n, x_{x+1}) + d(x_{n+1}, x_{n+2}) + \cdots + d(x_{n+k-1}, x_{n+k}) \\
			  &\leq \frac{\ell}{2^{n}} + \frac{\ell}{2^{n+1}} + \cdots + \frac{\ell}{2^{n+k-1}} \\
			  &= \ell \left[ \frac{1}{2^{n}} + \frac{1}{2^{n+1}} + \cdots + \frac{1}{2^{n+k-1}} \right]
\end{align*} 

\begin{lemma}
	$\frac{1}{2^n} + \cdots + \frac{1}{2^{n+k-1}} \leq \frac{1}{2^{n-1}} $
\end{lemma}
\begin{proof}
	We first show that the sequence $s_k \doteq \sum_{i=1}^{k} \frac{1}{2^i} \leq 1$, and we will then extend it to the more general case that is desired. Fix $k$, then we have
	\[
		2 s_k = 2 \sum_{i=1}^{k} \frac{1}{2^i} = \sum_{i=0}^{k-1} \frac{1}{2^i} = 1 + s_k - \frac{1}{2^k} 
	\]
	Slight re-arranging then gives $s_k = 1 - 1/2^k $. Since $1/2^k$ is positive for any $k \geq 0$, $s_k \leq 1$ for any $k \geq 0$. The extension to the more general case is straightforward. Note that
	\[
	\frac{1}{2^n}  + \cdots \frac{1}{2^{n+k-1}} \leq \frac{1}{2^{n-1}} \iff \frac{1}{2}  + \cdots \frac{1}{2^{k}} \leq 1
	\] 
	Since we have already proven the latter inequality, we know that the desired general inequality holds.
\end{proof}

This inequality shows that $d(x_{n}, x_{n+k}) \leq \frac{\ell}{2^{n-1}} $.
Since this does not depend on $ k$, this holds for any $m \geq n$. Then for any given $\varepsilon > 0$, choose $N$ such that $\frac{\ell}{2^{N-1}} < \varepsilon$, then for all $n,m > N$, we have $d(x_n, x_m) < \frac{\ell}{2^{N-1}} < \varepsilon$. This means the sequence $\left\{ x_n \right\}$ is a Cauchy sequence.

\begin{exercise}{1.31}
	Let $A,B \subset \mathbb{R}$ and $f:a\times B \to \mathbb{R}$ be bounded. Is it true that
	\[
		\sup_{(x,y) \in A \times B} f(x,y) = \sup_{y\in B} \left( \sup_{x \in A}f(x,y) \right)
	\] 
\end{exercise}
Let $L = \sup_{(x,y) \in A \times B} f(x,y)$. Then by the definition of the supremum, for fixed $y'$, $\sup_x f(x,y') \leq L$. Now since $f(x',y') \leq L$ for all $x' \in A, y' \in B$, taking the supremum over $B$ of $\sup_x f(x,y')$ does not affect the inequality.
\[
	\sup_y \left( \sup_x f(x,y) \right) \leq L = \sup_{x,y} f(x,y)
\] 
Now we show that flipping the inequality is still valid. For every $x' \in A, y' \in B$, we know $f(x',y') \leq \sup_y f(x',y)$. Thus for $(\tilde{x}, \tilde{y})$ in the pre-image of $L$ under $f$,
\[
	f(\tilde{x},\tilde{y}) = L \leq \sup_x f(x,\tilde{y}) \leq \sup_y \left( \sup_x f(x,y) \right)
\] 
Based on these two inequalities, it must be the case that
\[
	\sup_{x,y} f(x,y) = \sup_y \left( \sup_x f(x,y) \right)
\] 
which is the desired result.

\begin{exercise}{2.1.2}
	Let $S=\left\{ (x,y) \in \mathbb{R}^2 \;|\; xy > 1 \right\}$. Show that $S$ is open.
\end{exercise}
The set $S$ can be written $S = S_+ \cup S_-$, where $S_+ = \left\{ (x,y) \;|\; xy>1, x,y>0 \right\}$ and $S_- = \left\{ (x,y) \;|\; xy>1, x,y<0 \right\}$. We will show that $S_+$ is an open set, then a similar argument will show that $S_-$ is also open. Since the union of a finite number of open sets is also open, it must then be the case that $S$ is open.

To find the greatest possible radius of an open disk around an arbitrary point $(x,y) \in S_+$, we can find the roots of $(x-\varepsilon)(y-\varepsilon) -1 = 0$. The smaller of the two possible roots is
\[
\varepsilon = \frac{x+y-\sqrt{x^2 -2xy+y^2+4}}{2} 
\]
We claim that $\varepsilon>0$ and $x-\varepsilon,y-\varepsilon>0$. Since
\[
	\sqrt{x^2 - 2xy + y^2 + 4} > \sqrt{(x-y)^2} = |x-y|
\]
we cound bound $x-\varepsilon$.
\begin{align*}
	x-\varepsilon &= x - \frac{x+y-\sqrt{x^2 -2xy+y^2+4}}{2} \\
		      &= \frac{x-y+\sqrt{x^2 -2xy+y^2+4}}{2} \\
		      &> \frac{x-y+|x-y|}{2} \\
		      &\geq 0
\end{align*}
A similar argument holds for $y-\varepsilon$, so we know $x-\varepsilon, y-\varepsilon>0$. Since $x$ and $y$ are known to be positive, $\varepsilon$ is as well.

We now claim that all points in $D( (x,y), \varepsilon)$ are also in $S_+$. To show this, we can take any point in the $\varepsilon$ ball $(\tilde{x},\tilde{y})$. Since by the definition of the ball, $(x-\tilde{x})^2 + (y-\tilde{y})^2 < \varepsilon^2$, we have
\begin{align*}
	(x-\tilde{x})^2 + (y-\tilde{y})^2 \leq (x-\tilde{x})^2 < \varepsilon^2 &\implies |{x-\tilde{x}}| < \varepsilon \\
									       &\implies x-\varepsilon < \tilde{x} < x+\varepsilon
\end{align*}
Similarly, $y-\varepsilon < \tilde{y} < y+\varepsilon$. From this it immediately follows that
\[
	\tilde{x}\tilde{y} > (x-\varepsilon)(y-\varepsilon) = 1
\]
Since we already showed $\tilde{x},\tilde{y}>0$, it must be the case that $(\tilde{x},\tilde{y}) \in S_+$. Similarly, $S_-$ is open. Then $S = S_+ \cup S_-$ must be open.

\begin{exercise}{2.1.4}
	Let $B \subset \mathbb{R}^n$ be any set. Define $C=\left\{ x\in \mathbb{R}^n \;|\; d(x,y) < 1 \text{ for some } y \in B \right\}$. Show that $C$ is open.
\end{exercise}
Let $D_y = \left\{ x \in \mathbb{R}^n \;|\; d(x,y) < 1 \right\}$, then $C = \cup_{y \in B} D_y$. It is known that the $\varepsilon$-ball $D(z,\varepsilon)$ is open for any $z$ in a metric space $M$. Letting $M = \mathbb{R}^n$, we have $D_y = D(y,1)$, so $D_y$ must be open. Then since the union of an arbitrary collection of open sets is open, it must also be the case that $C = \cup_{y \in B} D_y$ is open.


\begin{exercise}{2.2.3}
	If $A \subset B$, is $A^o \subset B^o$?
\end{exercise}
Let $a \in A^o$, then there exists an open set $U$ such that $a \in U \subset A \subset B$. Since $U$ lies entirely in $B$, this implies that $a \in B^o$ as well. Since $a$ was arbitrary, this implies $A^o \subset B^o$.


\end{document}

