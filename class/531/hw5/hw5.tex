\documentclass[10pt]{amsart}
\usepackage{latexsym} 
\usepackage{amscd,amsthm,amssymb,amsfonts,amsmath}
\usepackage{xcolor}
%\usepackage{epsfig}
%\usepackage{graphicx}
%\usepackage[dvips]{graphicx}

\usepackage[matrix,tips,graph,curve]{xy}

\newcommand{\mnote}[1]{${}^*$\marginpar{\footnotesize ${}^*$#1}}
\linespread{1.065}
% \setlength{\parskip}{1em}

\makeatletter

\setlength\@tempdima  {5.5in}
\addtolength\@tempdima {-\textwidth}
\addtolength\hoffset{-0.5\@tempdima}
\setlength{\textwidth}{5.5in}
\setlength{\textheight}{8.75in}
\addtolength\voffset{-0.625in}

\makeatother

\makeatletter 
\@addtoreset{equation}{section}
\makeatother


\renewcommand{\theequation}{\thesection.\arabic{equation}}

\theoremstyle{plain}
\newtheorem{theorem}[equation]{Theorem}
\newtheorem{corollary}[equation]{Corollary}
\newtheorem{lemma}[equation]{Lemma}
\newtheorem{proposition}[equation]{Proposition}
\newtheorem{conjecture}[equation]{Conjecture}
\newtheorem{fact}[equation]{Fact}
\newtheorem{facts}[equation]{Facts}

\newtheorem{manualtheoreminner}{Exercise}
\newenvironment{exercise}[1]{%
  \renewcommand\themanualtheoreminner{#1}%
  \manualtheoreminner
}{\endmanualtheoreminner}

\theoremstyle{definition}
\newtheorem{definition}[equation]{Definition}
\newtheorem{definitions}[equation]{Definitions}
%\theoremstyle{remark}

\newtheorem{remark}[equation]{Remark}
\newtheorem{remarks}[equation]{Remarks}
\newtheorem{example}[equation]{Example}
\newtheorem{examples}[equation]{Examples}
\newtheorem{notation}[equation]{Notation}
\newtheorem{question}[equation]{Question}
\newtheorem{assumption}[equation]{Assumption}
\newtheorem*{claim}{Claim}
\newtheorem{answer}[equation]{Answer}

\newcommand{\p}{\partial}

%%%%%%%%%%%%% new definitions for the positive mass paper %%%%%%%%%

\newcommand{\sperp}{{\scriptscriptstyle \perp}}

%%%%%%%%%%%%%%%%%%%%%%%%%%%%%%%%%%%%%%%%%%%%%


%
\begin{document}
%

\title{Math 531 Homework 5}
\author{Braden Hoagland}


\date{\today}

\maketitle

% ========================================
\begin{exercise}{3.1.5}
	Let $M$ be a set with the discrete metric. Show that any infinite subset of $M$ is noncompact. Why does this not contradict the statement in Execise 4?
	\hrulefill
\end{exercise}
Let $A$ be an infinite subset of discrete metric space $M$, then $U = \left\{ D(a, 1/2) \;|\; a \in A \right\}$ is clearly an infinite open cover of $A$. Now select arbitrary $a' \in A$ and remove its corresponding ball to yield $U' = \left\{ D(a,1/2) \;|\; a \in A, a \neq a' \right\}$. Then $U'$ does \textit{not} cover $A$ since by definition of the discrete metric, none of the balls in $U'$ cover $a'$. Since $a'$ was arbitrary, we cannot remove any of the open sets from $U$, meaning that we cannot find a finite subcover for $A$. Thus $A$ is noncompact.

Exercise 4 required that we find a convergent sequence $x_n \to x$. In a discrete metric space, the only convergent sequence is a sequence that eventually becomes constant. So even though the sequence has infinite terms, it is in fact still only a finite subset of $M$. Thus exercise 4 does not contradict the result of this problem.

% ========================================
\begin{exercise}{3.3.2}
	Is the nested set property true if ``compact nonempty" is replaced by ``open bounded nonempty"?
	\hrulefill
\end{exercise}
No. In $\mathbb{R}$, let $U_n = (0,1/n)$, then $\left\{ U_n \right\}_{n=1}^\infty$ is a sequence of open bounded nonempty decreasing sets. Assume $x \in \cap_{n=1}^\infty U_n$, then $0 < x < 1/n$ for all $n \in \mathbb{N}$. Since $\mathbb{R}$ is Archimedean, this is impossible, so no such $x$ exists and $\cap_{n=1}^\infty U_n = \varnothing$.

% ========================================
\begin{exercise}{3.3.4}
	Let $x_k \to x$ be a convergent sequence in a metric space. Let $\mathcal{A}$ be a family of closed sets with the property that for each $A \in \mathcal{A}$, there is an $N$ such that $k \geq N$ implies $x_k \in A$. Prove that $x \in \cap \mathcal{A}$.
	\hrulefill
\end{exercise}
Let $A \in \mathcal{A}$ be arbitrary. We know there exists some $N$ such that if $k \geq N$, then $x_k \in A$. Thus we have a convergent sequence $\left\{ x_k \right\}_{k=N}^\infty \subset A$. Since $A$ is closed, it contains its limit points, in this case $x$. Since $A$ was arbitrary, $x$ must lie in all $A$. Thus $x \in \cap \mathcal{A}$.

% ========================================
\begin{exercise}{3.5.2}
	Is $\left\{ (x,y) \in \mathbb{R}^2 \;|\; 0 \leq x \leq 1 \right\} \cup \left\{ (x,0) \;|\; 1<x<2 \right\}$ connected? Prove or disprove.
	\hrulefill
\end{exercise}
Let $A = \left\{ (x,y) \in \mathbb{R}^2 \;|\; 0 \leq x \leq 1 \right\}$ and $B = \left\{ (x,0) \;|\; 1<x<2 \right\}$ We will show that $A \cup B$ is connected by showing that it is path-connected. Take two points $x,y \in \mathbb{R}^2$. There are three cases we must consider when constructing a continuous path from $x$ to $y$ that lies in $A \cup B$.

\begin{enumerate}
	\item Assume $v,w \in A$. Now let $\varphi_1:[0,1]\to A$ be defined by $\varphi(t) = (w-v)t + v$. This is a continuous mapping since for any sequence $z_k$ and constant $\lambda$, $\lambda z_k \to \lambda z$ and $z_k + \lambda \to z + \lambda$, implying $\varphi_1(t_k) \to \varphi_1(t)$ if $t_k \to t$.

		Note that $\varphi_1(0) = v$ and $\varphi_1(1) = w$. Now let $x_v$ be the $x$-component of $v$ and $x_w$ be the $x$-component of $w$, then $x_{\varphi(t)} = (x_w - x_v)t + x_v = x_w t + (1-t) x_v$. We can show that this is in $A$ for any $t \in [0,1]$ since
		\begin{gather*}
			x_w t + (1-t) x_v \geq 1 - t \geq 0
		\end{gather*}
		and
		\begin{gather*}
			x_w t + (1-t) x_v \leq t + (1-t) = 1 \leq 1.
		\end{gather*}
		Thus if $v$ and $w$ are both in $A$, we can construct a continuous path between them that also lies in $A$.

	\item Assume $v,w \in B$. We can define $\varphi_2(t) = wt + (1-t)v$, the same as the previous case. Similarly, $\varphi_2$ is a continuous map from $v$ to $w$ that lies in $B$.

	\item Without loss of generality, assume $v \in A$ and $w \in B$. Let $z$ be the point $(1,0)$, then define $\varphi_3:[0,2]\to A \cup B$ by
		\[
			\varphi_2(t) =
			\begin{cases}
				(z-v) t + v & \text{ if } t \leq 1 \\
				(w-z)(t-1) + z & \text{ if } t \geq 1
			\end{cases}
		\] 
		Informally, this can be thought of as a concatenation of the two continuous maps presented earlier. It is then itself a continuous map from $v$ to $w$ that lies entirely in $A \cup B$.
\end{enumerate}
We have found continuous maps between any two points in $A \cup B$, so it is path-connected and, subsequently, connected.

% ========================================
\begin{exercise}{3.17}
	Let $K$ be a nonempty closed set in $\mathbb{R}^n$ and $x \in K^c$. Prove that there is a $y \in K$ such that $d(x,y) = \inf\left\{ d(x,z) \;|\; z\in K \right\}$. Is this true for open sets? Is it true in general metric spaces?
	\hrulefill
\end{exercise}
Let $L = \inf\left\{ d(x,z) \;|\; z\in K \right\}$. Consider $K' = K \cap \left\{ x' \;|\; d(x',x) \leq L + 1 \right\}$. $K'$ is the intersection of closed sets, so it is also closed. Moreover, it is bounded since the closed ball of radius $L+1$ around $x$ is clearly bounded. Since we are operating in $\mathbb{R}^n$, the Heine-Borel theorem shows that $K'$ is then compact. This will allow us to construct a sequence with a convergent subsequence.

Let $y_1 \in K'$ such that $d(y_1,x) > L$. If no such $y_1$ exists, then the problem is trivial since $K$ is nonempty, which implies that the only points of $K'$ are distance $L$ away from $x$. Now select $y_2 \in K'$ such that $d(y_1,x) > d(y_2,x) > L$. If no such $y_2$ exists, then the problem is once again trivial since $L$ being an infimum of the distances implies that there must exist some point $\tilde{y}$ satsifying $d(\tilde{y},x) = L$. Continuining in this manner and assuming we run into no trivial cases, construct the sequence $\left\{ y_k \right\}_{k=1}^\infty$ such that $d(y_k,x) > d(y_{k+1},x) > L$. Since $K'$ is compact, this sequence has a convergent subsequence $y_{\sigma(k)} \to y \in K'$.

We have in fact created two sequences: a sequence of points and a sequence of distances. Denote the latter by $\left\{ d(y_{\sigma(k)},x) \right\}_{k=1}^\infty$. Since this lies in $\mathbb{R}$ and is strictly decreasing and bounded below, it must converge. If it converges to any point other than $L$, we contradict the fact that $L$ is the infimum of the distances, so it must converge to $L$.

Since $y_{\sigma(k)} \to y$, for all $\varepsilon>0$ there exists $N_1$ such that if $k > N_1$, then $d(y_{\sigma(k)}, y) < \varepsilon$. Similarly, for all $\varepsilon>0$ there exists $N_2$ such that if $k > N_2$, then $|d(y_{\sigma(k)},x)-L| < \varepsilon$. Since $d(y_{\sigma(k)},x)>L$ by construction, this implies $d(y_{\sigma(k)},x)<L+\varepsilon$.

Now let $\varepsilon' > 0$, then we know that for $k$ large enough,
\begin{align*}
	d(y,x) &\leq d(y,y_k) + d(y_k,x) \\
	       &< \frac{\varepsilon'}{2} + L + \frac{\varepsilon'}{2} \\
	       &= L + \varepsilon'.
\end{align*}
Since $d(y,x)$ is strictly lower than any $L+\varepsilon$, it must be the case that $d(y,x) \leq L$. Since $L$ is the infimum of the distances, i.e. $d(y,x) \geq L$, this implies $d(y,x) = L$.

This would not work if $K$ had been open, as $y$ could have been an element of $K^c$ instead. This will hold in any metric space in which closed balls are compact, as this requirement is enough to find a convergent subsequence in $K'$.

% ========================================
\begin{exercise}{3.29}
	Let $A = \left\{ (x,y) \in \mathbb{R}^2 \;|\; x^4 + y^4 = 1 \right\}$. Show that $A$ is compact. Is it connected?
	\hrulefill
\end{exercise}
Since $A \subset \mathbb{R}^2$, by the Heine-Borel theorem it suffices to show that $A$ is closd and bounded.

\textbf{Closed:} Let $(x_n,y_n) \to (x,y)$ for $(x_n,y_n) \in A$. Since convergence in $\mathbb{R}^n$ implies pointwise convergence, we know $x_n \to x$ and $y_n \to y$. Then by limit arithmetic in $\mathbb{R}$, $1 = x_n^4 + y_n^4 = x^4 + y^4$. Thus $(x,y) \in A$, so $A$ is closed.

\textbf{Bounded:} If $|x| > 1$ or $|y| > 1$, then the sum $x^4 + y^4$ would be strictly greater than 1, since both $x^4$ and $y^4$ are non-negative. Thus any point in $A$ satisfies $|x|,|y| \leq 1$. The distance from any point in $A$ to the origin can then be bounded by
\[
	\Vert{(x,y)}\Vert = \sqrt{x^2+y^2} \leq \sqrt{1 + 1} = \sqrt{2},
\] 
so clearly $A \subset D(0, 2)$.

To show that $A$ is connected, it suffices to show that it is path-connected. We will do so by constructing two maps: one from $[0,2\pi]$ to an intermediate set, and a second from this intermediate set to $A$. Let $\tilde{A}= \left\{ (x,y) \;|\; x^2 + y^2 = 1 \right\}$ be the unit circle in $\mathbb{R}^2$, then define $\phi_1 :[0,2\pi] \to \tilde{A}$ and $\varphi_2:\tilde{A}\to A$ by
\begin{align*}
	\varphi_1(t) &= (\cos t, \sin t) \\
	\varphi_2( (x,y)) &= \frac{1}{\left( x^4+y^4 \right)^{1/4}} (x,y),
\end{align*}
then their composition $\varphi = \varphi_2 \circ \varphi_1$ is a continuous map $[0,2\pi] \to A$. To see that $\varphi_1$ indeed maps to $\tilde{A}$, note that $\cos^2 t + \sin^2 t = 1$ by a trigonometric identity. To see that $\varphi_2$ indeed maps to $A$, note that
\[
	\Big\Vert{\frac{(x,y)}{\Vert{(x,y)}\Vert_4} }\Big\Vert_4 = \frac{\Vert{(x,y)}\Vert_4}{\Vert{(x,y)}\Vert_4} = 1.
\] 
Given two points in $A$, we can then find a continuous map between them. Let $x,y \in A$, then there exist $a,b \in [0,2\pi]$ such that $\varphi(a)=x$ and $\varphi(b)=y$. Then define the continuous map $\varphi_3:[0,1]\to[0,2\pi]$ by
\[
	\varphi_3(t) = a + t(b-a).
\] 
Then $\varphi_2 \circ \varphi_1 \circ \varphi_3$ is a continuous path between $x$ and $y$ that lies entirely in $A$. Thus $A$ is path-connected and, subsequently, connected.

% ========================================
\begin{exercise}{3.33}
	A set $S$ in a metric space is called \textbf{nowhere dense} if for any nonempty open set $U$, we have $\overline{S}\cap U \neq U$, or equivalently, $(\overline{S})^o = \varnothing$. Show that $\mathbb{R}^n$ cannot be written as the countable union of nowhere dense sets.
	\hrulefill
\end{exercise}
Let $\mathbb{R}^n = \cup_{n=1}^\infty A_n$. Assume that each $A_n$ is a nowhere dense set in $\mathbb{R}^n$, then none of them contain nonempty open subsets. This means we can find a nonempty open subset in $A_1^c$, so let
\[
	D_1 \doteq D(x_1, \varepsilon_1) \subset A_1^c
\] 
for some $x_1 \in A_1^c$ and $0<\varepsilon_1<1$. Similarly, there must be a nonempty open subset in $D_1 \cap A_2^c$, so let
\[
	D_2 \doteq D(x_2, \varepsilon_2) \subset D_1 \cap A_2^c
\] 
for some $x_2 \in D_2 \cap A_2^c$. Inductively construct a sequence satisfying
\[
D_{k+1} \subset D_{n} \cap A_{n+1}^c, \quad \varepsilon_{n}<\frac{1}{2^n}. 
\] 
Then by construction, $\overline{D_1}\supset D_1 \supset \overline{D_2} \supset D_2 \supset \cdots$. Since we are working in $\mathbb{R}^n$ and each $\overline{D_k}$ is closed and bounded, they are compact. Then by the Nested Set Property, there exists some $x \in \mathbb{R}^n$ in the intersection $\cap_n \overline{D_n}$. Since $x$ lies in every $D_n$, $x \not\in A_n$ for any $n$. Thus $x \not\in \cup_n A_n$. Then by contradiction, we have that $\mathbb{R}^n$ cannot be constructed as the countable union of nowhere dense sets.

% ========================================
\begin{exercise}{3.34}
	Prove that any closed set $A \subset M$ is an intersection of a countable family of open sets.
	\hrulefill
\end{exercise}
Let $U_n = \cup_{a\in A}D(a,1/n)$ and let $\mathcal{U}=\left\{ U_n \;|\; n\in \mathbb{N} \right\}$, then we claim $\cap\;\mathcal{U} =A$. Clearly we have $A \subset \cap\;\mathcal{U}$ since by definition, every point of $A$ lies in every $U_n$.

We can prove the reverse inclusion by contrapositive. Let $x \not\in A$, then we must show $x \not\in \cap\;\mathcal{U}$. Since $x \not\in A$, we must have $x \in A^c$. Since $A$ is open, $A^c$ is closed, so there exists $\varepsilon>0$ such that $D(x,\varepsilon) \subset A^c$. This implies that there exists $n \in \mathbb{N}$ such that $D(x,1/n) \subset A^c$, meaning that $D(x,1/n) \cap A = \varnothing$. This implies that every point of $A$ is at least a distance of $1/n$ away from $x$, so $x \not\in D(a,1/n)$ for any $a \in A$. Thus $x \not\in \cup_{a\in A}D(a,1/n) = U_n$, so $x \not\in \cap\;\mathcal{U}$.

% ========================================
\begin{exercise}{3.36}
	Let $A \subset \mathbb{R}^n$ be uncountable. Prove that $A$ has an accumulation point.
	\hrulefill
\end{exercise}
We can first find a closed and bounded infinite subset of $\mathbb{R}^n$. Let $F_l^k$ be the set of all points in $A$ whose $k$-th coordinate lies in the closed interval $[l, l+1]$, then $\mathbb{R}^n = \cup_{k=1}^n \cup_{l\in \mathbb{Z}} F_l^k$. At least one $F_l^k$ must be uncountable, otherwise $\mathbb{R}^n$ would be countable. We can thus find a closed, bounded, uncountable set $F_l^k \subset \mathbb{R}^n$. For simplicity, denote this set $B$.

$B \subset \mathbb{R}^n$ is closed and bounded, so it is compact by the Heine-Borel theorem. By Bolzano-Weierstrass, it is sequentially compact. Thus every infinite sequence $\left\{ x_k \right\}\subset B$ has a subsequence $\left\{ x_{\sigma(k)} \right\}$ that converges to some point $b \in B$.

Since there are infinite points in $B$, we can form our sequence $\left\{ x_k \right\}$ using unique elements. By the definition of convergence, for every open neighborhood $U$ of $b$, there exists $\sigma(k)$ such that $x_{l}\in U$ when $l > \sigma(k)$. Since every element of our sequence is unique, this implies that at least one such $x_l$ is not equal to $b$. Thus $U \cap A\backslash \left\{ b \right\}$ is nonempty for every $U$, so $b$ is an accumulation point of $A$.

% ========================================
\begin{exercise}{3.37}
	Let $A,B \subset M$ with $A$ compact, $B$ closed, and $A \cap B = \varnothing$.
	\begin{enumerate}
		\item Show that there is an $\varepsilon>0$ such that $d(x,y) > \varepsilon$ for all $x \in A$ and $y \in B$.
		\item Is this true if $A$ and $B$ are merely closed?
	\end{enumerate}
	\hrulefill
\end{exercise}
\begin{enumerate}
	\item 
		We first state two facts that we will use to prove this statement.
		\begin{enumerate}
			\item[(*)] Since $A \cap B = \varnothing$, $A$ must lie in $B^c$, which is open since $B$ is closed. Then for each $a \in A$, there exists $\delta_a$ such that $D(a,\delta_a) \subset B^c$. Thus for any $a\in A$, $d(a, b) \geq \delta_a$ for any $b \in B$.
			\item[(**)] Additionally, we can take the open cover $\left\{ D(a, \delta_a/2) \right\}$ of $A$ and use the compactness of $A$ to find a finite open subcover $\left\{ D(a_k, \delta_{a_k}/2) \right\}_{k=1}^N$. By definition, any element in one of the balls in the subcover is at most $\delta_a/2$ away from its corresponding $a_k$.
		\end{enumerate}

Let $a \in A$, then $a$ lies in at least one of the sets in the finite subcover, i.e. $a \in D(a_k, \delta_{a_k}/2)$ for some $k \in \left\{ 1,\dots,N \right\}$. Then by the triangle inequality we have
\begin{align*}
	d(a_k,b) &\leq d(a,b) + d(a_k,a) \\
	d(a,b) &\geq d(a_k, b) - d(a_k,a)
	\intertext{Using facts $(*)$ and $(**)$ gives}
		&> \delta_{a_k}-\delta_{a_k}/2 \\
		&= \delta_{a_k}/2
\end{align*}
Set $\varepsilon=\min\left\{ \delta_{a_1}/2, \dots, \delta_{a_N}/2 \right\}$, then $d(a,b) > \varepsilon$ for all $a\in A$ and $b \in B$.

	\item 
		No. Let $A = \left\{ (x,0) \;|\; x\in \mathbb{R} \right\}$ and $B = \left\{ (x, 1/x) \;|\; x\in \mathbb{R}, x > 0 \right\}$ be closed subsets of $\mathbb{R}^2$. Fix $x > 0$, then consider $a=(x,0) \in A$ and $b = (x,1/x) \in B$. The distance between them is $d(a,b) = |1/x| = 1/x $ under the usual metric. Assume satisfactory $\varepsilon>0$ exists, then $1/x > \varepsilon$ for all $x$. Since $\mathbb{R}$ is archimedean, however, we can find $x$ such that $1/x < \varepsilon$ for any $\varepsilon$. Thus $A$ and $B$ just being closed is not enough to get the same result.

\end{enumerate}


\end{document}


