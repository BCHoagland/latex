\documentclass[10pt]{amsart}
\usepackage{latexsym} 
\usepackage{amscd,amsthm,amssymb,amsfonts,amsmath}
\usepackage{xcolor}
%\usepackage{epsfig}
%\usepackage{graphicx}
%\usepackage[dvips]{graphicx}

\usepackage[matrix,tips,graph,curve]{xy}

\newcommand{\mnote}[1]{${}^*$\marginpar{\footnotesize ${}^*$#1}}
\linespread{1.065}

\makeatletter

\setlength\@tempdima  {5.5in}
\addtolength\@tempdima {-\textwidth}
\addtolength\hoffset{-0.5\@tempdima}
\setlength{\textwidth}{5.5in}
\setlength{\textheight}{8.75in}
\addtolength\voffset{-0.625in}

\makeatother

\makeatletter 
\@addtoreset{equation}{section}
\makeatother


\renewcommand{\theequation}{\thesection.\arabic{equation}}

\theoremstyle{plain}
\newtheorem{theorem}[equation]{Theorem}
\newtheorem{corollary}[equation]{Corollary}
\newtheorem{lemma}[equation]{Lemma}
\newtheorem{proposition}[equation]{Proposition}
\newtheorem{conjecture}[equation]{Conjecture}
\newtheorem{fact}[equation]{Fact}
\newtheorem{facts}[equation]{Facts}
\newtheorem*{theoremA}{Theorem A}
\newtheorem*{theoremB}{Theorem B}
\newtheorem*{theoremC}{Theorem C}
\newtheorem*{theoremD}{Theorem D}
\newtheorem*{theoremE}{Theorem E}
\newtheorem*{theoremF}{Theorem F}
\newtheorem*{theoremG}{Theorem G}
\newtheorem*{theoremH}{Theorem H}

\newtheorem{manualtheoreminner}{Exercise}
\newenvironment{exercise}[1]{%
  \renewcommand\themanualtheoreminner{#1}%
  \manualtheoreminner
}{\endmanualtheoreminner}

\theoremstyle{definition}
\newtheorem{definition}[equation]{Definition}
\newtheorem{definitions}[equation]{Definitions}
%\theoremstyle{remark}

\newtheorem{remark}[equation]{Remark}
\newtheorem{remarks}[equation]{Remarks}
\newtheorem{example}[equation]{Example}
\newtheorem{examples}[equation]{Examples}
\newtheorem{notation}[equation]{Notation}
\newtheorem{question}[equation]{Question}
\newtheorem{assumption}[equation]{Assumption}
\newtheorem*{claim}{Claim}
\newtheorem{answer}[equation]{Answer}
%%%%%% letters %%%%

\newcommand{\fA}{\mathfrak{A}}
\newcommand{\fB}{\mathfrak{B}}
\newcommand{\fC}{\mathfrak{C}}
\newcommand{\fD}{\mathfrak{D}}
\newcommand{\fE}{\mathfrak{E}}
\newcommand{\fF}{\mathfrak{F}}
\newcommand{\fG}{\mathfrak{G}}
\newcommand{\fH}{\mathfrak{H}}
\newcommand{\fI}{\mathfrak{I}}
\newcommand{\fJ}{\mathfrak{J}}
\newcommand{\fK}{\mathfrak{K}}
\newcommand{\fL}{\mathfrak{L}}
\newcommand{\fM}{\mathfrak{M}}
\newcommand{\fN}{\mathfrak{N}}
\newcommand{\fO}{\mathfrak{O}}
\newcommand{\fP}{\mathfrak{P}}
\newcommand{\fQ}{\mathfrak{Q}}
\newcommand{\fR}{\mathfrak{R}}
\newcommand{\fS}{\mathfrak{S}}
\newcommand{\fT}{\mathfrak{T}}
\newcommand{\fU}{\mathfrak{U}}
\newcommand{\fV}{\mathfrak{V}}
\newcommand{\fW}{\mathfrak{W}}
\newcommand{\fX}{\mathfrak{X}}
\newcommand{\fY}{\mathfrak{Y}}
\newcommand{\fZ}{\mathfrak{Z}}

\newcommand{\fa}{\mathfrak{a}}
\newcommand{\fb}{\mathfrak{b}}
\newcommand{\fc}{\mathfrak{c}}
\newcommand{\fd}{\mathfrak{d}}
\newcommand{\fe}{\mathfrak{e}}
\newcommand{\ff}{\mathfrak{f}}
\newcommand{\fg}{\mathfrak{g}}
\newcommand{\fh}{\mathfrak{h}}
%\newcommand{\fi}{\mathfrak{i}}

\newcommand{\fj}{\mathfrak{j}}
\newcommand{\fk}{\mathfrak{k}}
\newcommand{\fl}{\mathfrak{l}}
\newcommand{\fm}{\mathfrak{m}}
\newcommand{\fn}{\mathfrak{n}}
\newcommand{\fo}{\mathfrak{o}}
\newcommand{\fp}{\mathfrak{p}}
\newcommand{\fq}{\mathfrak{q}}
\newcommand{\fr}{\mathfrak{r}}
\newcommand{\fs}{\mathfrak{s}}
\newcommand{\ft}{\mathfrak{t}}
\newcommand{\fu}{\mathfrak{u}}
\newcommand{\fv}{\mathfrak{v}}
\newcommand{\fw}{\mathfrak{w}}
\newcommand{\fx}{\mathfrak{x}}
\newcommand{\fy}{\mathfrak{y}}
\newcommand{\fz}{\mathfrak{z}}


\newcommand{\sA}{\mathcal{A}\,}
\newcommand{\sB}{\mathcal{B}\,}
\newcommand{\sC}{\mathcal{C}}
\newcommand{\sD}{\mathcal{D}\,}
\newcommand{\sE}{\mathcal{E}\,}
\newcommand{\sF}{\mathcal{F}\,}
\newcommand{\sG}{\mathcal{G}\,}
\newcommand{\sH}{\mathcal{H}}
\newcommand{\sI}{\mathcal{I}\,}
\newcommand{\sJ}{\mathcal{J}\,}
\newcommand{\sK}{\mathcal{K}\,}
\newcommand{\sL}{\mathcal{L}\,}
\newcommand{\sM}{\mathcal{M}\,}
\newcommand{\sN}{\mathcal{N}}
\newcommand{\sO}{\mathcal{O}}
\newcommand{\sP}{\mathcal{P}\,}
\newcommand{\sQ}{\mathcal{Q}\,}
\newcommand{\sR}{\mathcal{R}}
\newcommand{\sS}{\mathcal{S}}
\newcommand{\sT}{\mathcal{T}\,}
\newcommand{\sU}{\mathcal{U}\,}
\newcommand{\sV}{\mathcal{V}\,}
\newcommand{\sW}{\mathcal{W}\,}
\newcommand{\sX}{\mathcal{X}\,}
\newcommand{\sY}{\mathcal{Y}\,}
\newcommand{\sZ}{\mathcal{Z}\,}

\newcommand{\IA}{\mathbb{A}}
\newcommand{\IB}{\mathbb{B}}
\newcommand{\IC}{\mathbb{C}}
\newcommand{\ID}{\mathbb{D}}
\newcommand{\IE}{\mathbb{E}}
\newcommand{\IF}{\mathbb{F}}
\newcommand{\IG}{\mathbb{G}}
\newcommand{\IH}{\mathbb{H}}
\newcommand{\II}{\mathbb{I}}
\newcommand{\IK}{\mathbb{K}}
\newcommand{\IL}{\mathbb{L}}
\newcommand{\IM}{\mathbb{M}}
\newcommand{\IN}{\mathbb{N}}
\newcommand{\IO}{\mathbb{O}}
\newcommand{\IP}{\mathbb{P}}
\newcommand{\IQ}{\mathbb{Q}}
\newcommand{\IR}{\mathbb{R}}
\newcommand{\IS}{\mathbb{S}}
\newcommand{\IT}{\mathbb{T}}
\newcommand{\IU}{\mathbb{U}}
\newcommand{\IV}{\mathbb{V}}
\newcommand{\IW}{\mathbb{W}}
\newcommand{\IX}{\mathbb{X}}
\newcommand{\IY}{\mathbb{Y}}
\newcommand{\IZ}{\mathbb{Z}}


 \newcommand{\tA}{\mathrm {A}}
 \newcommand{\tB}{\mathrm {B}}
 \newcommand{\tC}{\mathrm {C}}
 \newcommand{\tD}{\mathrm {D}}
 \newcommand{\tE}{\mathrm {E}}
 \newcommand{\tF}{\mathrm {F}}
 \newcommand{\tG}{\mathrm {G}}
 \newcommand{\tH}{\mathrm {H}}
 \newcommand{\tI}{\mathrm {I}}
 \newcommand{\tJ}{\mathrm {J}}
 \newcommand{\tK}{\mathrm {K}}
 \newcommand{\tL}{\mathrm {L}}
 \newcommand{\tM}{\mathrm {M}}
 \newcommand{\tN}{\mathrm {N}}
 \newcommand{\tO}{\mathrm {O}}
 \newcommand{\tP}{\mathrm {P}}
 \newcommand{\tQ}{\mathrm {Q}}
 \newcommand{\tR}{\mathrm {R}}
 \newcommand{\tS}{\mathrm {S}}
 \newcommand{\tT}{\mathrm {T}}
 \newcommand{\tU}{\mathrm {U}}
 \newcommand{\tV}{\mathrm {V}}
 \newcommand{\tW}{\mathrm {W}}
 \newcommand{\tX}{\mathrm {X}}
 \newcommand{\tY}{\mathrm {Y}}
 \newcommand{\tZ}{\mathrm {Z}}
%%%%%%% macros %%%%%

%% my definitions %%%

\newcommand{\End}{\mathrm{End}}
\newcommand{\tr}{\mathrm{tr}}
%\newcommand{\ind}{\mathrm{ind}}

\renewcommand{\index}{\mathrm{index \,}}
\newcommand{\Hom}{\mathrm{Hom}}
\newcommand{\Aut}{\mathrm{Aut}}
\newcommand{\Trace}{\mathrm{Trace}\,}
\newcommand{\Res}{\mathrm{Res}\,}
\newcommand{\rank}{\mathrm{rank}}
%\renewcommand{\dim}{\mathrm{dim}}

\renewcommand{\deg}{\mathrm{deg}}
\newcommand{\spin}{\rm Spin}
\newcommand{\Spin}{\rm Spin}
\newcommand{\erfc}{\rm erfc\,}
\newcommand{\sgn}{\rm sgn\,}
\newcommand{\Spec}{\rm Spec\,}
\newcommand{\diag}{\rm diag\,}
\newcommand{\Fix}{\mathrm{Fix}}
\newcommand{\Ker}{\mathrm{Ker \,}}
\newcommand{\Coker}{\mathrm{Coker \,}}
\newcommand{\Sym}{\mathrm{Sym \,}}
\newcommand{\Hess}{\mathrm{Hess \,}}
\newcommand{\grad}{\mathrm{grad \,}}
\newcommand{\Center}{\mathrm{Center}}
\newcommand{\Lie}{\mathrm{Lie}}


\newcommand{\ch}{\rm ch} % Chern Character

\newcommand{\rk}{\rm rk} 
%\renewcommand{\c}{\rm c}  % Chern class

\newcommand{\sign}{\rm sign}
\renewcommand\dim{{\rm dim\,}}
\renewcommand\det{{\rm det\,}}
\newcommand{\dimKrull}{{\rm Krulldim\,}}
\newcommand\Rep{\mathrm{Rep}}
\newcommand\Hilb{\mathrm{Hilb}}
\newcommand\vol{\mathrm{vol}}

\newcommand\QED{\hfill $\Box$} %{\bf QED}} 

\newcommand\Pf{\nonintend{\em Proof. }}


\newcommand\reals{{\mathbb R}} 
\newcommand\complexes{{\mathbb C}}
\renewcommand\i{\sqrt{-1}}
\renewcommand\Re{\mathrm Re}
\renewcommand\Im{\mathrm Im}
\newcommand\integers{{\mathbb Z}}
\newcommand\quaternions{{\mathbb H}}


\newcommand\iso{{\cong}} 
\newcommand\tensor{{\otimes}}
\newcommand\Tensor{{\bigotimes}} 
\newcommand\union{\bigcup} 
\newcommand\onehalf{\frac{1}{2}}
%\newcommand\Sym[1]{{Sym^{#1}(\complexes^2)}}

\newcommand\lie[1]{{\mathfrak #1}} 
\renewcommand\fk{\mathfrak{K}}
\newcommand\smooth{\mathcal{C}^{\infty}}
\newcommand\trivial{{\mathbb I}}
\newcommand\widebar{\overline}

%%%%%Delimiters%%%%

\newcommand{\<}{\langle}
\renewcommand{\>}{\rangle}

%\renewcommand{\(}{\left(}
%\renewcommand{\)}{\right)}


%%%% Different kind of derivatives %%%%%

\newcommand{\delbar}{\bar{\partial}}
\newcommand{\pdu}{\frac{\partial}{\partial u}}
%\newcommand{\pd}[1][2]{\frac{\partial #1}{\partial #2}}

%%%%% Arrows %%%%%
%\renewcommand{\ra}{\rightarrow}                   % right arrow
%\newcommand{\lra}{\longrightarrow}              % long right arrow
%\renewcommand{\la}{\leftarrow}                    % left arrow
%\newcommand{\lla}{\longleftarrow}               % long left arrow
%\newcommand{\ua}{\uparrow}                     % long up arrow
%\newcommand{\na}{\nearrow}                      %  NE arrow
%\newcommand{\llra}[1]{\stackrel{#1}{\lra}}      % labeled long right arrow
%\newcommand{\llla}[1]{\stackrel{#1}{\lla}}      % labeled long left arrow
%\newcommand{\lua}[1]{\stackrel{#1}{\ua}}      % labeled  up arrow
%\newcommand{\lna}[1]{\stackrel{#1}{\na}}      % labeled long NE arrow

\newcommand{\into}{\hookrightarrow}
\newcommand{\tto}{\longmapsto}
\def\llra{\longleftrightarrow}

\def\d/{/\mspace{-6.0mu}/}
\newcommand{\git}[3]{#1\d/_{\mspace{-4.0mu}#2}#3}
\newcommand\zetahilb{\zeta_{{\text{Hilb}}}}
\def\Fy{\sF \mspace{-3.0mu} \cdot \mspace{-3.0mu} y}
\def\tv{\tilde{v}}
\def\tw{\tilde{w}}
\def\wt{\widetilde}
\def\wtilde{\widetilde}
\def\what{\widehat}

%%%%%%%%%%%%%%%%%%% Mark's definitions %%%%%%%%%%%%%%%%%%%%

\newcommand{\frakg}{\mbox{\frakturfont g}}
\newcommand{\frakk}{\mbox{\frakturfont k}}
\newcommand{\frakp}{\mbox{\frakturfont p}}
\newcommand{\q}{\mbox{\frakturfont q}}
\newcommand{\frakn}{\mbox{\frakturfont n}}
\newcommand{\frakv}{\mbox{\frakturfont v}}
\newcommand{\fraku}{\mbox{\frakturfont u}}
\newcommand{\frakh}{\mbox{\frakturfont h}}
\newcommand{\frakm}{\mbox{\frakturfont m}}
\newcommand{\frakt}{\mbox{\frakturfont t}}
\newcommand{\G}{\Gamma}
\newcommand{\g}{\gamma}
\newcommand{\fraka}{\mbox{\frakturfont a}}
\newcommand{\db}{\bar{\partial}}
\newcommand{\dbs}{\bar{\partial}^*}
\newcommand{\p}{\partial}

%%%%%%%%%%%%% new definitions for the positive mass paper %%%%%%%%%

\newcommand{\sperp}{{\scriptscriptstyle \perp}}

%%%%%%%%%%%%%%%%%%%%%%%

%%%%%%%%%%%%%%%%%%%%%%%%%%%%%%%%%%%%%%%%%%%%%



%
\begin{document}
%

\title{Math 531 Homework 1}
\author{Braden Hoagland}


\date{\today}

\maketitle

%\setcounter{secnumdepth}{1} 

%%%%%%%%%%%%%%%%%%%%%%%%%%%%%%%%%%%%%%%%
\begin{exercise}{1.1.3}
	In an ordered field, if $a>b$, show that $a^2b < ab^2 + (a^3 - b^3)/3$.
\end{exercise}
After each line, a reference to the particular axiom or proposition from the textbook is given to justify the arithmetic.

Since multiplication distributes over addition, we can factor $(a^3-b^3) = (a-b)(a^2+ab+b^2).$ We then have
\begin{align*}
	a^2 b-ab^2 &< (a-b)(a^2+ab+b^2)/3 && \text{III.15} \\
	3 (a^2 b- ab^2) &< (a-b)(a^2+ab+b^2) && \text{1.1.2.xi} \\
	\intertext{Since $(a^2b - ab^2) = ab(a-b)$ by the distributive and commutative properties of multiplication, this becomes}
	3 ab(a-b) &< (a-b) (a^2+ab+b^2) && \text{II.5, II.9} \\
	\intertext{Now since $a > b \iff a-b>0$, we can use the cancellation law for multiplication}
	3ab &< a^2 + ab + b^2 && \text{1.1.2.v} \\
	0 &< a^2 - 2ab + b^2 && \text{III.15} \\
	\intertext{Since we can factor $a^2 - 2ab + b^2 = (a-b)^2$, this simplifies to}
	0 &< (a-b)^2 && \text{I.2, I.4, II.5, II.9, 1.1.2.viii}
\end{align*}
By 1.1.2.xiv, we know that $(a-b)^2 \geq 0$. Thus to show that the derived inequality holds true, we must show that $(a-b)^2 \neq 0$. 
Since fields have no zero-divisors (1.1.2.iii), we know that $(a-b)^2=0 \iff a-b=0 \iff a=b$. Since we were given that $a>b$, we know that this cannot be the case. Thus $(a-b)^2 \neq 0$ and we have shown that the original inequality holds.

%%%%%%%%%%%%%%%%%%%%%%%%%%%%%%%%%%%%%%%%
\begin{exercise}{1.1.4}
	Prove that in an ordered field, if $\sqrt{2}$ is a positive number whose square is $2$, then $\sqrt{2}<3/2$.
\end{exercise}
Assume $\sqrt{2} \geq 3/2$, then $2 \geq \frac{3}{2} \sqrt{2}$ and $\sqrt{2} \frac{3}{2} \geq \frac{9}{4}$. By commutivity of multiplication, we have the chain of inequalities
\begin{align*}
	2 \geq \frac{3}{2} \sqrt{2} \geq \frac{9}{4} \\
	8 \geq 6 \sqrt{2} \geq 9
\end{align*}
This yields a contradiction, namely $8 \geq 9$. Thus our initial assumption must be wrong, and it must instead be the case that $\sqrt{2} < \frac{3}{2}$.

%%%%%%%%%%%%%%%%%%%%%%%%%%%%%%%%%%%%%%%%
\begin{exercise}{1.1.5}
	Give an example of a field with only three elements. Prove that it cannot be made into an ordered field.
\end{exercise}
A field with only three elements is the set of integers modulo 3, denoted $\mathbb{Z}_3$ with elements $\left\{ 0, 1, 2 \right\}$.

When addition is defined $x+y \doteq (x+y) \mod 3$, then axioms $1-4$ for fields are satisfied. Addition is commutative and associative (1 and 2), which can be shown by case analysis for every combination of terms. The element 0 is the additive identity (3). Each element has an additive inverse (4), since $0+0=0, 1+2=0$, and  $2+1=0$.

The multiplication axtioms $5-10$ are also satisfied when multiplication is defined as usual. Case analysis can again be used to show that it is commutative and associative (5 and 6). 1 is the mlutiplicative identity (7). Both 1 and 2 have inverses (8), since $1\cdot 1 = 1$ and $2 \cdot 2 = 1$. Multiplication distributes over addition (9). It is also true that the multiplicative and additive identities are distinct (10), since 1 and 0 are different elements.

We can now show that $\mathbb{Z}_3$ cannot be made into an ordered field by assuming that it is one and then showing that this leads to a contradiction. Assume $\mathbb{Z}_3$ is an ordered field, then it must satisfy the Law of Trichotomy. We know $1 \neq 0$, so it must be true that either $0 < 1$ or $1 < 0$. However, we can show that either of these leads to a contradiction.
\begin{enumerate}
	\item
	$0 < 1 \implies 0\cdot 2 < 1\cdot 2 \implies 0 < 2 \implies 0+1 < 2+1 \implies 1 < 0$
	\item
	$1 < 0 \implies 1+(-1) < 0+(-1) \implies 0 < 2 \implies 0\cdot 2 < 2\cdot 2 \implies 0 < 1$
\end{enumerate}
We have shown that the Law of Trichotomy is violated if $\mathbb{Z}_3$ is an ordered field, which itself is a condtradiction. Thus $\mathbb{Z}_3$ cannot be an ordered field.

%%%%%%%%%%%%%%%%%%%%%%%%%%%%%%%%%%%%%%%%
\begin{exercise}{1.2.2}
	Show that $3^n / n!$ converges to 0.
\end{exercise}
We can use the squeeze theorem to bound this sequence above and below by sequences that both converge to 0, which shows that $\lim_{n \to \infty} 3^n/n! = 0$.

First note that $\frac{3^n}{n!} \geq 0$ for all $n \in \mathbb{N}$ since $3^n$ and $n!$ are themselves always non-negative. This allows us to bound our sequence below by 0. Next we must find an upper bound for our sequence that converges to 0, then we can apply the squeeze theorem.

A sufficient bound is
\begin{align*}
	\frac{3^n}{n!} &= \frac{3}{1} \frac{3}{2} \cdots \frac{3}{n} \\
		       &\leq \frac{3}{1} \frac{3}{2} \frac{3}{n} \\
		       &= \frac{27}{2n}
\end{align*}
since all terms from $\frac{3}{3}$ to $\frac{3}{n-1}$ are less than or equal to 1.

To satisfy $|\frac{27}{2n}| = \frac{27}{2n} < \varepsilon$, set $n > \frac{27}{2 \varepsilon}$. Thus this sequence converges to 0. Now we have $0 \leq \frac{3^n}{n!} \leq \frac{27}{2n}$ where $0 \to 0$ and $\frac{27}{2n} \to 0$. Then by the squeeze theorem, $\frac{3^n}{n!} \to 0$ as well.

%%%%%%%%%%%%%%%%%%%%%%%%%%%%%%%%%%%%%%%%
\begin{exercise}{1.2.3}
	Let $x_n = \sqrt{n^2+1} - n$. Compute $\lim_{n \to \infty}x_n$.
\end{exercise}
We hypothesize that the limit for this sequence is 0. We can use the squeeze theorem to bound this sequence above and below by sequences that both converge to 0, which shows that $\lim_{ \to \infty}x_n = 0$.

First we show that $\{x_n\}_{n=1}^\infty$ is bounded below by 0.
\begin{align*}
	\sqrt{n^2+1} - n &\geq 0 \\
	\sqrt{n^2+1} &\geq n \\
	\intertext{From this inequality we have both $n \sqrt{n^2+1} \geq n^2$ and $n^2 + 1 \geq n \sqrt{n^2+1}$. This gives us the single inequality}
	n^2 + 1 &\geq n^2 \\
	1 &\geq 0
\end{align*}
This is always true, so we have shown that every term $x_n$ is bounded below by 0.

Now we show that our sequence is bounded above by the sequence $\{ \frac{1}{2n}\}_{n=1}^\infty$.
\begin{align*}
	\sqrt{n^2+1} - n &= (\sqrt{n^2+1} -n) \frac{\sqrt{n^2+1} + n}{\sqrt{n^2+1} + n} \\
			 &= \frac{1}{\sqrt{n^2+1} + n} \\
			 &\leq \frac{1}{2n}
\end{align*}
where the last line comes from the inequality $\sqrt{n^2+1} \geq n$. It is also worth noting that this inequality only holds when $n > 0$, as $\frac{1}{2n} $ is undefined when $n=0$ and negative when $n<0$. Since we only care about the behavior of our sequence as $n\to \infty$, this has no effect on the limit.

Since $0 \leq \sqrt{n^2+1} -n \leq \frac{1}{2n}$, we need only show that $\frac{1}{2n}$ converges to $0$ as $n\to\infty$. Let $\varepsilon > 0$, then we need to find $N \in \mathbb{N}$ such that for all $n > N$, $|\frac{1}{2n}| = \frac{1}{2n} < \varepsilon$. This is true if we set $N = \frac{1}{2\varepsilon}$. Thus this sequence converges to 0, and by the squeeze theorem, so does the sequence $\{x_n\}_{n=1}^\infty$.

%%%%%%%%%%%%%%%%%%%%%%%%%%%%%%%%%%%%%%%%
\begin{exercise}{1.2.4}
	Let $x_n$ be a monotone increasing sequence such that $x_{n+1}-x_n \leq 1/n$. Must $x_n$ converge?
\end{exercise}
The sequence $\left\{ x_n \right\}$ is in fact not necessarily bounded, and thus does not have to converge. This ends up being a consequence of the divergence of the harmonic series.

We start by finding the difference between an arbitrary term in the sequence $x_{n+1}$ and the first term $x_1$.
\begin{align*}
	x_{n+1} - x_1 &= (x_{n+1} - x_n) + (x_n - x_{n-1})  \cdots + (x_2 - x_1) \\
		      &\leq \frac{1}{n} + \frac{1}{n-1} + \cdots + \frac{1}{2}  + 1
\end{align*}
This is the harmonic series, which we can show diverges with a classic proof.
\begin{align*}
	1 + \frac{1}{2} + \frac{1}{3} + \frac{1}{4} + \cdots &\geq 1 + \frac{1}{2} + \left( \frac{1}{4} + \frac{1}{4}  \right) + \left( \frac{1}{8} + \frac{1}{8} +\frac{1}{8} + \frac{1}{8}   \right) + \cdots \\
						    &= 1 + \left( \frac{1}{2} + \frac{1}{2} + \cdots \right)
\end{align*}
Since this sum is unbounded, this means that the difference $x_{n+1} - x_1$ is not necessarily bounded. By extension, the entire sequence  $\left\{ x_n \right\}$ is not necessarily bounded.

In the worst case (i.e. $x_{n+1} - x_n = \frac{1}{n} $), divergence of the sequence is guaranteed since $x_{n+1} - x_n = \sum_{i=1}^n \frac{1}{i} $, which we just proved diverges as $n \to \infty$. For other sequences satisfiying $x_{n+1}-x_n \leq \frac{1}{n} $, convergence is possible but not guaranteed because of the lack of a guaranteed upper bound.

%%%%%%%%%%%%%%%%%%%%%%%%%%%%%%%%%%%%%%%%
\begin{exercise}{1.2.5}
	Let $\mathbb{F}$ be an ordered field in which every strictly monotone increasing sequence bounded above converges. Prove that $\mathbb{F}$ is complete.
\end{exercise}
The general strategy of this proof is to take a monotone increasing sequence and strip it of repeated terms, thus transforming it into a strictly increasing sequence. Using the convergence guarantees for this new sequence, we can show that the repeated terms in the original sequence do not prevent the convergence from occurring.

Let $\left\{ x_n \right\}_{n=1}^\infty$ satisfy $x_{n+1} \geq x_n$. We can form a strictly increasing sequence $\left\{ y_n \right\}$ from $\left\{ x_n \right\}$ by simply removing any repeated terms from $\left\{ x_n \right\}$. The formal construction is as follows:
\begin{enumerate}
	\item Set $y_1 = x_1$.
	\item For each member of $\left\{ x_n \right\}$ in order, add it to the sequence $\left\{ y_n \right\}$ if no member in $\left\{ y_n \right\}$ has an equivalent value.
\end{enumerate}
There are two possibilities: the first is that $\left\{ y_n \right\}$ is finite because $\left\{ x_n \right\}$ at some point becomes fixed at a single value, and the second is that $\left\{ y_n \right\}$ is infinite with all repeats from  $\left\{ x_n \right\}$ removed. In the first case, $x_n$ trivially converges. In the second case, $\left\{ x_n \right\}$ can once again be shown to converge.

Since $\left\{ y_n \right\}$ satisfies $y_{n+1} > y_n$, it converges by assumption to some limit $L$. This means that for any $\varepsilon > 0$, there exists $N \in \mathbb{N}$ such that for all $n > N$, $|y_n - L| < \varepsilon$. We wish to show that the sequence $\left\{ x_n \right\}$ satisfies this same property and thus also converges to $L$.

For given $\varepsilon > 0$, we must find $N$ such that for any $n > N$, $|x_n - L| < \varepsilon$. By construction, any term $x_n$ must appear in $\left\{ y_n \right\}$ (and vice versa). We can thus construct a surjective map $\phi: \left\{ x_n \right\} \twoheadrightarrow \left\{ y_n \right\}$ by $\phi(x_n) \mapsto y_{n'}$, where $y_{n'}$ has the same value as $x_n$. Additionally by construction, all terms appearing after $x_n$ must also appear after the corresponding $y_{n'}$. Thus to find our desired $N$, simply find any $n$ satisfying $\phi(x_n) = y_{n'}$, where $|y_i - L| < \varepsilon$ for all $i > n'$. This is guaranteed to exist since $\left\{ y_n \right\}$ itself converges to $L$ and our map $\phi$ is surjective.

Since $\left\{ x_n \right\}$ converges, our field $\mathbb{F}$ satisfies the Monotone Convergence Property and is subsequently complete.


\end{document}
