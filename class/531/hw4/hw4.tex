\documentclass[10pt]{amsart}
\usepackage{latexsym} 
\usepackage{amscd,amsthm,amssymb,amsfonts,amsmath}
\usepackage{xcolor}
%\usepackage{epsfig}
%\usepackage{graphicx}
%\usepackage[dvips]{graphicx}

\usepackage[matrix,tips,graph,curve]{xy}

\newcommand{\mnote}[1]{${}^*$\marginpar{\footnotesize ${}^*$#1}}
\linespread{1.065}
% \setlength{\parskip}{1em}

\makeatletter

\setlength\@tempdima  {5.5in}
\addtolength\@tempdima {-\textwidth}
\addtolength\hoffset{-0.5\@tempdima}
\setlength{\textwidth}{5.5in}
\setlength{\textheight}{8.75in}
\addtolength\voffset{-0.625in}

\makeatother

\makeatletter 
\@addtoreset{equation}{section}
\makeatother


\renewcommand{\theequation}{\thesection.\arabic{equation}}

\theoremstyle{plain}
\newtheorem{theorem}[equation]{Theorem}
\newtheorem{corollary}[equation]{Corollary}
\newtheorem{lemma}[equation]{Lemma}
\newtheorem{proposition}[equation]{Proposition}
\newtheorem{conjecture}[equation]{Conjecture}
\newtheorem{fact}[equation]{Fact}
\newtheorem{facts}[equation]{Facts}

\newtheorem{manualtheoreminner}{Exercise}
\newenvironment{exercise}[1]{%
  \renewcommand\themanualtheoreminner{#1}%
  \manualtheoreminner
}{\endmanualtheoreminner}

\theoremstyle{definition}
\newtheorem{definition}[equation]{Definition}
\newtheorem{definitions}[equation]{Definitions}
%\theoremstyle{remark}

\newtheorem{remark}[equation]{Remark}
\newtheorem{remarks}[equation]{Remarks}
\newtheorem{example}[equation]{Example}
\newtheorem{examples}[equation]{Examples}
\newtheorem{notation}[equation]{Notation}
\newtheorem{question}[equation]{Question}
\newtheorem{assumption}[equation]{Assumption}
\newtheorem*{claim}{Claim}
\newtheorem{answer}[equation]{Answer}

\newcommand{\p}{\partial}

%%%%%%%%%%%%% new definitions for the positive mass paper %%%%%%%%%

\newcommand{\sperp}{{\scriptscriptstyle \perp}}

%%%%%%%%%%%%%%%%%%%%%%%%%%%%%%%%%%%%%%%%%%%%%


%
\begin{document}
%

\title{Math 531 Homework 4}
\author{Braden Hoagland}


\date{\today}

\maketitle

% ========================================
\begin{exercise}{2.12}
	Prove the following properties for subsets $A$ and $B$ of a metric space:
	\begin{enumerate}
		\item $(A^o)^o = A^o$
		\item $(A \cup B)^o \supset A^o \cup B^o$ 
		\item $(A \cap B)^o = A^o \cap B^o$
	\end{enumerate}
	\hrulefill
\end{exercise}
\begin{enumerate}
	\item
		For any set $X$, by definition $X^o = X$ if and only if $X$ is open. Now for any $a \in A^o$, there exists open neighborhood $U \subset A$ such that $a \in U$. This shows $A^o$ is open, so $(A^o)^o = A^o$.

	\item
		Let $x \in A^o \cup B^o$. If $x \in A^o$, then there exists an open neighborhood $U$ of $x$ that lies within $A$ and, by extension, $A \cup B$. Thus $x \in (A \cup B)^o$. A similar argument holds for when $x \in B^o $, so we conclude $A^o \cup B^o \subset (A \cup B)^o$.

	\item 
		Let $x \in A^o \cap B^o$, then there exist open neighborhoods $U_A \subset A$ and $U_B \subset B $ of $x$. Their intersection $U_A \cap U_B \subset A \cap B$ is still an open neighborhood of $x$, so $x \in (A \cap B)^o$. Thus $A^o \cap B^o \subset (A \cap B)^o$.

		Now let $x \in (A \cap B)^o$, then there exists open neighborhood $U \subset A \cap B$ of $x$. Since $U \subset A$ and $U \subset B$, $x$ is in both $A^o$ and $B^o$. Thus $(A \cap B)^o \subset A^o \cap B^o$.

		These two inclusions show $(A \cap B)^o = A^o \cap B^o$.
\end{enumerate}


% ========================================
\begin{exercise}{2.15}
	Prove the following for subsets of a metric space $M$:
	\begin{enumerate}
		\item $\p A = \p(A^c)$ 
		\item $\p (\p A) \subset \p A$
		\item $\p (A \cup B) \subset \p A \cup \p B \subset \p(A \cup B) \cup A \cup B$
		\item $\p (\p (\p A)) = \p (\p A)$
	\end{enumerate}
	\hrulefill
\end{exercise}
\begin{enumerate}
	\item $\p A = \overline{A} \cap \overline{A^c} = \overline{A^c} \cap \overline{A} = \p(A^c)$.
	\item Let $a \in \p(\p A) = \overline{\p A} \cap \overline{(\p A)^c} = \p A \cap \overline{(\p A)^c} $, then it is clear that $a \in \p A$ as well. Thus  $\p(\p A) \subset \p A$.
	\item
		\textbf{First inclusion:} $\p (A \cup B) = \overline{A \cup B} \cap \overline{(A \cup B)^c} = (\overline{A} \cup \overline{B}) \cap \overline{A^c \cap B^c}$. Let $x \in \p (A \cup B)$. If $x \in \overline{A}$, then it must also be an element of $\overline{A^c \cap B^c} \subset \overline{A^c} $. This means $x \in \overline{A} \implies x \in \overline{A} \cap \overline{A} = \p A$. Similarly, if $x \in \overline{B}$, then it is also in $\p B$. Thus $\p(A \cup B) \subset \p A \cup \p B$.

		\textbf{Second inclusion:} We start by proving a helpful implication. We can show by contrapositive that $x \in \p A \cup \p B \implies x \in \overline{A \cup B} $. Assume $x \not\in \overline{A \cup B} = \overline{A} \cup \overline{B}$, then $x$ is not in $\overline{A}$ or $\overline{B}$, which further implies that $x$ is not in $\p A$ or $\p B$. This shows $x \in \p A \cup \p B \implies x \in \overline{A \cup B} $.

		Using this implication, we can prove the desired inclusion. Suppose $x \in \p A \cup \p B$ such that $x \not\in \p(A \cup B)$ (if this is not possible, then $\p(A \cup B) = \p A \cup \p B$, in which case the desired inclusion is trivial). Then by using the identity $\p(A \cup B) = \overline{A \cup B} \cap \overline{(A \cup B)^c}$ and the implication we just proved, we know that the only way $x$ is not in $\p(A \cup B)$ is if $x \not\in \overline{(A \cup B)^c}$. This means  $x \in \overline{(A \cup B)^c}^c = (A \cup B)^o \subset A \cup B \subset \p(A \cup B) \cup A \cup B$. This is the desired result.
	\item 
		For any closed set $A$, $\overline{A}=A$. We can use this to simplify the definition of the boundary of a boundary, since the boundary is defined to be the intersection of two closed sets and is thus also closed.
		\[
			\p(\p A) = \overline{\p A} \cap \overline{(\p A)^c} = \p A \cap \overline{(\p A)^c} 
		\] 
		Moreover, since $\p\p A$ is also closed, we can do something similar for $\p\p\p A$ and show that it reduces to this same quantity.
		\begin{align*}
			\p(\p(\p A)) &= \overline{\p(\p A)} \cap \overline{(\p(\p A))^c} \\
				     &= \p(\p A) \cap \overline{(\p A)^c \cup \p A} \\
				     &= \p A \cap \overline{(\p A)^c} \cap \left( \overline{(\p A)^c} \cup \overline{\p A} \right) \\
				     &= \p A \cap \overline{(\p A)^c} \cap \left( \overline{(\p A)^c} \cup \p A \right) \\
				     &= \p A \cap \overline{(\p A)^c} \\
				     &= \p (\p A)
		\end{align*}
		which is the desired equality.

\end{enumerate}


% ========================================
\begin{exercise}{2.20}
	For a set $A$ in a metric space $M$ and $x \in M$, let
	\[
		d(x,A) = \inf\left\{ d(x,y) \;|\; y \in A \right\},
	\] 
	and for $\varepsilon>0$, let $D(A,\varepsilon) = \left\{ x \;|\; d(x,A) < \varepsilon \right\}$.
	\begin{enumerate}
		\item Show that $D(A,\varepsilon)$ is open.
		\item Let $A \subset M$ and $N_\varepsilon = \left\{ x\in M \;|\; d(x,A) \leq \varepsilon \right\}$, where $ \varepsilon>0$. Show that $N_\varepsilon$ is closed and that $A$ is closed if and only if $A = \cap \left\{ N_\varepsilon \;|\; \varepsilon>0 \right\}$.
	\end{enumerate}
	\hrulefill
\end{exercise}
\begin{enumerate}
	\item
		The $ \varepsilon$-ball around a point $a\in A$ is $D(a,\varepsilon) = \left\{ x\in M \;|\; d(x,a) < \varepsilon \right\}$, so we can write $D(A,\varepsilon)$ as
		 \[
			 D(A,\varepsilon) = \bigcup_{a \in A} D(a,\varepsilon).
		\] 
		Since the union of an arbitrary colleciton of open sets is itself open, it suffices to prove that $D(a,\varepsilon)$ is open. It is known that $\varepsilon$-balls are open, and $D(a,\varepsilon)$ is a specific instance of an $\varepsilon$-ball around $a$, so the desired result immediately follows.
	
	\item 
		First we show that $N_\varepsilon^c$ is open. Let $x \in N_\varepsilon^c$, then the $\varepsilon$-ball $D(x,\varepsilon)$ lies entirely in $N_\varepsilon^c$. Then by definition, $N_\varepsilon$ is closed. Now we can show that $A$ is closed if and only if $A=\cap\left\{ N_\varepsilon\;|\;\varepsilon>0 \right\}$.

		\textbf{Backward:} Assume $A = \cap\left\{ N_\varepsilon\;|\;\varepsilon>0 \right\}$. Since each  $N_\varepsilon$ is closed and the intersection of an arbitrary collection of closed sets is itself closed, $A$ must be closed.

		\textbf{Forward:} Assume $A$ is closed, and let $a \in A$ be arbitrary. Then $a \in \cap_\varepsilon \left\{ N_\varepsilon \right\}$ since $d(a,A) = 0 \leq \varepsilon$ for every $\varepsilon>0$. This shows $A \subset \cap_\varepsilon \left\{ N_\varepsilon \right\}$. We now show the reverse inclusion.

		Let $n \in \cap_\varepsilon \left\{ N_\varepsilon \right\}$, then we can show $n \in A$ by contradiction. Assume $n \not\in A$. Since $A$ is closed, it contains all its accumulation points, so $d(n,A) > 0$. Now let $\varepsilon<\delta$. If $n \in N_\varepsilon$, then $n \not\in \left\{ N_\varepsilon \right\}$ and, subsequently, $n \not\in \cap_\varepsilon \left\{ N_\varepsilon \right\}$. This is a contradiction, so $n$ must be in $A$. Thus $\cap_\varepsilon \left\{ N_\varepsilon \right\} \subset A$.

		These two inclusions show $A = \cap_\varepsilon \left\{ N_\varepsilon \right\}$.
\end{enumerate}


% ========================================
\begin{exercise}{2.21}
	Prove that a sequence $\left\{ x_k \right\}$ in a normed vector space is a Cauchy sequence if and only iff for every neighborhood $U$ of 0, there is an $N$ such that $k,l \geq N$ implies $x_k - x_l\in U$.
	\hrulefill
\end{exercise}
\textbf{Backward:} Fix $\varepsilon>0$, then consider the $\varepsilon$-ball $D(0,\varepsilon)$. Since every $\varepsilon$-ball is open, then by assumption there exists $N$ such that if $k,l\geq N$, then $x_k -x_l \in D(0,\varepsilon)$. This implies $\Vert{x_k-x_l}\Vert<\varepsilon$, which shows that $\left\{ x_n \right\}$ is a Cauchy sequence.

\textbf{Forward:} Since by assumption $\left\{ x_n \right\}$ is a Cauchy sequence, we know that for every $\varepsilon>0$, there exists $N$ such that if $k,l >  N$, then $\Vert{x_k-x_l}\Vert<\varepsilon$. This implies that the element $x_k-x_l$ is contained in $D(0,\varepsilon)$.

Now let $U$ be an open set containing 0, then by definition there exists an $\varepsilon>0$ such that $D(0,\varepsilon) \subset U$. We just showed that there exists some $N$ such that if $k,l > N$, then $x_k-x_l \in D(0,\varepsilon)$ for all $\varepsilon>0$, so clearly $x_k - x_l \in U$.


% ========================================
\begin{exercise}{2.28}
	Give examples of:
	\begin{enumerate}
		\item An infinite set in $\mathbb{R}$ with no accumulation points.
		\item A nonempty subset of $\mathbb{R}$ that is contained in its set of accumulation points.
		\item A subset of $\mathbb{R}$ that has infinitely many accumulation points but contains none of them.
		\item A set $A$ such that $\p A = \bar{A}$.
	\end{enumerate}
	\hrulefill
\end{exercise}
\begin{enumerate}
	\item An example is $\mathbb{Z}$. Take $D(z,1/2)$ for any $z \in \mathbb{Z}$ to show this.
	\item An example is $[0,1]$, since $\text{acc}([0,1]) = [0,1]$.
	\item Let $A_b = \left\{ b + 1/n \;|\; n \in \mathbb{N}\right\}$ and $A = \bigcup_{b\in\mathbb{Z}} A_b$. Each $A_b$ has one accumulation point, namely $b$; however $b \not\in A_b$ for any $b$. There are an infinite number of $A_b$'s, so there are an infinite number of accumulation points in $A$, meaning we have a set with infinite accumulation points that contains none of them.
	\item An example is $\varnothing$. Let $A=\varnothing$, then $\overline{A}=\varnothing$ and $\p A = \overline{A} \cap \overline{A^c} = \varnothing \cap \overline{A^c} = \varnothing$, so $\p A = \overline{A}$.
\end{enumerate}


% ========================================
\begin{exercise}{2.38}
	Let $x_k \in \mathbb{R}^n$ satisfy $\Vert{x_k-x_l}\Vert\leq \frac{1}{k} +\frac{1}{l} $. Prove that $x_k$ converges.
	\hrulefill
\end{exercise}
Note that $\Vert{x_n + x_{n+k}}\Vert \leq \frac{1}{n} + \frac{1}{n+k} \leq \frac{2}{n} $ for any $k\geq 0$. Then to make $\Vert{x_k + x_{n+k}}\Vert<\varepsilon$ for some given $\varepsilon>0$, we can find $n$  such that $n > 2 / \varepsilon$. Let $ N > 2 / \varepsilon$ be some integer, then for all $n > N$, we have $\Vert{x_n - x_{n+k}}\Vert<\varepsilon$ for all $k\geq 0$. Since this holds for arbitrary $k$, we have $\Vert{x_k-x_l}\Vert<\varepsilon$ for all $k,l > N$. This shows that $\left\{ x_k \right\}$ is a Cauchy sequence. Since we are operating in $\mathbb{R}^n$, this is sufficient to show that $ \left\{ x_k \right\}$ converges.


% ========================================
\begin{exercise}{2.40}
	Suppose in $\mathbb{R}$ that for all $n, a_n \leq b_n, a_n \leq a_{n+1}$, and $b_{n+1}\leq b_n$. Prove that $a_n$ converges.
	\hrulefill
\end{exercise}
Since $b_1 \geq b_n \geq a_n $ for all $n$, the sequence $\left\{ a_n \right\}$ is bounded above by $b_1$. Since $\mathbb{R}$ satisfies the monotone convergence property and since $\left\{ a_n \right\}$ is a monotone non-decreasing sequence, $ \left\{ a_n \right\}$ must converge.


% ========================================
\begin{exercise}{2.42}
	Let $A \subset \mathbb{R}^n$ and $x \in \mathbb{R}^n$. Define $d(x,A) = \inf\left\{ d(x,y) \;|\; y \in A \right\}$. Must there be a $z \in A$ such that $d(x,A) = d(x,z)$?
	\hrulefill
\end{exercise}
No. Consider the open interval $A=(0,\infty) \subset \mathbb{R}$ with the usual metric $d(x,y) = |x-y|$ and the point $x=-1$. We claim $d(x,A) = 1$.

Since $a > 0$ for all $a \in A$, the distance between $x=-1$ and any point in $A$ satisfies $d(-1,a) = |a+1| \geq 1$. Thus 1 is a lower bound of $d(x,a)$. Now take $1+\varepsilon$ for some $\varepsilon>0$. This cannot be a lower bound on $d(x,a)$ since for $\varepsilon/2 \in A$, the distance from $x$ is $d(-1,\varepsilon/2) = |1 + \varepsilon/2| < 1 + \varepsilon$. Thus $d(x,A) = 1$.

However, the only points $z$ that satisfy $d(x,z) = d(-1,z) = 1$ are $-2$ and $0$, and neither of these points are in $A$.


% ========================================
\begin{exercise}{2.44}
	A set $A \subset \mathbb{R}^n$ is said to be \textbf{dense} in $B \subset \mathbb{R}^n$ if $B \subset \bar{A} $. If $A$ is dense in $\mathbb{R}^n$ and $U$ is open, prove that $A \cap U$ is dense in $U$. Is this true if $U$ is not open?
	\hrulefill
\end{exercise}
Let $u \in U$ and let $V$ be any open neighborhood of $u$. Then since $U$ and $V$ are both open, their intersection $U \cap V$ is also open. Then there eixsts $\varepsilon>0$ such that $D(u,\varepsilon) \subset U \cap V$. Assume $A \cap (D(u,\varepsilon) \backslash \left\{ u \right\}) = \varnothing$ for some $\varepsilon$, then let $y \in D(u,\varepsilon)$ be any member of this disk that is not equal to $u$. Since $D(u,\varepsilon)$ has an empty intersection with $A$, the point $y$ is clearly not in $A$. It is also not an accumulation point of $A$, as the disk $D(y,  \varepsilon-d(u,y))$ does not contain any points of $A$. This implies $y \not\in \overline{A}$, which is a contradiction since we know $\mathbb{R}^n \subset \overline{A}$.

Thus by contradiction, $A \cap (D(u,\varepsilon) \backslash \left\{ u \right\}) \neq \varnothing$ for any $\varepsilon>0$. This implies $A \cap (U \cap V \backslash \left\{ u \right\})$ is nonempty. Rearranged, this says $(A \cap U) \cap (V \backslash \left\{ u \right\})$ is nonempty for any open neighborhood $V$ of $u$. Then by definition, $u$ is an accumulation point of $A \cap U$, meaning $u \in \overline{A \cap U}$. This implies $U \subset \overline{A \cap U}$, so $A \cap U$ is dense in $U$.

The crux of this proof relied on the observation that no open holes can exist in $A$. Thus if $U$ had \textit{not} been open, it could have been a subset of the holes that $A$ does contain. Then the original claim ``$A \cap U$ is dense in $U$'' would become ``the empty set is dense in a nonempty set U", which is clearly false. Thus the requirement that $U$ be open was necessary.


% ========================================
\begin{exercise}{2.51}
	\
	\begin{enumerate}
		\item If $u_n > 0, n = 1,2,\dots$, show that
			\[
				\lim \inf \frac{u_{n+1}}{u_n} \leq \lim\inf \sqrt[n]{u_n} \leq \lim\sup \sqrt[n]{u_n}  \leq \lim\sup \frac{u_{n+1}}{u_n}.
			\] 
		\item Deduce that if $\lim(u_{n+1}/u_n)=A$, then $\lim\sup \sqrt[n]{u_n}  = A$.
		\item Show that the converse of part (b) is false by use of the seuqence $u_{2n}=u_{2n+1}=2^{-n}$.
		\item Calculate $\lim\sup  \frac{\sqrt[n]{n!}}{n}$.
	\end{enumerate}
	\hrulefill
\end{exercise}
\begin{enumerate}
	\item 
		Let $L = \lim\sup \frac{u_{n+1}}{u_n} $. If $L=\infty$, then the desired inequality is trivial, so assume $L < \infty$. Now fix $\varepsilon>0$, then there exists $N$ such that $u_n < L+\varepsilon$ whenever $n \geq N$. Then for $n \geq N$,
		\begin{align*}
			\frac{u_n}{u_N} = \frac{u_n}{u_{n-1}} \frac{u_{n-1}}{u_{n-2}} \cdots \frac{u_{N+1}}{u_{N}} \leq (L+\varepsilon)^{n-N}
		\end{align*}
		Rearranging gives
		\begin{align*}
			u_n &\leq (L+\varepsilon)^n \frac{u_N}{(L+\varepsilon)^N} \\
			\sqrt[n]{u_n} &\leq (L+\varepsilon) \left( \frac{u_N}{(L+\varepsilon)^N}  \right)^{1/n} 
		\end{align*}
		Since the fraction on the RHS is strictly greater than 0 and strictly less than 1, we get the final inequality
		\[
		\sqrt[n]{u_n} \leq (L+\varepsilon)
		\] 
		Taking the limit superior of both sides then gives
		\[
		\lim \sup \sqrt[n]{u_n} \leq (L+\varepsilon) = \lim \sup \frac{u_{n+1}}{u_n} 
		\] 
		Similarly, we can show 
		\[
                \lim \inf \frac{u_{n+1}}{u_n} \leq (L+\varepsilon) = \lim \inf \sqrt[n]{u_n}
		\]
		Combining these two inequalities with the fact that $\lim\inf x_n \leq \lim\sup x_n$ for any sequence $x_n$, we get the desired inequality
		\[
		\lim \inf \frac{u_{n+1}}{u_n} \leq \lim\inf \sqrt[n]{u_n} \leq \lim\sup \sqrt[n]{u_n}  \leq \lim\sup \frac{u_{n+1}}{u_n}.
		\] 

	\item $\lim(u_{n+1}/u_n)=A$ if and only if $\lim\inf (u_{n+1}/u_n) = \lim\sup (u_{n+1}/u_n) = A$, so we can use the inequality from part (a) to get
		 \[
			 A \leq \lim \sup \sqrt[n]{u_n} \leq A
		\] 
		which implies $\lim \sup \sqrt[n]{u_n} = A$.

	\item 
		For this series, $\sqrt[n]{u_n} = 1/2$ for all $n$, so clearly $\sup \sqrt[n]{u_n} =1/2$. However, $u_{n+1}/u_n$ alternates between $1/2$ and $1$, so there is clearly no limit. Thus the converse of part (b) is false.

	\item 
		Let $x_n = \frac{n!}{n^n} $, then
		\begin{align*}
			\lim_{n \to \infty} \frac{x_{n+1}}{x_n} &= \lim_{n \to \infty} \frac{(n+1)!}{(n+1)^{n+1}} \frac{n^n}{n!} \\
								&= \lim_{n \to \infty} \frac{n^n}{(n+1)^n} \\
								&= \lim_{n \to \infty} \left( \frac{n}{n+1} \right)^n \\
								&= \lim_{n \to \infty} \left( \frac{n+1}{n} \right)^{-n} \\
								&= \lim_{n \to \infty} \left( 1+\frac{1}{n}  \right)^{-n}
		\end{align*}
		By definition, this is $1/e$. Then by the result from part (b), we know
		\[
			\lim \sup \frac{\sqrt[n]{n!} }{n} = \frac{1}{e} .
		\] 
\end{enumerate}


\end{document}


