\documentclass[10pt]{amsart}
\usepackage{latexsym}
\usepackage{amscd,amsthm,amssymb,amsfonts,amsmath}
\usepackage{xcolor}
\usepackage[matrix,tips,graph,curve]{xy}

\newcommand{\mnote}[1]{${}^*$\marginpar{\footnotesize ${}^*$#1}}
\linespread{1.065}
\setlength{\parskip}{0.5em}

\makeatletter

\setlength\@tempdima  {5.5in}
\addtolength\@tempdima {-\textwidth}
\addtolength\hoffset{-0.5\@tempdima}
\setlength{\textwidth}{5.5in}
\setlength{\textheight}{8.75in}
\addtolength\voffset{-0.625in}

\newtheorem{manualtheoreminner}{}
\newenvironment{exercise}[1]{%
        \vspace{10mm}
        \renewcommand\themanualtheoreminner{#1}%
  \manualtheoreminner
}\hrulefill{\endmanualtheoreminner}

\newcommand{\p}{\partial}

\begin{document}
\title{Math 531 Homework 8}
\author{Braden Hoagland}
\date{\today}
\maketitle

\begin{exercise}{Page 210, Exercise 4.8.4}
	Let $f:[a,b] \subset \mathbb{R} \to \mathbb{R}$ be integrable and $f \leq M$. Prove that \[
		\int_{a}^{b} f(x) \;dx \leq (b-a) M.
	\] 
\end{exercise}

Since $f$ is Riemann integrable, we know $\int_{a}^{b} f(x) \;dx = \overline{\int_{a}^{b}} f(x) \;dx$. Expanding the definition of the upper intergral gives
\begin{align*}
	\int_{a}^{b} f(x) \;dx &= \overline{\int_{a}^{b}} f(x) \;dx \\
			       &= \inf_P \left\{ U(f,P) \right\}.
			       \intertext{Replacing the infimum over partitions with any fixed partition $P$, we get the bound}
				&\leq U(f, P) \\
			       &= \sum_{i=0}^{n-1} \sup_{x \in [x_i, x_{i+1}]} f(x) (x_{i+1}-x_i).
			       \intertext{Finally, we use the fact that $f$ is bounded above by $M$ to get}
			       &\leq \sum_{i=0}^{n-1} M (x_{i+1}-x_i) \\
			       &= (b-a) M.
\end{align*}
This is the desired bound.

%====================
\begin{exercise}{Page 211, Exercise 4.8.7}
	Let $f:[0,1] \to \mathbb{R}$, $f(x)=1$ if $x = 1/n$, $n$ an integer, and $f(x) = 0$ otherwise.
	\begin{enumerate}
		\item Prove that $f$ is integrable.
		\item Show that $\int_{0}^{1} f(x) \;dx = 0$.
	\end{enumerate}
\end{exercise}

\begin{enumerate}
	\item To show that $f$ is Riemann integrable, we must show that its upper and lower integrals are equal. First, note that any subinterval of $[0,1]$ contains an irrational number since the irrationals are dense in $\mathbb{R}$. Since $1/n$ is the form of a rational number, this means that every subinterval of $[0,1]$ contains at least one point $x$ satisfying $f(x)=0$. Thus for any partition $P$ of $[0,1]$, the lower sum is
		\begin{align*}
			L(f,P) &= \sum_{i=1}^{n-1} \inf_{x \in [x_i, x_{i+1}]} f(x) (x_{i+1}-x_i) \\
			       &= \sum_{i=1}^{n-1} 0 x_{i+1}-x_i) \\
			       &= 0.
		\end{align*}
		Thus $\sup_P \left\{ L(f,P) \right\} = \underline{\int_{0}^{1}} f(x) \;dx = 0$.

		Now fix $n \in \mathbb{N}$, and let $[x_0,x_1] = [0, \sqrt{2} /n]$. Construct a partition $P_n$ of $[0,1]$ containing $x_0$ and $x_1$ as its first two points such that all subsequent intervals have length no more than $1/n^2$. Then we can bound the upper sum as follows.
		\begin{align*}
			U(f,P_n) &= \sum_{i=0}^{n-1} \sup_{x \in [x_i, x_{i+1}]} f(x) (x_{i+1}-x_i) \\
				 &= \sup_{x \in [x_0,x_1]} (x_1-x_0) + \sum_{i=1}^{n-1} \sup_{x \in [x_i, x_{i+1}]} f(x) (x_{i+1}-x_i).
				 \intertext{We can bound every supremum term by 1, resulting in the upper bound}
				 &\leq (x_1-x_0) + \sum_{i=1}^{n-1} (x_{i+1}-x_i) \\
				 &\leq \frac{\sqrt{2} }{n} + \sum_{i=1}^{n-1} \frac{1}{n^2} \\
				 &= \frac{\sqrt{2} }{n} + \frac{n-1}{n^2} \\
				 &\leq \frac{\sqrt{2} +1}{n} .
		\end{align*}
		Taking the limit as $n$ increases, we get
		\[
			\lim_{n \to \infty} U(f,P_n) \leq \lim_{n \to \infty} \frac{\sqrt{2} +1}{n} = 0.
		\]
		Thus $\inf_P \left\{ U(f,P) \right\} = \overline{\int_{0}^{1}} f(x) \;dx = 0 = \underline{\int_{0}^{1}} f(x) \;dx$, so $f$ is Riemann integrable.

	\item Since both the lower and upper integral are 0, the Riemann integral $\int_{0}^{1} f(x) \;dx$ is also 0.
\end{enumerate}

%====================
\begin{exercise}{Page 211, Exercise 4.8.8}
	Let $f:[a,b] \to \mathbb{R}$ be Riemann integrable and $|f(x)| \leq M$. Let $F(x) = \int_{a}^{x} f(t) \;dt$. Prove that $|F(y) - F(x)| \leq M |y-x|$. Deduce that $F$ is continuous. Does this check with Example 4.8.10?
\end{exercise}

The case $x=y$ is trivial, so assume $x$ and $y$ are distinct. If $x< y$, then we have
\begin{align*}
	\left| F(y)-F(x) \right| &= \left| \int_{a}^{y} f(t) \;dt - \int_{a}^{x} f(t) \;dt \right| \\
				 &= \left| \int_{x}^{y} f(t) \;dt \right| \\
				 &= \left| \inf_P U(f,P) \right|
				 \intertext{where the infimum is over all partitions $P$ of $[x,y]$. Selecting an arbitrary partition $\tilde{P}$ of the form $\left\{ x_0=x, x_1, \dots, x_n = y \right\}$ gives the bound}
				 &\leq \left| U(f,\tilde{P}) \right| \\
				 &= \left| \sum_{i=0}^{n-1} \sup_{x \in [x_i, x_{i+1}]} f(x) (x_{i+1}-x_i) \right| \\
				 &\leq \sum_{i=0}^{n-1} \sup_{x \in [x_i, x_{i+1}]} |f(x)| \; (x_{i+1}-x_i) \\
				 &\leq \sum_{i=0}^{n-1} M (x_{i+1}-x_i) \\
				 &= M (y-x).
\end{align*}
So $|F(y)-F(x)| \leq M(y-x)$ when $x < y$. Similarly, if $y<x$, then $|F(y)-F(x)| \leq M(x-y)$. Putting these together yields the desired inequality
\[
	|F(y)-F(x)| \leq M|y-x|.
\] 
Now fix $\varepsilon>0$ and set $\delta= \varepsilon/M$. If $|y-x| < \delta$, then
\[
	|F(y)-F(x)| \leq M|y-x| < M \frac{\varepsilon}{M} = \varepsilon,
\] so $F$ is continuous on $[a,b]$. Furthermore, since $\delta$ does not depend on the specific $x$ and $y$ being used, $F$ is uniformly continuous on $[a,b]$.

This checks with Example 4.8.10, as continuity of a function is not enough to guarantee differentiability.

%====================
\begin{exercise}{Page 235, Exercise 4.41}
	Prove that
	\[
		\sum_{k=1}^{n} k = \frac{n(n+1)}{2} 
	\] and
	\[
		\sum_{k=1}^{n} k^2 = \frac{n(n+1)(2n+1)}{6} .
	\] 
\end{exercise}

We can rewrite $\sum_{k=1}^{n} k$ as
\begin{align*}
	\sum_{k=1}^{n} k &= \frac{1}{2} \left( \sum_{k=1}^{n} k + \sum_{k=1}^{n} k \right) \\
			 &= \frac{1}{2} \left( \sum_{k=1}^{n} k + \sum_{k=1}^{n} (n-k+1) \right) \\
			 &= \frac{n(n+1)}{2},
\end{align*}
as desired.

We can prove the second identity by induction. When $n=1$, we have $\sum_{k=1}^{n} k^2 = 1$ and $n(n+1)(2n+2) = 1(2)(3)/6 = 1$. Assuming the identity holds for some arbitrary $n$, we must show it holds for $n+1$. We have
\begin{align*}
	\sum_{k=1}^{n=1} k^2 &= (n+1)^2 + \sum_{k=1}^{n} k^2 \\
			     &= \frac{6(n+1)^2 +n(n+1)(2n+1)}{6} \\
			     &= \frac{(n+1)(2n^2 + 7n + 6)}{6} \\
			     &= \frac{(n+1)(n+2)(2n+3)}{6} \\
			     &= \frac{(n+1)( (n+1)+1) (2(n+1)+1)}{6},
\end{align*}
so the identity holds for all $n$.

%====================
\begin{exercise}{Page 236, Exercise 4.42}
	For $x>0$, define $L(x)= \int_{1}^{x} (1/t) \;dt$. Prove the following, using this definition:
	\begin{enumerate}
		\item $L$ is increasing in $x $.
		\item $L(xy) = L(x) + L(y)$.
		\item $L'(x) = 1/x$.
		\item $L(1) = 0$.
		\item Properties \textbf{c} and \textbf{d} uniquely determine $L$. What is $L$?
	\end{enumerate}
\end{exercise}

\begin{enumerate}
	\item Let $x' \geq x > 0$, then
		\begin{align*}
			L(x') - L(x) &= \int_{1}^{x'} \frac{1}{t} \;d t - \int_{1}^{x} \frac{1}{t} \;d t \\
				     &= \int_{1}^{x} \frac{1}{t} \;d t + \int_{x}^{x'} \frac{1}{t} \;d t - \int_{1}^{x} \frac{1}{t} \;d t \\
				     &= \int_{x}^{x'} \frac{1}{t} \;d t.
		\end{align*}
		Since $1/t \geq 0$ for all $t$ and $x' \geq x$, this integral is nonzero, so $L$ is increasing in $x$.

	\item We have
		\begin{align*}
			L(xy) &= \int_{1}^{xy} \frac{1}{t} \;d t \\
			      &= \int_{1}^{x} \frac{1}{t} \;d t + \int_{x}^{xy}\frac{1}{t} \;d t.
			      \intertext{Now let $u = t/x$, then this becomes}
			      &= \int_{1}^{x} \frac{1}{t} \;d t + \int_{1}^{y} \frac{1}{u} \;du \\
			      &= L(x) + L(y),
		\end{align*}
		as desired.
	
	\item Let $G$ be any antiderivative of $1/x$, then the derivative of $L$ is
		\begin{align*}
			L'(x) &= \frac{d }{d x} \int_{1}^{x} \frac{1}{t} \;d t \\
			      &= \frac{d }{d x} \left[ G(x) - G(1) \right] \\
			      &= \frac{1}{x} + 0,
		\end{align*}
		so $L'(x) = 1/x$.

	\item Let $G$ again be any antiderivative of $1/x$, then $L(1)$ is
		\begin{align*}
			L(1) &= \int_{1}^{1} \frac{1}{t} \;d t \\
			     &= G(1) - G(1) \\
			     &= 0,
		\end{align*}
		so $L(1) = 0$.

	\item $L$ is $\ln$. In class we showed that the derivative of $\ln(x)$ is $1/x$, so by $(c)$, $L(x) = \ln(x) + C(x)$, where $C'(x)=0$. But by $(d)$, we know $L(1) = \ln(1) + C(1) = C(1) = 0$. Since $C(1)$ is just a constant, this means all terms in $C$ must be 0. Thus $L(x) = \ln(x)$.
\end{enumerate}


%====================
\begin{exercise}{Page 236, Exercise 4.44}
	Let $f:[0,1] \to \mathbb{R}$ be Riemann integrable and suppose for every $a,b$ with $0 \leq a < b \leq 1$ there is a $c$ with $a < c < b$ and $f(c) = 0$. Prove $\int_{0}^{1} f = 0$. Must $f$ be zero? What if $f$ is continuous?
\end{exercise}

Consider any partition $P$ of $[0,1]$ of the form $\left\{ x_0=0, x_1, \dots, x_n=1 \right\}$. By assumption, the interval $[x_i, x_{i+1}]$ contains a point $c_i$ such that $f(c_i) = 0$. Thus the infimum of $f$ on this interval is 0.

Since $f$ is Riemann integrable, the value of the integral is equal to $\sup_P L(f,P)$. Expanding the lower sum into its full definition gives
\[
	\int_{0}^{1} f(x) \;dx  = \sup_P \sum_{i=0}^{n-1} \inf_{[x_i, x_{i+1}]} f(x) (x_{i+1} - x_i).
\] We just showed that each infimum term in this summation is 0, so $L(f,P) = 0$ for every partition $P$. Since it is true for every partition, $\sup_P L(f,P) = 0$ as well, so $\int_{0}^{1} f$ must itself be 0.

With no further constraints, $f$ need not be the zero function. Consider the function
\[
	f(x) =
	\begin{cases}
		0 & x \in \mathbb{Q} \\
		1 & \text{otherwise}
	\end{cases}
\] defined for all $x \in [0,1]$. Since $\mathbb{Q}$ is dense in $\mathbb{R}$, this function satisfies the conditions from the original problem, and thus $\int_{0}^{1} f = 0$ even though $f$ in this case is clearly not the zero function.

If $f$ is continuous, however, it must be the zero function, which we show by contradiction. To begin, note that since $f$ is continuous and it is defined on the compact set $[0,1]$, $f$ is in fact uniformly continuous on $[0,1]$. Thus for all $\varepsilon>0$, there exists $\delta>0$ such that $|f(y)-f(x)| < \varepsilon$ when $|y-x| < \delta$.

Suppose $f(z) \neq 0$ for some $z \in [0,1]$, i.e. $|f(z)| = \varepsilon$ for some $\varepsilon>0$. Now take $x \in [0,1]$ such that $|z-x| < \delta$, then $|f(x)-f(z)| < \varepsilon$. This means that $f(x)$ is not 0 either. Moreover, $f(x)$ must be the same sign as $f(z)$. If $x<z$, consider the interval $[x,z]$, and if $z < x$, consider the interval $[z,x]$. For all $y$ in this interval, $|y-z| < \delta$, so $|f(x)-f(z)| < \varepsilon$. Since $f(x)$ and $f(z)$ are the same sign, $f(y)$ is also nonzero. We have found an interval with no roots of $f$, which contradicts the assumption that all intervals of $[0,1]$ contain a root of $f$. Thus $f(z)=0$ for all $z \in [0,1]$, i.e. $f$ is the zero function.

%====================
\begin{exercise}{Page 236, Exercise 4.45}
	Prove the following \textbf{second mean value theorem}. Let $f$ and $g$ be defined on $[a,b]$ with $g$ continuous, $f \geq 0$, and $f$ integrable. Then there is a point $x_0 \in (a,b)$ such that
	\[
		\int_{a}^{b} f(x) g(x) \;dx = g(x_0) \int_{a}^{b} f(x) \;dx.
	\] 
\end{exercise}

Let $m = \inf\left\{ g([a,b]) \right\}$, and let $M = \sup\left\{ g([a,b]) \right\}$, then clearly $m \leq g(x) \leq M$ for all $x \in [a,b]$. Since $f \geq 0$, the integral $\int_{a}^{b} f$ is also non-negative. Thus we have
\[
	m \int_{a}^{b} f(x) \;dx \leq \int_{a}^{b} f(x)g(x) \;dx \leq M \int_{a}^{b} f(x) \;dx.
\] 

Now consider $h(t) = t \int_{a}^{b} f(x) \;dx$. For fixed $f$, this is continuous with respect to $t$ (it is just a linear function). Since $\int_{a}^{b} f(x)g(x) \;dx$ lies in the connected set $[h(m), h(M)]$, by the intermediate value theorem we know there exists $t_0 \in [m,M]$ such that
\[
	t_0 \int_{a}^{b} f(x) \;dx = \int_{a}^{b} f(x) g(x) \;dx.
\]

Since $[a,b]$ is connected, $g$ is continuous, and $m \leq t_0 \leq M$, by the intermediate value theorem again we know there exists $x_0 \in [a,b]$ such that $g(x_0) = t_0$. Thus we have
\[
	t_0 \int_{a}^{b} f(x) \;dx = g(x_0) \int_{a}^{b} f(x) \;dx = \int_{a}^{b} f(x) g(x) \;dx,
\] which gives us the desired equality.


\end{document}
