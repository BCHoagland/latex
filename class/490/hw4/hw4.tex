\documentclass{article}
\usepackage[utf8]{inputenc}
\usepackage[a4paper, total={6.5in, 9in}]{geometry}
\usepackage{fancyhdr}
\usepackage{amsmath, amsthm, amsfonts, mathrsfs}
\usepackage{xcolor}
\usepackage{mathtools, float, subfig}
\renewcommand{\labelenumii}{(\roman{enumii})}

\newcommand{\rr}{\mathscr{R}_0}

\setlength{\headsep}{4em}
\pagestyle{fancy}
\setlength{\parindent}{0cm}
\setlength{\parskip}{1em}

% ASSIGNMENT INFORMATION
\newcommand{\class}{MATH 490}
\newcommand{\hwnumber}{4}

\lhead{Braden Hoagland (bch29)}
\chead{{\class} - HW {\hwnumber}}
\rhead{\today}

\begin{document}


\begin{enumerate}
	\item
		Taking the derivative of the proposed solution $\mathbf{y}(t) = \sum_{i=1}^{k} c_i e^{\lambda}$ gives
		\begin{align*}
			\mathbf{y}'(t) &= \sum_{i=1}^{k} c_i \lambda_i \mathbf{v}_i e^{\lambda_i t} \\
			\intertext{Since $\mathbf{v}_i$ is the eigenvector corresponding to eigenvalue $\lambda_i$, this becomes}
				       &= \sum_{i=1}^{k} c_i A \mathbf{v}_i e^{\lambda_i t} \\
				       &= A \sum_{i=1}^{k} c_i \mathbf{v}_i e^{\lambda_i t} \\
				       &= A \mathbf{y}(t)
		\end{align*}
		Thus $\mathbf{y}$ is a solution of $\mathbf{y}'=A\mathbf{y}$.
	
	\item 
	\begin{enumerate}
		\item
			The differential of $\mathbf{F}$ is
			\[
				D\mathbf{F}=
				\begin{pmatrix}
					\frac{d \mathbf{F}_1}{d S} & \frac{d \mathbf{F}_1}{d E} & \frac{d \mathbf{F}_1}{d I} & \frac{d \mathbf{F}_1}{d R} \\
					\frac{d \mathbf{F}_2}{d S} & \ddots & & \vdots \\
					\frac{d \mathbf{F}_3}{d S} & & \ddots & \vdots \\
					\frac{d \mathbf{F}_4}{d S} & \frac{d \mathbf{F}_4}{d E} & \frac{d \mathbf{F}_4}{d I} & \frac{d \mathbf{F}_4}{d R}
				\end{pmatrix}=
				\begin{pmatrix}
					-\beta I & 0 & -\beta S & 0 \\
					\beta I & -a & \beta S & 0 \\
					0 & a & -\gamma & 0 \\
					0 & 0 & \gamma & 0
				\end{pmatrix}
			\] 

		\item 
			The particular matrix $D \mathbf{F}(y_0)$, where $y_0=(1-v,0,0,v)$, is
			\[
				D\mathbf{F}(y_0) =
				\begin{pmatrix}
					0 & 0 & -\beta(1-v) & 0 \\
					0 & -a & \beta(1-v) & 0 \\
					0 & a & -\gamma & 0 \\
					0 & 0 & \gamma & 0
				\end{pmatrix}
			\] 

		\item 
			When $v=0$, the submatrix $M_0$ corresponding to the $E$ and $I$ rows is
			\[
			M_0=
			\begin{pmatrix}
				-a&\beta\\
				a&-\gamma
			\end{pmatrix}
			\] 
			The trace of $M_0$ is $\text{tr}(M_0)= \lambda_1+\lambda_2=-a-\gamma$, and the determinant is $\text{det}(M_0)=\lambda_1 \lambda_2=a(\gamma-\beta)$. Since $a$ and $\gamma$ are both positive, the trace must be negative. There are then two possible cases
			\begin{itemize}
				\item If the determinant is positive, then both eigenvalues have the same sign. Since two positives cannot add together to yield a negative, both eigenvalues must be positive.
				\item If the determinant is negative, then the eigenvalues are opposite signs. This means one is positive and one is negative.
			\end{itemize}
			If $\beta<\gamma$, then the determinant $a(\gamma-\beta)$ is positive. Thus both eigenvalues are negative. If $\beta>\gamma$ instead, then the determinant is negative and we have one positive and one negative eigenvalue.

		\item 
			For a linear homogeneous ODE $\mathbf{y}'=A\mathbf{y}$, the origin is unstable if $\text{Re}(\lambda_i) > 0$ for any eigenvalue $\lambda_i$ of $A$. For the given two-dimensional system $\mathbf{z}'=M_v \mathbf{z}$, the matrix
			\[
			M_v =
			\begin{pmatrix}
				-a & \beta(1-v) \\
				a & -\gamma
			\end{pmatrix}
			\] 
			has trace $\lambda_1+\lambda_2 = -a-\gamma < 0$ and determinant $\lambda_1 \lambda_2 = a(\gamma-\beta(1-v))$. We can now use the cases from the previous question to determine the signs of the eigenvalues of $M_v$.
		
			As before, since the trace is negative, a negative determinant implies eigenvalues with opposite signs, which is the only way to a positive eigenvalue for this matrix. Thus we have
			\begin{align*}
				\text{Re}(\lambda_i) \text{ for some } i &\iff \det(M_v) \\
									 &\iff a(\gamma-\beta(1-v)) < 0 \\
										 \intertext{Since $a>0$, this becomes}
									 &\iff \gamma - \beta(1-v)<0
			\end{align*}
			If $v=1$, then we have the inequality $\gamma<0$, which is always false. This means that if $v=1$, then the origin is stable. If $v\neq 1$, then the inequality becomes $\beta>\gamma/(1-v)$. So if $\beta$ is greater than this threshold, i.e. $\beta\in(\gamma/(1-v), \infty)$, the origin is unstable.

		\item 
			If the flow rate from $S$ to $E$ is high enough, i.e. enough people get exposed to the disease on a regular basis, then the disease will persist. This threshold increases as $\gamma$ (the flow rate from $I$ to $R$) increases. This makes sense, as a faster recovery rate would lead to fewer overall cases). The threshold also increases as $v$ increases. This also makes sense, as an increased proportion of vaccinated individuals will lead to fewer people being susceptible.

			The threshold does \textit{not} depend on the latency parameter $a$. Thus when determining if a disease will die out (stable equilibrium at 0 cases) or continue to grow (unstable equilibrium at 0 cases), what matters is the rate of exposure, not the rate of infection for those already exposed.
	\end{enumerate}
\end{enumerate}	


\end{document}

