\documentclass{article}
\usepackage[utf8]{inputenc}
\usepackage[a4paper, total={6.5in, 9in}]{geometry}
\usepackage{fancyhdr}
\usepackage{amsmath, amsthm, amsfonts}
\usepackage[mathscr]{euscript}
\let\euscr\mathscr \let\mathscr\relax
\usepackage{xcolor}
\usepackage{mathtools}

% \renewcommand{\headrulewidth}{0pt}
\newtheorem{defn}{Definition}
\newtheorem{prop}{Proposition}
\newtheorem{thrm}{Theorem}
\newtheorem{corr}{Corollary}

\setlength{\headsep}{4em}
\pagestyle{fancy}
\setlength{\parindent}{0cm}

% ASSIGNMENT INFORMATION
\newcommand{\class}{MATH 490}
\newcommand{\hwnumber}{1}

\lhead{Braden Hoagland (bch29)}
\chead{{\class} - HW {\hwnumber}}
\rhead{\today}

\begin{document}

\section{Simple growth model}
\begin{enumerate}
	\item[a.]
		This is false. If $y$ is negative, then $y'$ will also be negative. This means that $y$ will be decreasing in $t$. In fact, whether or not the solution is increasing or decreasing in $t$ depends on the initial condition $y (0)$. Since we know the solution to be $y(0) e^{rt}$, a negative $y(0$ will result in a solution that decreases monotonically in $t$.
	\item[b.] If $y = c$ is a solution, then then $y'=0$, meaning $ry=rc=0$. Since $r\neq 0$, this means $c=0$.
	\item[c.]
		\begin{align*}
			\frac{d y}{d t} &= r y(t) \\
			\frac{1}{y(t)} \frac{d y}{d t} &= r \\
			\frac{d }{d t} \ln y(t) &= r
		\end{align*}
		Thus $G = \ln y(t)$.
	\item[d.] The FTC says that for function $f$ with anti-derivative $F$, $\int_{a}^{b} f(x) dx = F(b) - F(a)$. This gives
		\begin{align*}
			\frac{d}{d t} \ln y(t) &= r \\
			\int_{0}^{T} \frac{d}{d t} \ln y(t) dt &= \int_{0}^{T} r dt \\
			\ln y(T) - \ln y(0) &= r T \\
			\ln \left( \frac{y(T)}{y(0)} \right) &= r t \\
			y(T) &= y(0) e^{rT}
		\end{align*}
	\item[e.] All solutions to the ODE are exponentials, differing only by the initial condition $y(0)$ and rate $r$.
	\item[f.] As $r$ increases linearly, $y(t)$ increases exponentially.
	\item[g.]
		\begin{align*}
			y(t) &= 2 y_0 \\
			y_0 e^{rt} &= 2 y_0 \\
			t &= \frac{\ln 2}{r}
		\end{align*}
		At $t=1/r$, $y$ evaluates to $y(1/r) = y_0 e$.
	\item[h.] The precision at time $T$ is the length of the interval $\left[ \left(b-\frac{\varepsilon}{2}\right) e^{rT}, \left(b+\frac{\varepsilon}{2}\right) e^{rT} \right]$, which is $\varepsilon e^{rT}$. To have precision $\delta$, we must set $\varepsilon$ to be $\varepsilon = \frac{\delta}{e^{rT}}$.
\end{enumerate}

\section{Logistic Growth}
\begin{enumerate}
	\item[a.] Since $r>0$, we can divide both sides of the inequality by it to get
		\begin{align*}
			r\left(1-\frac{y}{K}\right) y &> 0 \\
			y - \frac{y^2}{K} &> 0
		\end{align*}
		This only holds when $0<y<K$. Thus $f(y) > 0$ when $0<y<K$.
		When $f(y) < 0$, the inequality is flipped. Thus $f(y) < 0$ when $y<0$ or $y>K$.
	\item[b.] A constant function is a solution to the ODE when $y' =0$, so
		\begin{align*}
			r \left( 1 - \frac{y}{K} \right) y &= 0 \\
			r \left( 1 - \frac{c}{K} \right) c &= 0
		\end{align*}
		which is clearly true only when $c=0$ or $c=K$. Interpreted in the context of logistic growth, this says that a population stagnates in size when there are either no people left or when the maximum resource consumption limit $K$ has been reached exactly.
	\item[c.] When $0 < y(t^*) < K$, the conditions from part (a) show that $y$ is increasing.
	\item[d.] Since $0 < y(0) < K$, $y$ will initially increase. As this occurs, the value of $y$ will approach $K$, at which point we know that $y'$ becomes 0. Thus we expect the rate of change of $y$ to slow down as $y(t)$ approaches $K$, convering to $K$ as $t \to \infty$.
	\item[e.]
		When $y(0) = 2K > K$, $y$ will initially decrease. Based on similar reasoning as before, we expect the magnitude of the rate of change of $y$ to decrease as $y$ approaches $K$. As $t \to \infty$, we expect $y(t) \to K$.

		When $y(0) < 0$, the conditions from part (a) show that $y'$ is always negative. Thus as $t \to \infty$, we expect $y(t)$ to diverge to $-\infty$.
\end{enumerate}

\section{Explosive Growth}
\begin{enumerate}
	\item[a.]
		As $t$ increases linearly, $y(t)$ increases very rapidly. This rate increases quadratically as time increases linearly. Unlike with the solution of $(0.1)$, it is impossible for this ODE to have negative growth. Another, more obvious, comparison is that this new ODE has much faster growth than the solution of $(0.1)$.
	
	\item[b.]
		\begin{align*}
			\frac{d y}{d t} &= y^2 \\
			\frac{1}{y^2} \frac{d y}{d t} &= 1 \\
			\frac{d }{d t} \left( - \frac{1}{y} \right) &= 1
		\end{align*}
		Thus $G(y(t)) = -\frac{1}{y(t)}$.
	
	\item[c.]
		Applying the FTC gives
		\begin{align*}
			\frac{d }{d t} \left( - \frac{1}{y} \right) &= 1 \\
			-\int_{0}^{T} \left( \frac{d }{d t} \frac{1}{y(t)} \right) dt  &= \int_{0}^{T} dt \\
			-\frac{1}{y(T)} + \frac{1}{y(0)} &= T \\
			y(T) &= \frac{y(0)}{1 - T y(0)}
		\end{align*}
		This cannot be a solution for all $t > 0$ since this function is undefined at $T = \frac{1}{y(0)}$. As $T$ approaches this value, we have a vertical asymptote as the function diverges to $\infty$.

	\item[d.]
		If $y(0) = 1$, there is no solution for all $t > 0$. In this case, the potential solution
		\[
			y(t) = \frac{1}{1 - t}
		\] 
		has a discontinuity at $t=1$. After this point in time, the function suddenly becomes negative, which we know to be incorrect behavior since the rate of change of $y$ is always positive.
\end{enumerate}

\section{Linear Inhomogeneous}
\begin{enumerate}
	\item[a.]
		The function $w(t) = y(t) - z(t)$ has derivative
	 	\begin{align*}
			 w' &= y' - z' \\
			    &= ry + k - rz - k \\
			    &= r(y-z) \\
			    &= rw
		\end{align*}
		which is the desired form.
	
	\item[b.]
		When trying to solve the ODE with the strategy from $(0.1)$, we find that the $y$-dependent terms cannot be expressed as the derivative of any function. What we must do instead is introduce an integrating factor.

	\item[c.]
		\begin{align*}
			\frac{d }{d t} (e^{-rt} y(t)) &= e^{-rt} y' - r e^{-rt} y \\
						      &= e^{-rt} (y' - ry) \\
						      &= e^{-rt} (ry + k(t) - ry) \\
						      &= k(t) e^{-rt}
		\end{align*}
	
	\item[d.]
		Applying the FTC to this relation gives
		\begin{align*}
			\int_{0}^{T} \left( \frac{d }{d t} (e^{-rt} y(t)) \right) dt &= \int_{0}^{T} k(t) e^{-rt} dt \\
			e^{-rT} y(T) - y(0) &= \int_{0}^{T} k(t) e^{-rt} dt \\
			y(T) &= e^{rT} \left[ y(0) + \int_{0}^{T} k(t) e^{-rt} dt \right]
		\end{align*}
\end{enumerate}

\section{Simple linear system}
\begin{enumerate}
	\item[a.]
		A simple system that these equations might describe is two friendly populations, where the first population supports the growth of the second. A toy example might be two small villages, where one village produces resources that benefit the second (perhaps the first village is a colony of the second). Both villages have their own base growth rates ($r$ and $\alpha$), but the first village's resources allow the second village to grow faster (the additional $\beta$ term). Another interpretation from this two-village example might be the reassignment of people. If the first vilage in fact grows at a rate of $r + \beta$ but ships $\frac{\beta}{r + \beta}$ of its newborns to the second village, then the first village will have the proper growth rate in the equation.
	
	\item[b.]
		The general solution for $y$ is $y(t) = y(0) e^{rt}$.

	\item[c.]
		We can use a similar strategy as with the linear inhomogeneous system. Let $k(t) \doteq \beta y_0 e^{rt}$, then we have
		\begin{align*}
			z' &= \alpha z + \beta y_0 e^{rt} \\
			z' - \alpha z &= k(t) \\
			\frac{d }{d t} e^{-\alpha t} z &= k(t) e^{-\alpha t} \\
			\int_{0}^{T} \left( \frac{d }{d t} e^{-\alpha t} z(t) \right) dt &= \int_{0}^{T} k(t) e^{-\alpha t} dt \\
			e^{-\alpha t} z(T) - z(0) &= \beta y(0) \int_{0}^{T} e^{rt} e^{-\alpha t} dt \\
			z(T) &= e^{\alpha T} \left[ z(0) + \beta y(0) \int_{0}^{T} e^{(r-\alpha) t} dt \right]
		\end{align*}

	\item[d.]
		Setting $z(0) = 0$, this becomes
		\[
		z(T) = \beta y(0) e^{\alpha T} \int_{0}^{T} e^{(r-\alpha) t} dt
		\] 
		When $y(0) > 0$, the parameters have the following qualitative effects on $z(t) $ 
		\begin{itemize}
			\item $\beta$ : As $\beta$ increases linearly, so does $z(t)$.
			\item $r$, $\alpha$ : $r$ and $\alpha$ balance each other out in this final equation. If $r$ is larger than $\alpha$, then we see an exponential increase in $z(t)$. As $\alpha$ increases, this growth becomes smaller. The function cannot be made arbitrarily small by increasing $\alpha$, however, due to the presence of the $e^{\alpha T}$ term outside of the integral. In fact, fixing the other parameters and increasing $\alpha$ still results in exponential growth, just at a lower rate than if we had just increased $r$.
		\end{itemize}
\end{enumerate}
\end{document}

