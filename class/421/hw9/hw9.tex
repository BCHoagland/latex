\documentclass[10pt]{report}
\usepackage{/Users/bradenhoagland/latex/math}

\lhead{Braden Hoagland}
\chead{HW 9}
\rhead{}

\renewcommand{\theenumi}{\alph{enumi}}

\begin{document}
%\tableofcontents

\begin{exer}[7.2: 1]
The Poincare half-plane has $K=-1$.
\end{exer}
Since $\left\langle \mathbf{v},\mathbf{w} \right\rangle = (\mathbf{v}\cdot \mathbf{w})/v^2$,
\begin{align*}
	E &= \left\langle \mathbf{x}_{u},\mathbf{x}_{u} \right\rangle = 1/v^2, \\
	F &= \left\langle \mathbf{x}_{u},\mathbf{x}_{v} \right\rangle = 0, \\
	G &= \left\langle \mathbf{x}_{v},\mathbf{x}_{v} \right\rangle = 1/v^2.
\end{align*}
Then
\begin{align*}
	(\sqrt{G} )_{u}&=(1/v)_{u}=0\\
	(\sqrt{E} )_{v}&=(1/v)_{v}=-1/v^2,
\end{align*}
so by the formula on page 297,
\begin{align*}
	K &= -\frac{1}{\sqrt{EG} } \left[ \left( \frac{(\sqrt{G} )_{u}}{\sqrt{E} }  \right)_{u}+ \left( \frac{(\sqrt{E} )_{v}}{\sqrt{G} }  \right)_{v} \right] \\
	  &= -v^2 \left[ 0 + (-1/v)_{v} \right] \\
	  &= -v^2 (1/v^2) \\
	  &= -1.
\end{align*}

\pagebreak
\begin{exer}[7.2: 2]
	Find the dual forms, connection forms, and $K$ for the conformal structure on the entire plane with $h = \text{sech}(uv)$.
\end{exer}
If we let $T = \text{tanh}(uv)$ and $S = \text{sech}(uv)$, then $\theta_1 = du/S$ and $\theta_2 = dv/S$.
\begin{align*}
	d\theta_1 &= d(1/S) \wedge du = \frac{T u}{S} du\wedge dv, \\
	d\theta_2 &= d(1/S) \wedge dv = -\frac{T v}{S} du\wedge dv.
\end{align*}
Then by the first structural equations,
\begin{align*}
	d\theta_1 &= \omega_{12}\wedge \theta_2 \\
	(Tu/S)du \wedge dv &= \omega_{12} dv/S
\end{align*}
and
\begin{align*}
	d\theta_2 &= -\omega_{12} \wedge \theta_1 \\
	(-Tv/S) du \wedge dv &= du/S \wedge \omega_{12}.
\end{align*}
This implies
\[
\omega_{12} = Tu \;du - Tv\;dv.
\] 
Then by the second structural equation,
\begin{align*}
	d\omega_{12}&=-K\theta_1\wedge \theta_2 \\
	-S^2(u^2+v^2)\;du \wedge dv &= -\frac{K}{S^2} \;du \wedge dv,
\end{align*}
so $K = S^4(u^2+v^2)$. Using Corollary 2.3 gives the same result: we calculate
\[
	h_{u}=-TSv, \quad h_{v}=-TSu, \quad h_{uu}=-Sv^2(S^2-T^2), \quad h_{vv}=-Su^2(S^2-T^2).
\] Then by the corollary,
\begin{align*}
	K &= h(h_{uu}+h_{vv}) - (h_{u}^2+h_{v}^2) \\
	  &= -S^2 \left[ (u^2+v^2)(S^2-T^2)-(T^2v^2 + t^2u^2) \right]\\
	  &= S^2 \left[ -v^2S^2 - u^2S^2 \right] \\
	  &= -S^4 (u^2+v^2).
\end{align*}

\pagebreak
\begin{exer}[7.2: 3]
	Find the area of the disk $u^2+v^2 \leq a^2$ in the hyperbolic plane.
\end{exer}
With the frame $hU_1, hU_2$, we have the dual frame $\theta_1 = du/h, \theta_2=dv/h$. The area form is then given by $\theta_1 \wedge \theta_2 = (1/h^2) \;du\wedge dv$. Converting to polar coordinates, the area form becomes
\[
\frac{r}{h^2}\; dr \wedge d\theta,
\] so the area of the disk is
\begin{align*}
	\int_{0}^{2\pi} \int_{0}^{a} \frac{r}{h^2} \;dr\;d\theta &= \int_{0}^{2\pi} \left[ \frac{2}{h}  \right]^{r=a}_{r=0}\;dv \\
								 &= 4\pi\left( \frac{1}{1-a^2/4} -1 \right) \\
								 &= \frac{\pi a^2}{1-\frac{a^2}{4} } .
\end{align*}
Then since the entire hyperbolic disk has radius 2, its area is the limit of this expression as $a \to 2$, which is $\infty$.

\pagebreak
\begin{exer}[7.2: 4]
	$H(r)$ has constant Gaussian curvature $K=-1/r^2$.
\end{exer}
Since $h = 1 - \frac{u^2+v^2}{4r^2}$, we calculate
\begin{align*}
	h_{u} &= -\frac{u}{2r^2}, \quad h_{uu} = -\frac{1}{2r^2}, \\
	h_{v} &= -\frac{v}{2r^2}, \quad h_{vv} = -\frac{1}{2r^2} .
\end{align*}
Then by Corollary 2.3,
\begin{align*}
	K &= h(h_{uu}+h_{vv})-(h_{u}^2+h_{v}^2) \\
	  &= \left( 1-\frac{u^2+v^2}{4r^2}  \right)\left( -\frac{1}{r^2}  \right)- \frac{u^2+v^2}{4r^4} \\
	  &= \frac{u^2+v^2-4r^2}{4r^4} - \frac{u^2+v^2}{4r^4} \\
	  &= -\frac{1}{r^2} .
\end{align*}

\pagebreak
\begin{exer}[7.2: 7]
Scale changes.
\end{exer}
\begin{enumerate}
	\item The norm on the scaled surface is
		\[
		{\Vert{\mathbf{v}}\Vert}^{-} = \sqrt{\left\langle \mathbf{v},\mathbf{v} \right\rangle^{-}} = \sqrt{c^2 \left\langle \mathbf{v},\mathbf{v} \right\rangle} = c{\Vert{\mathbf{v}}\Vert}.
		\] Then if $\theta$ is the angle between $\mathbf{v},\mathbf{w} \in M$ and $\overline{\theta} $ is the corresponding angle in $\overline{M}$,
		\[
		\cos \overline{\theta} = \frac{\left\langle \mathbf{v},\mathbf{w} \right\rangle}{{\Vert{\mathbf{v}}\Vert}^{-}{\Vert{\mathbf{w}}\Vert}^{-}} = \frac{c^2\left\langle \mathbf{v},\mathbf{w} \right\rangle}{c^2 {\Vert{\mathbf{v}}\Vert}{\Vert{\mathbf{w}}\Vert}} = \cos \theta.
		\] Thus angles are preserved.

	\item The length of $\alpha$ in $\overline{M}$ is
		\[
			\overline{L}(\alpha) = \int_{\alpha} {\Vert{\alpha'}\Vert}^{-} = c \int_{\alpha} {\Vert{\alpha'}\Vert} = c L(\alpha).
		\] 

	\item We need $\overline{\theta}_i(\overline{E}_{j})= \overline{\theta}_i(E_j)/c =\delta_{ij}$, so $\overline{\theta}_i = c\theta_i$. Then by the first structural equations,
		\[
			c(\overline{\omega}_{12} \wedge \theta_2) = \overline{\omega}_{12} \wedge \overline{\theta}_2 = d \overline{\theta}_1 = c\;d\theta_1 = c(\omega_{12}\wedge \theta_2),
		\] so $\overline{\omega}_{12}=\omega_{12}$.

	\item The area form on $\overline{M}$ is
		\[
			d\overline{M} = \overline{\theta}_1 \wedge \overline{\theta}_2 = c^2\;\theta_1 \wedge \theta_2 = c^2 \;dM,
		\] so the area of $\mathscr{R}$ is
		\[
		\int_{\mathscr{R}} d\overline{M} = c^2 \int_{\mathscr{R}} dM = c^2 A.
		\] 
		Thus $\mathscr{R}$ has area $A$ in $M$ if and only if it has area $c^2 A$ in $\overline{M}$.

	\item By the second structural equation and part (c),
		\[
			-c^2 \overline{K} \; \theta_1 \wedge \theta_2 = -\overline{K}\;\overline{\theta}_1 \wedge \overline{\theta}_2 = d \overline{\omega}_{12} = d\omega_{12} = -K\;\theta_1\wedge \theta_2.
		\] Thus $\overline{K}=K/c^2$.
\end{enumerate}

\pagebreak
\begin{exer}[7.2: 8]
	\begin{enumerate}
		\item $S(r)$ and $\Sigma$ scaled by $r$ are isometric.
		\item $H(r)$ and $H(1)$ scalred by $r$ are isometric.
	\end{enumerate}
\end{exer}
\begin{enumerate}
	\item
		Based on the standard parameterization of $S(r)$
		\[
			\mathbf{x} = (r \sin u, \cos v,r\sin u \sin v,r \sin u),
		\] we get partials
		\begin{align*}
			\mathbf{x}_{u} &= (r\cos u\cos v,r\cos u\sin v,r\cos u) \\
			\mathbf{x}_{v} &= (r\sin u\sin v, r\sin u \cos v,0).
		\end{align*}
		We can then calculate
		\[
		\mathbf{x}_{u}\cdot \mathbf{x}_{u} = 2r^2\cos^2 u, \quad \mathbf{x}_{u}\cdot \mathbf{x}_{v} = 0, \quad \mathbf{x}_{v}\cdot \mathbf{x}_{v} = r^2\sin^2 u.
		\] 
		From this we see that the dot product between \textit{any} two tangent vectors on $S(r)$ is a linear combination of these terms. Now define the map
		\begin{align*}
			F:S(r) &\to \Sigma \text{ scaled by } r \\
			\mathbf{p}&\mapsto \mathbf{p}/r.
		\end{align*}
		This is clearly bijective. Additionally, in $\Sigma$ scaled by $r$ we have
		\begin{align*}
			F_{*}\mathbf{x}_{u}\cdot F_{*}\mathbf{x}_{u} &= r^2 (2 \cos^2 u), \\
			F_{*}\mathbf{x}_{v}\cdot F_{*}\mathbf{x}_{v} &= r^2 (\sin^2 u),
		\end{align*}
		which are equivalent to $\mathbf{x}_{u} \cdot \mathbf{x}_{u}$ and $\mathbf{x}_{v} \cdot \mathbf{x}_{v}$ in $S(r)$. Thus $F$ is an isometry.

	\item We claim that the map
		\begin{align*}
			F:H(r)&\to H(1) \text{ scaled by } r\\
			\mathbf{p}&\mapsto \mathbf{p}/r
		\end{align*}
		is an isometry. It is clearly bijective, so we must show that it is metric preserving. Note that $F_{*}$ maps $\mathbf{v} \mapsto \mathbf{v}/r$. Then for all $\mathbf{v},\mathbf{w}$,
		\begin{align*}
			\left\langle F_{*}\mathbf{v},F_{*}\mathbf{w} \right\rangle &= r^2\frac{(1/r^2)\mathbf{v}\cdot \mathbf{w}}{\left(1 - \frac{(u^2+v^2)(1/r^2)}{4}\right)^2} \\
			&= \frac{\mathbf{v}\cdot \mathbf{w}}{\left(1 - \frac{(u^2+v^2)}{4r^2}\right)^2} \\
			&= \left\langle \mathbf{v},\mathbf{w} \right\rangle,
		\end{align*}
		where the inner product on the first line is in $H(1)$ scaled by $r$ and the inner product on the last line is in $H(r)$. Thus $F$ is an isometry.
\end{enumerate}

\pagebreak
\begin{exer}[7.2: 9]
Classical tensor formula for Gaussian curvature.
\end{exer}
\begin{enumerate}
	\item We have
		\begin{align*}
			\left\langle E_1,E_1 \right\rangle&=\frac{\left\langle \mathbf{x}_{u},\mathbf{x}_{u} \right\rangle}{E} = \frac{E}{E} =1,\\
			\left\langle E_2,E_2 \right\rangle&=\frac{1}{W^2 E} \left( E^2\left\langle \mathbf{x}_{v},\mathbf{x}_{v} \right\rangle-2EF\left\langle \mathbf{x}_{u},\mathbf{x}_{v}\right\rangle+ F^2\left\langle \mathbf{x}_{u},\mathbf{x}_{u} \right\rangle \right) \\
							  &= \frac{(EG-F^2)E}{(EG-F^2)E} =1,\\
			\left\langle E_1,E_2 \right\rangle &= \frac{1}{WE} \left( E\left\langle \mathbf{x}_{u},\mathbf{x}_{v} \right\rangle -F\left\langle \mathbf{x}_{u},\mathbf{x}_{u} \right\rangle \right) \\
							   &= \frac{1}{WE} (EF-EF)=0,
		\end{align*} so $E_1,E_2$ are orthonormal.

	\item Since $\theta_i = \left\langle E_i,\mathbf{x}_{u} \right\rangle du + \left\langle E_i,\mathbf{x}_{v} \right\rangle dv$,
		\begin{align*}
			\theta_1 &= \sqrt{E} du+\frac{F}{\sqrt{E} } dv \\
			\theta_2 &= \frac{1}{W\sqrt{E} } \left[ \left\langle E\mathbf{x}_{v}-F\mathbf{x}_{u},\mathbf{x}_{u} \right\rangle du + \left\langle E\mathbf{x}_{v}-F\mathbf{x}_{u},\mathbf{x}_{v} \right\rangle dv \right] \\
				 &= \frac{1}{W\sqrt{E} } \left[ (EF-EF)du + (EG-F^2)dv \right]\\
				 &= \frac{W}{\sqrt{E} } dv.
		\end{align*}
	\item By part (b) and the first structural equations,
		\begin{align*}
			d\theta_1 &= (P\,du+Q\,dv) \wedge \frac{W}{\sqrt{E} } \,dv \\
			d\theta_2 &= (-P\,du-Q\,dv) \wedge (\sqrt{E} \,du + \frac{F}{\sqrt{E} } \,dv).
		\end{align*}
		Then manually calculating $d\theta_1$ and $d\theta_2$ and solving for $P$ and $Q$ yields
		\begin{align*}
			P &= \frac{2EF_{u}-FE_{u}-EE_{v}}{2EW} \\
			Q &= -\frac{FE_{v}-EG_{u}}{2EW} .
		\end{align*}

	\item By the second structural equation,
		\begin{align*}
			d\omega_{12} &= -K\theta_1 \wedge \theta_2 \\
			(P_{v}-Q_{u})du \wedge dv &= (KW)\,du \wedge dv,
		\end{align*}
		which implies
		\begin{align*}
			K &= \frac{P_{v}-Q_{u}}{W} \\
			  &= \frac{1}{2W} \left[ \frac{\partial }{\partial v} \left( \frac{2EF_{u}-FE_{u}-EE_{v}}{EW}  \right)+\frac{\partial }{\partial u} \left( \frac{FE_{v}-EG_{u}}{EW} \right) \right].
		\end{align*}

	\item If $\mathbf{x}$ is orthogonal, then $F=0$ (and by extension, $W = \sqrt{EG} $), so
		\begin{align*}
			K &= -\frac{1}{2W} \left[ \left( \frac{E_{v}}{W}  \right)_{v} + \left( \frac{G_{u}}{W}  \right)_{u} \right] \\
			  &= -\frac{1}{2\sqrt{EG} } \left[ \left( \frac{G_{u}}{\sqrt{EG} }  \right)_{u} + \left( \frac{E_{v}}{\sqrt{EG} }  \right)_{v} \right].
		\end{align*}
		But $(2\sqrt{f} )_{x} = f_{x}/\sqrt{f} $ for any function $f$, so this becomes
		\[
			K = -\frac{1}{\sqrt{EG} } \left[ \left( \frac{(\sqrt{G} )_{u}}{\sqrt{E} }  \right)_{u} + \left( \frac{(\sqrt{E} )_{v}}{\sqrt{G} }  \right)_{v} \right],
		\] which matches proposition 6.3 from chapter 6.

	\item Since $E=1+v^2,F=uv,G=1+u^2$, we calculate $E_{u}=0,E_{v}=2v,F_{u}=v, G_{u}=2u$, and $W = \sqrt{EG-F^2} =\sqrt{1+u^2+v^2} $. Then substituting into part (d) yields
		\begin{align*}
			K &= \frac{1}{2W} \left[ (0)_{v} - \left( \frac{2u}{(1+v^2)W}  \right)_{u}  \right] \\
			  &= \frac{1}{(1+v^2)W} \left[ \frac{W=u^2W^{-1}}{W^2}  \right] \\
			  &= \frac{1}{1+v^2} \left[ \frac{W^2-u^2}{W^4}  \right] \\
			  &= \frac{1}{W^4} \\
			  &= \frac{1}{(1+u^2+v^2)^2},
		\end{align*}
		as desired.
\end{enumerate}

\end{document}
