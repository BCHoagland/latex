\documentclass[10pt]{report}
\usepackage{/Users/bradenhoagland/latex/math}

\lhead{Braden Hoagland}
\chead{HW 5}
\rhead{}

\renewcommand{\theenumi}{\alph{enumi}}

\begin{document}
%\tableofcontents

\begin{exer}[4.5: 6]
Prove that a regular mapping of surfaces is a local diffeomorphism.
\end{exer}
Since $F$ is regular, then so is $\mathbf{y}^{-1}F\mathbf{x}_{*}$. Fix $\mathbf{p} \in M$, then by the inverse function theorem, $\mathbf{y}^{-1}F\mathbf{x}$ maps some neighborhood $U$ of $\mathbf{x}^{-1}(\mathbf{p})$ onto some neighborhood $\mathcal{V}$ of $\mathbf{y}^{-1}F\mathbf{x}(\mathbf{x}^{-1}(\mathbf{p})) = \mathbf{y}^{-1}F(\mathbf{p})$. Then
\begin{align*}
	\mathbf{y}^{-1}F\mathbf{x}(\mathcal{U}) &= \mathcal{V} \\
	F(\mathbf{x}(\mathcal{U})) &= \mathbf{y}(\mathcal{V}),
\end{align*}
so $F$ maps the neighborhood $\mathbf{x}(\mathcal{U})$ of $\mathbf{p}$ onto the neighborhood $\mathbf{y}(\mathcal{V})$ of $F(\mathbf{p})$. Thus $F$ is a local diffeomorphism.

\begin{exer}[4.6: 2]
	Let $\alpha:[-1,1]\to \mathbb{R}^2$ be the curve segment given by $\alpha(t)=(t,t^2)$.
	\begin{enumerate}
		\item If $\phi=v^2\;du+2uv\;dv$, compute $\int_{\alpha} \phi$.
		\item Find a function $f$ such that $df=\phi$ and check Theorem 6.2 in this case.
	\end{enumerate}
\end{exer}
\begin{enumerate}
	\item Since $\alpha(t)=(t,t^2)$ and $\alpha'(t) = (1, 2t)$, we have
		\[
			\int_{\alpha} \phi = \int_{-1}^{1} \alpha^* \phi = \int_{-1}^{1} \phi(\alpha'(t)\;dt = \int_{-1}^{1} \phi(1, 2t)\;dt = \int_{-1}^{1} 5t^4\;dt = 2.
		\] 

	\item Let $f=uv^2$, then $df=\phi$. Then
		\[
			\int_{\alpha} df = f(\alpha(1))-f(\alpha(-1))=f(1,1,)-f(-1,1) = 2,
		\] which matches the result from part (a).
\end{enumerate}
\pagebreak

\begin{exer}[4.6: 4b]
The 1-form
\[
\psi = \frac{u\;dv - v\;du}{u^2+v^2} 
\] is well-defined on the plane $\mathbb{R}^2$ with the origin $\mathbf{0}$ removed. Show that the restriction of $\psi$ to the right half-plane $u>0$ is exact.
\end{exer}
By part (a), $\psi$ is closed on $\mathbb{R}^2-\mathbf{0}$, so it is closed on the half-plane $P \doteq \left\{ (u,v) \;|\; u>0 \right\}$. For any two points $\mathbf{p},\mathbf{q} \in P$, the curve $\gamma:[0,1] \to  P$ given by
\[
	\gamma(t) = t\mathbf{p} + (1-t)\mathbf{q}
\] lies entirely in $P$, so P is path connected. Additionally, given a loop $\alpha:[a,b] \to P$ at $\mathbf{p}$, the function
\[
	\mathbf{x}(u,v) = v\alpha(a) + (1-v)\alpha(u)
\] lies entirely in $P$ for $0 \leq v \leq 1$, so $P$ is homotopic to a constant. Thus $P$ is simply connected, so by the Poincar\'e Lemma, $\psi$ is exact on $P$.

\begin{exer}[4.6: 5]
\begin{enumerate}
	\item Show that every curve $\alpha$ in $\mathbb{R}^2$ that does not pass through the origin has an (orientation-preserving) reparameterization in the polar form
		\[
			\alpha(t) = (r(t) \cos \theta(t), r(t) \sin \theta(t)).
		\] 
	\item If the curve $\alpha:[a,b]\to \mathbb{R}^2-\mathbf{0}$ is closed, prove that
		\[
			\text{wind}(\alpha) = \frac{\theta(b)-\theta(a)}{2\pi} 
		\] is an integer.
	\item If $\psi$ is the 1-form in Exercise 4, then $\text{wind}(\alpha)=\frac{1}{2\pi} \int_{\alpha} \psi$.
	\item If $\alpha=(f,g)$, then
		\[
			\text{wind}(\alpha)=\frac{1}{2\pi } \int_{a}^{b} \frac{fg'-gf'}{f^2+g^2} dt = \frac{1}{2\pi } \int_{a}^{b} \frac{\det(\alpha(t), \alpha'(t))}{\alpha(t)\cdot \alpha(t)} dt.
		\] 
\end{enumerate}
\end{exer}
\begin{enumerate}
	\item Let $r(t) = {\Vert{\alpha(t)}\Vert}$, $f=U_1\cdot \alpha/{\Vert{\alpha}\Vert}$, and $g=U_2\cdot \alpha/{\Vert{\alpha}\Vert}$. Then since $f$ and $g$ are differentiable with $f^2+g^2=1$, by \S 2.1 Exercise 12, there exists a function $\theta$ such that
		\[
		f=\cos \theta, g = \sin \theta.
	\] Then \[
	\alpha = (\alpha_1, \alpha_2) = \left( {\Vert{\alpha}\Vert}U_1\cdot\frac{\alpha}{{\Vert{\alpha}\Vert}}, {\Vert{\alpha}\Vert}U_2\cdot\frac{\alpha}{{\Vert{\alpha}\Vert}}  \right) = (rf, rg) = (r \cos\theta, r\sin \theta).
	\]
\item Since $\alpha$ is closed, i.e. $\alpha(a)=\alpha(b)$, the quantities $\theta(a)$ and $\theta(b)$ measure the same angle, so $2\pi$ divides $|\theta(b)-\theta(a)|$, i.e. $|\theta(b)-\theta(a)|=2\pi m$ for some nonnegative integer $m$. Then $\text{wind}(\alpha) = \pm \frac{2\pi m}{2\pi} =m$, which is an integer.
\item We can use the parameterization of $\alpha$ from part (a) to show this. For notational simplicity, I'll use $r,f$, and $g$ in place of $r, \cos \theta$, and $\sin \theta$.
	\begin{align*}
		\int_{\alpha} \psi &= \int_{a}^{b} \psi(\alpha'(t)\;dt \\
				   &= \int_{a}^{b} \frac{\alpha_1 d\alpha_2-\alpha_2 d\alpha_1}{\alpha_1^2 + \alpha_2^2} \\
				   &= \int_{a}^{b} \frac{rf \;d(rg) - rg \; d(rf)}{r^2(f^2+g^2)} dt.
				   \intertext{Expanding the derivatives and simplying yields}
				   &= \int_{a}^{b} (f\;dg-g\;df) dt.
				   \intertext{By the definition of $\theta(t)$ in \S 2.1 Exercise 12, the expression in parentheses is actually $d\theta$, so this becomes}
				   &= \int_{a}^{b} (d\theta)\;dt \\
				   &= \theta(b)-\theta(a).
	\end{align*}
	Thus $\frac{1}{2\pi} \int_{\alpha}\psi = \text{wind}(\alpha) $.

	\item If $\alpha=(f,g)$, then
		\[
			\text{wind}(\alpha) = \frac{1}{2\pi} \int_{\alpha} \psi = \frac{1}{2\pi} \psi(\alpha')\;dt = \frac{1}{2\pi} \int_{a}^{b} \frac{fg'-gf'}{f^2+g^2} dt.
		\] 
		The second desired equality follows from
		\[
		\frac{\det(\alpha,\alpha')}{\alpha\cdot\alpha} = \left|
		\begin{matrix}
			f & g \\
			f' & g'
		\end{matrix} \right| / (f^2+g^2) = \frac{fg'-gf'}{f^2+g^2} .
		\] 
\end{enumerate}
\pagebreak

\begin{exer}[4.6: 13]
Interpret the classical Stokes' theorem
\[
	\int_{\partial_{}{\mathbf{x}} } V\cdot ds = \int_{} \int_{\mathbf{x}} U \cdot (\nabla_{}\times V) \; dA
\] as a special case of Theorem 6.5.
\end{exer}
The vector field $V$ can be written $V = \sum v_i U_i$, so by \S 1.6 Exercise 8, there is a one-to-one correspondence (type (1)) between $V$ and the 1-form $\phi = \sum v_i \;dx_i$. Then by part (b) of that exercise, there is a one-to-one correspondence (type (2)) between $d\phi$ and $\text{curl} V = \nabla_{}\times V$. Thus there is a correspondence
\begin{align*}
	\int_{\partial_{}{\mathbf{x}} } \phi &= \int_{} \int_{\mathbf{x}} d\phi \\
	\updownarrow (1) & \quad\quad\updownarrow (2) \\
	\int_{\partial_{}{\mathbf{x}} } V \cdot ds &= \int_{} \int_{\mathbf{x}} (\nabla_{}\times V)\cdot dA.
\end{align*}

\begin{exer}[4.7: 1]
Which of the following surfaces are compact and which are connected?
\end{exer}
\begin{enumerate}
	\item Just connected.
	\item Neither connected nor compact.
	\item Neither connected not compact.
	\item Both connected and compact.
	\item Both connected and compact.
\end{enumerate}
\pagebreak

\begin{exer}[4.7: 2]
Let $F$ be a mapping of a surface $M$ onto a surface $N$. Prove:
\begin{enumerate}
	\item If $M$ is connected, then $N$ is connected.
	\item If $M$ is compact, then $N$ is compact (try both the covering definition and Lemma 7.2).
\end{enumerate}
\end{exer}
\begin{enumerate}
	\item Since $F$ is onto $N$, any 2 points in $N$ have the form $F(\mathbf{p})$, $F( \mathbf{q})$ for $\mathbf{p},\mathbf{q} \in M$. Since $M$ is path connected, there is a curve $\gamma:[0,1]\to M$ such that $\gamma(0)=\mathbf{p}$ and $\gamma(1)=\mathbf{q}$. Then $\gamma' \doteq F \circ \gamma:[0,1]\to N$ is a curve such that $\gamma'(0)=F(\mathbf{p})$ and $\gamma'(1)=F(\mathbf{q})$, so $N$ is path connected.
	\item \textbf{Open Cover:} Let $\mathcal{U}$ be an arbitrary open cover of $N$, then since $F$ is onto $N$, $F^{-1}(\mathcal{U})$ is an open cover of $M$ (by the definition of continuity, preimages of open sets under continuous functions remain open). Since $M$ is compact, there is a finite subcover $F^{-1}(\overline{\mathcal{U}} )$ of $M$. Then $\overline{\mathcal{U}} $ is a finite subcover of $N$, so $N$ is compact.

		\textbf{Lemma 7.2:} Since $M$ is compact, it is covered by the images of a finite collection $X$ of 2-segments. Since $F$ is onto $N$, $F(X)$ covers $N$, so $N$ is compact.
\end{enumerate}

\begin{exer}[4.7: 3]
Let $F:M\to N$ be a regular mapping. Prove that if $N$ is orientable, then $M$ is orientable.
\end{exer}
Since $N$ is orientable, there is a nonvanishing 2-form $\mu$ on $N$. Consider the pullback $F^{*}\mu$ of $\mu$ on $M$. then for all $\mathbf{p} \in M$ and for all $\mathbf{v},\mathbf{w} \in T_{\mathbf{p}}(M)$, the pullback evaluates to $(F^{*}\mu)(\mathbf{v},\mathbf{w}) = \mu(F_{*}(\mathbf{v},\mathbf{w}))$. Then since $\mu$ is never zero, neither is $F^{*}\mu$, so $M$ has a nonvanishing 2-form and is thus orientable.

\begin{exer}[4.7: 4]
Let $f$ be a differentiable real-valued function on a connected surface. Prove:
\begin{enumerate}
	\item If $df=0$, then $f$ is constant.
	\item If $f$ is never zero then either $f>0$ or $f<0$.
\end{enumerate}
\end{exer}
\begin{enumerate}
	\item Let $\mathbf{p},\mathbf{q} \in M$, then since $M$ is path connected, there is a path $\gamma$ from $\mathbf{p}$ to $\mathbf{q}$. Then
		\[
			f(\mathbf{q})-f(\mathbf{p})= \int_{\gamma} df = \int_{\gamma} 0 = 0,
		\] so $f(\mathbf{p}) = f(\mathbf{q})$. Since $\mathbf{p}$ and $\mathbf{q}$ were arbitrary, this shows that $f$ is constant.
	\item Suppose $\mathbf{p},\mathbf{q} \in M$, then since $M$ is path connected, there is a path $\gamma$ from $\mathbf{p}$ to $\mathbf{q}$. Then the path $\gamma' = f \circ \gamma$ is a path from $f(\mathbf{p})$ to $f(\mathbf{q})$. If these two points have opposite signs, say $f(\mathbf{p}) < 0$ and $f(\mathbf{q}) > 0$, then by the intermediate value theorem, there is a real number $t$ in the domain of $\gamma'$ such that $\gamma'(t) = f(\gamma(t)) = 0$. This contradicts the fact that $f$ is never zero, so $f$ must either always be positive or always be negative.
\end{enumerate}

\begin{exer}[4.7: 10]
The Hausdorff axiom asserts that distinct points $\mathbf{p}\neq \mathbf{q}$ have disjoint neighborhoods. Prove:
\begin{enumerate}
	\item $\mathbb{R}^3$ obeys the Hausdorff axiom.
	\item A surface $M$ in $\mathbb{R}^3$ obeys the Hausdorff axiom.
\end{enumerate}
\end{exer}
\begin{enumerate}
	\item Let $\mathbf{p} \neq \mathbf{q}$ be points in $\mathbb{R}^3$, then $d \doteq {\Vert{\mathbf{p}-\mathbf{q}}\Vert} \neq 0$. Then the open balls $B(\mathbf{p},d/4)$ and $B(\mathbf{q},d/4)$ are disjoint neighborhoods of $\mathbf{p}$ and $\mathbf{q}$, respectively. Thus $\mathbb{R}^3$ is Hausdorff.
	\item Let $\mathbf{p} \in M$, then by definition of a surface in $\mathbb{R}^3$, its neighborhoods are of the form $B(\mathbf{p},\varepsilon) \cap M$. So if $\mathbf{q} \neq \mathbf{p}$, then the Euclidean distance between them is $d = {\Vert{\mathbf{p}-\mathbf{q}}\Vert}$, then $B(\mathbf{p}, d/4) \cap M$ and $B(\mathbf{q},d/4) \cap M$ are disjoint neighborhoods of $\mathbf{p}$ and $\mathbf{q}$. Thus $M$ is Hausdorff.
\end{enumerate}

\begin{exer}[]
	Let $S$ be a circle and $\theta$ the angle parameter in this circle.
	\begin{enumerate}
		\item Prove $S$ is a 1-dimensional manifold.
		\item Is $d\theta$ a 1-form?
		\item Is $d\theta$ closed?
		\item Is $d\theta$ exact?
	\end{enumerate}
\end{exer}
\begin{enumerate}
	\item We can cover $S$ with four diffeomorphisms, from which the necessary properties of being a manifold follow. Let $\mathcal{D} = (-1,1) \subset \mathbb{R}$, then define four patches from $\mathcal{D}$ to $S$ as follows:
		\begin{align*}
			\mathbf{x}_{1}: x&\mapsto (x,\sqrt{1-x^2} ), \\
			\mathbf{x}_{2}: x&\mapsto (x,-\sqrt{1-x^2} ), \\
			\mathbf{x}_{3}: x&\mapsto (\sqrt{1-x^2},x ), \\
			\mathbf{x}_{4}: x&\mapsto (-\sqrt{1-x^2},x ).
		\end{align*}
		The inverses of $\mathbf{x}_1$ and $\mathbf{x}_2$ just project onto the first coordinate, and the inverses of $\mathbf{x}_3$ and $\mathbf{x}_4$ project onto the second coordinate. These are differentiable operations, so these patches are diffeomorphisms from $\mathcal{D}$ onto their images. This means $\mathbf{x}_{i}^{-1}\mathbf{x}_{j}$ will be differentiable on the overlap of the two images.

		As in \S 4.7 Exercise 10 (b), the use of diffeomorphisms in this construction is enough to satisfy the Hausdorff axiom, so $S$ is a manifold. Since the patches all lie in $\mathbb{R}$, it is 1-dimensional.

	\item Defined on the individual patches, $d\theta$ is a 1-form. We can solve for it explicitly by considering the system
		\begin{align*}
			dx &= \cos \theta \;dr - r \sin \theta\;d\theta \\
			dy &= \sin \theta\;dr + r \cos \theta\;d\theta,
		\end{align*}
		which we derived on the first homework. Solving for $d\theta$ in terms of $dx,dy$ yields
		\[
		d\theta = \frac{\cos\theta\;dy-\sin\theta\;dx}{r} = \frac{x\;dy-y\;dx}{r^2} = \frac{x\;dy-y\;dx}{x^2+y^2},
		\] 
		which shows that $d\theta$ has the structure of a 1-form.

	\item We calculate
		\[
			d(d\theta) = \frac{(y^2-x^2)dx\;dy + (x^2-y^2)dx\;dy}{(x^2+y^2)^2} = \frac{0}{1} =0,
		\] so $d\theta$ is closed.

	\item If $d\theta$ were exact, then it would be equal to $df$ for some form $f$. Then for any closed curve $\alpha \subset \mathbb{R}-\mathbf{0}$, by Stokes' theorem we would have
		\[
		\int_{\alpha} d\theta = \int_{\partial_{}{\alpha} } f = 0
		\] since closed curves have no boundary. But by \S 4.6 Exercise 4, the integral of $d\theta$ along the curve that goes clockwise around the circle once is $\theta(b)-\theta(a)=2\pi\neq 0$, so $d\theta$ is not exact.
\end{enumerate}


\end{document}
