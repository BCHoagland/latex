\documentclass[10pt]{report}
\usepackage{/Users/bradenhoagland/latex/math}

\lhead{Braden Hoagland}
\chead{HW 3}
\rhead{}

\begin{document}
%\tableofcontents

\begin{exer}[2.5: 2]
	Let $V=-yU_1+xU_3$ and $W=\cos xU_1+\sin xU_2$. Express the following covariant derivatives in terms of $U_1,U_2,U_3$.
\end{exer}
\begin{enumerate}
	\item $V[\cos x] = -y U_1[\cos x] = y \sin x$ and $V[\sin x] = -y U_1[\sin x] = -y \cos x$, so \begin{align*}
			\nabla_{V}W &= V[\cos x]U_1 + V[\sin x]U_2 \\
				    &= y\sin x U_1 - y\cos x U_2.
	\end{align*}
\item $V[-y]$ is 0 since $V$ has no $U_2$ component, so
	\begin{align*}
		\nabla_{V}V &= V[-y] U_1 + V[x] U_3 \\
			    &= -y U_1[x] U_3 \\
			    &= -y U_3.
	\end{align*}

\item $V[z^2 \cos x] = yz^2 \sin x + 2xz \cos x$ and $V[z^2 \sin x] = -yz^2 \cos x + 2xz \sin x$, so
	\begin{align*}
		\nabla_{V}(z^2 W) &= V[z^2 \cos x] U_1 + V[z^2 \sin x] U_2 \\
				  &= (yz^2 \sin x + 2xz \cos x) U_1 + (-yz^2 \cos x + 2xz \sin x) U_2.
	\end{align*}
\item Since $W[-y] = -\sin x$ and $W[x] = \cos x$,
	\begin{align*}
		\nabla_{W}V &= W[-y]U_1 + W[x] U_3 \\
			    &= -\sin x U_1 + \cos x U_3.
	\end{align*}

\item Since $V[y\sin x]=-y^2\cos x$ and $V[-y\cos x] = -y^2\sin x$,
	\begin{align*}
		\nabla_{V}(\nabla_{V}W)) &= \nabla_{V}(y\sin x\;U_1-y\cos x\;U_2) \\
					 &=V[y\sin x]U_1 + V[-y\cos x]U_2 \\
					 &= -y^2\cos x\;U_1-y^2\sin x\;U_2.
	\end{align*}

	\item This is
		\begin{align*}
			\nabla_{V}(xV-zW) &= \nabla_{V}\big( (-xy-z\cos x)\;U_1-z\sin x\;U_2+x^2\;U_3 \big) \\
					  &= V[-xy-z\cos x]U_1+V[-z\sin x]U_2+V[x^2]U_3 \\
					  &= (y^2-yz\sin x-x\cos x)U_1+(yz\cos x-x\sin x)U_2\\&\quad+(-2xy)U_3.
		\end{align*}
\end{enumerate}
\pagebreak

\begin{exer}[2.5: 3]
	If $W$ is a vector field with constant length $\Vert{W}\Vert$, prove that for any vector field $V$, the covariant derivative $\nabla_{V}W$ is everywhere orthogonal to $W$.
\end{exer}
Since $\Vert{W}\Vert$ is constant, $\Vert{W}\Vert^2$ is constant, so $\nabla_{V}\Vert{W}\Vert^2 = \nabla_{V}(W \cdot W) =0$; however, we can manually calculate this derivative to be
\begin{align*}
	\nabla_{V}\Vert{W}\Vert^2 &= \nabla_{V}(W \cdot W) \\
				  &= \nabla_{V}W \cdot W + W \cdot \nabla_{V}W \\
			      &= 2(\nabla_{V}W \cdot W),
\end{align*}
so $\nabla_{V}W \cdot W=0$. Thus for any vector field $V$, $\nabla_{V}W$ is everywhere orthogonal to $W$.

\begin{exer}[2.5: 5]
	Let $W$ be a vector field defined on a region containing a regular curve $\alpha$. Then $t\to W(\alpha(t))$ is a vector field on $\alpha$ called the restriction of $W$ to $\alpha$ and is denoted by $W_{\alpha}$.
	\begin{enumerate}
		\item Prove that $\nabla_{\alpha'(t)}W=(W_{\alpha})'(t)$.
		\item Deduce that the straight line in Definition 5.1 may be replaced by any curve with initial velocity $\mathbf{v}$.
	\end{enumerate}
\end{exer}

\begin{enumerate}
	\item The vector field $W$ can be written $W = \sum w_i U_i$. Using this we have
		\[
			\nabla_{\alpha'(t)}W = \sum \alpha'(t)[w_i]U_i(\alpha(t))
		\] and
		\[
			W_{\alpha}(t) = W(\alpha(t)) = \sum w_i(\alpha(t)) U_i(\alpha(t)),
		\] from which it follows that
		\[
			(W_{\alpha})'(t) = \sum w_i'(\alpha(t))\alpha'(t) U_i(\alpha(t)).
		\] Thus $\nabla_{\alpha'(t)}W = (W_{\alpha})'(t)$ if $\alpha'(t)[w_i]=w_i'(\alpha(t))\alpha'(t)$. This is true, since
		\begin{align*}
			\alpha'(t)[w_i] &= \frac{d }{d s} w_i(\alpha(t)+s\alpha'(t))\Big|_{s=0}\\
					&= w_i'(\alpha(t)+s\alpha'(t))a'(t) \Big|_{s=0}\\
					&= w_i'(\alpha(t))\alpha'(t).
		\end{align*}

	\item Then if $\alpha'(0)=\mathbf{v}$, then
		\begin{align*}
			\nabla_{\mathbf{v}}W = \nabla_{\alpha'(0)}W &= (W_{\alpha})'(0)\\
			&=W'(\alpha(0))\alpha'(0) \\
			&= \frac{d }{d s} W(\alpha(t)+s\alpha'(t))\Big|_{s=0}.
		\end{align*}
		This is exactly the covariant derivative, except now we have $\alpha(t)$ instead of $\mathbf{p}$ and $\alpha'(t)$ instead of $\mathbf{v}$.
\end{enumerate}

\begin{exer}[2.7: 2]
	Find the connection forms of the natural frame field $U_1, U_2, U_3$.
\end{exer}
The attitude matrix for the natural frame field is simply $I_3$, so $dA=0$. Then the matrix of connection forms is $\omega=dA\;A^T = 0$. Thus every connection form is the zero function.

\begin{exer}[2.7: 4]
	Prove that the connection forms of the spherical frame field are
	\[
	\omega_{12}=\cos\varphi \;d\theta,\quad \omega_{13}=d\varphi,\quad \omega_{23}=\sin\varphi \;d\theta.
	\] 
\end{exer}
Given the spherical frame fields
\begin{align*}
	F_1&=\cos\varphi(\cos\theta U_1+\sin\theta U_2)+\sin\varphi U_3, \\
	F_2&=-\sin\theta U_1+\cos\theta U_2, \\
	F_3&=-\sin\varphi(\cos\theta U_1+\sin\theta U_2)+\cos\varphi U_3,
\end{align*}
we can form the attitude matrix
\[
A =
\begin{pmatrix}
	\cos\varphi\cos\theta & \cos\varphi\sin\theta & \sin\varphi \\
	-\sin\theta&\cos\theta&0 \\
	-\sin\varphi\cos\theta&-\sin\varphi\sin\theta&\cos\varphi
\end{pmatrix}.
\]
Applying $d$ to the entries of this matrix yields $dA=$
\[
\begin{pmatrix}
	-\sin\varphi\cos\theta d\varphi -\cos\varphi\sin\theta d\theta&-\sin\varphi\sin\theta d\varphi+\cos\varphi\cos\theta d\theta&\cos\varphi d\varphi \\
	-\cos\theta d\theta&-\sin\theta d\theta&0\\
	-\cos\varphi\cos\theta d\varphi+\sin\varphi\sin\theta d\theta&-\cos\varphi\sin\theta d\varphi-\sin\varphi\cos\theta d\theta&-\sin\varphi d\varphi
\end{pmatrix}.
\] 
We then find the connection forms by $\omega=dA\;A^T$. The entries for $\omega_{12},\omega_{13},$ and $\omega_{23}$ are then
\begin{align*}
	\omega_{12}&=\sin\varphi\sin\theta\cos\theta\;d\varphi+\cos\varphi\sin^2\theta\;d\theta\\&\quad-\sin\varphi\sin\theta\cos\theta\;d\varphi+\cos\varphi\cos^2\theta\;d\theta \\
		   &= \cos\varphi(\sin^2\theta+\cos^2\theta)d\theta \\
		   &= \cos\varphi\;d\theta. \\
	\omega_{13}&= \sin^2\varphi\cos^2\theta\;d\varphi+\sin\varphi\cos\varphi\sin\theta\cos\theta\;d\theta\\&\quad+\sin^2\varphi\sin^2\theta\;d\varphi-\sin\varphi\cos\varphi\sin\theta\cos\theta\;d\theta+\cos^2\varphi\;d\varphi \\
		   &=(\sin^2\varphi(\cos^2\theta+\sin^2\theta)+\cos^2\varphi)d\varphi \\
		   &=d\varphi. \\
	\omega_{23}&=\sin\varphi\cos^2\theta\;d\theta+\sin\varphi\sin^2\theta\;d\theta \\
		   &=\sin\varphi(\sin^2\theta+\cos^2\theta)\;d\theta \\
		   &=\sin\varphi\;d\theta.
\end{align*}

\begin{exer}[2.7: 5]
	If $E_1,E_2,E_3$ is a frame field and $W=\sum f_i E_i$, prove the covariant derivative formula
	\[
		\nabla_{V}W=\sum_j \left\{ V[f_j]+\sum_i f_i \omega_{ij}(V) \right\}E_j.
	\] 
\end{exer}

Using the linearity of the covariant derivative, its Leibniz property, and the decomposition
\[
	\nabla_{V}E_i = \sum_j \omega_{ij}(V)E_j,
\] we have
\begin{align*}
	\nabla_{V}W &= \nabla_{V}\left( \sum_i f_i E_i \right) \\
		    &= \sum_i \nabla_V f_i E_i \\
		    &= \sum_i \Big\{ V[f_i]E_i +f_i\nabla_{V}E_i \Big\} \\
		    &= \sum_i \left\{ V[f_i]E_i +f_i \sum_j \omega_{ij}(V)E_j \right\} \\
		    &= \sum_i V[f_i]E_i + \sum_{i,j} f_i \omega_{ij}(V)E_j.
		    \intertext{Now we can change the first summation to use $j$ instead of $i$, since it's just a symbol and doesn't change what we're actually summing over. This then becomes}
		    &= \sum_j \left\{ V[f_j] + \sum_i f_i \omega_{ij}(V) \right\}E_j.
\end{align*}


\begin{exer}[2.7: 8]
	Let $\beta$ be a unit-speed curve in $\mathbb{R}^3$ with $\kappa>0$, and suppose that $E_1,E_2,E_3$ is a frame field on $\mathbb{R}^3$ such that the restriction of these vector fields to $\beta$ gives the Frenet frame field $T,N,B$ of $\beta$. Prove that
	\[
		\omega_{12}(T)=\kappa,\quad \omega_{13}(T)=0,\quad \omega_{23}(T)=\tau.
	\] Then deduce the Frenet formulas from the connection equations.
\end{exer}
In \S 2.5:5, we showed that $\nabla_{\alpha'}V=\nabla_{T}V=\frac{d }{d t} V_{\alpha}$ for any vector field $V$. The Frenet formulas then allow us to write the covariant derivative of our frame field with respect to $T$ as
\[
\nabla_{T}
\begin{pmatrix}
	E_1\\E_2\\E_3
\end{pmatrix} =
\frac{d }{d t} 
\begin{pmatrix}
	T\\N\\B
\end{pmatrix} =
\begin{pmatrix}
	0 & \kappa & 0 \\
	-\kappa & 0 & \tau \\
	0 &-\tau &0
\end{pmatrix}
\begin{pmatrix}
	T\\N\\B
\end{pmatrix}.
\] 
Thus $\omega_{12}(T), \omega_{13}(T)=0$, and $\omega_{23}(T)$. The Frenet formulas are clear from the above matrix.

\begin{exer}[2.8: 1]
For a 1-form $\phi=\sum f_i \theta_i$, prove
\[
d\phi=\sum_j \left\{ df_j + \sum_i f_i \omega_{ij} \right\}\wedge \theta_j.
\] 
\end{exer}
Applying $d$ to $\phi$ gives
\begin{align*}
	d\phi&=d \sum f_i \theta_i \\
	     &= \sum d(f_i\theta_i) \\
	     &= \sum \left\{ df_i \wedge \theta_i + f_i d\theta_i \right\}.
	     \intertext{Then by the Cartan structure equation for $d\theta_i$, this becomes}
	     &= \sum_i df_i \wedge \theta_i + \sum_{i,j}f_i \omega_{ij}\wedge \theta_j.
	     \intertext{If we use $j$ instead of $i$ as the indexing variable in the leftmost sum, this becomes}
	     &= \sum_j \left\{ df_j + \sum_i f_i \omega_{ij} \right\}\wedge \theta_j.
\end{align*}


\end{document}
