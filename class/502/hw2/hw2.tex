\documentclass[10pt]{report}
\usepackage{/Users/bradenhoagland/latex/math}

\lhead{Braden Hoagland}
\chead{HW 2}
\rhead{}

\begin{document}
%\tableofcontents

\begin{exer}[DF 13.2: 8]
	Let $F$ be a field of characteristic $\neq 2$. Let $D_1$ and $D_2$ be elements of $F$, neither of which is a square in $F$. Prove that $F(\sqrt{D_1} , \sqrt{D_2} )$ is of degree 4 over $F$ if $D_1D_2$ is not a square in $F$ and is of degree 2 over $F$ otherwise.
\end{exer}
Since neither $D_1$ nor $D_2$ are squares in $F$, the extensions $F(\sqrt{D_1} )$ and $F(\sqrt{D_2} )$ are both quadratic extensions, so we know that their composite satisfies
\[
	\left[ F(\sqrt{D_1} ,\sqrt{D_2} ):F \right]\leq \left[F(\sqrt{D_1} ):F\right]\left[F(\sqrt{D_2}) :F\right] = 2 \cdot 2 = 4.
\] This gives us the following tower.
\begin{center}
\begin{tikzcd}
                                                            & {F(\sqrt{D_1} ,\sqrt{D_2} )} \arrow[rd, no head] \arrow[dd, "\leq 4", no head, dashed] &                \\
F(\sqrt{D_1} ) \arrow[rd, "2"', no head] \arrow[ru, no head] &                                                                                        & F(\sqrt{D_2} ) \\
                                                            & F \arrow[ru, "2"', no head]                                                             &               
\end{tikzcd}	
\end{center}
Then since degrees multiply in towers, we know that the degree of $F(\sqrt{D_1} ,\sqrt{D_2} )$ over $F$ must be divisible by 2, i.e. it must be either 2 or 4.

\textbf{Case 1 - $D_1D_2$ is a square in $F$:} If $\sqrt{D_1D_2} \in F$, then
\begin{align*}
	\sqrt{D_1} &=\sqrt{D_1D_2} /\sqrt{D_2} \in F(\sqrt{D_2} ) \\
	\sqrt{D_2} &=\sqrt{D_1D_2} / \sqrt{D_1} \in F(\sqrt{D_1} ),
\end{align*}
so $F(\sqrt{D_1} ,\sqrt{D_2} )=F(\sqrt{1} )=F(\sqrt{2} )$, i.e. the degree of $F(\sqrt{D_1} ,\sqrt{D_2} )$ over both intermediate extensions in the tower is 1. This forces the degree of $F(\sqrt{D_1} ,\sqrt{D_2} )$ over $F$ to be 2.

\textbf{Case 2 - $D_1D_2$ is not a square in $F$:} If $\sqrt{D_1D_2} $ is not in $F$, then the polynomial $x^2-D_1D_2$, which has roots $\pm \sqrt{D_1D_2}$, is irreducible over $F(\sqrt{D_1} )$ and $F(\sqrt{D_2} )$. This means the degree of $F(\sqrt{D_1} ,\sqrt{D_2} )$ over both intermediate extensions in the tower is 2, which forces the degree of $F(\sqrt{D_1} ,\sqrt{D_2} )$ over $F$ to be 4.

\begin{exer}[DF 13.2: 14]
	Prove that if $[F(\alpha):F]$ is odd then $F(\alpha)=F(\alpha^2)$.
\end{exer}
	The polynomial $x^2-\alpha^2$ over $F(\alpha^2)$ has $\alpha$ as a root, so $\alpha$ has degree at most 2 over $F(\alpha^2)$. Then
	\[
		[F(\alpha,\alpha^2):F(\alpha^2)] = [F(\alpha):F(\alpha^2)] \leq 2.
	\] 
	But $[F(\alpha):F]=[F(\alpha):F(\alpha^2)][F(\alpha^2):F]$ is odd, which, since anything multiplied by 2 is even, forces $[F(\alpha):F(\alpha^2)]$ to be 1, so $F(\alpha)=F(\alpha^2)$.


\begin{exer}[DF 13.2: 17]
	Let $f(x)$ be an irreducible polynomial of degree $n$ over a field $F$. Let $g(x)$ be any polynomial in $F[x]$. Prove that every irreducible factor of the composite polynomial $f(g(x))$ has degree divisible by $n$.
\end{exer}
If either $f(x)$ or $g(x)$ is 0, then so is $f(g(x))$, so it is already irreducible and with degree 0, which every integer $n$ divides. Thus we consider the case when both $f(x)$ and $g(x)$ are nonzero.

Suppose $h(x)$ is an irreducible factor of $f(g(x))$. If $h$ has some root $\alpha$, then $f(g(\alpha))=0$, so $g(\alpha)$ is a root of $f$. This means the extension $F(g(\alpha))$ over $F$ has degree $n$. Additionally, since $g(\alpha)$ is a function of $\alpha$, we know that $F(g(\alpha)) \subset F(\alpha)$. These two facts allow us to express $\deg(h)$ as
\[
	\deg(h) = [F(\alpha):F] = [F(\alpha):F(g(\alpha))][F(g(\alpha)):F] = [F(\alpha):F(g(\alpha))] n.
\] Thus the degree of $h$ is divisible by $n$.

\begin{exer}[DF 13.4: 1]
	Determine the splitting field and its degree over $\mathbb{Q}$ for $x^4-2$.
\end{exer}
The polynomial $x^4-2$ has roots
\[
	\sqrt[4]{2}, \sqrt[4]{2} \;\zeta_4, \sqrt[4]{2} \;\zeta_4^2, \sqrt[4]{2} \;\zeta_4^3,
\] so its splitting field is clearly $\mathbb{Q}(\sqrt[4]{2}, \zeta_4)$. But $\zeta_4=i$, so the splitting field is actually $\mathbb{Q}(\sqrt[4]{2}, i)$.

By Eisenstein's criterion for $p=2$, $x^4-2$ is irreducible over $\mathbb{Q}$, so \[[\mathbb{Q}(\sqrt[4]{2}):\mathbb{Q}]=4.\] The polynomial $x^2+1$ (which has $i$ as a root) is also irreducible over $\mathbb{Q}$ since it has no roots in $\mathbb{Q}$, so \[[\mathbb{Q}(i):\mathbb{Q}]=2.\] Thus $[Q(\sqrt[4]{2},i):\mathbb{Q}] \leq 8$ and is divisible by 2 and 4, so the only possibilities for it are 4 and 8. If it is 4, then
\[
	4 = [\mathbb{Q}(\sqrt[4]{2},i):\mathbb{Q}] = [\mathbb{Q}(\sqrt[4]{2},i):\mathbb{Q}(\sqrt[4]{2})][\mathbb{Q}(\sqrt[4]{2}):\mathbb{Q}] = [\mathbb{Q}(\sqrt[4]{2},i):\mathbb{Q}(\sqrt[4]{2})] \cdot 4,
\] which implies that $\mathbb{Q}(\sqrt[4]{2},i)=\mathbb{Q}(\sqrt[4]{2})$. But they are clearly distinct fields, so $[\mathbb{Q}(\sqrt[4]{2},i):\mathbb{Q}]=8$.

\newpage
\begin{exer}[DF 13.4: 3]
	Determine the splitting field and its degree over $\mathbb{Q}$ for $x^4+x^2+1$.
\end{exer}
We can reduce the given polynomial into
\[
	x^4+x^2+1 = (x^2+x+1)(x^2-x+1).
\] A rational number $c/d$, if its a root for either of these factors, must satisfy $c,d \;|\; 1$. Since $1$ and $-1$ are not roots of either factor, they are then both irreducible over $\mathbb{Q}$. By the quadratic formula, the roots of the factors are
\[
\frac{-1\pm i\sqrt{3} }{2} , \frac{1\pm i\sqrt{3} }{2} .
\] The splitting field for this polynomial is then clearly $\mathbb{Q}(i\sqrt{3} )$.

Since $i\sqrt{3} $ is a root of $x^2+3$, and since this is irreducible over $\mathbb{Q}$ by Eisenstein's criterion for $p=3$, the degree of this extension over $\mathbb{Q}$ is $[\mathbb{Q}(i\sqrt{3} ):\mathbb{Q}]=2$.

\begin{exer}[DF 13.4: 4]
	Determine the splitting field and its degree over $\mathbb{Q}$ for $x^6- 4$.
\end{exer}
We can factor $x^6-4$ into
\[
	x^6-4 = (x^3+2)(x^3-2).
\] Both of these factors are irreducible over $\mathbb{Q}$ by Eisenstein's criterion for $p=2$, but we can still manually calculate their roots as
\begin{align*}
	x^3 + 2: &\quad \sqrt[3]{-2}, \sqrt[3]{-2}\;\zeta_3, \sqrt[3]{-2}\;\zeta_3^2 \\
	x^3-2: &\quad \sqrt[3]{2}, \sqrt[3]{2}\;\zeta_3, \;\zeta_3^2.
\end{align*}
Since $\zeta_3 = (-1 +i\sqrt{3})/2 $, this means the splitting field of $x^6-4$ over $\mathbb{Q}$ needs to include $\sqrt[3]{-2} , \sqrt[3]{2} ,$ and $i\sqrt{3} $. We can actually prune this list slightly: since $\left( -\sqrt[3]{2}  \right)^3 = -2$, we know $\sqrt[3]{-2} =-\sqrt[3]{2} $, so the splitting field is actually just
\[
	\mathbb{Q}(\sqrt[3]{2} , i\sqrt{3} ).
\] 
Now $\sqrt[3]{2} $ is a root of $x^3-2$ and $i\sqrt{3} $ is a root of $x^2+3$, which are both irreducible by Eisenstein's criterion for $p=2$ and $p=3$, respectively. Thus $[\mathbb{Q}(\sqrt[3]{2} ):\mathbb{Q}]=3$ and $[\mathbb{Q}(i\sqrt{3} ):\mathbb{Q}]=2$, giving us the following tower.
\begin{center}
	\begin{tikzcd}
                                              & {\mathbb{Q}(\sqrt[3]{2},i\sqrt{3})} \arrow[rd, no head] \arrow[dd, no head, dashed] &                       \\
{\mathbb{Q}(\sqrt[3]{2})} \arrow[ru, no head] &                                                                                     & \mathbb{Q}(i\sqrt{3}) \\
                                              & \mathbb{Q} \arrow[ru, "2"', no head] \arrow[lu, "3", no head]                       &                      
\end{tikzcd}
\end{center}
Since 2 and 3 are relatively prime, this means that the splitting field, which is the composite of the two intermediate field extensions in the tower, has degree 6 over $\mathbb{Q}$.

\begin{exer}[DF 13.4: 5]
	Let $K$ be a finite extension of $F$. Prove that $K$ is a splitting field over $F$ if and only if every irreducible polynomial in $F[x]$ that has a root in $K$ splits completely in $K[x]$ (use theorems 8 and 27).
\end{exer}
\textbf{Forward:} Suppose $K$ is the splitting field for some $f(x) \in F[x]$. Now let $g(x)$ be any irreducible polynomial in $F[x]$ with a root $\alpha_i \in K$. We know there exists \textit{some} splitting field of $g(x)$, so we can write it as
\[
	g(x) = \prod_{i=1}^n (x-\alpha_i),
\] where $\alpha_1$ is for sure an element of $K $. We must show that all the other $\alpha_i$ are also in $K$, and then $g(x)$ will split in $K[x]$. We will show that $\alpha_2$ is in $K$, then appeal to induction for the rest of the $\alpha_i$.

By Theorem 8, there is an isomorphism
\begin{align*}
	\phi:F(\alpha_1) &\to F(\alpha_2) \\
	\alpha_1&\mapsto \alpha_2.
\end{align*}
We can now use this isomorphism $\phi$ and the fields $F(\alpha_1)$ and $F(\alpha_2)$ as the inputs to Theorem 27. Consider the polynomials $h(x)=(x-\alpha_1)f(x) \in F(\alpha_1)[x]$ and $\phi(h(x)) = (x-\alpha_2)f(x)\in F(\alpha_2)[x]$. They clearly have splitting fields $K(\alpha_1)=K$ and $K(\alpha_2)$, respectively. Then by theorem 27, $K \cong K(\alpha_2)$, which can only be true if $\alpha_2 \in K$. After passing to induction, we get that all the $\alpha_i$ are in $K$, so $g(x)$ splits in $K$.

\textbf{Backward:} Since $K$ is a finite extension of $F$, 
\[
	K = F(\alpha_1, \dots, \alpha_n),
\] for some fixed $n$ and distinct $\alpha_i \in K$ algebraic over $F$. By assumption, all the minimal polynomials $m_{\alpha_i,F}(x)$, since they are irreducible over $F$ with a root in $K$, split in $K[x]$. Their product
\[
	f(x) = \prod_{i=1}^n m_{\alpha_i,F}(x) \in F[x]
\] then also splits in $K[x]$. Since the $\alpha_i$ are roots of $f(x)$, any field in which $f(x)$ splits must contain the $\alpha_i$. Thus $K = F(\alpha_1, \dots, \alpha_n)$ is the smallest field in which $f(x)$ splits, i.e. the splitting field of $f(x)$.

\begin{exer}[]
	Prove that $[K:F]=1 \iff K=F$.
\end{exer}
\textbf{Backward:} if $K=F$, then $K$ has a basis $\left\{ 1 \right\}$ as an $F$-vector space, so $[K:F]=1$.

\textbf{Forward:} If $[K:F]=1$, then $K$ has a basis $\{ \tilde{k} \}$ as an $F$-vector space. So for all nonzero $k \in K$,
\[
k = f_k \tilde{k}
\] for some nonzero $f_k \in F$. In particular, this holds for $k=1$, so we have
\begin{align*}
	1 &= f_1\tilde{k} \\
	f_1^{-1}&=\tilde{k}.
\end{align*}
Now since $F$ is closed under nonzero inverses, this means $\tilde{k} \in F$. Then for all $k \in K$, $k = f_k \tilde{k}$ is the product of two elements of $F$, so $k \in F$. This shows $K \subset F$. Since $K$ is an extension of $F$, we also have $F \subset K$, so $K=F$.

\begin{exer}[]
	Find the degree of $\sqrt[5]{2} $ over $\mathbb{Q}$. Then prove for each $a \in \mathbb{Q}(\sqrt[5]{2}) - \mathbb{Q}$, we have $\mathbb{Q}(a) = \mathbb{Q}(\sqrt[5]{2})$.
\end{exer}
$\sqrt[5]{2} $ is a root of $x^5-2$, which is irreducible over $\mathbb{Q}$ by Eisenstein's criterion for $p=2$, so the degree of $\sqrt[5]{2} $ over $\mathbb{Q}$ is 5.

Let $a \in \mathbb{Q}(a) - \mathbb{Q}$. Since $a$ is an element of $\mathbb{Q}(\sqrt[5]{2} )$, we have $\mathbb{Q}(a) \subset \mathbb{Q}(\sqrt[5]{2} )$. We then have the following tower.
\begin{center}
	\begin{tikzcd}
{\mathbb{Q}(\sqrt[5]{2})} \arrow[d, no head] \arrow[dd, "5"', no head, bend right] \\
\mathbb{Q}(a) \arrow[d, no head]                                                   \\
\mathbb{Q}                                                                        
\end{tikzcd}
\end{center}
Since degrees multiply in towers and $5$ is prime, the other two extensions in this tower must be of degree 1 and 5. Now $[\mathbb{Q}(a):\mathbb{Q}]$ cannot be 1, since $a \not\in \mathbb{Q}$ forces $\mathbb{Q}(a)$ to be distinct from $\mathbb{Q}$. Thus $\mathbb{Q}(a)$ has degree 5 over $\mathbb{Q}$ and $\mathbb{Q}(\sqrt[5]{2} )$ has degree $1$ over $\mathbb{Q}(a)$, meaning that $\mathbb{Q}(\sqrt[5]{2} )=\mathbb{Q}(a)$.

\newpage
\begin{exer}[]
	Prove that a finite field cannot be algebraically closed.
\end{exer}
Suppose $F$ is a finite field with elements $a_1, \dots, a_m$ for some arbitrary $m$. We can construct the polynomial
\[
	p(x) = (x-a_1) \cdots (x-a_m) + 1,
\] where 1 represents whichever of the $a_i$ is the multiplicative identity of $F$. Since in the last homework we proved that any finite field has $p^n$ elements, where $p$ is a prime, and since 1 is not a prime, we know that any finite field has at least 2 elements, i.e. the multiplicative and additive identities are distinct.

Thus $p(a_i) =1 \neq 0$ for all $a_i$, meaning that $p(x)$ has no roots in $F$, so $F$ is not algebraically closed.

\end{document}
