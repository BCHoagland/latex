\documentclass[10pt]{report}
\usepackage{/Users/bradenhoagland/latex/math}

\lhead{Braden Hoagland}
\chead{HW 5}
\rhead{}

%\renewcommand{\theenumi}{\alph{enumi}}

\begin{document}
%\tableofcontents

\begin{exer}[14.4: 2]
	Find a primitive generator for $\mathbb{Q}(\sqrt{2} ,\sqrt{3} ,\sqrt{5} )$ over $\mathbb{Q}$.
\end{exer}
The Galois group of $K \doteq \mathbb{Q}(\sqrt{2} ,\sqrt{3} ,\sqrt{5} )$ over $\mathbb{Q}$ is clearly all 8 distinct maps determined by
\begin{align*}
	\sqrt{2} &\mapsto \pm \sqrt{2} \\
	\sqrt{3} &\mapsto \pm \sqrt{3} \\
	\sqrt{5} &\mapsto \pm \sqrt{5} .
\end{align*}
Since $K$ is a degree 8 extension over $\mathbb{Q}$, this means $K$ is Galois over $\mathbb{Q}$. Now Consider $\theta=\sqrt{2} +\sqrt{3} +\sqrt{5} $. It is not fixed by any nontrivial element of the above Galois group, so by the Fundamental Theorem of Galois Theory, $\mathbb{Q}(\theta)$ cannot lie in any proper subfield of $K$. But since $\theta \in K$, we know $\mathbb{Q}(\theta)$ is a subfield of $K$. Thus $K = \mathbb{Q}(\theta)$.

\pagebreak
\begin{exer}[14.5: 7]
	Complex conjugation restricts to the automorphism $\sigma_{-1} \in \text{Gal}_{\mathbb{Q}}(\mathbb{Q}(\zeta_{n})$ of the cyclotomic field of $n$-th roots of unity. The field $K^{+}=\mathbb{Q}(\zeta_{n}+\zeta_{n}^{-1})$ is the subfield of real elements in $K=\mathbb{Q}(\zeta_{n})$, called the maximal real subfield of $K$.
\end{exer}
\textbf{Complex conjugation:} Using the definitions of roots of unity on page 539 of DF, any primitive $n$-th root of unity $\zeta_{n}$ can be written
\[
	\zeta_{n} = e^{2\pi k i/ n} = \cos\left( \frac{2\pi k}{n} \right)+i \sin \left( \frac{2\pi k}{n}  \right)
\] for some $0 \leq k \leq n-1$ (not all $k$ in this range will give a primitive root of unity, but the actual value of $k$ won't be needed in this proof). Then using the trigonometirc identities $\sin(-x)=-\sin(x)$ and $\cos(-x)=\cos(x)$, complex conjugation gives
\begin{align*}
	\overline{\zeta_{n}} &= \cos\left( \frac{2\pi k}{n} \right)-i \sin \left( \frac{2\pi k}{n}  \right) \\
			     &= \cos\left( -\frac{2\pi k}{n} \right)+i \sin \left( -\frac{2\pi k}{n}  \right) \\
			     &= e^{-2\pi k i/ n} \\
			     &= \zeta_{n}^{-1}.
\end{align*}
Since the elements of $\mathbb{Q}$ have no imaginary component, $\mathbb{Q}$ is fixed by complex conjugation. Thus it restricts to $\sigma_{-1} \in \text{Gal}_{\mathbb{Q}}(\mathbb{Q}(\zeta_{n}))$.

\textbf{Maximal Real Subfield:} Since $\zeta_{n}+\zeta_{n}^{-1} = \zeta_{n}+\overline{\zeta_{n}}= 2\cos(2\pi k /n)$ (the imaginary components cancel out), the extension $K^{+} = \mathbb{Q}(\zeta_{n}+\zeta_{n}^{-1})$ is real. This shows $K^{+}\subset K \cap\mathbb{R}$. This gives us the tower
\begin{figure}[H]
	\centering
\begin{tikzcd}
K                                                                                      \\
\mathbb{R} \cap K \arrow[u, no head]                                                   \\
K^+ \arrow[u, no head] \arrow[uu, "2"', no head, bend right=50]                           \\
\mathbb{Q} \arrow[u, "\phi(n)/2"', no head] \arrow[uuu, "\phi(n)", no head, bend left=50]
\end{tikzcd}
\end{figure}
where $[K^{+}:\mathbb{Q}]=\phi(n)/2$ was proven on the first midterm and $[K:K^{+}]=2$ follows from degrees multiplying in towers. This forces one of $[K:\mathbb{R}\cap K]$ and $[\mathbb{R}\cap K:K^{+}]$ to be 1. But the former cannot be 1, as $K$ contains complex elements and $\mathbb{R} \cap K$ does not. Thus $[\mathbb{R}\cap K:K^{+}]=1$, so $\mathbb{R}\cap K = K^{+}$.

\pagebreak
\begin{exer}[14.5: 10]
	$\mathbb{Q}(\sqrt[3]{2} )$ is not a subfield of any cyclotomic field over $\mathbb{Q}$.
\end{exer}
Because it was easier to type out, I denote $\sqrt[3]{2} $ by $\theta$.

First we show that $\mathbb{Q}(\theta)$ is not Galois over $\mathbb{Q}$. We know that $\theta$ is the root of the minimal polynomial $x^3-2$ over $\mathbb{Q}$ (this shows that $\mathbb{Q}(\theta)$ is a degree 3 extension of $\mathbb{Q}$). But this polynomial has roots
\[
\theta, \theta\zeta_3,\theta\zeta_3^2,
\] and $\theta\zeta_3,\theta\zeta_3^2$ aren't real (this follows from the explicit form of $\zeta_3$ on page 540 of DF). Since $\theta$, on the other hand, \textit{is} real, this means that $\theta$ is the only one of these roots contained in $\mathbb{Q}(\theta)$. Thus the Galois group can only map $\theta$ to itself, i.e. the group is trivial. Since the Galois group is order 1 but  $\mathbb{Q}(\theta)$ is a degree 3 extension over $\mathbb{Q}$, this means $\mathbb{Q}(\theta)$ is not Galois over $\mathbb{Q}$.

Now we can show the desired result. We know $\text{Gal}_{\mathbb{Q}}(\mathbb{Q}(\zeta_{n})) \cong \mathbb{Z}_{n}^{\times}$, which in particular shows that it is abelian. And every subgroup of an abelian group is necessarily normal, so by the Fundamental Theorem of Galois Theory, all subfields of $\mathbb{Q}(\zeta_{n})$ must be Galois over $\mathbb{Q}$. But we just showed that $\mathbb{Q}(\theta)$ is not Galois over $\mathbb{Q}$, so it cannot be one of these subfields.

\pagebreak
\begin{exer}[14.5: 13]
\end{exer}
\begin{enumerate}
	\item Since $a$ is relatively prime to $n$, it is relatively prime to the components of the prime decomposition. Then we have
		\[
			\sigma_{a}(\zeta_{p_i^{a_i}}) = \sigma_{a}(\zeta_{n}^{d_i}) = \sigma_{a}(\zeta_{n})^{d_i}= (\zeta_{n}^{a})^{d_i}= (\zeta_{n}^{d_i})^a=\zeta_{p_i^{a_i}}^{a}.
		\] It is also an automorphism of $\mathbb{Q}(\zeta_{p_i^{a_i}})/\mathbb{Q}$ since it still fixes $\mathbb{Q}$ and also maps $\mathbb{Q}(\zeta_{p_i^{a_i}})$ onto itself. It depends only on where $\zeta_{p_i^{a_i}}$ is mapped to, and this is determined by $a \; (\text{mod } p_i^{a_i})$.

	\item Then by Theorem 26, $\sigma_{a \; (\text{mod }p_i^{a_i})}$ defines an isomorphism
		\[
			(\mathbb{Z}/p_i^{a_i}\mathbb{Z})^{\times} \to \text{Gal}_{\mathbb{Q}}(\mathbb{Q}(\zeta_{p_i^{a_i}})).
		\] When we apply these isomorphisms componentwise to the decomposition of $\text{Gal}_{\mathbb{Q}}(\mathbb{Q}(\zeta_{n}))$, we get that the two products in Corollary 27 are isomorphic.
\end{enumerate}

\pagebreak
\begin{exer}[14.6: 2]
Determine the Galois groups of the following polynomials.
\begin{enumerate}
	\item $x^3-x^2-4$.
	\item $x^3-2x+4$.
	\item $x^3-x+1$.
	\item $x^3+x^2-2x-1$.
\end{enumerate}
\end{exer}
In all these, I denote the current polynomial's Galois group by $G$.
\begin{enumerate}
	\item $x^3-x^2-4$ factors into $(x-2)(x^2+x+2)$, and since $2 \in \mathbb{Q}$, its splitting field is determined by the second factor. The second factor's discriminant is $b^2-4ac=-7$. Since this isn't a rational square, DF Proposition 34 says that $G$ isn't a subgroup of $A_2$. Since $S_2$ has no proper subfields containing $A_2$, this means $G \cong S_2$.

	\item $x^3-2x+4$ factors into $(x+2)(x^2-2x+2)$, and since $-2 \in \mathbb{Q}$, its splitting field is determined by the second factor. The second factor's discriminant is $b^2-4ac=-4$, so by the same logic as in part (a), $G \cong S_2$.

	\item This polynomial is irreducible by the rational root test. Then by DF Equation 14.18', the discriminant of this polynomial is
		\[
		D=a^2b^2-4b^3-4a^3c-27c^2+18abc,
		\] where $a=0,b=-1,c=1$. This comes out to $D=-23$, and since this isn't a square, $G$ is not a subgroup of $A_3$. Once again, $S_3$ has no proper subsets containing $A_3$, so $G \cong S_3$.
	
	\item This polynomial is irreducible by the rational root test. Then again by Equation 14.18', for $a=1, b=-2, c=-1$, we get $D=49$. This is a square in $\mathbb{Q}$, so $G$ is a subgroup of $A_3$. But $|A_3|=3$, so its only proper subgroup is the trivial group. But since our polynomial's roots are not contained in $\mathbb{Q}$, its Galois group cannot be trivial, so $G \cong A_3$.
\end{enumerate}

\pagebreak
\begin{exer}[14.6: 6]
Determine the Galois group of $x^4+3x^3-3x-2$.
\end{exer}
This polynomial is irreducible by Eisenstein's Criterion for $p=2$. By the long ugly formula on the bottom of DF page 614, for $a=3,b=0,c=-3,d=-2$, we calculate the discriminant to be $D=-2183$. This isn't a square in $\mathbb{Q}$, so by DF Proposition 34, the polynomial's Galois group $G$ is not a subgroup of $A_4$.

Additionally, we can manually calculate the resolvent cubic to be
\[
	h(x) = \frac{1}{64} \left( x^3+432x^2+908x+9 \right).
\] This is irreducible by the rational root test, so now we're in case (a) of DF page 615. This means $G \cong S_4$.

\pagebreak
\begin{exer}[14.6: 11]
Let $F$ be a non-Galois degree 4 extension of $\mathbb{Q}$. Prove that the Galois closure of $F$ has Galois group either $S_4,A_4$, or $D_8$. Prove that the Galois group is dihedral if and only if $F$ contains a quadratic extension of $\mathbb{Q}$.
\end{exer}
\textbf{First bit:} By the primitive element theorem, since $F$ is a finite extension of a character 0 field, $F$ must be simple, i.e. $F = \mathbb{Q}(\alpha)$ for some $\alpha \in \mathbb{Q}$. Since it's a degree 4 extension, this means the minimal polynomial $m_{\alpha}(x)$ of $\alpha$ is quartic. Denote the splitting field of $m_{\alpha}(x)$ by $K$.

Since $K$ is a splitting field, it's Galois over $\mathbb{Q}$. Since there are 4 roots of $m_{\alpha}(x)$, this means $G \doteq \text{Gal}_{\mathbb{Q}}(K)$ is a subgroup of $S_{4}$.

Since $F$ is not Galois, we know $K \neq F$, so we have the following tower.
\begin{figure}[H]
	\centering
	\begin{tikzcd}
K \arrow[d, no head]      \\
F \arrow[d, "4", no head] \\
\mathbb{Q}
\end{tikzcd}
\end{figure}
Since $\left| S_{4} \right|=24$ and the order of a subgroup divides the order of the group, this means $[K:\mathbb{Q}] = |\text{Gal}_{\mathbb{Q}}(K)| $ divides 24. Then since degrees multiply in towers, this means $[K:F]$ divides $6$ (and isn't 1, since $K \neq F$). This means $[K:\mathbb{Q}]=8, 12, $ or $24$. Using the list at the top of DF page 614, the only possible subgroups of $S_4$ of these sizes are $D_8, A_4$, and $S_4$.

\textbf{Second bit:} Suppose $F$ contains a quadratic extension $\mathbb{Q}(\sqrt{D} )$. We then know $[F:\mathbb{Q}(\sqrt{D} )]=[\mathbb{Q}(\sqrt{D} ):\mathbb{Q}]=2$. Since $A_4$ has no subgroups of index 2, we know $G \neq A_4$. The only index 2 subgroup of $S_4$ is $A_4$, but since $A_4$ has no index 2 subgroups, we can't get 2 index 2 subgroups in a row, as required by the Galois correspondence. Thus $G = D_8$.

Conversely, suppose $ G=D_8$. All subgroups in the subgroup lattice of $D_8$ has index 2, so by the Galois correspondence and the fact that $F$ is degree 4 over $\mathbb{Q}$ instead of degree 2, $F$ must contain a proper subfield. This proper subfield must then be degree 2 over $\mathbb{Q}$, i.e. it is a quadratic extension.

\pagebreak
\begin{exer}[14.6: 44]
	Let $\alpha_1,\alpha_2,\alpha_3,\alpha_4$ be the roots of a quartic $f(x)$ over $\mathbb{Q}$. Show that the quantities $\alpha_1\alpha_2+\alpha_3\alpha_4, \alpha_1\alpha_3+\alpha_2\alpha_4,$ and $\alpha_1\alpha_4+\alpha_2\alpha_3$ are permuted by the Galois group of $f(x)$. Conclude that these elements are the roots of a cubic polynomial over $\mathbb{Q}$.
\end{exer}
The Galois group $G$ of $f(x)$ permutes the $\alpha_i$, so if $\sigma \in G$, then
\[
	\sigma(\alpha_1 \alpha_2+\alpha_3\alpha_4) = \sigma(\alpha_1)\sigma(\alpha_2)+\sigma(\alpha_3)\sigma(\alpha_4)
\] must be one of the other two expressions (note that since the roots all lie in some field, they commute under multiplication). Using similar logic for applying $\sigma$ to the other two expressions, we get that all three are permuted by $G$.

If we denote these three expressions by $a,b,c$, then they're the roots of the polynomial
\[
	(x-a)(x-b)(x-c).
\] We just showed that this polynomial is symmetric, so by DF Corollary 31 it is a rational function, i.e. it is a cubic polynomial over $\mathbb{Q}$.

\pagebreak
\begin{exer}[14.7: 3]
	Let $\text{char}(F)\neq 2$. State and prove a necessary and sufficient condition on $\alpha,\beta\in F$ so that $F(\sqrt{\alpha} ) = F(\sqrt{\beta} )$. Use this to determine whether $\mathbb{Q}(\sqrt{1-\sqrt{2} } )=\mathbb{Q}(i,\sqrt{2} )$.
\end{exer}
\textbf{First bit:} There are two trivial cases:
\begin{enumerate}
	\item If $\alpha$ and $\beta$ are both squares in $F$, then $\sqrt{\alpha} ,\sqrt{\beta} \in F$, so $F(\sqrt{\alpha} )=F = F(\sqrt{\beta} )$.
	\item If one is a square and the other isn't, then one of the extensions is equal to $F$ and the other isn't, so the extensions aren't equal.
\end{enumerate}
The final case is when neither $\alpha$ nor $\beta$ is a square in $F$, i.e. $\sqrt{\alpha} ,\sqrt{\beta} \not\in F$. Supposing $F(\sqrt{\alpha} )=F(\sqrt{\beta} )$, we can write $\sqrt{\alpha} $ as $\sqrt{\alpha} =c + d \sqrt{\beta} $ for some $c,d \in F$. Squaring gives $\alpha=c^2+2cd\sqrt{\beta} +d^2\beta$. Since $\alpha \in F$ but $\sqrt{\beta} \not\in F$, this forces $2cd=0$. Since $\text{char}(F)\neq 2$, this means either $c$ or $d$ is 0. If $d=0$, then $\sqrt{\alpha} =c \in F$, but $\sqrt{\alpha} \not\in F$ by assumption, so it must be the case that $c=0, d\neq 0$. This gives $\alpha = d^2\beta$, or
\[
\frac{\alpha}{\beta} =d^2.
\] 
So if $F(\sqrt{\alpha})=F(\sqrt{\beta} )$, then $\alpha/\beta$ is a square in $F$. This condition is also sufficient: if $\alpha/\beta = d^2$ for some $d \in F$, then $\sqrt{\alpha} =d \sqrt{\beta} $, so $F(\sqrt{\alpha} )=F(\sqrt{\beta} )$.

\textbf{Second bit:} Let $F=\mathbb{Q}(\sqrt{2} )$, $\alpha=1-\sqrt{2} , \beta=-1$, then
\begin{align*}
	F(\sqrt{\alpha} )&=\mathbb{Q}(\sqrt{1-\sqrt{2} } ,\sqrt{2} )=\mathbb{Q}(\sqrt{1-\sqrt{2} } ), \\
	F(\sqrt{\beta} )&=\mathbb{Q}(i, \sqrt{2} ).
\end{align*}
Note that the last equality on the first line above comes from this extension being of degree higher than 2, so $((1-\sqrt{2} )^{1/2})^{2}= 1-\sqrt{2}$ is a basis element of that extension, so $\sqrt{2} \in \mathbb{Q}((1-\sqrt{2} )^{1/2}$.

Suppose $\alpha/\beta=\sqrt{2} -1$ is a square in $F=\mathbb{Q}(\sqrt{2} )$, then
\[
	\sqrt{2} -1=(c+d\sqrt{2} )^2=c^2+2cd\sqrt{2} +2d^2
\] for some $c,d \in \mathbb{Q}$. Since $c,d,c^2,d^2 \in \mathbb{Q}$ and $\sqrt{2} \neq \mathbb{Q}$, this implies
\begin{align*}
	c^2+2d^2&=-1,\\
	2cd&=1.
\end{align*}
The first equality in this system is impossible, though, so the two extensions are not equal.

\end{document}
