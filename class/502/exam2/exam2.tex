\documentclass[twoside,10pt]{report}
\newcommand{\docTitle}{Exam 2}
\usepackage{/Users/bradenhoagland/latex/math2}

%\renewcommand{\theenumi}{\alph{enumi}}

\begin{document}
%\tableofcontents

\begin{exer}[]
	Let $K=\mathbb{Q}(\sqrt{2+\sqrt{2} } )$. Prove that $K$ is Galois over $\mathbb{Q}$. Explicitly describe the $\mathbb{Q}$-automorphisms of $K$ to determine the Galois group of this extension, and draw the corresponding subgroup and subfield lattices.
\end{exer}
\textbf{$K$ is Galois over $\mathbb{Q}$:} Define $\theta \doteq \sqrt{2+\sqrt{2} } $, then $\theta^2=2+\sqrt{2} $ and $\theta^{4}=6+4\sqrt{2} =4\theta^2-2$. Thus $\theta$ is a root of $f(x) = x^{4}-4x^2+2$. Since $f(x)$ is irreducible over $\mathbb{Q}$ by Eisenstein's criterion for $p=2$, $[K:\mathbb{Q}]=4$.

If we let $\theta' \doteq \sqrt{2-\sqrt{2} } $, then we can check that $\pm \theta,\pm\theta'$ are the roots of $f(x)$. Since these roots are all distinct, $f(x)$ is separable. Then by \S 14.1 Corollary 6, if we can show that $K$ is actually the splitting field of $f(x)$, then $K$ is Galois over $\mathbb{Q}$.

To start, note that $\theta^{2}-2=\sqrt{2} $, so $\sqrt{2} \in K$. Also, $\theta^{-1}$ must necessarily be in $K$. Then
\[
	\sqrt{2} \theta^{-1} = \frac{\sqrt{2} }{\sqrt{2+\sqrt{2} } } = \frac{\sqrt{2} }{\sqrt{2+\sqrt{2} } } \frac{\sqrt{2-\sqrt{2} } }{\sqrt{2} -\sqrt{2} } = \sqrt{2-\sqrt{2} } =\theta'\in K.
\] 
Thus $\pm \theta,\pm\theta'$ (all the roots of $f(x)$) are in $K$, so $K$ is the splitting field of a separable polynomial and thus Galois over $\mathbb{Q}$.

\textbf{Galois Group of $K$ over $\mathbb{Q}$:} Let $G \doteq \text{Gal}_{\mathbb{Q}}(K)$. Since $K/\mathbb{Q}$ is Galois, we know $|G|=[K:\mathbb{Q}]=4$. Then by the list on DF page 614, the only possible subgroups of $S_4$ with order 4 are $V$ (the Klein four-group) and $C$ (the cyclic group of order 4).

We will now show that $G$ has an order 4 element, meaning that $G=C$. Since $\theta$ and $\theta'$ are roots of the same irreducible polynomial, $G$ permutes them. Suppose $\sigma\in G$ maps $\theta \mapsto \theta'$, then since $\sqrt{2} =\theta^{2}-2$,
\[
	\sigma(\sigma(\theta)) = \sigma(\theta') = \sigma\left( \frac{\theta^{2}-2 }{\theta }  \right) = \frac{\sigma(\theta)^{2}-2}{\sigma(\theta)} = \frac{\theta'^{2}-2}{\theta'} = \frac{-\sqrt{2} }{\sqrt{2-\sqrt{2} } } =-\theta.
\] 
Thus the order of $\theta$ is greater than 2, but we also know that it must divide 4 (the order of the whole group). This forces $|\theta|=4$, so $G$ is cyclic, i.e. $G \cong C \cong \mathbb{Z}_{4}$.

\textbf{Subgroup and subfield lattices:} We know the subgroups of $\mathbb{Z}_{4}$, so we can use the Galois correspondence to determine the orders of the subfields of $K$. Since $\mathbb{Z}_2$ is the only nontrivial proper subgroup of $\mathbb{Z}_{4}$ and it has order 2, we know that there is only one intermediate field in the subfield lattice of $K$ and that it has degree 2 over $\mathbb{Q}$.

As remarked earlier, $\sqrt{2} \in K$, so $\mathbb{Q}(\sqrt{2} )\subset K$. Since $[\mathbb{Q}(\sqrt{2} ):\mathbb{Q}]=2$, we have found the subfield of $K$. The two lattices are then as pictured below.

\begin{center}
	\begin{tikzcd}
		G=\mathbb{Z}_4\arrow[d,dash]&K\arrow[d,dash]\\
		\mathbb{Z}_2\arrow[d,dash]&\mathbb{Q}(\sqrt{2})\arrow[d,dash]\\
		\mathbb{Z}_1=\{0\}&\mathbb{Q}
	\end{tikzcd}
\end{center}

\newpage
\begin{exer}[]
	Let $f(x) \in \mathbb{Q}[x]$ be a cubic polynomial and let $K\subset \mathbb{C}$ be a splitting field of $f$ over $\mathbb{Q}$. If $[K:\mathbb{Q}]=3$, then all the roots of $f$ are real.
\end{exer}
Supopse $c \in \mathbb{C}-\mathbb{R}$ is a complex root of $f(x)$, then its complex conjugate $\overline{c}$ is known to also be a root of $f(x)$. Since odd degree polynomials always have at least one root, this forces the third root to be real. Thus we can represent $f(x)$ by
\[
	f(x) = (x-c)(x-\overline{c})(x-\alpha),
\] where $\alpha$ is the real root. This shows $f$ is separable, so by \S 14.1 Corollary 6, $K$ is Galois over $\mathbb{Q}$. Let $G \doteq \text{Gal}_{\mathbb{Q}}(K)$, then by the Galois correspondence, since $[K:\mathbb{Q}]=3$, we know $|G|=3$.

If $f$ had complex roots, then $c \mapsto \overline{c}$ would be a $\mathbb{Q}$-automorphism and thus belong to $G$. But this particular map has order 2, and the order of a group element must divide the order of the group, so this is impossible. Thus all the roots of $f(x)$ are real.

\newpage
\begin{exer}[]
	Let $K$ be a splitting field of $f(x)=x^{4}-5$ over $\mathbb{Q}$. Show that there cannot be a $\mathbb{Q}$-automorphism of $K$ that fixes exactly one root of $f$.
\end{exer}
Let $\theta \doteq \sqrt[4]{5} $, then the roots of $f(x)$ are $\theta,\theta\zeta_4,\theta\zeta_4^{2},\theta\zeta_{4}^{3}$, so its splitting field is $K=\mathbb{Q}(\theta,\zeta_4)$. But by the list on DF page 540, $\zeta_4=i$, so the splitting field is really $K=\mathbb{Q}(\theta,i)$.

Since $f(x)$ is irreducible over $\mathbb{Q}$ by Eisenstein's criterion for $p=5$, we know $[\mathbb{Q}(\theta):\mathbb{Q}]=4$. Since each $\zeta^{n}$ is either complex or an integer, $\pm\sqrt{2} \not\in \mathbb{Q}(\theta)$. This means the polynomial $x^2-2$ has no roots in $\mathbb{Q}(\theta)$, but since this polynomial is quadratic, that's equivalent to it being irreducible over $\mathbb{Q}(\theta)$. Then since $\sqrt{2} $ is a root of this polynomial, $[K:\mathbb{Q}(\theta)]=2$. Then since degrees multiply in towers, $[K:\mathbb{Q}]=8$. Since $f(x)$ has four distinct roots (i.e. is separable), by \S 14.1 Corollary 6, its splitting field $K$ is Galois over $\mathbb{Q}$. Then by the Galois correspondence, we know its Galois group has 8 elements.

If we define maps that permute the roots of $f(x)$ by
\[
\sigma:\zeta^{n}\mapsto \zeta^{n+1}, \quad \tau:\zeta^{n}\mapsto \zeta^{n+2}, \quad \pi:\zeta^{n}\mapsto \zeta^{n+3},
\] (where $\theta=\theta\zeta^{0}$), then the subgroup they generate is
\[
\left\langle \sigma,\tau,\pi \right\rangle = \left\{ 1,\sigma,\sigma^{2},\sigma^{3},\tau,\pi,\pi^{2},\pi^{3} \right\}.
\] Furthermore, since each $\zeta$ is complex and the $\zeta$'s are all that change, each of these maps is a $\mathbb{Q}$-automorphism. But since there are 8 of these and we know that the Galois group has 8 elements, we have found all possible $\mathbb{Q}$-automorphisms. Since none of these fix only one root of $f(x)$, we are done.

\newpage
\begin{exer}[]
	Determine the Galois group of $\mathbb{Q}(\zeta_{7})$ over $\mathbb{Q}$ and find all intermediate fields. What is the minimal polynomial of $\zeta_7+\zeta_7^{-1}$ over $\mathbb{Q}$?
\end{exer}
\textbf{Galois group and subfields:} Let $\zeta\doteq \zeta_{7}$. The \S 14.5 Theorem 26, $G \doteq \text{Gal}_{\mathbb{Q}}(\mathbb{Q}(\zeta))\cong \mathbb{Z}_{7}^{\times}\cong \mathbb{Z}_{6}$. We know that the subgroups of $\mathbb{Z}_n$ correspond to the divisors of $n$, which gives us the structure of the subgroup lattice of $G$. Using the map $\sigma_{a}: \zeta\mapsto \zeta^{a}$ (this map was defined for $a$ relatively prime to 7, but 7 is prime so any $a< 7$ will work), we get an isomorphic copy of the lattice.
\begin{center}
	\begin{tikzcd}
		&G\cong\mathbb{Z}_6\arrow[dl,dash]\arrow[dr,dash]&&&\langle\sigma_3\rangle=\langle\sigma_5\rangle\arrow[dl,dash]\arrow[dr,dash]\\
		\mathbb{Z}_2\arrow[dr,dash]&&\mathbb{Z}_3\arrow[dl,dash]&\langle\sigma_6\rangle\arrow[dr,dash]&&\langle\sigma_2\rangle=\langle\sigma_4\rangle\arrow[dl,dash]\\
					   &\mathbb{Z}_1 = \{0\}&&&\langle\sigma_1\rangle
	\end{tikzcd}
\end{center}
By the Galois correspondence, we know there are two proper subfields of $\mathbb{Q}(\zeta_{7})$: the fixed fields of $\left\langle \sigma_{6} \right\rangle$ and $\left\langle \sigma_{2} \right\rangle=\left\langle \sigma_{4} \right\rangle$ (from now on, I work with $\left\langle \sigma_2 \right\rangle$ instead of $\left\langle \sigma_4 \right\rangle$ since it doesn't matter which one I choose).

Following example 2 on DF page 597, since 7 is odd and we're working with $\mathbb{Q}(\zeta_7)$, we know that the fixed fields of $\left\langle \sigma_6 \right\rangle$ and $\left\langle \sigma_2 \right\rangle$ are given by $\mathbb{Q}(\alpha)$ and $\mathbb{Q}(\beta)$, respectively, where
\begin{align*}
	\alpha &= \sum_{\tau \in \left\langle \sigma_6 \right\rangle}\tau \zeta \\
	\beta &= \sum_{\tau \in \left\langle \sigma_2 \right\rangle}\tau \zeta .
\end{align*}

Since $\left\langle \sigma_6 \right\rangle= \left\{ \sigma_1,\sigma_6 \right\}$ and $\left\langle \sigma_2 \right\rangle=\left\{ \sigma_1,\sigma_2,\sigma_4 \right\}$, these evaluate to $\alpha = \zeta+\zeta^{6}$ and $\beta=\zeta+\zeta^{2}+\zeta^{4}$. Thus the subfields of $\mathbb{Q}(\zeta)$ are $\mathbb{Q}(\zeta+\zeta^{6})$ and $\mathbb{Q}(\zeta+\zeta^{2}+\zeta^{4})$. The subfield lattice is pictured below.

\begin{center}
	\begin{tikzcd}
		&\mathbb{Q}(\zeta)\arrow[dl,dash]\arrow[dr,dash]\\
		\mathbb{Q}(\zeta+\zeta^6)\arrow[dr,dash]&&\mathbb{Q}(\zeta+\zeta^2+\zeta^6)\arrow[dl,dash]\\
						       &\mathbb{Q}
	\end{tikzcd}
\end{center}

\textbf{Minimal polynomial:} Let $\alpha \doteq \zeta+\zeta^{-1}=\zeta+\zeta^{6}$. Then we manually calculate $\alpha^{2}=\zeta^{5}+\zeta^{2}+5$ and $\alpha^{3}=3\zeta^{6}+\zeta^{4}+\zeta^{3}+3\zeta$. Now $\zeta$ is a root of the 7th cyclotomic polynomial, which we can express in terms of $\alpha,\alpha^{2},$ and $\alpha^3$. We have
\[
	0=\Phi_{7}(\zeta)=\zeta^{6}+\zeta^{5}+\cdots+\zeta^{1}+1 = \alpha^{3}+\alpha^{2}-2\alpha-1,
\] so $\alpha = \zeta+\zeta^{-1}$ is a root of the polynomial $x^{3}+x^2-2x-1$. Since this is irreducible over $\mathbb{Q}$ by the rational root test, it is the minimal polynomial of $\zeta+\zeta^{-1}$ over $\mathbb{Q}$.

\newpage
\begin{exer}[]
	Construct (with justification) an example of a Galois extension whose Galois group is $Z_2\times Z_6$.
\end{exer}
Before constructing the extension, we note that such an extension must exist. This is because $Z_{2}\times Z_{6}$, as the product of finite abelian groups, is itself finite abelian. Then by \S 14.5 Corollary 28, there is some subfield of a cyclotomic extension whose Galois group is $Z_{2}\times Z_6$.

Now consider $\mathbb{Q}(\zeta_{21})$, which we know to be Galois over $\mathbb{Q}$. Since $21$ has prime decomposition $21=3 \cdot 7$, by \S 14.5 Corollary 27,
\begin{align*}
	\text{Gal}_{\mathbb{Q}}(\mathbb{Q}(\zeta_{21})) &\cong \text{Gal}_{\mathbb{Q}}(\mathbb{Q}(\zeta_3))\times \text{Gal}_{\mathbb{Q}}(\mathbb{Q}(\zeta_7)) \\
							&\cong \mathbb{Z}_{3}^{\times}\times \mathbb{Z}_{7}^{\times}\\
							&\cong Z_{2}\times Z_{6}.
\end{align*}
Thus $\mathbb{Q}(\zeta_{21})$ is a Galois extension whose Galois group is $Z_2\times Z_6$.

\newpage
\begin{exer}[]
	If $K$ is a root extension of $F$ and $E$ is an intermediate field, then $K$ is a root extension of $E$.
\end{exer}
By assumption,
\[
F = K_0\subset K_1 \subset \cdots \subset K_s=K,
\] for some $s$, where $K_{i+1}=K_i\left( \sqrt[n_i]{a_i}  \right)$ for some $a_i \in K_i$. If we let $\theta_i \doteq \sqrt[n_i]{a_i}$, then
\[
	K = K_s = F(\theta_1, \dots, \theta_s).
\] 
Now suppose $E$ is an intermediate field, i.e. $F \subset E \subset K$. If $E$ happens to be one of the $K_i$ above, then $K$ is a root extension of $E$: we just append to $E$ all $\theta_j$ for $j>i$.

If $E$ is not one of the $K_i$, then $K$ is still a root extension of $E$. If we append all $\theta_i$ to $E$, we get the chain
\[
E = E_0 \subset E_1 \subset \cdots \subset E_s,
\] where $E_{i+1}=E_i(\theta_i)$. Since $E \subset K$ and $\theta_i \in K$ for all $i$, we know $E_s \subset K$. Conversely, since $F \subset E$, we get $K = F_s =F(\theta_1,\dots,\theta_s)\subset E(\theta_1,\dots,\theta_s)=E_s$. Thus $E_s=K$, so $K$ is a root extension of $E$.

\newpage
\begin{exer}[]
If $f:A\to A$ is an $R$-module homomorphism such that $ff=f$, then $A = \ker f \oplus \im f$.
\end{exer}
Let $a \in A$ be arbitrary, then consider $a-f(a)$. Mapping this under $f$ and using the condition $f \circ f=f$ along with the fact that $f$ is a homomorphism gives
\[
	f(a-f(a)) = f(a) - f(f(a)) = f(a)-f(a)=0.
\] Thus $a-f(a) \in \ker f$. But $a=a-f(a)+f(a)$, so we have written $a$ as a sum of an element of the kernel of $f$ and an element of the image of $f$. Thus $A = \ker f + \im f$.

Now we show that $\ker f$ and $\im f$ have trivial intersection. Suppose $\tilde{a} \in \ker f \cap \im f$, then $f(\tilde{a})=0$ and $\tilde{a}=f(a)$ for some $a \in A$. Then since $f\circ f=f$,
\[
	\tilde{a} = f(a) = f(f(a)) = f(\tilde{a}) = 0.
\] 
Thus $\tilde{a}$ is 0, so the intersection of $\ker f$ and $\im f$ is trivial. This shows that $A = \ker f \oplus \im f$.

\newpage
\begin{exer}[]
	Let $R$ be a commutative ring with 1 and let $M$ be a left $R$-module. Show that $\text{Hom}_{R}(R,M) \cong M$ (as $R$-modules).
\end{exer}
Define the map
\begin{align*}
	\phi:\hom_{R}(R,M) &\to M \\
	f&\mapsto f(1).
\end{align*}
This is well-defined since $R$ is assumed to have 1. We claim that $\phi$ is an isomorphism.

\textbf{Homomorphism:} By the definitions of function addition and the $R$ action on $\hom_{R}(R,M)$, for $r \in R$, $f,g \in \hom_R(R,M)$,
\begin{align*}
	\phi(rf+g) &= (rf+g)(1) \\
			    &= (rf)(1)+g(1) \\
			    &= rf(1)+g(1) \\
			    &= r\phi(f)+\phi(g).
\end{align*}
Thus $\phi$ is an $R$-module homomorphism.

\textbf{Bijective:} Let $m \in M$ be arbitrary and consider the map $f_{m}(r) = rm$. By the definition of a module, for $s,r_1,r_2 \in R$,
\begin{align*}
	f_{m}(sr_1+r_2) &= (sr_1+r_2)m \\
			      &= (sr_1)m+r_2m \\
			      &= s(r_1m)+r_2m \\
			      &= sf_{m}(r_1)+f_{m}(r_2),
\end{align*}
so $f_{m} \in \hom_{R}(R,M)$. Since $\phi(f_{m})=f_{m}(1)=m$ and $m$ was arbitrary, this means $\phi$ is surjective.

Now suppose $g \in \ker \phi$, i.e. $\phi(g) = g(1)=0$. Then since $g$ is by assumption a homomorphism, for all $r \in R$,
\[
	g(r) = g(1\cdot r) = g(1)\cdot g(r)=0 \cdot g(r) =0.
\]
Thus $g$ is the trivial homomorphism, so the kernel of $\phi$ is trivial, so $\phi$ is injective. This shows that $\phi$ is a bijective $R$-module homomorphism, i.e. an $R$-module isomorphism, so $\hom_{R}(R,M) \cong M$.


\end{document}
