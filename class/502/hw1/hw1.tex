\documentclass[10pt]{report}
\usepackage{/Users/bradenhoagland/latex/math}

\lhead{Braden Hoagland}
\chead{HW 1}
\rhead{}

\renewcommand{\theenumi}{\alph{enumi}}

\begin{document}
%\tableofcontents

\begin{exer}[DF 9.2: 5]
	Exhibit all the ideals in the ring $F[x]/(p(x))$, where $F$ is a field and $p(x) \in F[x]$ (describe them in terms of the factorization of $p(x)$).
\end{exer}
By the fourth isomorphism theorem for rings, a subring $A$ of $F[x]$ containing $(p(x))$ is an ideal of $F[x]$ if and only if $A/(p(x))$ is an ideal of $F[x]/(p(x))$. Thus this problem reduces to describing the ideals of $F[x]$.

Since $F$ is a field, $F[x]$ is a unique factorization domain and a principal ideal domain. Then we can write $p(x)$ as
\[
	p(x) = \prod_{i=1}^m p_i(x)
\] for some $m$, and we can generate any ideal of $F[x]$ with a single polynomial. Now $(p(x)) \subset (q(x))$ if and only if $q(x)$ divides $p(x)$, so all ideals of $F[x]$ containing $(p(x))$ can be generated the factors of $p(x)$. Thus the ideals of $F[x]$ are of the form $(p_{i_1}\cdots p_{i_k})$ for $k \leq m$, and the ideals of $F[x]/(p(x))$ are of the form
\[
	(p_{i_1}\cdots p_{i_k}) / (p(x)) .
\] 

\begin{exer}[DF 9.4: 1]
	Determine whether the following polynomials are irreducible in the rings indicated. For those that are reducible, determine their factorization into irreducibles.
	\begin{enumerate}
		\item $x^2+x+1$ in $\mathbb{F}_2[x]$.
		\item $x^3+x+1$ in $\mathbb{F}_3[x]$.
		\item $x^4+1$ in $\mathbb{F}_5[x]$.
		\item $x^4+10x^2+1$ in $\mathbb{Z}[x]$.
	\end{enumerate}
\end{exer}
\begin{enumerate}
	\item Since this polynomial is degree 2, its reducibility coincides with the existence of roots in $\mathbb{F}_2$. Since the polynomial is nonzero when evaluated at both $0$ and $1$, it is irreducible.
	\item This polynomial has the root $1 \in \mathbb{F}_3$, so we can write it as $x^3+x+1 = (x-1)(x^2+x+2)$. Then since $x^2+x+2$ has no roots in $\mathbb{F}_3$, we have expressed the original polynomial in terms of irreducibles.
	\item We can write this polynomial as $(2x^2+1)(3x^2+1)$. Neither of these quadratics have roots in $\mathbb{F}_5$, so they are both irreducible.
	\item This polynomial is irreducible in $\mathbb{Q}[x]$ (and by extension in $\mathbb{Z}[x]$) by Eisenstein's criterion for $p=2$.
\end{enumerate}

\begin{exer}[DF 13.1: 1]
	Show that $p(x)=x^3+9x+6$ is irreducible in $\mathbb{Q}[x]$. Let $\theta$ be a root of $p(x)$. Find the inverse of $1+\theta$ in $\mathbb{Q}(\theta)$.
\end{exer}
The polynomial $p(x)$ is irreducible in $\mathbb{Q}[x]$ by Eisenstein's criterion for the prime $3$. We can use the fact that $\theta^3+9\theta+6=0$ to calculate the inverse of $1+\theta$ in $\mathbb{Q}(\theta)$.

Any element in $\mathbb{Q}(\theta)$ can be written
\[
a_0+a_1\theta+a_2\theta^2,
\] for $a_i \in \mathbb{Q}$. We wish to find such an element of $\mathbb{Q}(\theta)$ such that it multiplies with $1+\theta$ to yield $1$. We have
\begin{align*}
	1 &= (1+\theta)(a_0+a_1\theta+a_2\theta^2) \\
	0 &= (a_0-1) + (a_0+a_1)\theta + (a_1+a_2)\theta^2 + a_2\theta^3.
	\intertext{Since, as noted earlier, $\theta^3 = -9\theta-6$, this becomes}
	0 &= (a_0-6a_2-1) + (a_0+a_1-9a_2)\theta+(a_1+a_2)\theta^2.
\end{align*}
This gives a system in $(a_0-6a_2-1)$, $(a_0+a_1-9a_2) $, and $(a_1+a_2)$ all equal 0. Solving the system yields
\[
	a_0=\frac{5}{2} ,\quad a_1=-\frac{1}{4} ,\quad\text{and } a_2=\frac{1}{4} .
\] Thus the inverse of $1+\theta$ in $\mathbb{Q}(\theta)$ is
\[
	\frac{5}{2} -\frac{1}{4} \theta + \frac{1}{4} \theta^2.
\] 



\begin{exer}[DF 13.1: 3]
	Show that $x^3+x+1$ is ireducible over $\mathbb{F}_2$ and let $\theta$ be a root. Compute the powers of $\theta$ in $\mathbb{F}_2(\theta)$.
\end{exer}
Let $p(x)=x^3+x+1$, then since $p(0)$ and $p(1)$ are both nonzero, $p(x)$ has no roots in $\mathbb{F}_2$, so it is irreducible in $\mathbb{F}_2[x]$. Since $p(x)$ has degree 3, the field $\mathbb{F}_2(\theta)$ has basis $\left\{ 1,\theta, \theta^2 \right\}$. Thus we need only compute the powers of $\theta$ that are greater than 2.

\begin{itemize}
	\item Since $p(\theta) = \theta^3+\theta+1=0$, we have $\theta^3=-\theta-1=\theta+1$.
	\item $\theta^4=\theta^3\theta=\theta^2+\theta$.
	\item $\theta^5 = \theta^4\theta=\theta^3+\theta^2=1+\theta+\theta^2$.
	\item $\theta^6=\theta+\theta^2+\theta^3=1=\theta^2$.
	\item $\theta^7=\theta+\theta^3=1$.
\end{itemize}
From here the pattern repeats.

\begin{exer}[DF 13.1: 7]
Prove that $x^3-nx+2$ is irreducible for $n \neq -1,3,5$.
\end{exer}
Let $f_n(x) = x^3-nx+2$. If a rational root $c/d$ of $f_n(x)$ exists, then it satisfies $c \;|\; a_0=2$ and $d\;|\;a_n=1$. This means that the possible values of $c$ and $d$ are $c = \pm 1, \pm 2$ and $d=\pm 1$, so the possible rational roots are $\pm 1$ and $\pm 2$. Evaluating $f_n(x)$ at these points yields
\begin{align*}
	f_n(1) &= 3-n \\
	f_n(-1) &= 1+n \\
	f_n(2)&=2(5-n) \\
	f_n(-2)&=-2(3-n).
\end{align*}
We are given that $n\neq -1,3,5$, however, so these quantities can never be 0. Thus $f_n(x)$ is irreducible over $\mathbb{Q}$.

\begin{exer}[DF 13.2: 1]
	Let $\mathbb{F}$ be a finite field of characteristic $p$. Prove that $|\mathbb{F}|=p^n$ for some positive integer $n$.
\end{exer}
We know that any field $\mathbb{F}$ is an extension of it prime subfield $F$, and since the character of $\mathbb{F}$ is $p$, $F$ is isomorphic to $\mathbb{F}_p$. Since $\mathbb{F}$ is finite, there is some basis $\left\{ \alpha_i \right\}_{i=1}^n$ for $\mathbb{F}$ as an $F-$ vector space. This means we can write all elements of $\mathbb{F}$ uniquely as
\[
\sum_{i=1}^{n} f_i \alpha_i,
\] where $f_i \in F$. Since there are $p$ different $f_i$, there are $p^n$ different ways of assigning the $f_i$, so this means that  $|\mathbb{F}|=p^n$.

\begin{exer}[DF 13.2: 3]
Determine the minimal polynomial over $\mathbb{Q}$ for the element $1+i$.
\end{exer}
The element $1+i$ is a root of $x-1+i$, but this is not over $\mathbb{Q}$. We can remove the $i$ by multiplying by the polynomial with the conjugate $1-i$ as a root, which will give us a polynomial which is over $\mathbb{Q}$ instead of $\mathbb{C}$. We get
\begin{align*}
	(x-(1-i)) (x-(1+i)) = x^2-2x+2,
\end{align*}
which is over $\mathbb{Q}$ and contians $1-i$ as a root. As it turns out, this is the exact polynomial we're looking for. By Eisenstein's criterion for $p=2$, it is irreducible over $\mathbb{Q}$. Thus it is the minimal polynomial over $\mathbb{Q}$ for $1+i$.

\begin{exer}[DF 13.2: 16]
Let $K/F$ be an algebraic extension and let $R$ be a ring contained in $K$ and containing $F$. Show that $R$ is a subfield of $K$ containing $F$.
\end{exer}
Let $r \in R$ be nonzero, then $r$ is necessarily in $K$, so it is algebraic over $F$. Then there exists a minimal polynomial $m_r(x) = \sum_{i=0}^{n} c_0 k^i$ over $F$ with $r$ as a root. Now $c_0$ must be nonzero, since otherwise we could factor out an $x$ from it $m_r(x)$, meaning it wouldn't be irreducible. Since $c_0$ is nonzero and is an element of a field, it has an inverse $c_0^{-1}$. We can then use this inverse to calculate an inverse for $r$:
\begin{align*}
	c_0+c_1r+\cdots+c_nr^n &= 0 \\
	(-c_0^{-1})(c_0+c_1r+\cdots+c_nr^n) &= 0 \\
	(-c_0^{-1})(c_1r+\cdots+c_nr^n) &= 1 \\
	r (-c_0^{-1})(c_1+\cdots+c_nr^{n-1}) &= 1
\end{align*}
Since we have found an inverse for an arbitrary nonzero element of $R$, it must be a field.

\begin{exer}[]
	If $a_0, \dots, a_n$ are distinct elements of a field $F$ and $b_0,\dots, b_n$ are any elements of $F$, then there is at most one polynomial $f\in F[x]$ with $\deg f\leq n$ such that $f(a_i) = b_i$ for $i=0, 1, \dots, n$.
\end{exer}
We will show this by contradiction. Suppose $f$ and $g$ are both polynomials of degree no greater than $n$ satisfying $f(a_i)=g(a_i)=b_i$ for all $i$. Then $(f-g)(a_i)=0$ for all $i$, meaning that it has $n+1$ roots; however, $f-g$ has degree no greater than $n$, so it can have no more than $n$ roots if it is nonzero. Thus $f-g$ is actually the zero polynomial, so $f=g$, so we can have no more than 1 polynomial satisfying the given condition.

\begin{exer}[]
	Construct a field of 27 elements.
\end{exer}
Let $p(x)=x^3+x^2+2$ be a polynomial over $\mathbb{F}_3$, then we claim that $F = \mathbb{F}_3/(p(x))$ is a field with 27 elements. The polynomial $p(x)$ is irreducible over $\mathbb{F}_3$ since it has no roots in $\mathbb{F}_3$, so the quotient $F$ is a degree 3 field extension of $\mathbb{F}_3$. This means the elements of $F$ can all be uniquely written in the form
\[
f_1 + f_2\theta + f_3\theta^2,
\] where $f_i \in \mathbb{F}_3$ and $\theta$ is a root of $p(x)$. Since there are $3^3=27$ possible ways of assigning the $f_i$, there are 27 elements of $F$.

\end{document}
