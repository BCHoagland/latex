\documentclass[10pt]{report}
\usepackage{/Users/bradenhoagland/latex/math}

\lhead{Braden Hoagland}
\chead{HW 2}
\rhead{}

\renewcommand{\theenumi}{\alph{enumi}}

\newmdtheoremenv{challenge}{Challenge}

\begin{document}

{\color{red}Problems completed: All (except challenge).}

\begin{exer}[6 points]
	Munkres, \S 13, pg. 83 \#7. You needn't compare $\mathcal{T}_1$ and $\mathcal{T}_3$ as this was an exercise on Homework 1.
\end{exer}
{\color{blue}Collaborators: None.}

Below is a figure showing the relations between each of the five topologies, with vertical position indicating which other topologies a certain topology contains.

\begin{figure}[H]
\centering	
\begin{tikzcd}
\mathcal{T}_2                     &                                                                           & \mathcal{T}_4 \\
                                  & \mathcal{T}_1 \arrow[rd, no head] \arrow[ru, no head] \arrow[lu, no head] &               \\
\mathcal{T}_3 \arrow[ru, no head] &                                                                           & \mathcal{T}_5
\end{tikzcd}
\end{figure}

In order to determine all these relations, transitivity of inclusion means that we only need to compare a select few of the toplogies. From homework 1, we already know that $\mathcal{T}_3$ is strictly coarser than $\mathcal{T}_1$.

\textbf{$\mathcal{T}_1$ vs $\mathcal{T}_2$:} Since the basis for the $K$-topology on $\mathbb{R}$ is $\left\{ (a,b) \;|\; a<b \right\} \cup \left\{ (a,b)-K \;|\; a<b \right\}$, it contains the entire basis for the standard topology. If $B=(-1,1) - K$, though, we cannot find an open interval of the form $(a,b)$ that contains $0$ but remains in $B$. Thus $\mathcal{T}_1$ is strictly coarser than $\mathcal{T}_2$.

\textbf{$\mathcal{T}_1$ vs $\mathcal{T}_4$:} If $B=(a,b)$ is in the standard basis and $x \in B$, then $(a,x]$ is an element of the upper limit basis that contains $x$ and remains in $B$, so $\mathcal{T}_1 \subset \mathcal{T}_4$. This inclusion is strict, since for $B'=(a,b]$, there is no interval of the form $(c,d)$ that contains $b$ and remains in $B'$.

\textbf{$\mathcal{T}_2$ vs $\mathcal{T}_4$:} As indicated in the diagram, these two topologies are not comparable. Let $B=(10,11]$ be an element of the upper limit basis, then there is no element of the $K$-basis that contains 11 and remains in $B$. On the other hand, if $B'=(-1,1)-K$ is an element of the $K$-basis, then there is no interval of the form $(a,b]$ containing 0 that also lies entirely in $B'$. Since neither topology is a subset of the other, they are not comparable.

\textbf{$\mathcal{T}_1$ vs $\mathcal{T}_5$:} For $x \in B=(-\infty,a)$, the set $(x-1, a)$ contains $x$ and lies entirely in $B$, so $\mathcal{T}_5 \subset \mathcal{T}_1$. This inclusion is strict, since for $B'=(a,b)$ in the standard basis, there is no interval of the form $(-\infty,c)$ that lies entirely inside $B'$.

\textbf{$\mathcal{T}_3$ vs $\mathcal{T}_5$:} The interval $(-\infty,a)$ is an element of $\mathcal{T}_5$, but $\mathbb{R}-(-\infty,a)=[a,\infty)$ is infinite, so it is not in the finite complement topology. On the other hand, $\mathbb{R}-\left\{ 0 \right\}$ is in the finite complement topology, but there is no way to take the union of elements of the form $(-\infty,a)$ and create such a hole at 0, so this set is not in $\mathcal{T}_5$. Thus the two topologies are not comarable.


\begin{exer}[4 points]
	Munkres, \S 16, pg. 91 \#1.
\end{exer}
{\color{blue}Collaborators: None.}

Denote the topology that $A$ inherits from $X$ by $\mathcal{T}_A^X$ and the topology it inherits from $Y$ by $\mathcal{T}_A^Y$.

We first show $\mathcal{T}_A^Y \subset \mathcal{T}_A^X$. Let $U \in \mathcal{T}_A^Y$, then $U = U'\cap A$ for some $U'$ open in $Y$. But since $U'$ is open in $Y$, we can write it as $U' = U'' \cap Y$, where $U''$ is open in $X$. Then since $A$ is a subset of $Y$, we have $U = U'' \cap Y \cap A = U'' \cap A$, so $U \in \mathcal{T}_X$.

We now show $\mathcal{T}_A^X \subset \mathcal{T}_A^Y$. Let $V \in \mathcal{T}_A^X$, then $V = V'\cap A$ for some $V'$ open in $X$. Since $A$ is a subset of $Y$, we can this as $V = (V' \cap Y) \cap A$. Now $V' \cap Y$ is an element of the topology on $Y$, so we have expressed $V$ as the intersection of $A$ and an open set of $Y$, so $V \in \mathcal{T}_Y$.


\begin{exer}[5 points]
	Prove Theorem 17.3 in \S 17, pg. 95 of Munkres.
\end{exer}
{\color{blue}Collaborators: None.}

Let $A$ be closed in $Y$, then since $Y$ is a subspace of $X$, $A = Y \cap B$ for some $B$ closed in $X$. Then since $Y$ is closed in $X$, $A$ is the intersection of closed sets of $X$, so $A$ is itself closed in $X$.

\begin{exer}[5 points]
	Munkres, \S 17, pg. 101 \#6.
\end{exer}
{\color{blue}Collaborators: None.}

\begin{enumerate}
	\item $A \subset B \subset \overline{B}$, so $\overline{B}$ is a closed set containing $A$. Now $\overline{A}$ is the intersection of all closed sets containing $A$, so $\overline{A} \subset \overline{B}$.
	\item First we show $\overline{A \cup B} \subset \overline{A}\cup \overline{B}$. Since $A \subset \overline{A}$ and $B\subset \overline{B}$, we have $A \cup B \subset \overline{A} \cup \overline{B}$. Then by part (a), $\overline{A \cup B} \subset \overline{\overline{A} \cup \overline{B}} = \overline{A} \cup \overline{B}$, where the last equality follows from the finite union of closed sets being closed.

		Now we show $\overline{A} \cup \overline{B} \subset \overline{A \cup B} $. By part (a), since $A \subset A \cup B$, we have $\overline{A} \subset \overline{A \cup B} $. Similarly, $\overline{B} \subset  \overline{A \cup B} $. Since $\overline{A}$ and $\overline{B}$ are both subsets of $\overline{A \cup B} $, their union $\overline{A} \cup \overline{B}$ is also a subset of $\overline{A \cup B} $.
	\item For all $\beta$, $A_{\beta}\subset \cup_{\alpha}A_{\alpha}$. So by part (a), $\overline{A_{\beta}} \subset \overline{\cup_{\alpha}A_{\alpha}} $ for all $\beta$. Then $\cup_{\alpha} \overline{A_{\alpha}} \subset \overline{\cup_{\alpha}A _{\alpha}} $.

		As a counterexample for the reverse inclusion, consider the set of intervals 
		\[
		\left\{ \left[ \frac{1}{n} , 1 \right] \right\}_{n \in \mathbb{Z}^+}
	\] in the standard topology on $\mathbb{R}$. The closure of their union is $\overline{\cup_{n}[1/n, 1]} = \overline{(0,1]} =[0,1]$, but their union of their closures is $\cup_{n}\overline{[1/n,1]} =\cup_{n}[1/n,1] = (0,1]$.
\end{enumerate}

\begin{exer}[5 points]
	Munkres, \S 17, pg. 101 \#8.
\end{exer}
{\color{blue}Collaborators: Rahul Ramesh, Saloni Bulchandani.}

\begin{enumerate}
	\item Equality does not hold, but we do have $\overline{A \cup B} \subset \overline{A} \cap \overline{B}$. Since $A \cap B \subset A, B$, by part (a) in the previous problem, $\overline{A \cap B} \subset \overline{A}, \overline{B}$. Then $\overline{A \cap B} \subset \overline{A} \cap \overline{B}$.

		As a counterexample for the reverse inclusion, consider the two intervals
		\begin{align*}
			A &= (0, 1) \\
			B &= (1, 2)
		\end{align*}
		in the standard topology on $\mathbb{R}$. For these two intervals, $\overline{A \cap B} = \overline{\varnothing} =\varnothing$, but $\overline{A} \cap \overline{B} = [0,1] \cap [1,2] = \left\{ 1 \right\}$.

	\item Again, we do not have equality, but we do have $\overline{\cap A_{\alpha}} \subset \overline{A_{\alpha}}$. For all $\beta$, $\cap A_{\alpha}\subset A_{\beta}$. Then by part(a) of the previous problem, $\overline{\cap A_{\alpha}} \subset \overline{A_{\beta}} $ for all $\beta$, so $\overline{\cap A_{\alpha}} \subset \cap \overline{A_{\alpha}} $.

		The counterexample from part (a) of this problem shows that the reverse inclusion does not hold.

	\item We don't have equality here, either, but this time the inclusion is reversed. Let $x \in \overline{A}-\overline{B}$, then every neighborhood of $x$ intersects $A$ but does not intersect $B$. In other words, every neighborhood of $x$ intersects $A-B$, so $x \in \overline{A-B}$. Thus $\overline{A}-\overline{B} \subset \overline{A-B} $.

		As a counterexample for the reverse inclusion, consider the intervals
		\begin{align*}
			A&=(0,3) \\
			B&=(1,2)
		\end{align*}
		in the standard topology on $\mathbb{R}$. For this example, $\overline{A-B} =\overline{(0,1] \cup [2,3)} =[0,1] \cup [2,3]$, but $\overline{A}-\overline{B}=[0,3]-[1,2] = [0,1) \cup (2,3]$.
\end{enumerate}

\newpage

\begin{challenge}
	This challenge problem is just for fun and worth no points, even if solved correctly. Please complete your solutions to all other problems before spending time on this question.
	
	Let $S$ be an arbitrary subset of a topological space $X$. What is the maximum number of distinct subsets of $X$ obtainable from $S$ using only the operations of closure and complement?

	As an example, suppose $S = (0, 1) \subseteq \mathbb{R}$. The complement of $S$ is $A_1 = (-\infty, 0] \cup [1, \infty)$, which is closed. The closure of $S$ is $A_2 = [0, 1]$, whose complement is $A_3 = (-\infty, 0) \cup (1, \infty)$. Since the closure of $A_3$ is back to $A_1$, we are done. Hence, $4$ sets are obtainable from $S$, including $S$ itself.
\end{challenge}
{\color{red}Collaborators: None.}

\end{document}
