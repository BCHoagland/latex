\documentclass[10pt]{report}
\usepackage{/Users/bradenhoagland/latex/math}

\lhead{Braden Hoagland}
\chead{HW 3}
\rhead{}

\renewcommand{\theenumi}{\alph{enumi}}

\begin{document}

{\color{blue}Problems completed: All.}

\begin{exer}[6 points]
Find a counterexample to the following statement. If every sequence $\left\{ x_n \right\}_{n=1}^\infty$ converges to at most 1 point in $X$, then $X$ is Hausdorff.
\end{exer}
{\color{blue}Collaborators: Saloni Bulchandani.}

We consider the topology on $\mathbb{R}$ given by
\[
	\mathcal{T} = \left\{ U \;|\; \mathbb{R} - U \text{ countable } \right\} \cup \left\{ \varnothing \right\}.
\] We first show that this is actually a topology. It contains $\varnothing$ by definition, and $\mathbb{R} \in \mathcal{T}$ since $\mathbb{R}-\mathbb{R}=\varnothing$ is countable. It is closed under arbitrary unions, since for $U_{\alpha}\in \mathcal{T}$, $\mathbb{R}-\bigcup_{\alpha}U_{\alpha} = \bigcap_{\alpha}(\mathbb{R}-U_{\alpha)}$, which is the intersection of countable sets and so is also countable. It is also closed under finite intersections, since for $U_{i} \in \mathcal{T}$, $\mathbb{R}-\bigcap_{i=1}^N U_i = \bigcup_{i=1}^N (\mathbb{R}-U_i)$, which is the finite union of countable sets and so is also countable.

\textbf{Unique Limits:} Now we show that $\mathcal{T}$ has unique limits by showing that any convergent sequence in $\mathcal{T}$ must eventually be constant. Suppose $x_n \to x$ with respect to $\mathcal{T}$. Define the set $U_x$ by
\[
	U_x \doteq \left(  \mathbb{R}-\left\{ x_n \right\}_{n=1}^\infty \right)\cup\left\{ x \right\},
\]
then its complement is
\[
\mathbb{R}-U_x = \left\{ x_n \;|\; x_n \neq x \right\}.
\] Since sequences are countable and this is a subset of the sequence $\left\{ x_n \right\}_{n=1}^\infty$, this means $U_x$ is an open set. In particular, it is a neighborhood of $x$. So if $\left\{ x_n \right\}_{n=1}^\infty$ is not eventually $x$, then it is never eventually in this neighborhood of $x$, so it cannot converge to $x$. Thus any convergent sequence in this topology must eventually be constant, meaning that it cannot converge to more than one point.

\textbf{Not Hausdorff:} Now we show that $\mathbb{R}$ with this topology is not Hausdorff. Suppose $x_1\neq x_2$ and $U_1$ and $U_2$ are neighborhoods of $x_1$ and $x_2$, respectively. Note that since $\mathbb{R}$ is uncountable and the complements of $U_1$ and $U_2$ are bouth countable, then both $U_1$ and $U_2$ must be uncountable. Now suppose $U_1$ and $U_2$ do not intersect, then $U_1 \subset \mathbb{R}-U_2$, but this is impossible, as $\mathbb{R}-U_2$ is countable and $U_2$ is uncountable. Thus $U_1$ and $U_2$ must intersect, so this space is not Hausdorff.
\pagebreak

\begin{exer}[6 points]
\begin{enumerate}
	\item Munkres \S 13, pg. 83 \#5.
	\item Equip $\mathbb{R}^{\infty}=\Pi_{i \in \mathbb{Z}^+}\mathbb{R}$ with the product topology. Prove or disprove that the function $f:\mathbb{R}\to \mathbb{R}^{\infty}$ defined by $f(x)=(x,x,\dots)$ is continuous.
\end{enumerate}
\end{exer}
{\color{blue}Collaborators: Saloni Bulchandani.}

\begin{enumerate}
	\item Denote the topology generated by $\mathcal{T}_{A}$, and denote the intersection of all topologies containing $\mathcal{A}$ by $\bigcap_{\beta}\mathcal{T}_{\beta}$.

		First we show that $\mathcal{T}_{A}$ is contained in $\bigcap_{\beta}\mathcal{T}_{\beta}$. Since each $\mathcal{T}_{\beta}$ is a topology containing $\mathcal{A}$, each contains arbitrary unions of finite intersections of elements of $\mathcal{A}$. If $\mathcal{A}$ is a basis, then $\mathcal{T}_A$ is the collection of all arbitrary unions of elements of $\mathcal{A}$, and if $\mathcal{A}$ is a subbasis, then $\mathcal{T}_{A}$ is the collection of all arbitrary unions of finite intersections of elements of $\mathcal{A}$. So in either case, each $\mathcal{T}_{\beta}$ contains $\mathcal{T}_{A}$, so $\mathcal{T}_{A} \subset \bigcap_{\beta}\mathcal{T}_{\beta}$.

		Now we show that $\bigcap_{\beta}\mathcal{T}_{\beta}$ is contained in $\mathcal{T}_{A}$. Whether $\mathcal{A}$ is a basis or subbasis, $\mathcal{T}_{A}$ is itself a topology containing $\mathcal{A}$, so it is one of the $\mathcal{T}_{\beta}$ in the intersection. Then if $U \in \bigcap_{\beta}\mathcal{T}_{\beta}$, $U$ must be in $\mathcal{T}_{A}$, so $\bigcap_{\beta}\mathcal{T}_{\beta}\subset \mathcal{T}_{A}$.

	\item It suffices to show that for all subbasis elements $S$ of the space $\mathbb{R}^\infty$, the set $f^{-1}(S)$ is open in $\mathbb{R}$. Let $\mathbb{R}_i$ denote the $i$-th copy of $\mathbb{R}$ in the cartesian product, then any $S$ is of the form
		\[
			\left\{ \pi_{i}^{-1}(U_{i}) \;|\; U_i \text{ open in } \mathbb{R}_i \right\}
		\] for some fixed $i$. But this is just the usual cartesian product with the single restriction that the functions that compose it must map $i$ into $U_i$ instead of all of $\mathbb{R}_i$.

		The preimage of $S$ under $f$ clearly contains $U_i$. It also can't contain any additional elements of $\mathbb{R}$, since $f$ being the identity map means that the $i$-th coordinate of the cartesian product would contain elements outside of $U_i$, which we know cannot be the case. Thus
		\[
			f^{-1}(S) = U_i \in \mathbb{R}.
		\] Since $U_i$ is open by definition, we have shown that $f$ is continuous.
\end{enumerate}
\pagebreak

\begin{exer}[7 points]
\begin{enumerate}
	\item Munkres \S 18, pg. 111 \#2.
	\item Munkres \S 18, pg. 111 \#6.
\end{enumerate}
\end{exer}
{\color{blue}Collaborators: Saloni Bulchandani.}

\begin{enumerate}
	\item Let $X$ be any topological space containing a limit point $x$ of some subset $A \subset X$, and let $Y$ be the singleton $\left\{ 0 \right\}$ endowed with the indiscrete topology.

		The only possible function $f:X\to Y$ is the zero function. It is continuous since it is onto a space that has the indiscrete topology: $f^{-1}(\varnothing)=\varnothing$ and $f^{-1}(Y) = X$ are both open in $X$; however, $f(x)$ is \textit{not} a limit point of $Y$. Since $f(x)=0$ and $0$ is the only element of $Y$, it is impossible for a neighborhood of $f(x)$ to intersect any subset of $Y$ at a point other than $f(x)$.

		\item Consider the function $f:\mathbb{R}\to \mathbb{R}$ given by
			\[
				f(x) =
				\begin{cases}
					x & x\in \mathbb{Q},\\
					0 & x\not\in\mathbb{Q}.
				\end{cases}
			\] 
			We first show that $f$ is continuous at 0. Fix an arbitrary neighborhood $U$ of $f(0)=0$. Since $U$ is open and contains 0, there must be some other neighborhood $V$ of 0 that is contained in $U$. Then
			\[
				f(V) = (V \cap \mathbb{Q}) \cup \left\{ 0 \right\} = V \cap \mathbb{Q} \;\subset\; U \cap \mathbb{Q} \;\subset\; U,
			\] so $f$ is continuous at 0.

			Now we show that $f$ is not continuous anywhere else. Let $x \neq 0$ be a rational number, and consider the neighborhood $U=B(x,x/2)$ of $f(x)=x$, which does not contain 0. If $f$ is continuous, then we can find some neighborhood $V$ of $x$ such that $f(V) \subset U$; however, any such $V$ contains an irrational number, so $f(V)$ will contain 0 and as such not be a subset of $U$. Thus $f$ is not continuous at any nonzero rational.

			Now suppose $x\neq 0$ is an irrational number, then consider the neighborhood $U=B(0,x/2)$ of $f(x)=0$. If we take any neighborhood $V$ of x, then it contains a rational number between $x/2$ and $x$, so $f(V)$ contains the same number and thus cannot be a subset of $U$. This shows that $f$ is not continuous at any nonzero irrational number, so it can only be continuous at 0.
\end{enumerate}
\pagebreak

\begin{exer}[4 points]
Munkres \S 18, pg. 111 \#7a.
\end{exer}
{\color{blue}Collaborators: Saloni Bulchandani.}

Fix arbitrary $a \in \mathbb{R}$. By assumption, for all $\varepsilon>0$, there exists some $\delta_{\varepsilon}>0$ such that $f(x) \in B(f(a),\varepsilon)$ when $x \in [a,a+\delta)$, i.e.
\[
	f([a,a+\delta_{\varepsilon})) \subset B(f(a), \varepsilon).
\]
Let $U$ be an arbitrary neighborhood of $f(a)$, then there is some $\varepsilon>0$ such that $B(f(a),\varepsilon) \subset U$. Then the open set $V=[a, a+\delta_\varepsilon)$ in $\mathbb{R}_l$ contains $a$ and satisfies $f(V) \subset U$, so $f$ is continuous.

\end{document}
