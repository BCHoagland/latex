\documentclass[10pt]{report}
\usepackage{/Users/bradenhoagland/latex/math}

\lhead{Braden Hoagland}
\chead{HW 5}
\rhead{}

\begin{document}
%\tableofcontents

{\color{blue}Exercises completed: All.}

\begin{exer}[]
Munkres \S 23, pg. 152 \#4.
\end{exer}
{\color{blue}Collaborators: None.}

Suppose $Y$ is a space with the finite complement topology and $X$ is an infinite set in $Y$. Note that since $X \subset Y$ is infinite, this forces $Y$ to also be infinite. Suppose $X$ is not connected, then there exist nonempty disjoint open sets $U$ and $V$ whose union is $X$.

Since $U \cap V = \varnothing$, by DeMorgan's laws the complement of their intersection is
\[
	Y - (U \cap V) = (Y-U) \cup (Y-V) = Y.
\] Since $Y$ is infinite, this means at least one of $Y-U$ and $Y-V$ is infinite. This is a contradiction, though, since open sets in the finite complement topology have finite complements. Thus $X$ must be connected.

\begin{exer}[]
Munkres \S 23, pg. 152 \#9.
\end{exer}
{\color{blue}Collaborators: None.}

Let $Z \doteq (X \times Y) - (A \times B)$, and fix $(a,b) \in Z$ such that $a \in X-A$ and $b \in Y-B$. Then the segments $\left\{ a \right\}\times Y$ and $X \times \left\{ b \right\}$ are both connected subsets of $Z$ (they are connected because they are homeomorphic to $Y$ and $X$, respectively). Since they share the point $(a,b)$, their union
\[
	T \doteq (X \times \left\{ b \right\}) \cup (\left\{ a \right\}\times Y)
\] is also connected. Now let $(x,y)$ be an arbitrary point of $Z$, and define
\[
T_{x,y} \doteq
\begin{cases}
	\{x\} \times Y & \text{if } x \in X-A, \\
	X \times \{y\} & \text{if } y \in Y-B.
\end{cases}
\] 
In the first case, $T_{x,y}$ intersects $T$ at the point $(x,b)$. In the second case, it intersects $T$ at the point $(a,y)$. Since $T_{x,y}$ is homeomorphic to either $X$ or $Y$, it is connected, and by definition it lies entirely in $Z$ and intersects $T$. Thus
\[
	\tilde{T}_{x,y} \doteq T_{x,y} \cup T
\] is a connected subset of $Z$ that contains the points $(x,y)$ and $(a,b)$. Finally,
\[
	\bigcup_{(x,y) \in Z} \tilde{T}_{x,y} = Z
\] is the union of connected sets that all contain the points $(a,b)$, so $Z$ is connected.

\begin{exer}[]
	Let $T = \left\{ (x,sin(1/x) \;|\; x>0 \right\} \subset \mathbb{R}^2$. Prove that $T$ is connected. You may assume the sine function is continuous.
\end{exer}
{\color{blue}Collaborators: Saloni Bulchandani, Rahul Ramesh.}

Since $(0,\infty)$ is connected and the function $x \mapsto 1/x$ is continuous on $(0,\infty)$, the set $\left\{ 1/x \;|\; x>0 \right\}$ is connected. Then since the sine function is continuous, the image of this set under the sine function $\left\{ \sin\left( 1/x \right)\;|\; x > 0 \right\}$ is connected. Then since the finite Cartesian product of connected spaces is connected, the set
\[
	T_1 \doteq \left\{ (x,\sin\left( 1/x \right)\;|\; x > 0 \right\}
\] is connected. Thus if $T = T_1 \cup \left\{ (0,0) \right\}$ is disconnected, then it is because $(0,0)$ can be separated from $T_1$.

Consider any neighborhood $U$ of $(0,0)$, which is of the form $B( (0,0), \varepsilon) \cap T$ for some $\varepsilon>0$. There is some $n \in \mathbb{N}$ such that $\frac{1}{\pi n} <\varepsilon$, then for $x = \frac{1}{\pi n} $, we have
\[
	\sin\left( 1/x \right) = \sin(\pi n) = 0.
\] The point $(1/\pi n, 0)$ is then in $T_1$, and its distance from the origin is
\[
	{\Vert{(1/\pi n},0)-(0,0)\Vert} = {\Vert{(1/\pi n,0)}\Vert}=|1/\pi n| < \varepsilon,
\]so it is also in $U$. Since $U$ was arbitrary, this shows that every neighborhood of the origin intersects $T_1$, so $(0,0)$ is a limit point of $T_1$. This gives
\[
T_1 \subset T \subset \overline{T_1},
\] so since $T_1$ is connected, $T$ is also connected.

\begin{exer}[]
Munkres \S 24, pg. 158 \#2.
\end{exer}
{\color{blue}Collaborators: Saloni Bulchandani, Rahul Ramesh.}

If $f$ is constant, then this is trivial, so assume $f$ is not constant. Let $g(x) = f(x)-f(-x)$. Then $g(-x) = f(-x)-f(x)=-g(x)$. Since $f$ is not constant, there is some $x$ such that $g(x) \neq 0$. Then $g(-x)$ has the opposite sign. Then since $S^1$ is connected, by the intermediate value theorem there is some $\tilde{x} \in S^1$ such that $g(\tilde{x})=0$, i.e. $f(\tilde{x})=f(-\tilde{x})$.
\pagebreak

\begin{exer}[]
Munkres \S 24, pg. 158 \#9.
\end{exer}
{\color{blue}Collaborators: Saloni Bulchandani, Rahul Ramesh.}

Fix arbitrary $x,y \in \mathbb{R}^2-A$, then we must show that there is a continuous path between them. Consider the collection of straight lines extending from $x$
\[
	\mathcal{R}_x \doteq \left\{ t \mapsto x+t(\cos \theta, \sin \theta) \;|\; \theta \in [0, \pi) \right\}.
\] Because of how we restricted $\theta$, all of these lines are disjoint. We claim that an uncountable number of these lines never intersect $A$.

Suppose that an uncountable number of the lines intersect $A$ at some point. Then since $\mathcal{R}_x$ is uncountable and the lines are all disjoint, $A$ contains an uncountable number of points. Since $A$ is countable, this is impossible, so there must be an uncountable number of lines in $\mathcal{R}_x$ that never intersect $A$.

Any two non-parallel straight lines in $\mathbb{R}^2$ will eventually intersect. Since we have uncountably many lines in both $\mathcal{R}_x$ and $\mathcal{R}_y$ that never intersect $A$, there must be two that are non-parallel and also never intersect $A$.

Take these two lines, then we can easily use them to construct a continuous path from $x$ to $y$: Suppose $L_x$ is the line extending from $x$ and $L_y$ is the line extending from $y$, and suppose they intersect at a point $z$. Then the path from $x$ to $z$ is continuous, and the path from $z$ to $y$ is continuous. Since they agree at $z$, by the pasting lemma, they can be combined into a continuous path from $x$ to $y$. Since this path never intersects $A$, the space $\mathbb{R}^2-A$ is connected.


\end{document}
