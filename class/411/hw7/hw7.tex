\documentclass[10pt]{report}
\usepackage{/Users/bradenhoagland/latex/math}

\lhead{Braden Hoagland}
\chead{HW 7}
\rhead{}

\begin{document}
%\tableofcontents

\renewcommand{\theenumi}{\alph{enumi}}

{\color{blue}Exercises completed: None.}

\begin{exer}[]
\S 24 \#8 a,b,d.
\end{exer}
{\color{blue}Collaborators: None.}

\begin{enumerate}
	\item Yes (if we're using the product topology). Suppose $X_{\alpha}$ is path connected for all $\alpha$ in some indexing set, and let $\mathbf{x},\mathbf{y} \in \prod X_{\alpha}$. Since each $X_{\alpha}$ is path connected, we can find a continuous path $\gamma_{\alpha}$ from $\mathbf{x}_{\alpha}$ to $\mathbf{y}_{\alpha}$ for all $\alpha$. Define a new function $\gamma = (\gamma_{\alpha})_{\alpha}$, then by Munkres Theorem 19.6, $\gamma$ is continuous since each $\gamma_{\alpha}$ is continuous. Since $\gamma$ is then a continuous path from $\mathbf{x}$ to $\mathbf{y}$, the space $\prod X_{\alpha}$ is path connected.
	
	\item No. If we let
		\[
			S \doteq \left\{ (x,\sin(1/x) \;|\; 0 < x \leq 1 \right\},
		\] then $\overline{S}$ is the topologist's sine curve. $S$ is the graph of a continuous function over a path connected domain, so it is itself path connected; however, $\overline{S}$ is known to not be path connected.

	\item[d.] Yes. Suppose $x,y \in \bigcup A_{\alpha}$, then $x \in A_{\alpha_x}$ and $y \in A_{\alpha_y}$ for some $\alpha_x,\alpha_y$. Since the intersection of all $A_{\alpha}$ is nonempty, we know that $A_{\alpha_x}$ and $A_{\alpha_y}$ have at least one common point. Let $z$ be a point in their intersection, then we can find paths $\gamma_1$ from $x$ to $z$ and $\gamma_2$ from $z$ to $y$. By the pasting lemma, we can use $\gamma_1$ and $\gamma_2$ to construct a single continuous path from $x$ to $y$. Thus $\bigcup_{}A_{\alpha}$ is path connected.
\end{enumerate}

\begin{exer}[]
\S 25 \#1.
\end{exer}
{\color{blue}Collaborators: None.}`

The connected components of $\mathbb{R}_{l}$ are its individual points. Suppose $x < y$ are in the same connected component $C$ of $\mathbb{R}_l$, then they can be separated by the disjoint open sets $C \cap (-\infty,y)$ and $C \cap [y,\infty)$. Thus no connected component of $\mathbb{R}_l$ can have more than 1 distinct point.

By Theorem 25.5, every path component lies in a component of $\mathbb{R}_l$. Thus every path component is also just a single point in $\mathbb{R}_l$.

Suppose $f:\mathbb{R}\to \mathbb{R}_l$ is continuous. Since $\mathbb{R}$ is connected and continuous maps preserve connectedness, $f$ must map $\mathbb{R}$ to a connected subset of $\mathbb{R}_l$. But the connected subsets of $\mathbb{R}_l$ are just single points, so $f$ must be constant. Thus the continuous maps from $\mathbb{R}$ to $\mathbb{R}_l$ are the constant maps.

\pagebreak

\begin{exer}[]
\S 26 \#5.
\end{exer}
{\color{blue}Collaborators: None.}

Fix $b \in B$. Since $X$ is Hausdorff, for all $a \in A$ we can find disjoint neighborhoods $U_{a,b}, V_{a,b}$ of $a,b$, respectively. Since $\left\{ U_{a,b} \right\}_{a \in A}$ covers $A$ and $A$ is compact, there is a finite subcover $\left\{ U_{a_i,b} \right\}_{i=1}^N$. Then $U_{b}\doteq \bigcup_{i=1}^N U_{a_i,b}$ contains $A$ and does not intersect $V_{b}\doteq \bigcap_{i=1}^N V_{a_i,b}$, which is a neighborhood of $b$ since the intersection is finite.

        Now $\left\{ V_{b} \right\}_{b \in B}$ is an open cover of $B$ and $B$ is compact, we can find a finite subcover $\left\{ V_{b_j} \right\}_{j=1}^M$. Since $U_b$ doesn't intersect $V_b$ for all $b$, we can define two disjoint open sets
        \[
                U \doteq \bigcap_{j=1}^M U_{b_j}, \quad V \doteq \bigcup_{j=1}^M V_{b_j}.
        \]
        Since each $U_{b_j}$ contains $A$, so does their intersection $U$, and $V$ is a cover of $B$ by definition. Thus we have found disjoint open sets containing $A$ and $B$.

\begin{exer}[]
\S 27 \#2.
\end{exer}
{\color{blue}Collaborators: None.}

\begin{enumerate}
	\item \textbf{Forward:} Suppose $d(x,A)=0$. If $x \not\in \overline{A}$, then there is a neighborhood $U$ of $x$ such that $U$ does not intersect $A$. Now there is some $\varepsilon>0$ such that $B(x,\varepsilon)\subset U$, which implies $d(x,A) \geq \varepsilon$, but this contradicts the assumption that $d(x,A)=0$, so $x \in \overline{A}$.

		\textbf{Backward:} Supose $x \in \overline{A}$. If $x \in A$, then $d(x,A)$ is clearly 0, so assume $x$ is a limit point of $A$ but not in $A$. Since it's a limit point, for all $\varepsilon>0$, $B(x,\varepsilon)$ intersects $A$. Since $\varepsilon$ was arbitrary, the only possibility for $d(x,A)$ is 0.

	\item Suppose $A$ is compact. Fix $x \in X$, then we will show that $f(a) \doteq d(x,a)$ is continuous, from which the desired result will follow. Fix $\varepsilon>0$ and set $\delta=\varepsilon$. If $d(a_1,a_2)<\delta=\varepsilon$, then by the triangle inequality,
		\[
			|f(a_1)-f(a_2)| = |d(a_1,x) - d(a_2,x)| \leq d(a_1,a_2) < \delta = \varepsilon.
		\] Thus $f$ is continuous. Then by the extreme value theorem, $f$ attains its infimum on $A$. This means there is some $a \in A$ such that $f(a) = d(x,a) = d(x,A)$.

	\item Fix $\varepsilon>0$. Let $x \in \bigcup_{a \in A} B(a,\varepsilon)$, then $x \in B(\tilde{a},\varepsilon)$ for some $\tilde{a} \in A$. Then $d(x,\tilde{a})<\varepsilon$, so $d(x,A) <\varepsilon$, so $x \in U(A,\varepsilon)$. Thus $\bigcup_{a\in A}B(a,\varepsilon) \subset U(a,\varepsilon)$.

		Conversely, let $x \in U(a,\varepsilon)$, then $d(x,A) < \varepsilon$. Then there is some $\tilde{a} \in A$ such that $d(x,\tilde{a}) < \varepsilon$, so $x \in B(\tilde{a},\varepsilon)$. Thus $U(A,\varepsilon) \subset \bigcup_{a \in A}B(a,\varepsilon)$.

	\item Suppose $A$ is compact and is contained in an open set $U$. Then for all $a \in A$, there is some $\varepsilon_{a}>0$ such that $B_{a} \doteq B(a,\varepsilon_{a}/2) \subset U$. Since $\left\{ B_{a} \right\}_{a \in A}$ is an open cover of $A$ and $A$ is compact, we can find a finite subcover $\left\{ B_{a_i} \right\}_{i=1}^N$ of $A$ (which still lies entirely in $U$).

		Let $\varepsilon \doteq \min_{i} \varepsilon_{a_i}$, then we claim that $U(A,\varepsilon/2)$ is contained $U$. Let $x \in U(A,\varepsilon/2)$, then by part (c), $x$ is in some ball $B(\tilde{a},\varepsilon/2)$. Since $\tilde{a} \in A$, we can use our finite open cover of $A$ to find some ball $B(a_i,\varepsilon_{a_i}/2)$ that contains $\tilde{a}$. Then by the triangle inequality,
		\[
			d(x,a_i) \leq d(x,\tilde{a}) + d(\tilde{a},a_{i}) < \frac{\varepsilon}{2} + \frac{\varepsilon_{a_i}}{2} \leq \varepsilon_{a_i}.
		\] Thus $x \in B(a,\varepsilon_{a_i}) \subset U$, so $U(A,\varepsilon/2)$ is an $\varepsilon$-neighborhood of $A$ that is contained in $U$.

	\item Let $A = \mathbb{Z} \subset \mathbb{R}$, which is closed since its complement $\mathbb{R} - \mathbb{Z} = \bigcup_{z \in \mathbb{Z}}(n,n+1)$ is open. Since it's an unbounded set in $\mathbb{R}$, it cannot be compact. Now consider the open set
		\[
			U = \bigcup_{n \in \mathbb{Z}} \left( n-\frac{1}{|n|} ,n+\frac{1}{|n|}  \right).
		\] Fix $\varepsilon>0$, then since we can always find $n \in \mathbb{Z}$ such that $1/n < \varepsilon$, the $\varepsilon$-neighborhood $U(A,\varepsilon)$ can never be fully contained in $U$. Since $\varepsilon$ was arbitrary, no $U(A,\varepsilon)$ can be fully contained in $U$. Thus $(d)$ does not necessarily hold if $A$ is not compact.
\end{enumerate}

\begin{exer}[]
\S 28 \#7 b.
\end{exer}
{\color{blue}Collaborators: None.}

I couldn't fill in all the gaps in the proof outlined in the book's hint, so I did something else entirely that's similar to what I did in Exercise 4(b). We claim that the function
\begin{align*}
	g:X&\to \mathbb{R} \\
	x&\mapsto d(x,f(x))
\end{align*}
is continuous. To show this, note that by the triangle inequality and the fact that $f$ is a shrinking map,
\begin{align*}
	d(x,f(x)) &\leq d(x,y)+d(y,f(y))+d(f(y),f(x)) \\
	g(x)-g(y) &\leq d(x,y) + d(f(x),f(y)) \\
	g(x)-g(y) &< 2d(x,y).
\end{align*}
Swapping $x$ and $y$ and using the symmetry of $d$ gives the same inequality, so
\[
	|g(x)-g(y)| < 2d(x,y).
\]
Now fix $\varepsilon>0$ and let $\delta = \varepsilon/2$. When $d(x,y)<\delta$, the inequality we just derived gives
\[
	|g(x)-g(y)| < 2d(x,y) < \varepsilon,
\] so $g$ is continuous. Since $X$ is compact, the extreme value theorem says that $g$ attains its infimum $I$ on $X$, say at a point $x$. Suppose $I>0$, then since $f$ is a shrinking map,
\[
	g(f(x)) = d(f(x), f^2(x)) < d(x,f(x)) = g(x) = I.
\] This contradicts the fact that $I$ is an infimum, so $I$ must be 0, i.e. $x$ is a fixed point of $f$. To show that $x$ is unique, suppose $y \neq x$ is also a fixed point of $f$. Then
\[
	d(x,y) = d(f(x),f(y))< d(x,y),
\] which is a contradiction, so $x=y$. Thus we've found a unique fixed point of $f$.

\end{document}
