\documentclass[10pt]{report}
\usepackage{/Users/bradenhoagland/latex/math}

\lhead{Braden Hoagland}
\chead{HW 4}
\rhead{}

\begin{document}
%\tableofcontents

{\color{blue}Problems completed: All.}

\begin{exer}[]
Munkres \S 19, pg. 118 \#6.
\end{exer}
{\color{blue}Collaborators: None.}

\begin{lem}
	In a general topological space, a sequence $x_n$ converges to $x$ if and only if $x_n$ is eventually in every subbasis element containing $x$.
\end{lem}
\begin{proof}
	Let $X$ be a topological space with topology generated by subbasis $\mathcal{S}$. Suppose $x_n \to x$ in $X$, then since each subbasis element is itself an open set, it follows from the definition of convergence that $x_n$ is eventually in each $S \in \mathcal{S}$.

	Conversely, suppose that $x_n$ is eventually in each $S \in \mathcal{S}$ containing $x$. Let $U$ be an arbitrary neighborhood of $x$, then
	\[
	U = \bigcup_{\alpha} \bigcap_{i=1}^N S_{\alpha,i},
\] where each $S_{\alpha,i}$ is a subbasis element containing $x$. Fix arbitrary $\alpha$, and let $N_i$ be the point at which the sequence crosses into $S_{\alpha,i}$ permanently. Then for $n > \max_i N_i$, the sequence lies in the entire intersection. Thus $x_n$ is eventually in every intersection, meaning that it's eventually in their union $U$.
\end{proof}

With this lemma, the desired characterization of convergence in the product topology almost follows directly from just the definitions.

\textbf{Forward:} Suppose $\mathbf{x}_{n}\to \mathbf{x}$, then $\mathbf{x}_{n}$ is eventually in every subbasis element $\pi_{\alpha}^{-1}(U_{\alpha})$ containing $\mathbf{x}$, so $\pi_{\alpha}(\mathbf{x}_{n})$ is eventually in all neighborhoods $ \pi_{\alpha}(\pi_{\alpha}^{-1}(U_{\alpha})) = U_{\alpha}$ of $\pi_{\alpha}(\mathbf{x})$. Thus $ \pi_{\alpha}(\mathbf{x}_{n})\to \pi_{\alpha}(\mathbf{x})$ for all $\alpha$.

\textbf{Backward:} Suppose $\pi_{\alpha}(\mathbf{x}_{n})\to \pi_{\alpha}(\mathbf{x})$ for all $\alpha$, then $\pi_{\alpha}(\mathbf{x}_{n})$ is eventually in every neighborhood $U_{\alpha}$ of $\pi_{\alpha}(\mathbf{x})$. Then $\mathbf{x}_{n}$ is eventually in every subbasic neighborhood $\pi_{\alpha}^{-1}(U_{\alpha})$ of $\mathbf{x}$, so $\mathbf{x}_{n}\to  \mathbf{x}$.

\textbf{Box topology part:} The box topology doesn't exhibit this behavior. Consider the sequence in $\prod_{i \in \mathbb{Z}^+}\mathbb{R}$ with components $\pi_{i}(\mathbf{x}_{n})=1/n$. Each component $\pi_{\alpha}(\mathbf{x}_{n})$ converges to 0, but we claim that $\mathbf{x}_{n}$ does \textit{not} converge to the zero sequence. To see this, take the open set
\[
	U = \prod_{i \in \mathbb{Z}^+}\left( -\frac{1}{i} ,\frac{1}{i}  \right),
\] which clearly contains the zero sequence. Fix $n$, then note that $\mathbf{x}_{n}$ is not contained in $U$ since $1/n$ is not in any of the intervals $(-1/i, 1/i)$ for $i \geq n$. Since $n$ was arbitrary, this means that $\mathbf{x}_{n}$ is never contained in $U$ for any $n$, so $\mathbf{x}_{n}$ cannot converge to the zero sequence.

\begin{exer}[]
Munkres \S 19, pg. 118 \#7.
\end{exer}
{\color{blue}Collaborators: None.}

\textbf{Box Topology:} In the box topology, $\overline{\mathbb{R}^{\infty}} =\mathbb{R}^{\infty}$. It suffices to show that $\mathbb{R}^{\infty}$ is closed, which is equivalent to showing that $\mathbb{R}^{\omega}-\mathbb{R}^{\infty}$ is open.

Let $x \in \mathbb{R}^{\omega}-\mathbb{R}^{\infty}$, then $x$ is not eventually 0. We can define an open set $U$ by $U = \prod U_i$, where
\[
U_i =
\begin{cases}
	B(x_i, |x_i|/2) & x_i \neq 0,\\
	\mathbb{R} & x_i=0
\end{cases}.
\] Note that since $U$ has infinitely many $U_i$ that are not $\mathbb{R}$, it is open in the box topology but not the product topology. It clearly contains $x$, and we claim that it lies entirely in $\mathbb{R}^{\omega}-\mathbb{R}^{\infty}$. Since any $y \in \mathbb{R}^{\infty}$ is eventually 0 and our $x$ isn't, there must be some $i$ such that $U_i=B(x_i, |x_i|/2)$ (which doesn't contain 0) and $y_i=0$. Since $y$ was arbitrary, $U$ does not contain any elements of $\mathbb{R}^{\infty}$. Thus $\mathbb{R}^{\omega}-\mathbb{R}^{\infty}$ is open, so $\mathbb{R}^{\infty}$ is closed, so $\overline{\mathbb{R}^{\infty}} =\mathbb{R}^{\infty}$.

\textbf{Product Topology:} In the product topology, $\overline{\mathbb{R}^{\infty}} =\mathbb{R}^{\omega}$. Let $x \in \mathbb{R}^{\omega}$ be arbitrary, then we will show that any neighborhood of $x$ intersects $\mathbb{R}^{\infty}$, making $x$ a limit point of $\mathbb{R}^{\infty}$.

Let $U$ be any neighborhood of $x$, then $U = \prod U_i$, where only finitely many of the $U_i$ are not $\mathbb{R}$. This means $U_i$ is eventually $\mathbb{R}$, so the sequence $y$ given by
\[
y_i =
\begin{cases}
	\text{any } u_i \in U_i & U_i \neq \mathbb{R},\\
	0 & U_i = \mathbb{R}
\end{cases}
\] is contained in $U$ and is eventually 0, i.e. in $\mathbb{R}^{\infty}$. Thus every neighborhood of $x$ intersects $\mathbb{R}^{\infty}$, and since $x$ was arbitrary, this means all points of $\mathbb{R}^{\omega}$ are limit points of $\mathbb{R}^{\infty}$, i.e. $\overline{\mathbb{R}^{\infty}}=\mathbb{R}^{\omega}$.

\newpage

\begin{exer}[]
	For any sequence of real numbers $x = \left\{ x_n \right\}_{n=1}^\infty$, define ${\Vert{x}\Vert}_{\infty}=\sup_{n}|x_n|$. Let $\ell^{\infty}$ be the collection of all sequences $x$ satisfying ${\Vert{x}\Vert}_{\infty}<\infty$ and define $d(x,y)={\Vert{x-y}\Vert}_{\infty}$. Prove $(\ell^{\infty}, d)$ is a metric space.
\end{exer}
{\color{blue}Collaborators: None.}

To begin with, since $\ell^{\infty}$ has sequences with bounded supremum norm, we know that $d$ is a function into $\mathbb{R}$ and not $\mathbb{R}\cup\left\{ \infty \right\}$. Now we show that $d$ is a metric, which will make $\left( \ell^{\infty}, d \right)$ a metric space.

\begin{enumerate}
	\item $d $ is non-negative: $d(x,y) = \sup_{i}|x_i-y_i|\geq 0$.
	\item $d(x,y)=\sup_{i}|x_i-y_i|=0$ if and only if $x_i-y_i=0$ for all $i$, which implies  $x=y$.
	\item $d$ is symmetric: $d(x,y) = \sup_{i}|x_i-y_i|=\sup_i|y_i-x_i| =d(y,x)$.
	\item Triangle inequality: For any sequence $z$, we have $d(x,y) = \sup_i|x_i-y_i|=\sup_i|x_i-z_i+z_i-y_i|\leq \sup_i|x_i-z_i|+\sup_i|z_i-y_i|=d(x,z)+d(z,y)$.
\end{enumerate}


\begin{exer}[]
Munkres, \S 21, pg. 133 \#1.
\end{exer}
{\color{blue}Collaborators: None.}

By definition of the metric topology, the topology on $X$ is given by the basis
\[
\mathcal{B}_{X}= \left\{B_{d}(x,\varepsilon) \;|\; x \in X, \varepsilon>0\right\}.
\] Then by definition of the subspace topology, the topology on $A$ as a subspace of $X$ is given by the basis
\[
	\mathcal{B}_{A} = \left\{ B_{d}(x,\varepsilon) \cap A \;|\; x \in X, \varepsilon>0 \right\}.
\] If $x \in A \subset X$, then $B_{d}(x,\varepsilon) \cap A$ can be written $B_{d|A \times A}(x,\varepsilon).$ And if $x \in X-A$, then there are two cases. If a particular $\varepsilon$-ball around $x$ does not intersect $A$, then since we've already shown that any $\varepsilon$-ball can be expressed as the union of balls contained in it, we can express $B_{d}(x,\varepsilon) \cap A$ as the union of balls with centers in $A$ based on the metric $d|A \times A$. Thus we can express $\mathcal{B}_{A}$ as
\[
	\mathcal{B}_{A} = \left\{ D_{d|A \times A}(x,\varepsilon) \;|\; x \in A, \varepsilon>0 \right\},
\] so the topology on $A$ as a subspace of $X$ is induced by the metric $d|A \times A$.

\begin{exer}[]
Wasserstein video.
\end{exer}
I watched the video.

\end{document}
