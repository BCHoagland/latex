\documentclass[twoside,10pt]{article}
\usepackage{/Users/bradenhoagland/latex/styles/toggles}
%\toggletrue{sectionbreaks}
%\toggletrue{sectionheaders}
\newcommand{\docTitle}{2-Choice Rules}
\usepackage{/Users/bradenhoagland/latex/styles/common}
\importStyles{modern}{rainbow}{boxy}

%\renewcommand{\theenumi}{\alph{enumi}}

\begin{document}
\tableofcontents

%--------------------------------------------------------------------------------
% Definitions
%--------------------------------------------------------------------------------
\section{Definitions}

To begin, we'll need to define some basic functions that we'll use over and over again. If $x_i$ is a vertex, then we denote its cluster size by $\kappa_i$. Denote the probability that the minimum of $m$ i.i.d. sampled vertices is $s$ by
\[
	Q_m(s) \doteq \mathbb{P}\left( \min\left\{ \kappa_1, \dots, \kappa_m \right\} = s \right) .
\] 
Note that $Q_m$ satisfies the identity $\sum_{s=1}^{\infty} Q_m(s) = 1-S^{m}$. Since they frequently show up in common examples, we give $m=1$ and $m=2$ shorthands:
\[
	P \doteq Q_1, \quad\quad Q \doteq Q_2.
\]
We also define
\[
	\ang{s^{k}}_{m} \doteq \sum_{s=1}^{\infty} s^{k} Q_m(s).
\] 
I'll use $\Ang{\;\cdot\;}_{P}$ and $\ang{\;\cdot\;}_{Q}$ instead of $\Ang{\;\cdot\;}_{1}$ and $\ang{\;\cdot\;}_{2}$, respectively.

Now for the main attraction. In these notes, we'll be discussing rules that add a single edge every $t=1/n$ units of time, gotten by selecting two vertices total from two separate groups of vertices that are sampled i.i.d. from the graph.

\begin{defn}[]
	Define a rule $\mathcal{R}$ as follows:
	\begin{itemize}
		\item Every $t=1/n$ units of time, choose two groups of vertices $\mathcal{V}_1$ and $\mathcal{V}_2$ by sampling vertices i.i.d. from the graph.
		\item For both $i$, follow some rule $\mathcal{F}_{i}$ to choose a vertex $x_i$ with cluster size $\kappa_w$ from group $\mathcal{V}_i$, subject to the condition that $\mathcal{F}_i$ induces a function $\phi_i(s) = \mathbb{P}\left( \kappa_i=s \right) $.
	\end{itemize}
We call $\mathcal{R}$ a \textbf{2-choice rule}.
\end{defn}
We'd like to restrict this vertex selection processes in each group as little as possible in order to get a more general theory, but in some cases we can perform much greater analysis if some information is known about them.

\begin{defn}[]
A 2-choice rule $\mathcal{R}$ is \textbf{minimizing} if $\phi_1 = Q_a$ and $\phi_2=Q_b$ for some $a,b$.
\end{defn}
Such a rule exhibits ``explosive" behavior in the sense that the critical time is significantly delayed and the giant component emerges incredibly quickly. Under the assumption that $P$ exhibits scaling behavior, minimizing 2-choice rules can be analyzed in a straightforward manner.

%--------------------------------------------------------------------------------
% The Scaling Assumption
%--------------------------------------------------------------------------------
\section{The Scaling Assumption}

Most of the results in these notes follow from the assumption that near the critical time $t_c$, $P$ has the form
\[
	P(s) = s^{1-\tau}f(s \delta^{1/\sigma})
\] for constants $\tau,\sigma$ and scaling function $f$. \warn{Motivation for this.} The following theorem gives relations between these constants if some regularity conditions hold for the scaling function $f$.

\begin{thrm}
	\label{thrm:crit-exp}
        Suppose a rule $\mathcal{R}$ has a scaling function $f$ such that
	\begin{enumerate}
		\item $\lim_{x \to \infty} x^{2-\tau}f(x) = 0$; and
		\item $\int_{0}^{\infty} x^{2-\tau} f'(x) \;dx$ is finite.
	\end{enumerate}
	Then there are \textbf{critical exponents}
	\begin{align*}
                \beta &= (\tau-2)/\sigma, \\
		\gamma_{m} &= (m(2-\tau)+1)/\sigma.
        \end{align*}
	such that $S \sim \delta^{\beta}$ and $\ang{s^{k}}_{m} \sim \delta^{-\gamma_m-(k-1)/\sigma}$.
\end{thrm}
\begin{proof}
	We'll begin by deriving $\beta$. Since
        \begin{align*}
                S &\approx \int_{0}^{\infty} s^{1-\tau}(f(0)-f(s \delta^{1/\sigma})) \;ds,
                \intertext{we can make the change of variable $s = x \delta^{-1/\sigma}$ to get}
                  &= \delta^{(\tau-2)/\sigma} \int_{0}^{\infty} x^{1-\tau} (f(0)-f(x)) \;dx.
                \intertext{Integrating by parts gives}
                &= \frac{\delta^{(\tau-2)/\sigma}}{\tau-2} \left[ \left[ -x^{2-\tau} (f(0) - f(x)) \right]^{x=\infty}_{x=0} - \int_{0}^{\infty} x^{2-\tau} f'(x) \;dx \right].
        \end{align*}
	So by our assumptions on $f$, we have $S \sim \delta^{\beta}$, where $\beta = (\tau-2)/\sigma$. \warn{Type up the rest of this.} \warn{Go over your concerns with the assumptions on $f$ and the relations between $f$ and $g$ in Appendex E of da Costa.}
\end{proof}

%--------------------------------------------------------------------------------
% Uniform Scaling Rules
%--------------------------------------------------------------------------------
\section{Uniform Scaling}

It would be nice to express all these critical exponents in terms of just one (in our case, we'll express everything in terms of $\beta$). This has two main applications:
\begin{enumerate}
	\item if we determine a single critical exponent, then we automatically know all others; and
	\item \warn{we can determine the limiting behavior of the critical exponents} as $m\to \infty$ (which drives $\beta\to 0$).
\end{enumerate}
The following property will be critical in establishing systems of equations that we can use to solve for the critical exponents in terms of $\beta$. As we will see later on, it always holds for minimizing 2-choice rules, and a partial version of it holds for general 2-choice rules.

\begin{defn}[]
	We say that a rule $\mathcal{R}$ that exhibits scaling behavior \textbf{scales uniformly} if for $S$ and all $\ang{s}_{m}$, every $\delta$ term comprising it has the same order. \warn{Can I fix this so that it's more formal, i.e. is the result of one thing that's more readily definable?}
\end{defn}

Note that since $\ang{s^{k}}_{m} \sim \delta^{-(\gamma_{m} + \frac{k-1}{\sigma})}$ differs from $\ang{s}_{m} \sim \delta^{-\gamma_{m}}$ by only an added constant, this property also applies to all $k$-th moments. Uniform scaling ends up giving us a systematic way of solving for all crtical exponents in terms of $\beta$ when we're working with minimizing 2-choice rules.

%--------------------------------------------------------------------------------
% Scaling Relations for Minimizing 2-Choice Rules
%--------------------------------------------------------------------------------
\section{Scaling Relations for Minimizing Rules}

%-------------------
% Uniform Scaling
%-------------------
\subsection{Uniform Scaling}

To start, note that $\p_{t}{P(s)} $ for any minimizing 2-choice rule has the form
\[
	\p_{t}{P(s)} = s \sum_{u+v=s} Q_a(u) Q_b(v) - s Q_a(s) - s Q_b(s).
\] As we've done in earlier notes, we can solve for $\p_{t}{S} $ and $\p_{t}{\ang{s}_{P}} $:
\begin{align*}
	\p_{t}{S} &= S^{b}\ang{s}_{a} + S^{a}\ang{s}_{b}, \tag{$\star$} \\
	\p_{t}{\ang{s}_{P}} &= 2 \ang{s}_{a}\ang{s}_{b} - S^{b}\ang{s^{2}}_{a} - S^{a}\ang{s^{2}}_{b}. \tag{$\star\star$}
\end{align*}
Using these calculations, we can show that all minimizing 2-choice rules scale uniformly.
\begin{prop}
All minimizing 2-choice rules scale uniformly.
\end{prop}
\begin{proof}
	The primary tool in this proof is the following identities that we proved earlier for rules with scaling behavior:
	\begin{align*}
		\beta &= \frac{\tau-2}{\sigma},\\
		\gamma_{m} &= \frac{m(2-\tau)+1}{\sigma} .
	\end{align*}
	We'll start with $\p_{t}{S}$. The terms on the RHS of $(\star)$ have orders
	\[
	\beta b - \gamma_{a}, \quad\quad \beta a - \gamma_{b}.
	\] But plugging in our identities for $\beta$ and $\gamma_{m}$ yields 
	\[
		\frac{(\tau-2)(a+b) - 1}{\sigma} 
	\] in both cases. Now we have to check $\p_{t}{\ang{s}_{P}}$. The three terms on the RHS of $(\star\star)$ have orders
	\[
	-\gamma_{a}-\gamma_{b}, \quad\quad \beta b - \gamma_{a} - \frac{1}{\sigma} , \quad\quad \beta a - \gamma_{b} - \frac{1}{\sigma} .
	\] As before, plugging in our identities for $\beta$ and $\gamma_{m}$ shows that all three of these are equal to
	\[
		\frac{(\tau-2)(a+b) - 2}{\sigma} .
	\] Thus all 2-choice rules scale uniformly.
\end{proof}

%-------------------
% Scaling Relations
%-------------------
\subsection{Scaling Relations}

Now, given any minimizing 2-choice rule $\mathcal{R}$, we can follow the same systematic approach for expressing its critical exponents in terms of just $\beta$. Since $S \sim \delta^{\beta}$, $(\star)$ gives
\[
\beta-1 \quad=\quad \beta b-\gamma_{a} \quad=\quad \beta a-\gamma_{b},
\] which gives us the two relations
\begin{align}
	\gamma_{a} &= 1+(b-1)\beta, \\
	\gamma_{b} &= 1+(a-1)\beta.
\end{align}
Since $\ang{s}_{P} \sim \delta^{-\gamma_{P}}$, our expression $(\star\star)$ gives
\[
-\gamma_p-1 \quad=\quad -\gamma_a-\gamma_b \quad=\quad \beta b-\frac{1}{\sigma} -\gamma_a \quad=\quad \beta a - \frac{1}{\sigma} -\gamma_b,
\] which gives us the relations
\begin{align}
	\gamma_P &= \gamma_{a}+\gamma_{b}-1,\\
	\gamma_{P} &= -\beta b + \frac{1}{\sigma} +\gamma_{a}-1,\\
	\gamma_{P} &= -\beta a + \frac{1}{\sigma} +\gamma_{b}-1,\\
	\gamma_{b} &= \frac{1}{\sigma} -\beta b,\\
	\gamma_{a} &= \frac{1}{\sigma} -\beta a,\\
	\gamma_{a}-\gamma_{b} &= \beta(b-a).
\end{align}
There's a bit of redundant information in this system, though, so we won't end up using all the relations. I'm including them all so I don't forget any of them, though. By (1), (2), and (3), we get
\begin{equation}
	\gamma_{P} = 1 + (a+b-2)\beta.
\end{equation}
By (1) and (7), or also by (2) and (6), we get
\begin{equation}
	\frac{1}{\sigma} = 1 + (a+b-1)\beta.
\end{equation}
And finally, by (10) and the relation $\beta = (\tau-2)/\sigma$, we get
\begin{equation}
	\tau = \frac{\beta}{1 + (a+b-1)\beta} +2.
\end{equation}
Formulas (1), (2), (9), (10), and (11) are our desired relations. As a sanity check, plugging in the values of $a$ and $b$ for our three understood rules gives the relations we previously derived for those.

%-------------------
% Observations
%-------------------
\subsection{Observations}

Suppose that $a=1$, then $\gamma_b=1$, no matter what $b$ is. A symmetric statement holds if $b=1$ instead. This matches what we saw with the adjacent edge rule, and reveals a somewhat surprising (at least to me) relationship. Here are some more scattered thoughts:
\begin{itemize}
	\item Unless we're using \ER, $\gamma_{P}$ will always have a dependence on $\beta$.
	\item $\sigma$ and $\tau$ will always depend on $\beta$.
\end{itemize}
So in summary, if neither of our groups has size 1, we can't know \textit{any} of the critical exponents until we've calculated $\beta$, which stinks.

%--------------------------------------------------------------------------------
% Scaling Relations for General 2-Choice Rules
%--------------------------------------------------------------------------------
\section{Scaling Relations for General 2-Choice Rules}

%-------------------
% Uniform Scaling
%-------------------
\subsection{Uniform Scaling}

General 2-choice rules have an ODE of the form
	\[
		\p_{t}{P(s)} = s \sum_{u+v=s} \phi_1(u) \phi_1(v) - s \phi_1(s) - s\phi_2(s).
	\] 
This lets us calculate $\p_{t}{S} $.
\begin{prop}
	\label{prop:2c-sdelS}
	For 2-choice rules,
	\[
		\p_{t}{S} = \ang{s}_{\phi_1}\left( 1-\ang{1}_{\phi_2} \right) + \ang{s}_{\phi_2}\left( 1-\ang{1}_{\phi_1} \right).
	\] 
\end{prop}
\begin{ex}[]
	If our rule is minimizing, then $\phi_1=Q_a$ and $\phi_2=Q_b$. Then using the identity $\sum_s Q_m(s) = 1-S^m$, this reduces to
	\[
		\p_{t}{S} = \ang{s}_{a}S^{b} + \ang{s}_{b}S^{a}.
	\] 
\end{ex}

\begin{lem}[]
For any $\phi_i$, there is an associated function $\zeta_i$ such that
\[
	\ang{1}_{\phi_i} = 1 - \zeta_i(S).
\] 
\end{lem}
\begin{proof}
	\warn{Note that $\zeta$ should have some kind of non-negativity condition or something like that.}

	\warn{I'd like to find a method for determining $\zeta_i$ explicitly.} The existence of such a $\zeta_i$ is straightforward. Since $\phi_i$ is a probability measure, we necessarily have $\sum_s \phi_i(s) \leq 1$. \warn{Seems like it has to be a function of $S$ since we're picking things from the space of finite clusters, but idk how to formalize that.}
\end{proof}

Note that $\zeta_i(S)$ induces another function $F_i:\beta \mapsto \alpha_i \beta$ that scales $\beta$ by the dominating (minimum) order $\alpha_i$ of $S$ in $\zeta_i(S)$.

\begin{ex}[]
	Consider the rule where in each of the 2 groups, we do the following:
	\begin{enumerate}
		\item Pick 3 verties $v_1, v_2, v_3$ i.i.d.
		\item Of $v_1$ and $v_2$, choose the one with the smaller cluster size and label it $\tilde{v}$.
		\item Choose between $\tilde{v}$ and $v_3$ randomly to get the vertex for the group.
	\end{enumerate}
	In terms of $\phi_i$, this rule is defined by $\phi_1=\phi_2=\frac{1}{2} (P+Q)$. A straightforward computation gives $\ang{1}_{\phi_i} = 1 - \frac{1}{2} (S+S^2)$, so $\zeta_i(S) = \frac{1}{2} (S+S^2)$. The minimum order of $S$ here is 1, so the induced map is $F_i:\beta\mapsto \beta$ for both $i$.
\end{ex}

\begin{thrm}
If $\mathcal{R}$ is a 2-choice rule, then it has two dominating terms with the same order.
\end{thrm}
\begin{proof}
	By Proposition \ref{prop:2c-sdelS} and the preceding lemma,
        \[
	\p_{t}{S} = \ang{s}_{\phi_1}\zeta_{2}(S) + \ang{s}_{\phi_2} \zeta_{1}(S).
	\]
	Suppose $F_i(\beta)$ is induced from $\zeta_i(S)$ \warn{(Check to  make sure that the two terms this applies to actually dominate)}. Then there are two dominating terms, and both have order $(\alpha_1 + \alpha_2) \beta - 1/\sigma$.
\end{proof}

Since we know $\p_{t}{S} \sim \delta^{\beta-1}$, this immediately implies
\[
	\frac{1}{\sigma} = (\alpha_1+\alpha_2-1)\beta+1,
\] so we can always determine $\sigma$ in terms of $\beta$ if we know both the $\alpha_i$.

\begin{cor}
	If $\mathcal{R}$ is a 2-choice rule such that $\zeta_i(S)$ is a single term for both $i$, then $\mathcal{R}$ scales uniformly.
\end{cor}
\begin{proof}
	If $\zeta_i(S)$ is a single term for both $i$, then $\p_{t}{S} $ has 2 total terms, which must necessarily have the same order.
\end{proof}

\begin{ex}[Uniform scaling]
	For minimizing rules, $\zeta_1(S) = S^{a}$, which induces $F_1(\beta) = a\beta$. Similarly, $\zeta_2(S) = S^{b}$ and $F_2(\beta) = b\beta$. So the terms in $\p_{t}{S} $ have order
	\[
		(a+b)\beta - \frac{1}{\sigma} ,
	\] which we can verify as true.
\end{ex}

\begin{ex}[Partial uniform scaling]
	Recall from Example 2 the rule that was defined by $\phi_1 = \phi_2 = \frac{1}{2} (P+Q)$, and how the induced scaling map for both $i$ was $F_i:\beta\mapsto \beta$. This means $\alpha_i=1$ for both $i$, so
	\[
	\frac{1}{\sigma} = \beta+1.
	\] Through the usual methods (just with a lot more algebra), we can derive
	\[
		\p_{t}{S} = \frac{1}{2} (S^2+2)(\ang{s}_{P}+\ang{s}_{Q}).
	\]
	Expanding this out and substituting in scaling forms gives
	\[
		\delta^{\beta-1} \sim \delta^{2\beta-\gamma_{P}}+\delta^{2\beta-\gamma_{Q}}+\delta^{\beta-\gamma_{P}}+\delta^{\beta-\gamma_{Q}}.
	\] This clearly cannot scale uniformly, since that would imply that $\beta=0$, i.e. the giant component doesn't grow at all near criticality; however, our theory tells us that since this is a 2-choice rule, the two dominating terms have the same order. Thus we have
	\[
	\beta -1 \quad=\quad \beta-\gamma_{P} \quad=\quad \beta-\gamma_{Q}.
\] This system implies
	\[
	\gamma_{P}=\gamma_{Q}=1.
	\]
\end{ex}

With all this in place, we can give a simpler proof that all minimizing 2-choice rules scale uniformly.
\begin{prop}
All minimizing 2-choice rules scale uniformly.
\end{prop}
\begin{proof}
	For both $i$, $\phi_i = Q_m$ for some $m$. Then $\ang{1}_{\phi_i} = 1 - S^{m}$, so $\zeta_i(S) = S^{m}$. Since this is a single term, the rule must scale uniformly.
\end{proof}

%-------------------
% Scaling Relations
%-------------------
\subsection{Scaling Relations}

We just saw that
\begin{equation}
	\frac{1}{\sigma} = (\alpha_1+\alpha_2-1)\beta+1,
\end{equation}
but we can derive other scaling relations for general 2-choice rules, too.

Since $\ang{1}_{\phi_i} = 1 - \zeta_i(S)$, it will have scaling behavior based on $\beta$. \warn{I'd like to somehow show this formally with the integral stuff, but I'm having trouble.} Thus it makes sense to define $\gamma_{\phi_i}$ as the constant satisfying
\[
\ang{s}_{\phi_i} \sim \delta^{-\gamma_{\phi_i}}.
\] 
Then since $\p_{t}{S} = \ang{s}_{\phi_1}\zeta_2(S) + \ang{s}_{\phi_2}\zeta_1(S)$, the two dominating terms near criticality give us the system
\[
\beta - 1 \quad=\quad -\gamma_{\phi_1} + \alpha_2 \beta \quad=\quad -\gamma_{\phi_2}+\alpha_1 \beta.
\] 
This system implies
\begin{align}
	\gamma_{\phi_1} &= (\alpha_2-1)\beta + 1,\\
	\gamma_{\phi_2} &= (\alpha_1-1)\beta + 1.
\end{align}
\begin{ex}[da Costa]
	If $\phi_1=\phi_2=Q_m$, then $\gamma_{\phi_1}=\gamma_{\phi_2}=\gamma_{m}$. We know that for da Costa $\alpha=m$, so $\gamma_{m} = (m-1)\beta+1$.
\end{ex}

One last constant that we care about is $\gamma_{P}$, which tells us how the average finite cluster size changes. In order to determine it, we need to differentiate $\ang{s}_{P}$.

\begin{prop}
For 2-choice rules,
\[
	\p_{t}{\ang{s}_{P}} = 2\ang{s}_{\phi_1}\ang{s}_{\phi_2} - \ang{s^2}_{\phi_1}\zeta_2(S) - \ang{s^2}_{\phi_2}\zeta_1(S).
\] 
\end{prop}

\warn{Check that the three dominating terms scale uniformly... although they definitely should.}

This gives us the system
\[
-\gamma_{P}-1 \quad=\quad -\gamma_{\phi_1}-\frac{1}{\sigma} +\alpha_2\beta \quad=\quad -\gamma_{\phi_2}-\frac{1}{\sigma} +\alpha_1\beta \quad=\quad -\gamma_{\phi}-\gamma_{\phi_2}.
\] 
Using (12), this system gives us
\begin{equation}
	\gamma_{P} = (\alpha_1+\alpha_2-2)\beta+1.
\end{equation}
Based on (12), we see
\[
\gamma_{P} = \frac{1}{\sigma} -\beta,
\] which coincidentally agrees with (6) and (7) (with $a=b=1$).

\warn{Is it possible to get similar statements for all $\gamma_{m}$?}

\begin{ex}[da Costa]
	Since $\alpha_i=m$, we have $\gamma_{P}=2(m-1)\beta+1$.
\end{ex}


\end{document}
