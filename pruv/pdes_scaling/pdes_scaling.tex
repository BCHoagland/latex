\documentclass[twoside,10pt]{report}
\usepackage{/Users/bradenhoagland/latex/toggles}
%\toggletrue{sectionbreaks}
%\toggletrue{sectionheaders}
\newcommand{\docTitle}{PDEs and Scaling}
\usepackage{/Users/bradenhoagland/latex/math2}

\setlength{\headheight}{15pt}

%\renewcommand{\theenumi}{\alph{enumi}}

\begin{document}
%\tableofcontents

%%%%%%%%%%%%%%%%%%%%
% PDEs
%%%%%%%%%%%%%%%%%%%%

\section{PDEs}

At every step we choose some finite collection of vertices $\left\{ v_i \right\}_{i=1}^{m}$. Let $\kappa_i$ denote the size of the cluster to which $v_i$ belongs. We'll use the following quantities a lot (all probabilities are implicitly taken at time $t$):
\begin{align*}
	X_{m}(k,t) &\doteq \mathbb{P}\left( \min\left\{ \kappa_1, \dots, \kappa_{m} \right\} = k \right); \\
	\hat{X}_{m}(k,t) &\doteq \mathbb{P}\left( \min\left\{ \kappa_1, \dots, \kappa_{m} \right\} \geq k \right)\\
	&= 1 - \sum_{j=1}^{k-1} X_{m}(j,t); \\
	R(k,t) &= \mathbb{P}\left( \kappa_1 + \kappa_2 = k \right); \\
	\hat{R}(k,t) &= \mathbb{P}\left( \kappa_1 + \kappa_2 \geq k \right).
\end{align*}
A common case for $X_{m}$ is $m=1$ or 2, so we can abbreviate those as
\[
P \doteq X_{1}, \quad\quad Q \doteq X_{2}.
\] Note that we can express $X_{m}$ as
\[
	X_{m}(k,t) = \hat{P}(k-1, t)^{m} - \hat{P}(k,t)^m,
\] 
\warn{(go over why)} so every $X_{m}$ is a function of $P$. As a final note, I will frequently suppress $t$ from now on.

We're interested in how $P$ changes throughout the percolation process. The following table gives the value of $\p_{t}{P} $, written in terms of the proper $X_{m}$, for each of our rules.
\begin{center}
	\begin{tabular}{ c | c }
		Rule & $\p_{t}{P(s,t)} $ \\
		\hline
		\ER & $\frac{s}{2} \sum_{u+v=s} P(u,t) P(v,t) - s P(s,t)$ \\
		Adjacent Edge & $s \sum_{u+v=s}P(u,t)Q(v,t) - s P(s,t)- s Q(s,t)$ \\
		DaCosta & $s \sum_{u+v=s}X_m(u,t)X_m(v,t) - 2s X_m(s,t)$ \\
		Sum & \warn{Do this.} \\
		Product & \warn{Do this.}
	\end{tabular}
\end{center}

%%%%%%%%%%%%%%%%%%%%
% Consequences
%%%%%%%%%%%%%%%%%%%%

\section{Consequences}

Let $S(t)$ denote the relative size (i.e. divided by $n$) of the percolation cluster at time $t$, and let $X_{m}(k,t) \doteq \mathbb{P}\left( \min\left\{ \kappa_1, \dots, \kappa_{m} \right\} = k \right) $.
\begin{prop}
	\[
		\sum_{k} X_{m}(k,t) = 1 - S^{m}(t).
	\] 
\end{prop}
	To justify this, we can interpret $\sum_{k}X_{m}(k, t)$ as the probability that, at time $t$, the minimum cluster size of $m$ vertex choices is finite (this is in the limit as $n\to \infty$). $S$ is then the probability that a single choice is from an ``infinite" cluster size. \warn{I kinda want to do this more rigorously, but that's not too important right now...}

	Differentiating this identity for $X_1=P$ gives
\[
\p_{t}{S} = - \sum_{s} \p_{t}{P},
\] 
so we can track the size of the percolation cluster by knowing $P(s)$ for all $s$. In the following computations, we'll express $\p_{t}{S} $ in terms of the moments of various $X_{m}$, which we denote by
\[
	\langle s^k \rangle_{X_{m}} \doteq \sum_{s} s^{k} X_{m}(s).
\] Sometimes I might denote $\ang{\;\cdot\;}_{X_m}$ by $\ang{\;\cdot\;}_m$. The below table gives $\p_{t}{S} $ for each of our rules. A derivation of this quantity is given afterwards for the \ER rule; the other quantities are derived similarly. \warn{We don't really do anything with this information, so it could be fun to figure out what it tells us.}
\begin{center}
	\begin{tabular}{ c | c }
		Rule & $\p_{t}{S} $ \\
		\hline
		ER & $S \langle s \rangle_{P}$ \\
		AE & $\langle s \rangle_{P}S^2 + S\langle s \rangle_{Q}$ \\
		DC & $2 S^{m}\langle s \rangle_{X_m}$ \\
		Sum & \warn{Do this.} \\
		Product & \warn{Do this.}
	\end{tabular}
\end{center}
\begin{prop}
	For the \ER rule, $\p_{t}{S} = S\langle s \rangle_{P}.$
\end{prop}
\begin{proof}
	In the below computation, I suppress the time $t$ for clarity.
	\begin{align*}
		\p_{t}{S} &= - \sum_{s} \p_{t}{P} \\
			  &= -\frac{1}{2} \sum_{s} s \sum_{u+v=s} P(u) P(v) + \sum_{s} s P(s) \\
			  &= -\frac{1}{2} \sum_{u}\sum_{v} (u+v)P(u)P(v) + \langle s \rangle_{P} \\
			  &= -\frac{1}{2} \left[ \sum_{u}uP(u)\sum_{v}P(v) + \sum_{u}P(u)\sum_{v}vP(v) \right] + \langle s \rangle_{P} \\
			  &= -\frac{1}{2} \left[ 2\left\langle s \right\rangle_{P}(1-S) \right] + \left\langle s \right\rangle_{P} \\
			  &= -\langle s \rangle_{P} (1-S) + \langle s \rangle_{P} \\
			  &= S \langle s \rangle_{P}.
	\end{align*}
\end{proof}

We're similarly able to calculate $\p_{t}{\langle s \rangle_{P}} $ for these rules, as summarized in the below table. As before, I incluce the derivation for the \ER rule afterwards, and the other derivations are similar.
\begin{center}
	\begin{tabular}{ c | c }
		Rule & $\p_{t}{\langle s \rangle_{P}} $ \\
		\hline
		ER & $\langle s \rangle_{P}^2 - \langle s^2 \rangle_{P}S$ \\
		AE & $2\langle s \rangle_{P}\langle s \rangle_{Q} - \langle s^2 \rangle_{P}S^2 - \langle s^2 \rangle_{Q}S$ \\
		DC & $2\langle s \rangle_{X_m}^2 - 2 \langle s^2 \rangle_{X_m}S^m$ \\
		Sum & \warn{Do this.} \\
		Product & \warn{Do this.}
	\end{tabular}
\end{center}
\begin{prop}
	For the \ER rule, $\p_{t}{\langle s \rangle_{P}} = \ang{s}^2_{P}-\ang{s^2}_{P}S$.
\end{prop}
\begin{proof}
	Once again, I suppress the time $t$ for clarity.
	\begin{align*}
		\p_{t}{\langle s \rangle_{P}} &= \sum_{s} s \p_{t}{P(s)} \\
			      &= \frac{1}{2} \sum_{s} s^2 \sum_{u+v=s} P(u) P(v) - \sum_{s} s P(s) \\
			      &= \frac{1}{2} \sum_{u}\sum_{v}(u+v)^2P(u)P(v) - \langle s^2 \rangle_{P} \\
			      &= \frac{1}{2} \left[ \sum_{u}u^2P(u)\sum_{v}P(v) + 2 \sum_{u}uP(u)\sum_{v}vP(v) + \sum_{u}P(u)\sum_{v}v^2P(v) \right] - \langle s^2 \rangle_{P} \\
			      &= \frac{1}{2} \left[ 2\langle s^2 \rangle_{P}(1-S) + 2\langle s \rangle_{P}^2 \right] - \langle s^2 \rangle_{P} \\
			      &= \langle s \rangle_{P}^2 - \langle s^2 \rangle_{P}S.
	\end{align*}
\end{proof}
\warn{NEED TO DEFINE $\sim$.}

If $\delta \doteq |t-t_c|$ is very small, then we have the scaling relationship
\[
	\langle s \rangle_{X_{m}} \sim \delta^{-\gamma}
\]
for some $\gamma$ dependent on $X_{m}$ \warn{(Include lots more details about this)}. Differentiating gives us the relation
\[
	\p_{t}{\langle s \rangle_{X_{m}}} \sim \delta^{-\gamma-1}.
\]
Given a particular rule, we can take these two relations and subsitute them into our earlier calculation of $\p_{t}{\langle s \rangle_{P}} $ to find out how the various $\gamma$ are related. The below table sumarizes this relationship for all our rules.

\warn{Right now, I'm using the fact that $S=0$ when $t<t_c$. I don't think it's necessary to be symmetric, though, since the behavior of the system seems to change after $t_c$ anyway.}
\begin{center}
	\begin{tabular}{ c | c }
		Rule & Scaling Relationship \\
		\hline
		ER & $\gamma_{P}=1$ \\
		AE & $\gamma_{Q}=1$ \\
		DaCosta & $\gamma_{P} + 1 = 2 \gamma_{X_m}$ \\
		Sum & \warn{Do this.} \\
		Product & \warn{Do this.}
	\end{tabular}
\end{center}
\warn{Do we get any special information when $\gamma=1$? At least in the Adjacent Edge case, I hope it gives us more since this method didn't give me $\gamma_{P}$.}

%%%%%%%%%%%%%%%%%%%%
% Generalizations
%%%%%%%%%%%%%%%%%%%%

\section{Generalizations}

\warn{I had some fun generalizing this somewhat. This particular version isn't useful whatsoever, but I'm keeping it here as a reminder to think about patterns in our rules that could actually be useful to generalize.}

\begin{prop}
If
\[
	\p_{t}{P(s)} = \zeta_0 \left[ s \sum_{u_1 + \cdots + u_m=s} \prod_{i}X_{m_i}(u_i) \right] - \sum_{i} \zeta_i s X_{m_i}(u_i),
\] then the time derivative of $S$ is
\[
	\p_{t}{S} = \sum_{i} \ang{s}_{m_i} \left[ -\zeta_0 \prod_{j \neq i}(1-S^{m_j}) + \zeta_{i} \right].
\] 
\end{prop}
\begin{proof}
	\begin{align*}
		\p_{t}{S} &= -\sum_{s}\p_{t}{P(s)} \\
			  &= -\zeta_0 \left[ \sum_{s} s \sum_{\sum_i u_i = s} \prod_{i}X_{m_i}(u_i) \right] + \sum_i \zeta_i \ang{s}_{m_i} \\
			  &= -\zeta_0 \left[ \sum_{u_1, \dots, u_m} \Big(\sum_i u_i\Big) \prod_{i}X_{m_i}(u_i) \right] + \sum_i \zeta_i \ang{s}_{m_i} \\
			  &= -\zeta_0 \left[ \sum_i \ang{s}_{m_i} \prod_{j \neq i} (1-S^{m_{j}}) \right] + \sum_i \zeta_i \ang{s}_{m_i} \\
			  &= \sum_i \ang{s}_{m_i} \left[ -\zeta_0 \prod_{j \neq i}(1-S^{m_j}) + \zeta_i \right].
	\end{align*}
\end{proof}


\end{document}
