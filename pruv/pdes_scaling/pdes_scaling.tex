\documentclass[twoside,10pt]{report}
\usepackage{/Users/bradenhoagland/latex/toggles}
%\toggletrue{sectionbreaks}
%\toggletrue{sectionheaders}
\newcommand{\docTitle}{PDEs and Scaling}
\usepackage{/Users/bradenhoagland/latex/math2}

\setlength{\headheight}{15pt}

%\renewcommand{\theenumi}{\alph{enumi}}

\begin{document}
%\tableofcontents

%%%%%%%%%%%%%%%%%%%%
% PDEs
%%%%%%%%%%%%%%%%%%%%

\section{PDEs}

At every step we choose some finite collection of vertices $\left\{ v_i \right\}_{i=1}^{m}$. Let $\kappa_i$ denote the size of the cluster to which $v_i$ belongs. We'll use the following quantities a lot (all probabilities are implicitly taken at time $t$):
\begin{align*}
	Q_{m}(k,t) &\doteq \mathbb{P}\left( \min\left\{ \kappa_1, \dots, \kappa_{m} \right\} = k \right); \\
	\hat{Q}_{m}(k,t) &\doteq \mathbb{P}\left( \min\left\{ \kappa_1, \dots, \kappa_{m} \right\} \geq k \right)\\
	&= 1 - \sum_{j=1}^{k-1} Q_{m}(j,t); \\
	R(k,t) &= \mathbb{P}\left( \kappa_1 + \kappa_2 = k \right); \\
	\hat{R}(k,t) &= \mathbb{P}\left( \kappa_1 + \kappa_2 \geq k \right).
\end{align*}
A common case for $Q_{m}$ is $m=1$ or 2, so we can abbreviate those as
\[
P \doteq Q_{1}, \quad\quad Q \doteq Q_{2}.
\] Note that we can express $Q_{m}$ as
\[
	Q_{m}(k,t) = \hat{P}(k-1, t)^{m} - \hat{P}(k,t)^m,
\] 
\warn{(go over why)} so every $Q_{m}$ is a function of $P$. Let $S(t)$ denote the relative size (i.e. divided by $n$) of the percolation cluster at time $t$. As a final note, I will frequently suppress the time $t$ from now on.

We're interested in how $P$ changes throughout the percolation process. The following table gives the value of $\p_{t}{P} $, written in terms of the proper $Q_{m}$, for each of our rules.
\begin{center}
	\begin{tabular}{ c | c }
		Rule & $\p_{t}{P(s,t)} $ \\
		\hline
		\ER & $\frac{s}{2} \sum_{u+v=s} P(u,t) P(v,t) - s P(s,t)$ \\
		Adjacent Edge & $s \sum_{u+v=s}P(u,t)Q(v,t) - s P(s,t)- s Q(s,t)$ \\
		DaCosta & $s \sum_{u+v=s}Q_m(u,t)Q_m(v,t) - 2s Q_m(s,t)$ \\
		Sum & \warn{Do this.} \\
		Product & \warn{Do this.}
	\end{tabular}
\end{center}

%%%%%%%%%%%%%%%%%%%%
% Consequences
%%%%%%%%%%%%%%%%%%%%

\section{Consequences}

\begin{prop}
	$\sum_{k} Q_{m}(k,t) = 1 - S^{m}(t)$.
\end{prop}
	To justify this, we can interpret $\sum_{k}Q_{m}(k, t)$ as the probability that, at time $t$, the minimum cluster size of $m$ vertex choices is finite (this is in the limit as $n\to \infty$). $S$ is then the probability that a single choice is from an ``infinite" cluster size. \warn{I kinda want to do this more rigorously, but that's not too important right now...}

	Differentiating this identity for $X_1=P$ gives
\[
\p_{t}{S} = - \sum_{s} \p_{t}{P},
\] 
so we can track the size of the percolation cluster by knowing $P(s)$ for all $s$. In the following computations, we'll express $\p_{t}{S} $ in terms of the moments of various $Q_{m}$, which we denote by
\[
	\langle s^k \rangle_{Q_{m}} \doteq \sum_{s} s^{k} Q_{m}(s).
\] Sometimes I might denote $\ang{\;\cdot\;}_{Q_m}$ by $\ang{\;\cdot\;}_m$. The below table gives $\p_{t}{S} $ for each of our rules. A derivation of this quantity is given afterwards for the \ER rule; the other quantities are derived similarly.
\begin{center}
	\begin{tabular}{ c | c }
		Rule & $\p_{t}{S} $ \\
		\hline
		ER & $S \langle s \rangle_{P}$ \\
		AE & $\langle s \rangle_{P}S^2 + S\langle s \rangle_{Q}$ \\
		DC & $2 S^{m}\langle s \rangle_{Q_m}$ \\
		Sum & \warn{Do this.} \\
		Product & \warn{Do this.}
	\end{tabular}
\end{center}
Using the assumption $S = \delta^{\beta}$ near $t_c$, these quantities can be used to relate $\beta$ to the various other exponents.
\begin{prop}
	For the \ER rule, $\p_{t}{S} = S\langle s \rangle_{P}.$
\end{prop}
\begin{proof}
	In the below computation, I suppress the time $t$ for clarity.
	\begin{align*}
		\p_{t}{S} &= - \sum_{s} \p_{t}{P} \\
			  &= -\frac{1}{2} \sum_{s} s \sum_{u+v=s} P(u) P(v) + \sum_{s} s P(s) \\
			  &= -\frac{1}{2} \sum_{u}\sum_{v} (u+v)P(u)P(v) + \langle s \rangle_{P} \\
			  &= -\frac{1}{2} \left[ \sum_{u}uP(u)\sum_{v}P(v) + \sum_{u}P(u)\sum_{v}vP(v) \right] + \langle s \rangle_{P} \\
			  &= -\frac{1}{2} \left[ 2\left\langle s \right\rangle_{P}(1-S) \right] + \left\langle s \right\rangle_{P} \\
			  &= -\langle s \rangle_{P} (1-S) + \langle s \rangle_{P} \\
			  &= S \langle s \rangle_{P}.
	\end{align*}
\end{proof}

We're similarly able to calculate $\p_{t}{\langle s \rangle_{P}} $ for these rules, as summarized in the below table. As before, I incluce the derivation for the \ER rule afterwards, and the other derivations are similar.
\begin{center}
	\begin{tabular}{ c | c }
		Rule & $\p_{t}{\langle s \rangle_{P}} $ \\
		\hline
		ER & $\langle s \rangle_{P}^2 - \langle s^2 \rangle_{P}S$ \\
		AE & $2\langle s \rangle_{P}\langle s \rangle_{Q} - \langle s^2 \rangle_{P}S^2 - \langle s^2 \rangle_{Q}S$ \\
		DC & $2\langle s \rangle_{Q_m}^2 - 2 \langle s^2 \rangle_{Q_m}S^m$ \\
		Sum & \warn{Do this.} \\
		Product & \warn{Do this.}
	\end{tabular}
\end{center}
\begin{prop}
	For the \ER rule, $\p_{t}{\langle s \rangle_{P}} = \ang{s}^2_{P}-\ang{s^2}_{P}S$.
\end{prop}
\begin{proof}
	Once again, I suppress the time $t$ for clarity.
	\begin{align*}
		\p_{t}{\langle s \rangle_{P}} &= \sum_{s} s \p_{t}{P(s)} \\
			      &= \frac{1}{2} \sum_{s} s^2 \sum_{u+v=s} P(u) P(v) - \sum_{s} s P(s) \\
			      &= \frac{1}{2} \sum_{u}\sum_{v}(u+v)^2P(u)P(v) - \langle s^2 \rangle_{P} \\
			      &= \frac{1}{2} \left[ \sum_{u}u^2P(u)\sum_{v}P(v) + 2 \sum_{u}uP(u)\sum_{v}vP(v) + \sum_{u}P(u)\sum_{v}v^2P(v) \right] - \langle s^2 \rangle_{P} \\
			      &= \frac{1}{2} \left[ 2\langle s^2 \rangle_{P}(1-S) + 2\langle s \rangle_{P}^2 \right] - \langle s^2 \rangle_{P} \\
			      &= \langle s \rangle_{P}^2 - \langle s^2 \rangle_{P}S.
	\end{align*}
\end{proof}
\warn{NEED TO DEFINE $\sim$.}

If $\delta \doteq |t-t_c|$ is very small, then we have the scaling relationship
\[
	\langle s \rangle_{Q_{m}} \sim \delta^{-\gamma}
\]
for some $\gamma$ dependent on $Q_{m}$. Differentiating gives us the relation
\[
	\p_{t}{\langle s \rangle_{Q_{m}}} \sim \delta^{-\gamma-1}.
\]
Given a particular rule, we can take these two relations and subsitute them into our earlier calculation of $\p_{t}{\langle s \rangle_{P}} $ to find out how the various $\gamma$ are related. The below table sumarizes this relationship for all our rules.

\warn{Right now, I'm using the fact that $S=0$ when $t<t_c$. I don't think it's necessary to be symmetric, though, since the behavior of the system seems to change after $t_c$ anyway.}
\begin{center}
	\begin{tabular}{ c | c }
		Rule & Scaling Relationship \\
		\hline
		ER & $\gamma_{P}=1$ \\
		AE & $\gamma_{Q}=1$ \\
		DaCosta & $\gamma_{P} + 1 = 2 \gamma_{Q_m}$ \\
		Sum & \warn{Do this.} \\
		Product & \warn{Do this.}
	\end{tabular}
\end{center}
\warn{Do we get any special information when $\gamma=1$? I wonder if that makes any other computations elsewhere easier...}

%%%%%%%%%%%%%%%%%%%%
% Scaling Relationships
%%%%%%%%%%%%%%%%%%%%

\section{Scaling Relationships}

There are relationships between the coefficients $\beta, \tau$, and $\sigma$ that all our models use, and we can use the $\gamma$ relationships from the previous table to express everything in terms of just two of these.

\begin{thrm}
	Suppose a rule has a scaling function $f$ such that
	\[
		\lim_{x \to \infty} x^{2-\tau}f(x) = 0
	\] and
	\[
		\int_{0}^{\infty} x^{2-\tau} f'(x) \;dx
	\] is finite. Then
	\begin{align*}
		\beta &= (\tau-2)/\sigma, \\
		\gamma_{P} &= (3-\tau)/\sigma, \\
		\gamma_{Q} &= (2m-m\tau+1)/\sigma.
	\end{align*}
\end{thrm}
\begin{proof}
	We'll begin with the relation for $\beta$. Since
	\begin{align*}
		S &\approx \int_{0}^{\infty} s^{1-\tau}(f(0)-f(s \delta^{1/\sigma})) \;ds,
		\intertext{we can make the change of variable $s = x \delta^{-1/\sigma}$ to get}
		  &= \delta^{(\tau-2)/\sigma} \int_{0}^{\infty} x^{1-\tau} (f(0)-f(x)) \;dx.
		\intertext{Integrating by parts gives}
		&= \frac{\delta^{(\tau-2)/\sigma}}{\tau-2} \left[ \left[ -x^{2-\tau} (f(0) - f(x)) \right]^{x=\infty}_{x=0} - \int_{0}^{\infty} x^{2-\tau} f'(x) \;dx \right].
	\end{align*}
	So by our assumptions on $f$, we have $S = \Theta\left( \delta^{(\tau-2)/\sigma} \right)$. Since we're already assuming $S \approx \delta^{\beta}$, this means $\beta = (\tau-2)/\sigma$. The proof is similar for $\gamma_{P}$. Since
	\[
		\ang{s}_{P} = \int_{0}^{\infty} s^{2-\tau}f(s \delta^{1/\sigma}) \;ds,
	\] we can once again make a change of variables and integrate by parts. \warn{(Actually, at this point I'm confused since there's another $x$ term floating around everywhere, but the paper doesn't seem to address this).} Then since $\ang{s}_{P} \approx \delta^{-\gamma_{P}}$, we get $\gamma_{P} = (3-\tau)/\sigma$. The derivation for $\gamma_{Q}$ is similar, with the relations in Appendix $E$ of DaCosta between $f'$ and $g'$ ensuring that the final integral will be finite with $g'$ instead of $f'$.
\end{proof}

%%%%%%%%%%%%%%%%%%%%
% Computations for Adjacent Edge Rule
%%%%%%%%%%%%%%%%%%%%

\section{Computations for Adjacent Edge Rule}

Results we've already shown:
\begin{align*}
	\p_{t}{S} &= \ang{s}_{P} S^2 + S\ang{s}_{Q}, \\
	\gamma_{Q} &= 1.
\end{align*}
If we have scaling behavior, then the previous theorem applies and we get
\begin{align*}
	\beta &= (\tau-2)/\sigma \tag{1} \\
	\gamma_{P} &= (3-\tau)/\sigma \tag{2} \\
	\gamma_{Q} &= (5-2\tau)/\sigma. \tag{3}
\end{align*}
Since we know $\gamma_{Q}=1$, we have three equations and four unknowns, so we can solve for all of them in terms of just one. Plugging $\gamma_{Q}=1$ into $(3)$ and rearranging gives $\sigma=5-2\tau$, which we can plug into $(1)$ to get $\tau = (5\beta+2)/(2\beta+1)$. Plug this new expression for $\tau$ into $\sigma=5-2\tau$ to get $\sigma = 1/(2\beta+1)$. Finally, plug these expressions for $\sigma$ and $\tau$ into $(2)$ to get $\gamma_{P} = \beta+1$. In summary,
\begin{align*}
	\sigma &= \frac{1}{2\beta+1} , \\
	\tau &= \frac{5\beta+2}{2\beta+1} , \\
	\gamma_{P} &= \beta+1, \\
	\gamma_{Q} &= 1.
\end{align*}


\end{document}
