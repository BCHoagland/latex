\documentclass[twoside,10pt]{report}
\usepackage{/Users/bradenhoagland/latex/breaks}
%\toggletrue{sectionbreaks}
\newcommand{\docTitle}{Emergence of Giant Component}
\usepackage{/Users/bradenhoagland/latex/math2}

%\renewcommand{\theenumi}{\alph{enumi}}

\begin{document}
%\tableofcontents

\begin{prop}
	For all $\varepsilon>0$, whp there is a giant component after at most $(1+\varepsilon)n$ steps.
\end{prop}
\begin{proof}
	We can assume $0 < \varepsilon \leq 2$. Then set $k = 2^{\ell}/\varepsilon^{\ell-1} \geq 1$, and $\Delta = 2 \left\lceil \frac{2^{\ell}n}{\varepsilon^{\ell-1}k} \right\rceil$.

	Let $W$ denote the union of all components of size at least $k$ at time $m=0$, and take $\tilde{\varepsilon} \leq \varepsilon$ such that $W \geq \tilde{\varepsilon}n$. Let $H$ denote the event that there are \textit{not} $\ell-1$ components together containing $|W| - \varepsilon n /2$ vertices. While $H$ holds, the probability of choosing $\ell$ vertices in distinct components of $W$ is at least $\frac{|W|}{n} \left( \frac{\tilde{\varepsilon}}{2}  \right)^{\ell-1}$. In this case, the number of components of $W$ must decrease by 1.

But as in the proof of Lemma 4, the probability of $H$ holding is exponentially small in $n/k$, so whp we have $\ell-1$ components of $W$ together containing at least
\[
	|W| - \varepsilon n /2 \quad\geq\quad \tilde{\varepsilon}n - \varepsilon n / 2 \quad\geq\quad \varepsilon n/2
\]
vertices after $\Delta/2 = n < (1+\varepsilon)n$ steps. At least one of these components must have at least $\frac{\varepsilon}{2(\ell-1)} n$ vertices, so it is a giant component.
\end{proof}


\end{document}
