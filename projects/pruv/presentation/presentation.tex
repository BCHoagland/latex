\documentclass{beamer}

%\usepackage[utf8]{inputenc}
\usepackage{fontspec}
\usepackage{amsmath,amssymb,tikz-cd}
\usepackage{xcolor, xspace}

\usefonttheme[onlymath]{serif}
%\usetheme{Warsaw}

\definecolor{darkblue}{HTML}{12335F}
\definecolor{myblue}{HTML}{00A2FF}

%--------------------------------------------------------------------------------
% Custom Template
%--------------------------------------------------------------------------------
\usecolortheme[named=myblue]{structure}

\setbeamertemplate{frametitle}{\vspace{5mm}\bfseries\insertframetitle}
\setbeamercolor{frametitle}{fg=darkblue}

\setbeamerfont{section in toc}{series=\bfseries}
\setbeamercolor{section in toc}{fg=darkblue}

\AtBeginEnvironment{theorem}{%
	\setbeamercolor{block title}{fg=white, bg=darkblue}
	\setbeamercolor{block body}{fg=black,bg=myblue!20}
}

%--------------------------------------------------------------------------------
% Custom Commands
%--------------------------------------------------------------------------------
\newcommand{\ER}{Erd\H{o}s R\'enyi\xspace}

%--------------------------------------------------------------------------------
% Fonts
%--------------------------------------------------------------------------------
\setsansfont{Helvetica Neue}

\begin{document}

%--------------------------------------------------------------------------------
% Title Page
%--------------------------------------------------------------------------------
{
\setbeamercolor{background canvas}{bg=darkblue}
\begin{frame}
	\bfseries
	{\color{white}
		\huge Percolation Phase Transitions on Dynamically Grown Graphs
	}
	\vspace{5mm}

	{\color{myblue}
		\large Braden Hoagland

		Advised by Rick Durrett
	}

	\vspace*{\fill}
	{\color{white}
		\small July 8, 2021
	}
\end{frame}
}

\begin{frame}
	\tableofcontents
\end{frame}

%--------------------------------------------------------------------------------
% Background
%--------------------------------------------------------------------------------
\section{Background}

{
\setbeamercolor{background canvas}{bg=darkblue}
\begin{frame}
        \bfseries
        {\color{white}
                \huge Background
        }
        \vspace{5mm}

	{\color{myblue}
		Dynamically grown graphs and percolation
	}
\end{frame}
}

%-------------------
% Dynamically Grown Graphs
%-------------------
\subsection{Dynamically Grown Graphs}

\begin{frame}{Dynamically Grown Graphs}
	Start with a graph with $n$ vertices and no edges.
	\vspace{5mm}

	Every $t=1/n$ units of time, add edges to the graph by sampling $m$ vertices i.i.d. and following some fixed rule.
	\vspace{5mm}

	\pause
	Let $n\to \infty$.
\end{frame}

%-------------------
% Percolation
%-------------------
\subsection{Percolation}

\begin{frame}{Percolation}
	A \textit{giant component} is a cluster comprising a finite fraction $\varepsilon n$ of the graph.
	\vspace{5mm}

	Percolation is the emergence of a giant component.
	\vspace{5mm}

	Percolation can have lots of different qualitative behaviors.
\end{frame}

\begin{frame}
	\frametitle{Explosive Percolation}

	For simple rules, the giant component might emerge is a predictable, linear manner.
	\vspace{5mm}

	If a rule prioritizes adding together smaller clusters, the giant component's emergence is delayed and happens very quickly (seemingly discontinuous). This is called \textit{explosive percolation}.
\end{frame}

%--------------------------------------------------------------------------------
% Basic Results
%--------------------------------------------------------------------------------
\section{Basic Results}

{
\setbeamercolor{background canvas}{bg=darkblue}
\begin{frame}
        \bfseries
        {\color{white}
                \huge Basic Results
        }
        \vspace{5mm}

        {\color{myblue}
                Continuity of the phase transition and scaling behavior
        }
\end{frame}
}

%-------------------
% Continuity of the Phase Transition
%-------------------
\subsection{Continuity of the Phase Transition}

\begin{frame}
	\frametitle{Continuity of the Phase Transition}

	\textit{$\ell$-vertex rule}: choose $\ell$ vertices i.i.d., and you're only required to add an edge if all $\ell$ of them are in distinct clusters.
	\vspace{5mm}

	Riordan, R., and Warnke, L. (2012): all $\ell$-vertex rules have a continuous phase transition. We just can't see it in simulation because the coefficients are insanely small.
	\vspace{5mm}

	\pause
	Their proof is by contradiction, so it doesn't give us much quantitative information about the clusters' behavior.
\end{frame}

%-------------------
% Scaling Behavior
%-------------------
\subsection{Scaling Behavior}

\begin{frame}
	\frametitle{Scaling Behavior}

	For rules with continuous phase transitions, we see \textit{scaling behavior}.
	\vspace{5mm}

	Let $\delta = t-t_c$ and let $P(s, t)$ be the probability that a randomly chosen vertex has cluster size $s$ at time $t$. Then near $t_c$, there are constants $\tau$ and $\sigma$ such that
	\[
		P(s) = s^{1-\tau}f(s \delta^{1/\sigma}).
	\] Scaling behavior lets us express several important quantities as powers of $\delta$.
	\vspace{5mm}

	\pause
	{\color{myblue}\bfseries From now on, we assume scaling behavior.}
\end{frame}

\begin{frame}{Scaling Behavior}
	Let $S$ be the size of the giant component, and let
	\[
		\chi_k(t) = \sum_s s^{k} P(s, t).
	\] Then
	\[
		S \approx \delta^{\beta}, \quad\quad
		\chi_1(t) \approx \delta^{-\gamma}, \quad\quad
		\frac{\chi_k(t)}{\chi_{k-1}(t)}  \approx \delta^{-\Delta}
	\]
	These unknowns are called \textit{critical exponents}.
\end{frame}

\begin{frame}
	\frametitle{Scaling Relations}

	Goal: determine all critical exponents in terms of one unknown.
	\vspace{5mm}

	Why is this useful?
	\vspace{5mm}

	\pause
	What kinds of rules can we do this for?
\end{frame}

%--------------------------------------------------------------------------------
% 2-Choice Rules
%--------------------------------------------------------------------------------
\section{2-Choice Rules}

{
\setbeamercolor{background canvas}{bg=darkblue}
\begin{frame}
        \bfseries
        {\color{white}
                \huge 2-Choice Rules
        }
        \vspace{5mm}

        {\color{myblue}
		Generalizing rules with useful properties
        }
\end{frame}
}

\begin{frame}{2-Choice Rules}
	Pick two groups of vertices i.i.d.
	\vspace{5mm}

	Select one vertex from each and add an edge between them.
	\vspace{5mm}

	$\phi_i(s) = \mathbb{P}\left( \text{vertex chosen from group } i \text{ has cluster size } s \right) $.
\end{frame}

%-------------------
% Examples
%-------------------
\subsection{Examples}

\begin{frame}{\ER}
	Pick two random vertices and add an edge between them.
	\vspace{5mm}

	\pause
	Percolation occurs after $t_c=1/2$.
	\vspace{5mm}

	$\beta=1$, so $S$ grows linearly near $t_c$.
\end{frame}

\begin{frame}{da Costa}
	Introduced by da Costa to disprove Achlioptas' discontinuity conjecture.
	\vspace{5mm}

	Pick two groups of vertices, both of size $m$. Pick the vertex with the smallest cluster size from the two groups and add an edge between them.
	\vspace{5mm}

	Same as \ER when $m=1$. As $m\to \infty$,
	\[
	\beta\to 0, \quad\quad t_c \to 1.
	\]
\end{frame}

%-------------------
% Uniform Scaling
%-------------------
\subsection{Uniform Scaling}

\begin{frame}{Uniform Scaling}
	For any 2-choice rule, the quantity $\partial_{t}{S} $ has a simple form that can be explicitly calculated.
	\vspace{5mm}

	Near $t_c$, $\partial_t S$ will look like
	\[
	\delta^{a}+\delta^{b}+\delta^{c}+\cdots
	\] 
	We say that a rule \textit{scales uniformly} if $a=b=c=\cdots$
	\vspace{5mm}
	
	\pause
	All 2-choice rules \textbf{almost} scale uniformly.
\end{frame}

\begin{frame}{Uniform Scaling}
	\begin{theorem}
		For any 2-choice rule, there will be two dominating terms of $\partial_t S$ with the same order. If some extra technical conditions hold, then the rule scales uniformly.
	\end{theorem}
\end{frame}

\begin{frame}{Uniform Scaling}
	Consequences:
	\begin{itemize}
		\item For all 2-choice rules, we can solve for all critical exponents, as well as the growth rate of the average cluster size, in terms of $\beta$.
		\item For a large family of 2-choice rules, we can do this algorithmically.
	\end{itemize}
\end{frame}

%--------------------------------------------------------------------------------
% Future Directions
%--------------------------------------------------------------------------------
\section{Future Directions}

{
\setbeamercolor{background canvas}{bg=darkblue}
\begin{frame}
        \bfseries
        {\color{white}
                \huge Future Directions
        }
\end{frame}
}

\begin{frame}{Future Directions}
	\begin{itemize}
		\item Interaction between the groups?
		\item When does scaling behavior actually occur?
		\item What about $n$-choice rules?
	\end{itemize}
\end{frame}


\end{document}
