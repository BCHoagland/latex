\documentclass[twoside,10pt]{article}
\usepackage{/Users/bradenhoagland/latex/styles/toggles}
%\toggletrue{sectionbreaks}
%\toggletrue{sectionheaders}
\newcommand{\docTitle}{Bounded Size Rules}
\usepackage{/Users/bradenhoagland/latex/styles/common}
\importStyles{modern}{rainbow}{boxy}

\newcommand{\BF}{Bohman-Frieze\xspace}


\begin{document}
%\tableofcontents

%--------------------------------------------------------------------------------
% Introduction
%--------------------------------------------------------------------------------
\section{Introduction}

Bounded size rules are a very general category of rule that eventually start acting like the \ER rule in the following sense: fix a constant $K \in \mathbb{N}$, then $\mathcal{R}$ is a bouded size rule if it treats all sampled points with $\kappa_i > K$ identically.

\warn{This stuff has already been proven by Riordan and Warnke, albeit in a very complicated manner.}

Intuitively, it makes sense that a bounded size rule eventually ``becomes" \ER. Eventually the number of clusters of size $\leq K$ will be so small that it won't have much of an effect on the growth of the giant component. The key to noticing the complexity here is actually twofold:
\begin{enumerate}
	\item for any $k \leq K$, the number of clusters of size $k$ will be \textit{nonzero} at percolation; \warn{(can I prove this for more than just BF?)}
	\item after all clusters of size $\leq K$ have stopped having a real effect on the giant component, we're in a position much different than that of \ER. Instead of lots of isolated nodes, we have lots of finite size clusters.
\end{enumerate}

We can use a simple bounded size rule to examine these problems, which we do using a variant of the well known \BF rule.

%--------------------------------------------------------------------------------
% Bohman-Frieze
%--------------------------------------------------------------------------------
\section{Bohman-Frieze}

The original \BF rule is as follows:
\begin{enumerate}
	\item At each step, pick two edges $e_i = \left\{ u_i, v_i \right\}$.
	\item If both $u_1$ and $v_1$ are isolated, pick $e_1$.
	\item Otherwise, pick $e_2$.
\end{enumerate}
Although well understood, this rule does not fit nicely into our frameworks. We can instead work with a slight variant (and also generalization) that makes this into a bona fide bounded size 2-choice rule:
\begin{enumerate}
	\item At each step, pick $m$ vertices. These are the group 1 vertices.
	\item If any of the $m$ vertices are isolated, pick the first such one as the group 1 representative. If not, sample one more random vertex and pick it no matter what.
	\item Repeat this for group 2, and connect the 2 group representatives with an edge.
\end{enumerate}

We can explicitly write out the probability $\phi(s)$ that a group representative cluster size is $s$:
\[
	\phi(s)=
	\begin{cases}
		1 - (1-P_1)^{m+1} & s = 1, \\
		(1-P_1)^{m}P_s & s > 1.
	\end{cases}
\] 
All our 2-choice analysis was based on the quantities $\ang{1}_{\phi}$ and $\ang{s}_{\phi}$, so we should compute those.

\begin{align*}
	\ang{1}_{\phi} &= \phi(1) + \sum_{s > 1} \phi(s) \\
		       &= 1 - (1-P_1)^{m+1} + \sum_{s>1}(1-P_1)^{m}P_s \\
		       &= 1 - (1-P_1)^{m} \left[ 1 - P_1 - \sum_{s>1}P_s \right] \\
		       &= 1 - (1-P_1)^{m} (1-\ang{1}_{P}) \\
		       &= 1 - (1-P_1)^{m} S.
\end{align*}
Since $t \mapsto (1-P_1)^{m}$ is continuous at $t_c$, this has the same critical exponents as $\ang{1}_{P} = 1-S$. Similarly,
\begin{align*}
	\ang{s}_{\phi} &= \phi_1(1) + \sum_{s>1}s \phi(s) \\
		       &= 1 - (1-P_1)^{m+1} + \sum_{s>1} s(1-P_1)^{m}P_s \\
		       &= 1 - (1-P_1)^{m}\left[ 1-P_1-\sum_{s>1}sP_s \right] \\
		       &= 1 - (1-P_1)^{m}\left[ 1 - \sum_s sP_s \right] \\
		       &= 1 - (1-P_1)^{m}(1-\ang{s}_{P}).
\end{align*}
For the same reason, this also has the same critical exponents as $\ang{s}_{P}$. The upshot of these two calculations is that our analysis of 2-choice rules relied only on the critical values of $\ang{1}_{\phi}$ and $\ang{s}_{\phi}$, so this \BF variant has all the same critical exponents as \ER.

For this particular rule, we can also explicitly track how the number of isolated clusters is changing through time, which will highlight problem 1 from the introduction. Let $m=1$, then the Smoluchowski equation gives
\begin{align*}
	\p_{t}{P_1} &= -2 \phi(1) \\
		    &= -2 + 2(1-P_1)^{m+1} \\
		    &= -2 + 2(1-P_1)^{2} \\
		    &= 2P_1^{2} - 4P_1.
\end{align*}
Solving this differential equation and using the initial condition $P_1(0)=1$ gives us the solution
\[
	P_1(t) = \frac{2}{e^{4t}+1} .
\] 
We know that percolation occurs at $t=1$ at the very latest. But plugging in $t=1$ gives $P_1(1) = \frac{2}{e^{4}+1} \approx 0.024$, so at criticality, a strictly positive proportion of our nodes are still isolated.

%--------------------------------------------------------------------------------
% Greedy Bounded Size Rules
%--------------------------------------------------------------------------------
\section{Greedy Bounded Size Rules}

We can generalize our \BF variant, and consequently get closer to the case of a general bounded size rule. Let's still sample $m$ points for each group, but now we'll take the first one whose cluster size is $\leq K$. If each of the $m$ cluster sizes are greater than $K$, we'll sample a new random vertex and choose it no matter what.

Calculating $\phi(s)$ for $s \leq K$ would be pretty messy, but we definitely know
\[
	\sum_{s \leq K} \phi(s) = 1 - (1-P_{\leq K})^{m+1}
\] and
\[
	\phi(s) = (1-P_{\leq K})^{m}P_{s} \qquad \text{ when } s>K.
\] This is enough to mimic the earlier calculation of $\ang{1}_{\phi}$ for the \BF rule:
\begin{align*}
	\ang{1}_{\phi} &= \sum_{s \leq K}\phi(s) + \sum_{s > K}\phi(s) \\
		       &= 1 - (1-P_{\leq K})^{m+1} + \sum_{s>K}(1-P_{\leq K})^{m}P_s \\
		       &= 1 - (1-P_{\leq K})^{m} \left[ 1 - P_{\leq K}-\sum_{s>K}P_s \right] \\
		       &= 1 - (1-P_{\leq K})^{m} \left[ 1 - \sum_s P_s \right] \\
		       &= 1 - (1-P_{\leq K})^{m} S.
\end{align*}
And since $t \mapsto (1-P_{\leq K})^{m}$ is continuous at $t_c$, this has the same critical exponents as $\ang{1}_{P} = 1-S$. This means the induced coefficient map for any greedy bounded size rule is the identity $F:\beta\mapsto \beta$. Since all our previous analysis for 2-choice rules was based on induced coeffient maps, this means any greedy bounded size rule shares all its critical exponents with \ER.


\end{document}
