\documentclass[twoside,10pt]{article}
\usepackage{/Users/bradenhoagland/latex/styles/toggles}
%\toggletrue{sectionbreaks}
%\toggletrue{sectionheaders}
\newcommand{\docTitle}{2-Choice Rules}
\usepackage{/Users/bradenhoagland/latex/styles/common}
\importStyles{formal}{rainbow}{boxy}

%\renewcommand{\theenumi}{\alph{enumi}}

\begin{document}
%\tableofcontents

\formalTitle{\docTitle}{Braden Hoagland}{}
\formalHeader{Braden Hoagland}{2-Choice Rules}

%--------------------------------------------------------------------------------
% Definitions
%--------------------------------------------------------------------------------
\section{Definitions}

To begin, we'll need to define some basic terms that we'll use over and over again. Let $S$ be the relative size of the dominating cluster as $n \to \infty$. If $x_i$ is a vertex, then we denote its absolute cluster size by $\kappa_i$. Denote the probability that the minimum of $m$ i.i.d. sampled vertices is $s$ by
\[
	Q_m(s) \doteq \mathbb{P}\left( \min\left\{ \kappa_1, \dots, \kappa_m \right\} = s \right) .
\] 
Note that $Q_m$ satisfies the identity $\sum_{s=1}^{\infty} Q_m(s) = 1-S^{m}$. Since they frequently show up in common examples, we give $m=1$ and $m=2$ shorthands:
\[
	P \doteq Q_1, \quad\quad Q \doteq Q_2.
\]
We also define
\[
	\ang{s^{k}}_{m} \doteq \sum_{s=1}^{\infty} s^{k} Q_m(s).
\] 
I'll use $\Ang{\;\cdot\;}_{P}$ and $\ang{\;\cdot\;}_{Q}$ instead of $\Ang{\;\cdot\;}_{1}$ and $\ang{\;\cdot\;}_{2}$, respectively.

Now for the main attraction. We'll be discussing rules that add a single edge every $t=1/n$ units of time, gotten by selecting two vertices total from two separate groups of vertices that are sampled i.i.d. from the graph.

\begin{defn}[]
	Define a rule $\mathcal{R}$ as follows:
	\begin{itemize}
		\item Every $t=1/n$ units of time, choose $\ell$ groups of vertices $\mathcal{V}_1, \dots, \mathcal{V}_{\ell}$ (of potentially different sizes) by sampling vertices i.i.d. from the graph.
		\item For each $i$, follow some rule $\mathcal{F}_{i}$ to choose a vertex $x_i$ with cluster size $\kappa_w$ from group $\mathcal{V}_i$, subject to the condition that $\mathcal{F}_i$ induces a function $\phi_i(s) = \mathbb{P}\left( \kappa_i=s \right) $ that is independent from all other $\phi_j$ for $j \neq i$.
	\end{itemize}
We call $\mathcal{R}$ an \textbf{$\ell$-choice rule}.
\end{defn}

We'd like to restrict this vertex selection processes in each group as little as possible in order to get a more general theory, but in some cases we can perform much greater analysis if some information is known about them.

\begin{defn}[]
	An $\ell$-choice rule is \textbf{minimizing} if $\phi_i = Q_{m_i}$ for each $i$. It is \textbf{symmetric} if each $\phi_i$ is the same.
\end{defn}

Minimizing rules exhibit ``explosive" behavior in the sense that the critical time is significantly delayed and the giant component emerges incredibly quickly. Under the assumption that $P$ exhibits scaling behavior, minimizing 2-choice rules in particular can be analyzed in a straightforward manner.

%--------------------------------------------------------------------------------
% The Scaling Assumption
%--------------------------------------------------------------------------------
\section{The Scaling Assumption}

\warn{NEED TO REDO THIS SECTION!}

Most of the results in these notes follow from the assumption that near the critical time $t_c$, $P$ has the form
\[
	P(s) = s^{1-\tau}f(s \delta^{1/\sigma})
\] for constants $\tau,\sigma$ and scaling function $f$. \warn{Motivation for this.} The following theorem gives relations between these constants if some regularity conditions hold for the scaling function $f$.

\begin{thrm}
	\label{crit-exp}
        Suppose a rule $\mathcal{R}$ has a scaling function $f$ such that
	\begin{enumerate}
		\item $\lim_{x \to \infty} x^{2-\tau}f(x) = 0$; and
		\item $\int_{0}^{\infty} x^{2-\tau} f'(x) \;dx$ is finite.
	\end{enumerate}
	Then there are \textbf{critical exponents}
	\begin{align*}
                \beta &= (\tau-2)/\sigma, \\
		\gamma_{m} &= (m(2-\tau)+1)/\sigma.
        \end{align*}
	such that $S \sim \delta^{\beta}$ and $\ang{s^{k}}_{m} \sim \delta^{-\gamma_m-(k-1)/\sigma}$.
\end{thrm}
\begin{proof}
	We'll begin by deriving $\beta$. Since
        \begin{align*}
                S &\approx \int_{0}^{\infty} s^{1-\tau}(f(0)-f(s \delta^{1/\sigma})) \;ds,
                \intertext{we can make the change of variable $s = x \delta^{-1/\sigma}$ to get}
                  &= \delta^{(\tau-2)/\sigma} \int_{0}^{\infty} x^{1-\tau} (f(0)-f(x)) \;dx.
                \intertext{Integrating by parts gives}
                &= \frac{\delta^{(\tau-2)/\sigma}}{\tau-2} \left[ \left[ -x^{2-\tau} (f(0) - f(x)) \right]^{x=\infty}_{x=0} - \int_{0}^{\infty} x^{2-\tau} f'(x) \;dx \right].
        \end{align*}
	So by our assumptions on $f$, we have $S \sim \delta^{\beta}$, where $\beta = (\tau-2)/\sigma$. \warn{Type up the rest of this.} \warn{Go over your concerns with the assumptions on $f$ and the relations between $f$ and $g$ in Appendex E of da Costa.}
\end{proof}

%-------------------
% Induced Coefficent Maps
%-------------------
\subsection{Induced Coefficent Maps}

\begin{defn}[]
Suppose we have a function $\zeta(S)$, then this necessarily induces a map $F(\beta)$ by taking the exponent in the dominating (minimum order) terms of $\zeta(S)$ in its scaling form. This map $F$ is called the \textbf{induced coefficient map} of $\zeta$.
\end{defn}

\begin{ex}
	Fix $a, b$, then let $\zeta(S) = S^{a} + S^{b}$. In scaling form, this is $\delta^{a \beta} + \delta^{b \beta}$. The induced coefficient map is $F(\beta) = \min{a,b} \cdot \beta$.
\end{ex}

%--------------------------------------------------------------------------------
% Scaling Relations for 2-Choice Rules
%--------------------------------------------------------------------------------
\section{Scaling Relations for 2-Choice Rules}

It would be nice to express all our critical exponents in terms of just one (in our case, we'll express everything in terms of $\beta$). This has two main applications for $\ell$-choice rules:
\begin{enumerate}
	\item if we determine a single critical exponent, then we automatically know all others; and
	\item we can determine the limiting behavior of the critical exponents as as the minimum group size goes to $\infty$ (which drives $\beta\to 0$).
\end{enumerate}

This ends up being possible for a large class of 2-choice rules. Suppose $\mathcal{R}$ is a general 2-choice rule, then, since neither of the $\phi_i$ depend on each other, it satisfies the differential equation
	\[
		\p_{t}{P(s)} = s \sum_{u+v=s} \phi_1(u) \phi_1(v) - s \phi_1(s) - s\phi_2(s).
	\] 
	The summation term represents two components merging into a new component of size $s$, and the last two terms each represent a component of size $s$ joining with another component. We can use this to calculate the growth rate of the giant component.
\begin{prop}
        \label{2c-sdelS}
        For 2-choice rules,
        \[
                \p_{t}{S} = \ang{s}_{\phi_1}\left( 1-\ang{1}_{\phi_2} \right) + \ang{s}_{\phi_2}\left( 1-\ang{1}_{\phi_1} \right).
        \]
\end{prop}
\begin{proof}
        Using the identity $\sum_s P(s) = 1-S$, we calculate
        \begin{align*}
                \p_{t}{S} &= - \sum_s \p_{t}{P(s)} \\
                          &= - \sum_s s \sum_{u+v=s}\phi_1(u)\phi_2(v) + \sum_s s \phi_1(s) \sum_s + s \phi_2(s) \\
                          &= - \sum_u \sum_v (u+v) \phi_1(u)\phi_2(v) + \ang{s}_{\phi_1} + \ang{s}_{\phi_2} \\
                          &= -\sum_{u}u \phi_1(u) \sum_{v} \phi_2(v) - \sum_{u}\phi_1(u)\sum_{v}v \phi_2(v) + \ang{s}_{\phi_1} + \ang{s}_{\phi_2} \\
                          &= -\ang{s}_{\phi_1}\ang{1}_{\phi_2} - \ang{1}_{\phi_1}\ang{s}_{\phi_2} + \ang{s}_{\phi_1}+\ang{s}_{\phi_2} \\
                          &= \ang{s}_{\phi_1}\left( 1-\ang{1}_{\phi_2} \right) + \ang{s}_{\phi_2}\left( 1-\ang{1}_{\phi_1} \right).
        \end{align*}
\end{proof}

\begin{lem}[]
For any $\phi_i$, there is an associated nonnegative function $\zeta_i$ such that
\[
	\ang{1}_{\phi_i} = 1 - \zeta_i(S).
\] 
\end{lem}
\begin{proof}
	$\zeta_i$ is just the probability that a vertex chosen from group $i$ belongs to a cluster of inifinite size, so it must be a nonnegative function of $S$. \warn{(Make more rigorous, perhaps making it explicit)}
\end{proof}

\warn{Need to establish scaling behavior of $\Ang{1}_{\phi}$ using this\^.}

\warn{Is $F_i(\beta)$ the scaling coefficient for $\ang{1}_{\phi_i}$?}

\begin{thrm}
	\label{2-dom-terms}
If $\mathcal{R}$ is a 2-choice rule, then it has two dominating terms with the same order.
\end{thrm}
\begin{proof}
	By Proposition \ref{2c-sdelS} and the preceding lemma,
        \[
	\p_{t}{S} = \ang{s}_{\phi_1}\zeta_{2}(S) + \ang{s}_{\phi_2} \zeta_{1}(S).
	\]
	Suppose $F_i(\beta)$ is the induced coefficient map of $\zeta_i(S)$. Then there are two dominating terms, and both have order $F_1(\beta) + F_2(\beta) - 1/\sigma$.
\end{proof}

%-------------------
% Scaling Relations
%-------------------
\subsection{Scaling Relations}

Since $\p_{t}{S} \sim \delta^{\beta-1}$, the proof of Theorem \ref{2-dom-terms} shows that
\begin{equation}
	\frac{1}{\sigma} = F_1(\beta) + F_2(\beta) - \beta + 1,
\end{equation}
but we can derive other scaling relations for general 2-choice rules, too.

Since $\ang{1}_{\phi_i} = 1 - \zeta_i(S)$, it will have scaling behavior based on $\beta$ \warn{(right? see if you can derive this rigorously using a scaling form for $\phi$ similar to that of $P$)}. Thus it makes sense to define $\gamma_{\phi_i}$ as the constant satisfying
\[
\ang{s}_{\phi_i} \sim \delta^{-\gamma_{\phi_i}}.
\] 
Then since $\p_{t}{S} = \ang{s}_{\phi_1}\zeta_2(S) + \ang{s}_{\phi_2}\zeta_1(S)$, the two dominating terms near criticality give us the system
\[
	\beta -1 \quad=\quad -\gamma_{\phi_1} + F_2(\beta) \quad=\quad -\gamma_{\phi_2} + F_1(\beta).
\] 
This system implies
\begin{align}
	\gamma_{\phi_1} &= F_2(\beta) - \beta + 1,\\
	\gamma_{\phi_2} &= F_1(\beta) - \beta + 1.
\end{align}

One last constant that we care about is $\gamma_{P}$, which tells us how the average finite cluster size changes. In order to determine it, we need to differentiate $\ang{s}_{P}$.

\begin{prop}
For 2-choice rules,
\[
	\p_{t}{\ang{s}_{P}} = 2\ang{s}_{\phi_1}\ang{s}_{\phi_2} - \ang{s^2}_{\phi_1}\zeta_2(S) - \ang{s^2}_{\phi_2}\zeta_1(S).
\] 
\end{prop}
\begin{proof}
	\warn{Generalize this?}
\end{proof}

\warn{Check that the three dominating terms scale uniformly... although they definitely should.}
This gives us the system
\[
	-\gamma_{P}-1 \quad=\quad -\gamma_{\phi_1}-\gamma_{\phi_2} \quad=\quad -\gamma_{\phi_1}-\frac{1}{\sigma} +F_2(\beta) \quad=\quad -\gamma_{\phi_2}-\frac{1}{\sigma} + F_1(\beta).
\] 
Using (12)-(14), this system gives us
\begin{equation}
	\gamma_{P} = F_1(\beta)+F_2(\beta) - 2\beta + 1.
\end{equation}
Based on (12), we see
\[
\gamma_{P} = \frac{1}{\sigma} -\beta,
\] which coincidentally agrees with (6) and (7) (with $a=b=1$).

\warn{Is it possible to get similar statements for all $\gamma_{m}$?}

\begin{ex}[da Costa]
	Since $F_1(\beta) = F_2(\beta) = \beta m$, we have $\gamma_{P}=2(m-1)\beta+1$.
\end{ex}

Finally, using the identity $\beta = (\tau-2)/\sigma$, we get
\begin{equation}
	\tau = \frac{\beta}{F_1(\beta)+F_2(\beta)-\beta+1} +2.
\end{equation}

%-------------------
% Summary
%-------------------
\subsection{Summary}

The important equations for general 2-choice rules are
\begin{align*}
	\p_{t}{P(s)} &= s \sum_{u+v=s} \phi_1(u) \phi_1(v) - s \phi_1(s) - s\phi_2(s),\\
	\p_{t}{S} &= \ang{s}_{\phi_1}\left( 1-\ang{1}_{\phi_2} \right) + \ang{s}_{\phi_2}\left( 1-\ang{1}_{\phi_1} \right),\\
	\p_{t}{\ang{s}_{P}} &= 2\ang{s}_{\phi_1}\ang{s}_{\phi_2} - \ang{s^2}_{\phi_1}\zeta_2(S) - \ang{s^2}_{\phi_2}\zeta_1(S),
\end{align*}
and the scaling relations are
\begin{align*}
	\gamma_{\phi_1} &= F_2(\beta) - \beta+1,\\
	\gamma_{\phi_2} &= F_1(\beta)-\beta+1,\\
	\gamma_{P} &= F_1(\beta) + F_2(\beta)-2\beta+1,\\
	\frac{1}{\sigma} &= F_1(\beta)+F_2(\beta)-\beta+1,\\
	\tau &= \frac{\beta}{F_1(\beta)+F_2(\beta)-\beta+1} +2.
\end{align*}

%-------------------
% Results for Minimizing 2-Choice Rules
%-------------------
\subsection{Results for Minimizing 2-Choice Rules}

Suppose that $\phi_1 = Q$, $\phi_2 = b$. Then the scaling relations take on simpler forms. In particular, note that since $\ang{1}_{m} = 1 - S^{m}$, the induced coefficient map for $Q_m$ is $\beta \mapsto m \beta$. Thus the important equations for minimizing 2-choice rules are
\begin{align*}
        \p_{t}{P(s)} &= s \sum_{u+v=s} Q_a(u) Q_b(v) - s Q_a(s) - s Q_b(s),\\
        \p_{t}{S} &= S^{b}\ang{s}_{a} + S^{a}\ang{s}_{b}\\        \p_{t}{\ang{s}_{P}} &= 2 \ang{s}_{a}\ang{s}_{b} - S^{b}\ang{s^{2}}_{a} - S^{a}\ang{s^{2}}_{b},
\end{align*}
and the scaling relations are
\begin{align*}
        \gamma_{a} &= 1 + (b-1)\beta,\\
        \gamma_{b} &= 1+(a-1)\beta,\\
        \gamma_{P} &= 1+(a+b-2)\beta,\\
        \frac{1}{\sigma} &= 1+(a+b-1)\beta,\\
        \tau &= \frac{\beta}{1+(a+b-1)\beta} +2.
\end{align*}

Suppose that $a=1$, then $\gamma_b=1$, no matter what $b$ is. A symmetric statement holds if $b=1$ instead. This matches what we saw with the adjacent edge rule, and reveals a somewhat surprising (at least to me) relationship. Here are some more scattered thoughts:
\begin{itemize}
        \item Unless we're using \ER, $\gamma_{P}$ will always have a dependence on $\beta$.
        \item $\sigma$ and $\tau$ will always depend on $\beta$.
\end{itemize}
So in summary, if neither of our groups has size 1, we can't know \textit{any} of the critical exponents until we've calculated $\beta$, which stinks.

%--------------------------------------------------------------------------------
% Limiting Behavior of 2-Choice Rules
%--------------------------------------------------------------------------------
\section{Limiting Behavior of 2-Choice Rules}

%-------------------
% Symmetric Rules
%-------------------
\subsection{Symmetric Rules}

Suppose $\mathcal{R}$ is a symmetric 2-choice rule with induced coefficient map $F$, then its scaling relations are
\begin{align*}
	\gamma_{\phi_1} = \gamma_{\phi_2} &= F(\beta)-\beta+1,\\
	\gamma_{P} &= 2F(\beta) - 2\beta+1,\\
	\frac{1}{\sigma}  &= 2F(\beta)-\beta+1,\\
	\tau &= \frac{\beta}{2F(\beta)-\beta+1} +2.
\end{align*}

\begin{prop}[]
	Suppose we have a symmetric 2-choice rule $\mathcal{R}$ with induced coefficient function $F$. If $F(\beta) \to 0$ as $a,b \to \infty$, then the scaling coefficients for $\mathcal{R}$ have limits
	\begin{align*}
		\gamma_{\phi_1} = \gamma_{\phi_2} = \gamma_{P} = \frac{1}{\sigma} &= 1,\\
		\tau &= 2.
	\end{align*}
\end{prop}
\begin{proof}
	We already know $\beta \to 0$, so if $F(\beta)\to 0$ too, then the above limits are straightforward computations.
\end{proof}

\begin{thrm}[]
	\label{m-beta-0}
$a \beta, b \beta \to 0$, i.e. any minimizing 2-choice rule's scaling coefficients have the above limits.
\end{thrm}
\begin{proof}
	\warn{Do this.}
\end{proof}

%--------------------------------------------------------------------------------
% Variance
%--------------------------------------------------------------------------------
\section{Variance}

The variance of the cluster size (at a fixed time) is $\var_i(s) = \ang{s^2}_{\phi_i} - \ang{s}^2_{\phi_i}$. For 2-choice rules, we can use our scaling relations to get that near criticality,
\begin{align*}
	\var_i(s) &= \delta^{-[\gamma_{\phi_i} + 1/\sigma]} - \delta^{-2\gamma_{\phi_i}} \\
		  &= \delta^{-[F_1(\beta)+2F_2(\beta)-2\beta+2]} + \delta^{-[2F_2(\beta)-2\beta+2]}.
\end{align*}
Then if $F_1(\beta) \geq 0$ \warn{(which I think should be the case?)}, this is dominated by the first term. So near $t_c$, we have
\[
	\var_i(s) \approx \delta^{-[F_1(\beta)+2F_2(\beta)-2\beta+2]}.
\] 

\begin{ex}[]
	Suppose we're working with a minimizing 2-choice rule, then $F_1(\beta) = a \beta$ and $F_2(\beta)=b \beta$. Then
	\begin{align*}
		\var_1(s) &= \delta^{(2-a-2b)\beta-2},\\
		\var_2(s) &= \delta^{(2-2a-b)\beta-2}.
	\end{align*}
	By Theorem \ref{m-beta-0}, $\var_i(s) \to \delta^{-2}$ for both $i$. \warn{(I wonder if the exponent is actually $\tau$)}
\end{ex}

\warn{Central limit theorem for cluster size?}

%--------------------------------------------------------------------------------
% Ideas and Questions
%--------------------------------------------------------------------------------
\section{Ideas and Questions}

\begin{itemize}
	\item The edge weighting idea seems very cool. If we have a graph whose vertices lie in some metric space, then we can weight them proportional to their distance from some fixed basepoint to have a stochastic process that grows ``cracks" from that base point. This seems like it would slow down percolation since the farther nodes from the basepoint are less likely to have edges added to them.
	\item Can $F_i$ ever negate $\beta$, or does any induced map of the form $\beta \mapsto A \beta$ need $A > 0$?
	\item Can we reformulate $F(\beta)\to 0$ in terms of $\zeta(S)$?
	\item I don't think it's necessary to have something of the form $1-\zeta(S)$ to get the 2 dominating terms to scale uniformly, at least as long as the two unknown scaling coefficients are non-positive.
	\item This is more of just a fun thing. It could be cool to describe different group actions on these graphs that relate to adding edges, then describe some things algebraically.
	\item Can we give a sequence of graphs a product/coproduct structure? If we can turn a sequence of graphs into a coproduct, then that means that whatever property we're tracking has to appear in finite time.
\end{itemize}


\end{document}
