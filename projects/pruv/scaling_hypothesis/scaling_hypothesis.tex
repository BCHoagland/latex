\documentclass[twoside,10pt]{report}
\usepackage{/Users/bradenhoagland/latex/toggles}
%\toggletrue{sectionbreaks}
%\toggletrue{sectionheaders}
\newcommand{\docTitle}{Scaling Hypothesis}
\usepackage{/Users/bradenhoagland/latex/math2}

%\renewcommand{\theenumi}{\alph{enumi}}

\begin{document}
%\tableofcontents

%%%%%%%%%%%%%%%%%%%%
% Moments and Typical Size
%%%%%%%%%%%%%%%%%%%%

\section{Moments and Typical Size}

Throughout, we'll be focused on solutions to the differential equation
\[
	\p_{t}{c(m,t)} = \int_{y=0}^{m} K(y,m-y) c(y,t) c(m-y,t) - \int_{z=0}^{\infty} K(z,m) c(z,t) c(m,t).\tag{$\star$}
\] 
{\color{red}Should there be a 1/2 at the beginning?} The disrete version of this is
\[
\p_{t}{c(m,t)} = \frac{1}{2} \sum_{j+k=m}K(j,k)c(j,t)c(k,t) - \sum_{k}K(k,m)c(k,t)c(m,t).
\] {\color{red}Why the 1/2?}

\begin{prop}
	if $K(m,m')$ is homogeneous of degree $\lambda$ and if $c(m,t)$ is a solution to $(\star)$, then
\[
	T_{a,b}c(m,t) \doteq a^{\lambda+1}bc(am,bt)
\] is also a solution. 
\end{prop}
\begin{proof}
	Since $c(m,t)$ is a solution to $(\star)$, we can expand
	\[
		\p_{t}{T_{a,b}c(m,t)} = a^{\lambda+1}b^{2} \p_{bt}{c(am,bt)} 
	\] into its integral expression. Making a change of variables $y=ay'$ and $z=az'$ and then using the homogeneity of $K$, we recover $(\star)$ but with $T_{a,b}c(m,t)$ in place of $c(m,t)$.
	{\color{red}If we're discrete instead, seems like we'd need $a^{\lambda+2}$ instead.}
\end{proof}

\end{document}
