\documentclass[twoside,10pt]{article}
\usepackage{/Users/bradenhoagland/latex/styles/toggles}
%\toggletrue{sectionbreaks}
%\toggletrue{sectionheaders}
\newcommand{\docTitle}{Percolation Processes on Dynamically Grown Graphs}
\usepackage{/Users/bradenhoagland/latex/styles/common}
\importStyles{formal}{rainbow}{plain}

\usepackage[
        backend=biber,
	autocite=superscript
]{biblatex}
\addbibresource{bib.bib}

%\renewcommand{\theenumi}{\alph{enumi}}

\begin{document}

\formalTitle{\docTitle}{Braden Hoagland}{PRUV Summer Research Report}
\formalHeader{Braden Hoagland}{Dynamic Percolation Processes}

%--------------------------------------------------------------------------------
% Introduction
%--------------------------------------------------------------------------------
\section{Introduction}

Suppose we have an empty graph on $n$ vertices, and at each step we add one edge chosen at random. If each step corresponds to time $1/n$, then as $n\to \infty$ the graph at time $t$ converges to an \ER graph in which the connection probability between any two edges is $2t/n$. Then by basic results on \ER graphs,\autocite{ER} there is a critical time $t_{c}=1/2$ at which point a giant component with order $n$ points emerges, which is referred to as ``percolation".

The properties of the \ER model are well understood, so it serves as a natural reference point when studying other models. For example, if $\theta(t)$ is the fraction of vertices in the giant component (after we have let $n\to \infty$), then $\theta(t)\to 0$ as $t\searrow t_{c}$. This shows that the phase transition is continuous. Additionally, $\theta(t) \sim (t-t_{c})^{\beta}$, where $\beta=1$. This gives us a ``critical exponent" $\beta$ that describes the behavior of the process near $t_{c}$.

It is then natural to ask which generalizations of \ER graphs also have these properties. At a Fields Institute workshop in 2000, Dmitris Achlioptas suggested a class of variants of the \ER model, now dubbed ``Achlioptas processes". One starts with an empty graph, and at each step two edges $(v_1,v_2)$ and $(v_3,v_4)$ are randomly chosen from the set of all possible edges. Only one of them is added to the graph, though, according to some rule that depends on the cluster sizes of the current graph.

If $\kappa_i$ is the cluster size of $v_{i}$, then two possible rules are the sum rule and product rule. In the sum rule, the first edge is added if $\kappa_1+\kappa_2 < \kappa_3+\kappa_4$, and the second edge is added otherwise. In the product rule, the first edge is added if $\kappa_1 \kappa_2 < \kappa_3 \kappa_4$, and the second edge is added otherwise.

These rules are just two of many rules desgined to delay the growth of clusters and, by extension, the eventual emergence of the giant component. In 2001, Bohman and Frieze were the first to describe an Achlioptas process that delayed percolation.\autocite{BF} At each step, the first edge is added if both $v_1$ and $v_2$ were isolated vertices, and the second edge is added otherwise. They showed that there was a constant $c_0>0.535$ such that the largest component at time $c_0$ has size bounded by $(\log n)^{O(1)}$.

In 2013, Bhamidi, Budhirja, and Wang showed that the behavior of the Bohman-Frieze process near the critical point was the same as in the \ER case.\autocite{coal} Explicitly, they showed that at time $t_c + r/n^{1/3}$ for $-\infty<r<\infty$, the system converges to the ``multiplicative coalesent", in which clusters of size $x$ and $y$ merge at rate $xy$. Aldous showed a corresponding result for \ER graphs.\autocite{Aldous} Through far from trivial, this result should not be surprising. After some time there will be very few isolated vertices, so the process will start to add edges almost uniformly at random.

The Bohman-Frieze rule is an example of a ``bounded size rule", in which all components of size greater than or equal to some constant $K$ are treated the same. In 2011, Riordan and Warnke showed that the phase transition at criticality is continuous for all $\ell$-vertex rules,\autocite{RW-cont} which are a generalization of Achlioptas processes. Then in 2017, Riordan and Warnke showed that all bounded size Achlioptas processes share (in a strong sense) all the features of the \ER phase transition.\autocite{RW-bounded} 

One unfortunate consequence of working in such generality is that Riordan and Warnke's work is largely based on contradiction, and thus it does not provide much quantitative information about the sizes of the graph clusters near criticality. Under the additional assumption of ``scaling behavior" that is supported by numerical evidence (see da Costa et al.\autocite{daCosta}), though, it is possible to calculate further critical exponents for these processes.

The focus of this work will then be to generalize existing methods of calculating critical exponents, determining the classes of rules for which this is possible, and studying the limiting behavior of the critical exponents as their corresponding rules' parameters tend to extreme values.

%--------------------------------------------------------------------------------
% $\ell$-Choice Rules
%--------------------------------------------------------------------------------
\section{\texorpdfstring{$\ell$}{l}-Choice Rules}

To begin, we will need to define some basic terms that will be used over and over again. Let $S$ be the relative size of the dominating cluster as $n \to \infty$. If $x_i$ is a vertex, then we denote its absolute cluster size by $\kappa_i$. Denote the probability that the minimum of $m$ i.i.d. sampled vertices is $s$ by
\[
	Q_m(s) \doteq \mathbb{P}\left( \min\left\{ \kappa_1, \dots, \kappa_m \right\} = s \right) .
\]
Note that $Q_m$ satisfies the identity $\sum_{s=1}^{\infty} Q_m(s) = 1-S^{m}$. Since they frequently show up in common examples, we give $m=1$ and $m=2$ shorthands:
\[
	P \doteq Q_1, \quad\quad Q \doteq Q_2.
\]
We also define
\[
	\ang{s^{k}}_{m} \doteq \sum_{s=1}^{\infty} s^{k} Q_m(s).
\]
We will use $\Ang{\;\cdot\;}_{P}$ and $\ang{\;\cdot\;}_{Q}$ instead of $\Ang{\;\cdot\;}_{1}$ and $\ang{\;\cdot\;}_{2}$, respectively.

With these definitions in hand, we can turn our attention to the main attraction. We will be discussing rules that add a single edge every $t=1/n$ units of time, gotten by randomly sampling two groups of vertices i.i.d. from the graph, then choosing an endpoint vertex from each group.

\begin{defn}[]
	Define a rule $\mathcal{R}$ as follows:
	\begin{itemize}
		\item Every $t=1/n$ units of time, choose $\ell$ groups of vertices $\mathcal{V}_1, \dots, \mathcal{V}_{\ell}$ (of potentially different sizes) by sampling vertices i.i.d. from the graph.
		\item For each $i$, follow some rule $\mathcal{F}_{i}$ to choose a vertex $x_i$ with cluster size $\kappa_w$ from group $\mathcal{V}_i$, subject to the condition that $\mathcal{F}_i$ induces a function $\phi_i(s) = \mathbb{P}\left( \kappa_i=s \right) $ that does not depend on any other $\phi_j$ for $j \neq i$.
	\end{itemize}
We call $\mathcal{R}$ an \textbf{$\ell$-choice rule}.
\end{defn}

If $\phi_{i}=Q_{m_i}$ for each $i$, then $\mathcal{R}$ is \textbf{minimizing}. We can similarly define \textbf{maximizing} rules. If each $\phi_i$ is the same, $\mathcal{R}$ is \textbf{symmetric}. We would like to restrict the vertex selection processes in each group as little as possible in order to get a more general theory, but for the aforementioned special rules, we can perform much greater analysis. 

Minimizing rules exhibit ``explosive" behavior in the sense that the critical time is significantly delayed and the giant component emerges incredibly quickly. Under the assumption that $P$ exhibits scaling behavior, minimizing 2-choice rules in particular can be analyzed in a straightforward manner.

%--------------------------------------------------------------------------------
% The Scaling Assumption
%--------------------------------------------------------------------------------
\section{The Scaling Assumption}

Most of the results in this work follow from the assumption that near the critical time $t_c$, the function $P$ is of the form
\[
	P(s) = s^{1-\tau}f(s \delta^{1/\sigma})
\] for constants $\tau,\sigma$ and scaling function $f$. The following theorem gives relations between these constants if some regularity conditions hold for the scaling function $f$.

\begin{thrm}
	\label{crit-exp}
        Suppose a rule $\mathcal{R}$ has a scaling function $f$ such that
	\begin{enumerate}
		\item $\lim_{x \to \infty} x^{2-\tau}f(x) = 0$; and
		\item $\int_{0}^{\infty} x^{2-\tau} f'(x) \;dx$ is finite.
	\end{enumerate}
	Then there are \textbf{critical exponents}
	\begin{align*}
                \beta &= (\tau-2)/\sigma, \\
		\gamma_{m} &= (m(2-\tau)+1)/\sigma.
        \end{align*}
	such that $S \sim \delta^{\beta}$ and $\ang{s^{k}}_{m} \sim \delta^{-\gamma_m-(k-1)/\sigma}$.
\end{thrm}
\begin{proof}
	We'll begin by deriving $\beta$. Since
        \begin{align*}
                S &\approx \int_{0}^{\infty} s^{1-\tau}(f(0)-f(s \delta^{1/\sigma})) \;ds,
                \intertext{we can make the change of variable $s = x \delta^{-1/\sigma}$ to get}
                  &= \delta^{(\tau-2)/\sigma} \int_{0}^{\infty} x^{1-\tau} (f(0)-f(x)) \;dx.
                \intertext{Integrating by parts gives}
                &= \frac{\delta^{(\tau-2)/\sigma}}{\tau-2} \left[ \left[ -x^{2-\tau} (f(0) - f(x)) \right]^{x=\infty}_{x=0} - \int_{0}^{\infty} x^{2-\tau} f'(x) \;dx \right].
        \end{align*}
	So by our assumptions on $f$, we have $S \sim \delta^{\beta}$, where $\beta = (\tau-2)/\sigma$. The derivation of $\gamma_{m}$ is similar.
\end{proof}

Suppose we have a function $\zeta(S)$, then this induces a map $F$ taking $\beta$ to the exponent in the dominating (minimum order) terms of $\zeta(S)$ in its scaling form. This map $F$ is called the \textbf{induced coefficient map} of $\zeta$.

\begin{ex}
	Fix $a, b$, then let $\zeta(S) = S^{a} + S^{b}$. In scaling form, this is $\delta^{a \beta} + \delta^{b \beta}$. The induced coefficient map is then $F(\beta) = \min\{a,b\} \cdot \beta$.
\end{ex}

%--------------------------------------------------------------------------------
% Scaling Relations for 2-Choice Rules
%--------------------------------------------------------------------------------
\section{Scaling Relations for 2-Choice Rules}

It would be nice to express all the critical exponents in terms of just one (in our case, we will express everything in terms of $\beta$). This has two main applications for $\ell$-choice rules:
\begin{enumerate}
	\item if we determine a single critical exponent, then we automatically know all others; and
	\item we can determine the limiting behavior of the critical exponents as as the minimum group size goes to $\infty$ (which drives $\beta\to 0$).
\end{enumerate}

This ends up being possible for a large class of 2-choice rules. Suppose $\mathcal{R}$ is some general 2-choice rule, then, since none of the $\phi_i$ depend on each other, it satisfies the differential equation
	\[
		\p_{t}{P(s)} = s \sum_{u+v=s} \phi_1(u) \phi_2(v) - s \phi_1(s) - s\phi_2(s).
	\]
The summation term represents two components merging into a new component of size $s$, and the last two terms each represent a component of size $s$ joining with another component. We can use this to calculate the growth rate of the giant component.
\begin{prop}
	\label{2c-sdelS}
	For 2-choice rules,
	\[
		\p_{t}{S} = \ang{s}_{\phi_1}\left( 1-\ang{1}_{\phi_2} \right) + \ang{s}_{\phi_2}\left( 1-\ang{1}_{\phi_1} \right).
	\]
\end{prop}
\begin{proof}
	Using the identity $\sum_s P(s) = 1-S$, we calculate
	\begin{align*}
		\p_{t}{S} &= - \sum_s \p_{t}{P(s)} \\
			  &= - \sum_s s \sum_{u+v=s}\phi_1(u)\phi_2(v) + \sum_s s \phi_1(s) \sum_s + s \phi_2(s) \\
			  &= - \sum_u \sum_v (u+v) \phi_1(u)\phi_2(v) + \ang{s}_{\phi_1} + \ang{s}_{\phi_2} \\
			  &= -\sum_{u}u \phi_1(u) \sum_{v} \phi_2(v) - \sum_{u}\phi_1(u)\sum_{v}v \phi_2(v) + \ang{s}_{\phi_1} + \ang{s}_{\phi_2} \\
			  &= -\ang{s}_{\phi_1}\ang{1}_{\phi_2} - \ang{1}_{\phi_1}\ang{s}_{\phi_2} + \ang{s}_{\phi_1}+\ang{s}_{\phi_2} \\
			  &= \ang{s}_{\phi_1}\left( 1-\ang{1}_{\phi_2} \right) + \ang{s}_{\phi_2}\left( 1-\ang{1}_{\phi_1} \right).
	\end{align*}
\end{proof}

\begin{lem}[]
	\label{zeta}
For any $\phi_i$, there is an associated nonnegative function $\zeta_i$ such that
\[
	\ang{1}_{\phi_i} = 1 - \zeta_i(S).
\]
\end{lem}
\begin{proof}
	$\zeta_i$ is just the probability that a vertex chosen from group $i$ belongs to a cluster of inifinite size, so it must be a nonnegative function of $S$.
\end{proof}

\begin{thrm}
	\label{2-dom-terms}
If $\mathcal{R}$ is a 2-choice rule, then $\p_{t}{S} $ has two dominating terms of the same order.
\end{thrm}
\begin{proof}
	By \Cref{2c-sdelS} and \Cref{zeta},
        \[
	\p_{t}{S} = \ang{s}_{\phi_1}\zeta_{2}(S) + \ang{s}_{\phi_2} \zeta_{1}(S).
	\]
	Suppose $F_i(\beta)$ is the induced coefficient map of $\zeta_i(S)$. Then there are two dominating terms, and both have order $F_1(\beta) + F_2(\beta) - 1/\sigma$.
\end{proof}

%-------------------
% Scaling Relations
%-------------------
\subsection{Scaling Relations}

Since $\p_{t}{S} \sim \delta^{\beta-1}$, the proof of \Cref{2-dom-terms} shows that
\begin{equation}
	\frac{1}{\sigma} = F_1(\beta) + F_2(\beta) - \beta + 1,
\end{equation}
but we can derive other scaling relations for general 2-choice rules, too.

Since $\ang{1}_{\phi_i} = 1 - \zeta_i(S)$, it will have scaling behavior based on $\beta$. Thus it makes sense to define $\gamma_{\phi_i}$ as the constant satisfying
\[
\ang{s}_{\phi_i} \sim \delta^{-\gamma_{\phi_i}}.
\]
Then since $\p_{t}{S} = \ang{s}_{\phi_1}\zeta_2(S) + \ang{s}_{\phi_2}\zeta_1(S)$, the two dominating terms near criticality give us the system
\[
	\beta -1 \quad=\quad -\gamma_{\phi_1} + F_2(\beta) \quad=\quad -\gamma_{\phi_2} + F_1(\beta).
\]
This system implies
\begin{align}
	\gamma_{\phi_1} &= F_2(\beta) - \beta + 1,\\
	\gamma_{\phi_2} &= F_1(\beta) - \beta + 1.
\end{align}

One last constant that we care about is $\gamma_{P}$, which tells us how the average finite cluster size changes. In order to determine it, we need to differentiate $\ang{s}_{P}$.

\begin{prop}
For 2-choice rules,
\[
	\p_{t}{\ang{s}_{P}} = 2\ang{s}_{\phi_1}\ang{s}_{\phi_2} - \ang{s^2}_{\phi_1}\zeta_2(S) - \ang{s^2}_{\phi_2}\zeta_1(S).
\]
\end{prop}
\begin{proof}
	This follows from our expression for $\p_{t}{P(s)} $ and from \Cref{zeta}.
	\begin{align*}
		\p_{t}{\ang{s}_{P}} &= \sum_s s \p_{t}{P(s)} \\
				    &= \sum_s s^2 \sum_{u+v=s}\phi_1(u)\phi_2(v) - \sum_s s^2 \phi_1(s) - \sum_s s^2 \phi_2(s) \\
				    &= \sum_{u}\sum_{v} (u+v)^2\phi_1(u) \phi_2(v) - \ang{s^2}_{\phi_1}-\ang{s^2}_{\phi_2} \\
				    &= \ang{s^2}_{\phi_1}\ang{1}_{\phi_2} + 2\ang{s}_{\phi_1}\ang{s}_{\phi_2} + \ang{1}_{\phi_1}\ang{s^2}_{\phi_2}- \ang{s^2}_{\phi_1}-\ang{s^2}_{\phi_2} \\
				    &= 2\ang{s}_{\phi_1}\ang{s}_{\phi_2} + \ang{s^2}_{\phi_1}(\ang{1}_{\phi_2} - 1) + \ang{s^2}_{\phi_2}(\ang{1}_{\phi_1}-1) \\
				    &= 2\ang{s}_{\phi_1}\ang{s}_{\phi_2} - \ang{s^2}_{\phi_1}\zeta_2(S) - \ang{s^2}_{\phi_2}\zeta_1(S).
	\end{align*}
\end{proof}

Now the three terms in this expression for $\p_{t}{\ang{s}_{P}} $ all have order $F_1(\beta)+F_2(\beta) - 2/\sigma$. This gives us the system
\[
	-\gamma_{P}-1 \quad=\quad -\gamma_{\phi_1}-\gamma_{\phi_2} \quad=\quad -\gamma_{\phi_1}-\frac{1}{\sigma} +F_2(\beta) \quad=\quad -\gamma_{\phi_2}-\frac{1}{\sigma} + F_1(\beta).
\]
Using (12)-(14), this system gives us
\begin{equation}
	\gamma_{P} = F_1(\beta)+F_2(\beta) - 2\beta + 1.
\end{equation}
Based on (12), we see
\[
\gamma_{P} = \frac{1}{\sigma} -\beta,
\] which coincidentally agrees with (6) and (7) (with $a=b=1$). Finally, using the identity $\beta = (\tau-2)/\sigma$, we get
\begin{equation}
	\tau = \frac{\beta}{F_1(\beta)+F_2(\beta)-\beta+1} +2.
\end{equation}

%-------------------
% Results for Minimizing 2-Choice Rules
%-------------------
\subsection{Results for Minimizing 2-Choice Rules}

Suppose that $\phi_1 = Q$, $\phi_2 = b$. Then the scaling relations take on simpler forms. In particular, note that since $\ang{1}_{m} = 1 - S^{m}$, the induced coefficient map for $Q_m$ is $\beta \mapsto m \beta$. Thus the important equations for minimizing 2-choice rules are
\begin{align*}
        \p_{t}{P(s)} &= s \sum_{u+v=s} Q_a(u) Q_b(v) - s Q_a(s) - s Q_b(s),\\
        \p_{t}{S} &= S^{b}\ang{s}_{a} + S^{a}\ang{s}_{b}\\        \p_{t}{\ang{s}_{P}} &= 2 \ang{s}_{a}\ang{s}_{b} - S^{b}\ang{s^{2}}_{a} - S^{a}\ang{s^{2}}_{b},
\end{align*}
and the scaling relations are
\begin{align*}
        \gamma_{a} &= 1 + (b-1)\beta,\\
        \gamma_{b} &= 1+(a-1)\beta,\\
        \gamma_{P} &= 1+(a+b-2)\beta,\\
        \frac{1}{\sigma} &= 1+(a+b-1)\beta,\\
        \tau &= \frac{\beta}{1+(a+b-1)\beta} +2.
\end{align*}

Suppose that $a=1$, then $\gamma_b=1$, no matter what $b$ is. A symmetric statement holds if $b=1$ instead. We also see that unless we are using the \ER rule, $\gamma_{P}$ will always have a dependence on $\beta$. Finally, $\sigma$ and $\tau$ will always depend on $\beta$, no matter the $a$ and $b$.

%--------------------------------------------------------------------------------
% Limiting Behavior of 2-Choice Rules
%--------------------------------------------------------------------------------
\section{Limiting Behavior of 2-Choice Rules}

We now briefly discuss the behavior of the critical exponents in the limit as $a$ and $b$ grow very large. Suppose $\mathcal{R}$ is a symmetric 2-choice rule with induced coefficient map $F$, then its scaling relations are
\begin{align*}
	\gamma_{\phi_1} = \gamma_{\phi_2} &= F(\beta)-\beta+1,\\
	\gamma_{P} &= 2F(\beta) - 2\beta+1,\\
	\frac{1}{\sigma}  &= 2F(\beta)-\beta+1,\\
	\tau &= \frac{\beta}{2F(\beta)-\beta+1} +2.
\end{align*}

\begin{prop}[]
	\label{symm-limits}
	Suppose we have a symmetric 2-choice rule $\mathcal{R}$ with induced coefficient function $F$. If $F(\beta) \to 0$ as $a,b \to \infty$, then the scaling coefficients for $\mathcal{R}$ have limits
	\begin{align*}
		\gamma_{\phi_1} = \gamma_{\phi_2} = \gamma_{P} = \frac{1}{\sigma} &= 1,\\
		\tau &= 2.
	\end{align*}
\end{prop}
\begin{proof}
	We already know $\beta \to 0$, so if $F(\beta)\to 0$ too, then the above limits are straightforward computations.
\end{proof}

\begin{conj}[]
	\label{m-beta-0}
	Suppose $\mathcal{R}$ is a minimizing 2-choice rule with group sizes $a$ and $b$. Then $a\beta \to 0$ and $b\beta\to 0$, i.e. the limits in \Cref{symm-limits} apply.
\end{conj}

%--------------------------------------------------------------------------------
% Cluster Size Variance
%--------------------------------------------------------------------------------
\section{Cluster Size Variance}

The variance of the cluster size (at a fixed time) is, by definition, $\var_i(s) = \ang{s^2}_{\phi_i} - \ang{s}^2_{\phi_i}$. For 2-choice rules, we can use our scaling relations to put this solely in terms of $\beta$ and the induced coefficient maps, at least near criticality.
\begin{align*}
	\var_1(s) &= \delta^{-[\gamma_{\phi_1} + 1/\sigma]} - \delta^{-2\gamma_{\phi_1}} \\
		  &= \delta^{-[F_1(\beta)+2F_2(\beta)-2\beta+2]} + \delta^{-[2F_2(\beta)-2\beta+2]}.
\end{align*}
Similarly,
\[
        \var_2(s) = \delta^{-[2F_1(\beta)+F_2(\beta)-2\beta+2]} + \delta^{-[2F_1(\beta)-2\beta+2]}.
\]
Then if $F_i(\beta) \geq 0$ for both $i$, this is dominated by the first term. So near $t_c$, we have
\begin{align*}
	\var_1(s) &\approx \delta^{-[F_1(\beta)+2F_2(\beta)-2\beta+2]}, \\
	\var_2(s) &\approx \delta^{-[2F_1(\beta)+F_2(\beta)-2\beta+2]}.
\end{align*}
If $F_{i}(\beta)\to 0$ for both $i$, then we get the asymptotic behavior $\var_i(s) \to \delta^{-2}$. If $F_{i}(\beta) \leq 0$ for either $i$, then a similar argument with more cases gives the same limit.

\begin{ex}[]
	If $\mathcal{R}$ is a minimizing 2-choice rule, then $F_1(\beta) = a \beta$ and $F_2(\beta)=b \beta$. Then
	\begin{align*}
		\var_1(s) &= \delta^{(2-a-2b)\beta-2},\\
		\var_2(s) &= \delta^{(2-2a-b)\beta-2}.
	\end{align*}
	By \Cref{m-beta-0}, $\var_i(s) \to \delta^{-2}$ for both $i$.
\end{ex}

%--------------------------------------------------------------------------------
% Future Directions
%--------------------------------------------------------------------------------
\section{Future Directions}

There are a number of questions I plan on answering during the coming semester, which I now detail. Induced coefficient maps are central to this work, so it makes sense to classify rules based on their induced maps. Given two induced maps from different rules and a transformation between them, we could then create algebraic properties invariant under this transformation. It might be possible to them use these invariants to determine if rules necessarily have different critical exponents and/or scaling relations. Developing proper classifications based on these rules and then developing algebraic tools to differentiate rules is what I will most likely work on first this semester.

Sometimes ties in a vertex selection process can complicate or even break the analysis, but we can avoid this issue entirely by weighting vertices from some continuous distribution. Then if we have a graph whose vertices lie in some metric space, we can weight them proportional to their distance from some fixed basepoint to have a stochastic process that grows ``cracks" from that basepoint. This seems like it would slow down percolation since the farther nodes from the basepoint are less likely to have edges added to them. I hope to at least develop some theory on this random graph variant.

I also plan on investigating the limiting behavior of 2-choice rules more. I would like to develop a theory on the decay rate of $\beta$ and what affects the limit of $t_{c}$. Since my current analysis is able to dispose of the induced coefficient maps, it might be possible to expand this limiting behavior analysis to a much larger class of rules.

Finally, this entire work is under the assumption of scaling behavior. Although this is a widely accepted assumption due to intuition and results in simulation, I would like to prove that it is true, even if only for a limited class of rules. I am unsure if I can do this, but it would be by far the most notable result.

\printbibliography


\end{document}
