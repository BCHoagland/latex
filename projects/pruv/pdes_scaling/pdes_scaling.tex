\documentclass[twoside,10pt]{article}
\usepackage{/Users/bradenhoagland/latex/styles/toggles}
%\toggletrue{sectionbreaks}
%\toggletrue{sectionheaders}
\newcommand{\docTitle}{PDEs and Scaling}
\usepackage{/Users/bradenhoagland/latex/styles/common}
\importStyles{formal}{rainbow}{boxy}

%\renewcommand{\theenumi}{\alph{enumi}}

\begin{document}
%\tableofcontents

%%%%%%%%%%%%%%%%%%%%
% PDEs
%%%%%%%%%%%%%%%%%%%%

\section{PDEs}

At every step we choose some finite collection of vertices $\left\{ v_i \right\}_{i=1}^{m}$. Let $\kappa_i$ denote the size of the cluster to which $v_i$ belongs. We'll use the following quantities a lot (all probabilities are implicitly taken at time $t$):
\begin{align*}
	Q_{m}(k,t) &\doteq \mathbb{P}\left( \min\left\{ \kappa_1, \dots, \kappa_{m} \right\} = k \right); \\
	\hat{Q}_{m}(k,t) &\doteq \mathbb{P}\left( \min\left\{ \kappa_1, \dots, \kappa_{m} \right\} \geq k \right)\\
	&= 1 - \sum_{j=1}^{k-1} Q_{m}(j,t); \\
	R(k,t) &= \mathbb{P}\left( \kappa_1 + \kappa_2 = k \right); \\
	\hat{R}(k,t) &= \mathbb{P}\left( \kappa_1 + \kappa_2 \geq k \right).
\end{align*}
A common case for $Q_{m}$ is $m=1$ or 2, so we can abbreviate those as
\[
P \doteq Q_{1}, \quad\quad Q \doteq Q_{2}.
\] Note that we can express $Q_{m}$ as
\[
	Q_{m}(k,t) = \hat{P}(k-1, t)^{m} - \hat{P}(k,t)^m,
\] 
\warn{(go over why)} so every $Q_{m}$ is a function of $P$. Let $S(t)$ denote the relative size (i.e. divided by $n$) of the percolation cluster at time $t$. As a final note, I will frequently suppress the time $t$ from now on.

We're interested in how $P$ changes throughout the percolation process. The following table gives the value of $\p_{t}{P} $, written in terms of the proper $Q_{m}$, for each of our rules.
\begin{center}
	\begin{tabular}{ c | c }
		Rule & $\p_{t}{P(s,t)} $ \\
		\hline
		\ER & $\frac{s}{2} \sum_{u+v=s} P(u,t) P(v,t) - s P(s,t)$ \\
		Adjacent Edge & $s \sum_{u+v=s}P(u,t)Q(v,t) - s P(s,t)- s Q(s,t)$ \\
		da Costa & $s \sum_{u+v=s}Q_m(u,t)Q_m(v,t) - 2s Q_m(s,t)$ \\
		Sum & \warn{Do this.} \\
		Product & \warn{Do this.}
	\end{tabular}
\end{center}

%%%%%%%%%%%%%%%%%%%%
% Consequences
%%%%%%%%%%%%%%%%%%%%

\section{Consequences}

\begin{prop}
	$\sum_{k} Q_{m}(k,t) = 1 - S^{m}(t)$.
\end{prop}
	To justify this, we can interpret $\sum_{k}Q_{m}(k, t)$ as the probability that, at time $t$, the minimum cluster size of $m$ vertex choices is finite (this is in the limit as $n\to \infty$). $S$ is then the probability that a single choice is from an ``infinite" cluster size. \warn{I kinda want to do this more rigorously, but that's not too important right now...}

	Differentiating this identity for $X_1=P$ gives
\[
\p_{t}{S} = - \sum_{s} \p_{t}{P},
\] 
so we can track the size of the percolation cluster by knowing $P(s)$ for all $s$. In the following computations, we'll express $\p_{t}{S} $ in terms of the moments of various $Q_{m}$, which we denote by
\[
	\langle s^k \rangle_{Q_{m}} \doteq \sum_{s} s^{k} Q_{m}(s).
\] Sometimes I might denote $\ang{\;\cdot\;}_{Q_m}$ by $\ang{\;\cdot\;}_m$. The below table gives $\p_{t}{S} $ for each of our rules. A derivation of this quantity is given afterwards for the \ER rule; the other quantities are derived similarly.
\begin{center}
	\begin{tabular}{ c | c }
		Rule & $\p_{t}{S} $ \\
		\hline
		ER & $S \langle s \rangle_{P}$ \\
		AE & $\langle s \rangle_{P}S^2 + S\langle s \rangle_{Q}$ \\
		DC & $2 S^{m}\langle s \rangle_{Q_m}$ \\
		Sum & \warn{Do this.} \\
		Product & \warn{Do this.}
	\end{tabular}
\end{center}
Using the assumption $S = \delta^{\beta}$ near $t_c$, these quantities can be used to relate $\beta$ to the various other exponents.
\begin{prop}
	For the \ER rule, $\p_{t}{S} = S\langle s \rangle_{P}.$
\end{prop}
\begin{proof}
	In the below computation, I suppress the time $t$ for clarity.
	\begin{align*}
		\p_{t}{S} &= - \sum_{s} \p_{t}{P} \\
			  &= -\frac{1}{2} \sum_{s} s \sum_{u+v=s} P(u) P(v) + \sum_{s} s P(s) \\
			  &= -\frac{1}{2} \sum_{u}\sum_{v} (u+v)P(u)P(v) + \langle s \rangle_{P} \\
			  &= -\frac{1}{2} \left[ \sum_{u}uP(u)\sum_{v}P(v) + \sum_{u}P(u)\sum_{v}vP(v) \right] + \langle s \rangle_{P} \\
			  &= -\frac{1}{2} \left[ 2\left\langle s \right\rangle_{P}(1-S) \right] + \left\langle s \right\rangle_{P} \\
			  &= -\langle s \rangle_{P} (1-S) + \langle s \rangle_{P} \\
			  &= S \langle s \rangle_{P}.
	\end{align*}
\end{proof}

We're similarly able to calculate $\p_{t}{\langle s \rangle_{P}} $ for these rules, as summarized in the below table. As before, I incluce the derivation for the \ER rule afterwards, and the other derivations are similar.
\begin{center}
	\begin{tabular}{ c | c }
		Rule & $\p_{t}{\langle s \rangle_{P}} $ \\
		\hline
		ER & $\langle s \rangle_{P}^2 - \langle s^2 \rangle_{P}S$ \\
		AE & $2\langle s \rangle_{P}\langle s \rangle_{Q} - \langle s^2 \rangle_{P}S^2 - \langle s^2 \rangle_{Q}S$ \\
		DC & $2\langle s \rangle_{Q_m}^2 - 2 \langle s^2 \rangle_{Q_m}S^m$ \\
		Sum & \warn{Do this.} \\
		Product & \warn{Do this.}
	\end{tabular}
\end{center}
\begin{prop}
	For the \ER rule, $\p_{t}{\langle s \rangle_{P}} = \ang{s}^2_{P}-\ang{s^2}_{P}S$.
\end{prop}
\begin{proof}
	Once again, I suppress the time $t$ for clarity.
	\begin{align*}
		\p_{t}{\langle s \rangle_{P}} &= \sum_{s} s \p_{t}{P(s)} \\
			      &= \frac{1}{2} \sum_{s} s^2 \sum_{u+v=s} P(u) P(v) - \sum_{s} s P(s) \\
			      &= \frac{1}{2} \sum_{u}\sum_{v}(u+v)^2P(u)P(v) - \langle s^2 \rangle_{P} \\
			      &= \frac{1}{2} \left[ \sum_{u}u^2P(u)\sum_{v}P(v) + 2 \sum_{u}uP(u)\sum_{v}vP(v) + \sum_{u}P(u)\sum_{v}v^2P(v) \right] - \langle s^2 \rangle_{P} \\
			      &= \frac{1}{2} \left[ 2\langle s^2 \rangle_{P}(1-S) + 2\langle s \rangle_{P}^2 \right] - \langle s^2 \rangle_{P} \\
			      &= \langle s \rangle_{P}^2 - \langle s^2 \rangle_{P}S.
	\end{align*}
\end{proof}
\warn{NEED TO DEFINE $\sim$.}

If $\delta \doteq |t-t_c|$ is very small, then we have the scaling relationship
\[
	\langle s \rangle_{Q_{m}} \sim \delta^{-\gamma}
\]
for some $\gamma$ dependent on $Q_{m}$. Differentiating gives us the relation
\[
	\p_{t}{\langle s \rangle_{Q_{m}}} \sim \delta^{-\gamma-1}.
\]
Given a particular rule, we can take these two relations and subsitute them into our earlier calculation of $\p_{t}{\langle s \rangle_{P}} $ to find out how the various $\gamma$ are related. The below table sumarizes this relationship for all our rules.

\warn{Right now, I'm using the fact that $S=0$ when $t<t_c$. I don't think it's necessary to be symmetric, though, since the behavior of the system seems to change after $t_c$ anyway.}
\begin{center}
	\begin{tabular}{ c | c }
		Rule & Scaling Relationship \\
		\hline
		ER & $\gamma_{P}=1$ \\
		AE & $\gamma_{Q}=1$ \\
		da Costa & $\gamma_{P} + 1 = 2 \gamma_{Q_m}$ \\
		Sum & \warn{Do this.} \\
		Product & \warn{Do this.}
	\end{tabular}
\end{center}
\warn{Do we get any special information when $\gamma=1$? I wonder if that makes any other computations elsewhere easier...}

%%%%%%%%%%%%%%%%%%%%
% Scaling Relationships
%%%%%%%%%%%%%%%%%%%%

\section{Scaling Relationships}

There are relationships between the coefficients $\beta, \tau$, and $\sigma$ that all our models use, and we can use the $\gamma$ relationships from the previous table to express everything in terms of just two of these.

\begin{thrm}
	Suppose a rule has a scaling function $f$ such that
	\[
		\lim_{x \to \infty} x^{2-\tau}f(x) = 0
	\] and
	\[
		\int_{0}^{\infty} x^{2-\tau} f'(x) \;dx
	\] is finite. Then
	\begin{align*}
		\beta &= (\tau-2)/\sigma, \\
		\gamma_{P} &= (3-\tau)/\sigma, \\
		\gamma_{Q} &= (2m-m\tau+1)/\sigma.
	\end{align*}
\end{thrm}
\begin{proof}
	We'll begin with the relation for $\beta$. Since
	\begin{align*}
		S &\approx \int_{0}^{\infty} s^{1-\tau}(f(0)-f(s \delta^{1/\sigma})) \;ds,
		\intertext{we can make the change of variable $s = x \delta^{-1/\sigma}$ to get}
		  &= \delta^{(\tau-2)/\sigma} \int_{0}^{\infty} x^{1-\tau} (f(0)-f(x)) \;dx.
		\intertext{Integrating by parts gives}
		&= \frac{\delta^{(\tau-2)/\sigma}}{\tau-2} \left[ \left[ -x^{2-\tau} (f(0) - f(x)) \right]^{x=\infty}_{x=0} - \int_{0}^{\infty} x^{2-\tau} f'(x) \;dx \right].
	\end{align*}
	So by our assumptions on $f$, we have $S = \Theta\left( \delta^{(\tau-2)/\sigma} \right)$. Since we're already assuming $S \approx \delta^{\beta}$, this means $\beta = (\tau-2)/\sigma$. The proof is similar for $\gamma_{P}$. Since
	\[
		\ang{s}_{P} = \int_{0}^{\infty} s^{2-\tau}f(s \delta^{1/\sigma}) \;ds,
	\] we can once again make a change of variables and integrate by parts. \warn{(Actually, at this point I'm confused since there's another $x$ term floating around everywhere, but the paper doesn't seem to address this).} Then since $\ang{s}_{P} \approx \delta^{-\gamma_{P}}$, we get $\gamma_{P} = (3-\tau)/\sigma$. The derivation for $\gamma_{Q}$ is similar, with the relations in Appendix $E$ of da Costa between $f'$ and $g'$ ensuring that the final integral will be finite with $g'$ instead of $f'$.
\end{proof}

%%%%%%%%%%%%%%%%%%%%
% Computations for Adjacent Edge Rule
%%%%%%%%%%%%%%%%%%%%

\section{Computations for Adjacent Edge Rule}

Results we've already shown:
\begin{align*}
	\p_{t}{S} &= \ang{s}_{P} S^2 + S\ang{s}_{Q}, \\
	\gamma_{Q} &= 1.
\end{align*}
If we have scaling behavior, then the previous theorem applies and we get
\begin{align*}
	\beta &= (\tau-2)/\sigma \tag{1} \\
	\gamma_{P} &= (3-\tau)/\sigma \tag{2} \\
	\gamma_{Q} &= (5-2\tau)/\sigma. \tag{3}
\end{align*}
Since we know $\gamma_{Q}=1$, we have three equations and four unknowns, so we can solve for all of them in terms of just one. Plugging $\gamma_{Q}=1$ into $(3)$ and rearranging gives $\sigma=5-2\tau$, which we can plug into $(1)$ to get $\tau = (5\beta+2)/(2\beta+1)$. Plug this new expression for $\tau$ into $\sigma=5-2\tau$ to get $\sigma = 1/(2\beta+1)$. Finally, plug these expressions for $\sigma$ and $\tau$ into $(2)$ to get $\gamma_{P} = \beta+1$. In summary,
\begin{align*}
	\sigma &= \frac{1}{2\beta+1} , \\
	\tau &= \frac{5\beta+2}{2\beta+1} , \\
	\gamma_{P} &= \beta+1, \\
	\gamma_{Q} &= 1.
\end{align*}
\warn{Prof's $\tau$ disagrees with mine.}

%%%%%%%%%%%%%%%%%%%%
% General Methods
%%%%%%%%%%%%%%%%%%%%

\section{General Methods}

The following is a collection of various methods of computing these different quantities. Some methods might product the same or completely different information for different rules, some might only work with some rules. It could be interesting to investigate why these have different behavior with different rules.

%-------------------
% Differentiating $S$ once
%-------------------

\subsection{Differentiating \texorpdfstring{$S$}{S} once}

Since $S \sim \delta^{\beta}$, we can differentiate once to get $\p_{t}{S} \sim \delta^{\beta-1}$. If $\p_{t}{S} $ is dependent only on $S$ and $\ang{s}_{m}$ for some fixed $m$, then we can get useful relations between $\gamma_{m}$ and $\beta$, or perhaps even determine one entirely.

If $\p_{t}{S} $ is only depends on $S$ and $\ang{s}_{m}$, then it induces a function $\varphi(\beta, \gamma_{m})$ such that $\delta^{\beta-1} \sim \p_{t}{S} \sim \delta^{\varphi(\beta, \gamma_{m})}$. Thus $\beta-1 = \varphi(\beta, \gamma_{m})$.

\begin{ex}
For \ER, $\p_{t}{S} = S \ang{s}_{P}$, so $\beta-1 = \beta - \gamma_{P}$, which implies $\gamma_{P}=1$. For da Costa, $\p_{t}{S} = 2S^{m} \ang{s}_{m}$, so $\beta-1 = \beta m - \gamma_{m}$, which implies $\gamma_{m} = 1 + (m-1)\beta$.
\end{ex}


%-------------------
% Differentiating $S$ twice
%-------------------

\subsection{Differentiating \texorpdfstring{$S$}{S} twice}

If the time derivative of $S$ is only dependent on $S$ and $\ang{s}_{P}$, then we can relate $\gamma_{P}$ and $\beta$ or even determine one. To start, suppose $\p_{t}{S} = \phi(S, \ang{s}_{P})$. In this case, both derivatives $\p_{S}{\phi} $ and $\p_{\ang{s}_{P}}{\phi} $ will necessarily exhibit scaling behavior. We can denote this by
\begin{align*}
	\p_{S}{\phi} &\sim \delta^{A}, \\
	\p_{\ang{s}_{P}}{\phi} &\sim \delta^{B}.
\end{align*}

\begin{prop}
	If $\p_{t}{S} = \phi(S, \ang{s}_{P})$, then
	\[
	\beta-2 = \min\left\{ A+\beta-1, B-\gamma_{P}-1 \right\}.
	\]
\end{prop}
\begin{proof}
	The second time derivative of $S$ is
	\begin{align*}
		\p_{tt}{S} &= \p_{t}{\phi} \\
			   &= \p_{S}{\phi} \cdot \p_{t}{S} + \p_{\ang{s}_{P}}{\phi} \cdot \p_{t}{\ang{s}_{P}} 
	\end{align*}
	Replacing all these terms with their respective scaling forms then yields
	\begin{align*}
		\delta^{\beta-2} &\sim \delta^{A} \cdot \delta^{\beta-1} + \delta^{B} \cdot \delta^{-\gamma_{P}-1} \\
						 &= \delta^{A + \beta - 1} + \delta^{B - \gamma_{P} - 1}.
	\end{align*}
	Since $\delta < 1$, the term on the right with the \textit{smaller} exponent will dominate. Thus $\beta-2 = \min\left\{ A+\beta-1, B-\gamma_{P}-1 \right\}$.
\end{proof}

\begin{ex}[\ER]
For \ER, $\phi = S \ang{s}_{P}$, so
\begin{align*}
	\p_{S}{\phi} = \ang{s}_{P}, &\quad\quad \p_{\ang{s}_{P}}{\phi} = S, \\
	A = -\gamma_{P}, &\quad\quad B = \beta.
\end{align*}
Now $A+\beta-1 = B - \gamma_{P} - 1 = \beta - \gamma_{P} - 1$, so setting this equal to $\beta-2$ and simplyfing yields $\gamma_{P} = 1$.
\end{ex}

%%%%%%%%%%%%%%%%%%%%
% Equal Term Orders
%%%%%%%%%%%%%%%%%%%%

\section{Equal Term Orders}

Here's a strange pattern in the previous computations for $\p_{t}{S} $ and $\p_{t}{\ang{s}_{P}} $. For every rule, each term in $\p_{t}{S} $ has the same scaling coefficient, and the same holds for every term in $\p_{t}{\ang{s}_{P}} $. This would have pretty big implications, as we wouldn't have to check to see which coefficient is smallest (i.e. which dominates) if there is more than one term. We would instead get multiple relations among the coefficients that have to hold, potentially making it much easier to solve for all of them.

To actually show this, we'll need the following result.

\begin{prop}
	Suppose $\ang{s}_{m} \sim \delta^{-\gamma_{m}}$, then
	\[
		\ang{s^{k}}_{m} \sim \delta^{-\left( \gamma_{m} + \frac{k-1}{\sigma}  \right)}.
	\]
\end{prop}
\begin{proof}
	This derivation is essentially the same as for $\ang{s}_{m}$, but with an extra constant $k$ floating around everywhere. We can expand the definition of $\ang{s^{k}}_{m}$ near $t_c$ using our scaling form assumption:
	\begin{align*}
		\ang{s^{k}}_{m} &= \int_{0}^{\infty} s^{k} Q_{m}(s) \;ds \\
				&= \int_{0}^{\infty} s^{k + 2m-m\tau - 1} g(s \delta^{1\sigma}) \;ds.
				\intertext{Then making the change of variable $s = x \delta^{-1/\sigma}$, this becomes}
				&= \delta^{-\frac{2m - m\tau + k}{\sigma} } \int_{0}^{\infty} x^{2m - m\tau -1 + k} g(x) \;dx.
	\end{align*}
	Assuming this integral is finite, then the identity $\gamma_{Q} = (2m-m\tau+1)/\sigma$ (see Theorem 1) means that
	\[
		\ang{s^{k}}_{m} \sim \delta^{-\left( \gamma_{m}+\frac{k-1}{\sigma} \right)}.
	\] 
\end{proof}

Showing that all terms in each different $\p_{t}{S} $ and $\p_{t}{\ang{s}_{P}} $ have the same $\delta$ coefficient is then just a tedious exercise in algebra. I found it helpful to use our identities for the different standard coefficients in terms of $\beta$, which we have already derived for \ER, da Costa, and the adjacent edge rule.

\begin{ex}[Adjacent Edge]
For reference, the adjacent edge rule has
\begin{align*}
	\p_{t}{S} &= \ang{s}_{P}S^2 + S\ang{s}_{Q}, \\
	\p_{t}{\ang{s}_{P}} &= 2\ang{s}_{P}\ang{s}_{Q} - \ang{s^2}_{P}S^2 - \ang{s^2}_{Q}S,
\end{align*}
and the various identities we'll be using all over the place are
\[
\frac{1}{\sigma} = 2\beta+1, \quad\quad \gamma_{P} = \beta+1, \quad\quad \gamma_{Q}=1,
\] which I derived in section 0.4. Now we compute the orders for the terms in $\p_{t}{S} $:
\begin{align*}
	\ang{s}_{P}S^2 &:\quad -\gamma_{P}+2\beta = \beta-1, \\
	S\ang{s}_{Q} &:\quad \beta-\gamma_{Q} = \beta-1.
\end{align*}
As claimed, the orders are the same (we also expected the dominating order to be $\beta-1$, so nothing here is broken). Now for $\p_{t}{\ang{s}_{P}} $ (the last two terms will require the use of the previous proposition):
\begin{align*}
	\ang{s}_{P}\ang{s}_{Q} &:\quad -\gamma_{P}-\gamma_{Q} = -\beta-2, \\
	\ang{s^2}_{P}S^2 &:\quad -\gamma_{P} - \frac{1}{\sigma} + 2\beta = -\beta-2, \\
	\ang{s^2}_{Q}S &:\quad -\gamma_{Q} - \frac{1}{\sigma} + \beta = -\beta-2.
\end{align*}
So they're also all the same, and they're the expected order since we know the dominating order is $-\gamma_{P}-1 = -\beta-2$.
\end{ex}
{\color{blue}When does this hold in general?}

%%%%%%%%%%%%%%%%%%%%
% Uniform Scaling Rules
%%%%%%%%%%%%%%%%%%%%

\section{Uniform Scaling Rules}

Motivated by the previous section, it makes sense to define a class of rules where all the terms in $\p_{t}{S} $ and in $\p_{t}{\ang{s}_{m}} $ have the same order. I think \textbf{uniform scaling rule} is a nice name, so that's what I'm sticking with.

So suppose $\mathcal{R}$ is a uniform scaling rule with ODE $\p_{t}{P}(s) $. If the derivative of $S$ is in terms of the $\ang{s}_{m}$ and $S$, then
\[
	\delta^{\beta-1} \sim \sum_{i=1}^{\ell} \delta^{A_i}
\] for some collection of unique coefficients $\left\{ A_i \right\}$. But since $\mathcal{R}$ scales uniformly, this gives us $\ell(\ell+1)/2$ relations between the various coefficients in play. Note that each $A_i$ is some linear combination of $\tau, \sigma, \{\gamma_{m}\}_m, \beta$. So if we have at most $1 + \ell(\ell+1)/2$ of these unknowns, then we can solve for them all in terms of $\beta$ \warn{(this makes sense to me, at least)}.

If $\p_{t}{\ang{s}_{P}} $ is also in terms of the $\ang{s}_{m}$ and $S$, then
\[
\delta^{-\gamma_{P}-1} \sim \sum_{i=1}^{k} \delta^{B_{i}}
\] for some collection of unique coefficients $\left\{ B_i \right\}$. There's a similar condition for when we're guaranteed to be able to solve for all our unknowns in terms of $\beta$.

\begin{ex}[Adjacent Edge]
For reference, the adjacent edge rule has
\begin{align*}
	\p_{t}{S} &= \ang{s}_{P}S^2 + S\ang{s}_{Q}, \\
	\p_{t}{\ang{s}_{P}} &= 2\ang{s}_{P}\ang{s}_{Q} - \ang{s^2}_{P}S^2 - \ang{s^2}_{Q}S.
\end{align*}
Under the assumption that the adjacent edge rule scales uniformly, the first equation allows us to derive $\gamma_{P}=\beta+1$ and $\gamma_{Q}=1$, and the second equation gives us enough extra information to derive $1/\sigma = 2\beta+1$ (it also has enough information to derive the first two identities, as well).
\end{ex}

The obvious question is then: \warn{when are a rule's $\p_{t}{S} $ and $\p_{t}{\ang{s}_{P}} $ in terms of only $S$ and the $\ang{s}_{m}$? If we know this, then any uniform scaling rule must obey the properties above, and we can guarantee that it can be solved.}


\end{document}
